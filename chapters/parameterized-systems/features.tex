%% ====================================================================
\section{Features}
\label{section:paramsys:features}
\KW{Features}%
\index{Parameterized Systems!Features}%
%% ====================================================================
% 
The parallel composition of several processes can be further
characterized with the following features, which make the task of
parameterized verification complex and interesting: %
\begin{itemize}
\item %[(i)]
  the topology,
\item %[(ii)]
  the way the process can communicate with each other and
\item \index{Finite-state}\index{Infinite-state}%[(iii)]
  the nature of each process itself, whether it is finite-state or
  not. In this chapter, %thesis
  each component is however restricted to be finite-state. %
  %, though they actually can be infinite-state.
\end{itemize}

\smallskip
%% ********************************************************************
\KW{Topology}%
\index{Topology}%
\index{Topology!linear}%
\index{Topology!ring}%
\index{Topology!tree}%
\index{Topology!multiset}%
%% ********************************************************************
The topology describes the way the processes are arranged and
implicitly how they can refer to each other, without necessarily
revealing their identity. %
% 
For the \emph{linear} topology, processes are organized in an array
and can distinguish between %any of
their left and right neighbors. %
For a \emph{ring}, they can inspect their immediate neighbor, while
for a~\emph{tree}, they inspect their parent and/or children
processes. %
Finally, the case where there is no particular structure and where
processes can refer to any other processes is called a \emph{multiset}
topology.
 
%% Topologies
%{\noindent\centering \tikzinput{img/topologies}\par}
%
\begin{wrapfigure}{r}{0.58\linewidth}
  \hfill%
  \tikzinput{img/topologies}
  \vspace{-1em}
\end{wrapfigure}
%
%% ********************************************************************
\KW{Global conditions}%
\index{Global Conditions}%
\index{Global Transitions}%
%% ********************************************************************
Processes can interact with each other and perform actions potentially
in any order or % and
simultaneously. %
Those actions are conditioned on the status of the other processes:
Before it performs its action, a process can inspect other processes
according to the topology, and their status can allow or prevent the
action from the process.
% 
We refer to these transitions as being \emph{guarded} by
a~\emph{global condition}, or just \emph{global transitions}. %
% 
For example, in a linear topology, a process (at position~$i$) may be
able to perform a transition only if all processes to its left
(i.e. with index~$j<i$) satisfy a particular property.
%
Sometimes, it is only required for \emph{some} rather than \emph{all},
as depicted in Figure~\ref{figure:example:transition}.
% These transitions are also called existentially and
% universally guarded transitions.
% 
%% ********************************************************************
\KW{Local transitions}%
\index{Local Transitions}%
%% ********************************************************************
On the other hand, it is occasionally not necessary to refer to other
processes at all before performing an action. %
% It is the case when a process performs its action regardless of the
% status of the other processes.
% 
These actions are by opposition called \emph{local transitions}.

\begingroup%
\setlength\intextsep{\dazintextsep}
\begin{figure}[b]%hb
  \begin{minipage}{0.56\linewidth}
    \noindent%
    \tikzinput{img/transition}
  \end{minipage}\begin{minipage}{0.44\linewidth}
    \caption{Example of global transition for a linear topology:
      a~process in state~\s[i]{s} may change to state~\s[i]{d} provided
      that there exists another process in state~\s[witness]{w} on
      its~right.}
    \label{figure:example:transition}
  \end{minipage}
\end{figure}
\endgroup


%% ********************************************************************
\KW{Non-atomic global conditions}%
\index{Global Conditions!non-atomic}%
%% ********************************************************************
A major issue %with such a transition system
is that global conditions are not necessarily checked atomically. %
% 
In case other processes can perform transitions, while a global
condition check is carried out, the system must in fact distinguish
intermediate states.
%
\index{State-space!explosion}%
These interleavings make the state-space grow even further.
%
We will see that the method from
Chapter~\ref{chapter:view:abstraction} %(page~\pageref{chapter:view:abstraction})
can handle elegantly this complex setting.
%, for which other techniques have troubles.

%% ********************************************************************
\KW{Communi-cation primitives}%
\index{Communication primitives}%
%% ********************************************************************
% Finally, besides local and global transitions, there are other types of
% \emph{communication primitives}, i.e.\ rules that a process must
% follow in order to perform an action, depending on the topology or
% not.
Apart from local and global transitions,
% belong to the class of \emph{asynchronous communication primitives}.
%
we complete the list with the other types of transitions, which might
depend on the topology or not.
%there are other rules that a process must follow in order to perform
%an action, depending on the topology or not.
% 
%% ********************************************************************
\KW{Broadcast and Rendez-vous}%
\index{Broadcast Transitions}%
\index{Rendez-vous Transitions}%
%% ********************************************************************
For a \emph{broadcast transition}, a process is forced to change state
synchronously with an arbitrary number of processes.
%
For a \emph{rendez-vous} transition, it is only required from one
extra process. %
%% ********************************************************************
\KW{Shared variable update}%
\index{Shared variable update}%
%% ********************************************************************
For \emph{shared variable} update, a~process communicates the updated
value to the other processes.
%(often modeled as a broadcast).
%% ********************************************************************
\KW{Process creation and deletion}%
\index{Process creation and deletion}%
%% ********************************************************************
Finally, process \emph{creation} and \emph{deletion} make the topology
dynamic, but it is often not a major issue, %
%as we will see in later chapters.
as shown in Paper~\ref{paper:VMCAI13}.
