
%% ====================================================================
\section{Formal Definition}
\label{section:paramsys:definition}
\index{Parameterized Systems!Formal definition}%
%% ====================================================================
% 
\index{Finite-state systems}%
\index{Topology!linear}%
To simplify the presentation, we will only focus, in this chapter, on
the case where processes are governed by a finite-state automaton and
organized in a~linear topology.
% 
Other topologies are presented in paper~\ref{paper:VMCAI13},
\ref{paper:MCC08} and \ref{paper:CAV08}.

A parameterized system is a pair $\parsys=(\locs,\rules)$ where
$\locs$ is a finite set of \emph{local states} of a process and
$\rules$ is a set of \emph{transition rules} over $\locs$. %
A transition rule is either {\it local} or {\it global}.
% 
A local rule is of the form $\src\trans\dst$, where the process
changes its local state from $\src$ to $\dst$ independently from the
local states of the other processes.
% 
\index{Universal transition}%
\index{Existential transition}%
A global rule is either \emph{universal} or \emph{existential}.
%
Recall that a global rule depends on the topology, %
% and the communication primitives, 
so it is here of the form:
$$\quantrule{\src}{\dst}{\quantify}{\sim}{\witnesses}$$
% 
where $\quantify$ is either $\exists$ or $\forall$, for existential
and universal conditions respectively, where $\sim$ is either $<$, $>$
or $\neq$, to indicate which processes are concerned, and %finally
where $S$ is a subset of $\locs$, describing the state of the other
\emph{witness} processes.
% 
\index{Global Transitions!source}%
\index{Global Transitions!destination}%
\index{Global Transitions!range}%
\index{Global Transitions!quantifier}%
We call $\src$ the~\emph{source}, $\dst$ the~\emph{destination},
$\quantify$ the~\emph{quantifier} and $\sim$ the~\emph{range}.
%
Informally, the $i^{th}$ process checks the local states (among
$\witnesses$) of the other processes when it makes the move.
%
%  (here only
% \emph{atomically}, i.e.\ the other processes do not change state while
% the global check is carried out).
% For the sake of presentation, we only consider, in this section, a
% version where each process checks \emph{atomically} the other
% processes. The more realistic and more difficult case, where the
% atomicity assumption is dropped, will be introduced in
% Section~\ref{section:non-atomic}.
We consider, in this section only, a version where each process checks
\emph{atomically} the other processes. The more realistic and more
difficult case, where the atomicity assumption is dropped, will be
introduced in the next section.
%
For instance, the condition $\forall\,j<i:\,\witnesses$ means that
“every process $j$, with a lower index than $i$, should be in a local
state that belongs to the set $\witnesses$.”
%
The condition $\exists\,j>i:\,\witnesses$ means that “there should be
a process $j$ with higher index than $i$ with a local state listed in
the set $\witnesses$”, etc\ldots%
% the condition
% $\forall\,j{\neq}i:\,\witnesses$ means that “the local state of all
% processes, except the one at position~$i$, should be in the set
% $\witnesses$”.

%% ********************************************************************
%\smallskip%
\bigskip%
\KW{Configurations}%
\index{Configuration}%
%% ********************************************************************
A \emph{configuration} in $\parsys$ is a word over the
alphabet~$\locs$, i.e.\ an array of process states.
%
We use \confs\ to denote the set of all configurations.
%
For a configuration~$c\in\confs$, we use $c[i]$ to denote %
the state at position~$i$ in the array
%the state of the $i^{th}$ process within the configuration~$c$ 
and $\sizeof c$~for its size.
%
% We use~$c[i]$ to denote the state at position~$i$ in the array and
% $\sizeof c$~for the size of the configuration.
%
We use $\range a b$ to denote the set of integers in the
interval~$[a{,}b]$ (i.e.\ $\range{a}{b} = [a{,}b]\cap\nat$).

%% ********************************************************************
%\paragraph{Successors.}
\KW{Successors}%
\index{Successors}%
\index{Post-image operator}%
%% ********************************************************************
%
For a configuration~$c\in\confs$, a position $i\leq\sizeof c$, and a
rule $\delta\in\rules$, we define how to transform the
configuration~$c$ into another configuration if we allow the
$i^{th}$ process in the array to perform the transition $\delta$.
%
% We call $c' = \delta(c,i)$ the immediate successor of $c$ under a
% $\delta$-move of the $i^{th}$ process.
%
Formally, we define the immediate successor of $c$ under a
$\delta$-move of the $i^{th}$ process, such that $\delta(c,i)=c'$ if
and only if $c[i] = \src$, $c'[i] = \dst$, $c[j]=c'[j]$ for all
$j:j\neq i$ and either (i)~$\delta$ is a local rule $\src\trans\dst$,
or (ii)~$\delta$ is a global rule %of the form
$\quantrule{\src}{\dst}{\quantify}{\sim}{\witnesses}$, and one of the
following two conditions is satisfied: %
\begin{itemize}
\item %
  $\quantify=\forall$ and for all $j\in\range{1}{\sizeof{c}}$ such
  that $j\sim i$, it holds that $c[j]\in \witnesses$,%
\item %
  $\quantify=\exists$ and there exists some
  $j\in\range{1}{\sizeof{c}}$ such that $j\sim i$ and
  $c[j]\in \witnesses$.
\end{itemize}
%
%
% There are two cases to consider, since $\delta$ is either a local or a
% global transition.
% %
% In the case where $\delta$ is a local transition, say, of the form
% $\src\trans\dst$, and $c[i] = \src$, we define $c'$ with $c'[i] =
% \dst$ and $c[j]=c'[j]$ for all $j:j\neq i$. 
% %
% On the other hand, when $\delta$ is a global transition, say, of the
% form $\quantrule{\src}{\dst}{\quantify}{\sim}{\witnesses}$, and $c[i] = \src$, we
% define $c'$ as follows: $c'[i] = \dst$ and $c[j]=c'[j]$ for all
% $j:j\neq i$ only if one of the following conditions is satisfied:
% \begin{itemize}
% \item $\quantify$ is $\forall$ and for all $j:1\leq j\leq |c|$ such
%   that $j\sim i$, it holds that $c[j]\in \witnesses$ %or
% \item $\quantify$ is $\exists$ and there exists $j:1\leq j\leq |c|$
%   such that $j\sim i$ and $c[j]\in \witnesses$.
% \end{itemize}
%
Note that $\delta(c,i)$ is not defined if $c[i]\neq\src$.
%
We use \mbox{$c\transof{\delta}c'$} when $c'=\delta(c,i)$ for some
$i\leq\sizeof c$, and $c \trans c'$ if $c\transof{\delta}c'$ for some
$\delta\in\rules$.
%
We define $\transof{*}$ as the reflexive transitive closure of
$\trans$ (i.e.\ the result of repeatedly applying $\trans$).
%
% For a~set of configurations $X\subseteq\confs$, we define the
% \emph{post-image} of $X$ as the set %
% $$
% \post(X) = \setcomp{\delta(c,i)}{c\in X, i\leq\sizeof{c},\delta\in\Delta} = \setcomp{c'}{c\in X, c\trans c'}
% $$
% % $$
% % \begin{array}{rcl}
% % \post(X) & = & \setcomp{\delta(c,i)}{c\in X, i\leq\sizeof{c},\delta\in\Delta} \\
% %          & = & \setcomp{c'}{c\in X, c\trans c'}
% % \end{array}
% % $$
