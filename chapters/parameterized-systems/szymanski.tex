%% ====================================================================
\section{A Challenging Example}
\label{section:paramsys:example}
%% ====================================================================
%
We illustrate the notion of a parameterized system with the example of
Szymanski's mutual exclusion protocol~\cite{Szymanski88,Szymanski90}.
\index{Szymanski}%
\index{Szymanski!Mutual exclusion protocol}%
%
% We introduce the mutual exclusion protocol designed by Dr.\ Boleslaw
% Szymanski~\cite{Szymanski88,Szymanski90}.
Among other properties, this protocol ensures exclusive access to a
shared resource in a system consisting of an unbounded number of
processes organized in an array.
%
\index{Critical section}%
The \emph{critical section} is the portion of code in the program
which threads are allowed to execute one at a time.
%
The source code for each process is presented in
Figure~\ref{figure:szymanski:implementation}. For Szymanski's
protocol, the critical section is composed of the statements on line 9
and 10.

Each process participating in a session of the protocol is represented
at a fixed position in the array with a local variable \texttt{flag}
and how far it has proceeded in its execution.
%
We encode the state of the $i^{th}$ process with a number, which
reflects the values of the program location and the local variable
$\mathit{flag}[i]$.

% The topology and communication primitives, together with the
% transition diagram, \emph{induce} a model of the system.
%
A~configuration of the induced transition system is a word over the
alphabet $\set{\w[i]{0}, \w[n]{1}, {\ldots}, \w[b]{11}}$ of local
process states.
%
The size of a configuration is the parameter of the system.
%
The initial configurations characterize the program before any
execution, e.g.\ using the initial values of the program variables.
% and structures. 
Here, all processes are initially in state\,\w[i]{0}, i.e.\
$\Inits~=~{\w[i]{0}}^+$.
%
The bad configurations are derived from the targeted property.
%
For a mutual exclusion protocol, a~configuration is considered to be
bad if it contains two occurrences of state\,\w[b]{9} or \w[b]{10},
i.e.\ at least two processes are in their critical section
simultaneously.
% --- here the states~\w[b]{9}~or~\w[b]{10}.
In other words, the bad configurations belong to an
infinite set characterized by the following patterns: %
{\badpattern{9}{9}}, %\mid
{\badpattern{9}{10}}, %\mid
{\badpattern{10}{9}} and %\mid
{\badpattern{10}{10}}.

\begin{wrapfigure}{r}[0pt]{0.61\linewidth}
  \vspace{-3pt}
  %\centering
  \hfill
  \tikzinput{experiments/szymanski-diagram}
  \caption{Szymanski's protocol transition system}
  \label{figure:szymanski:transitions}
  \label{figure:szymanski:diagram}
\end{wrapfigure}
%
A transition moves a process from a state to another, provided that
the other processes respect the global condition. %condition guard. %
Here, for example, processes move from state~\w{1} to \w{2}, if the
other processes are all in state $\w{0}, \w{1}, \w{2}, \w{5}$
or~$\w{6}$. % Using the math env, in order to not break the , and . afterwards.
If not, the process stays in state~$\w{1}$.

The intuition that Szymanski gives is the presence of a ``waiting
room'', with an entering and exiting door, before processes move into
their critical section.
%
The transition rules are depicted in
Figure~\ref{figure:szymanski:transitions} using a simple graphical
representation, often called a~transition diagram: The label on an
edge represent the global conditions that other processes must respect
in order to perform the transition. There is no label when the
transition is local.
%
For example, consider the edges between the nodes %
$\w{3}, \w{4}\text{ and }\w{7}$. %
They depict the transition as in Figure~\ref{figure:non:atomic}\\%
\parbox{\textwidth}{\centering $\forrule{\neq}{\neg\set{1,2}}{3}{7}{4}$.}

% Once the first process is in state\,\w{2}, it “closes the door”
% on the processes which are still in\,\w{0}. They can no longer
% leave state\,\w{0} until the door is opened again (when no
% process is in state\,\w{2} or \w{3}). %
% %
% Moreover, a process is allowed to cross from state\,\w{3} to
% \w{4} only if there is at least one process still in
% state\,\w{2} (i.e., the door is still closed on the processes
% in state\,\w{0}). %
% %
% This prevents a process to first reach state\,\w{4} along with
% a process to its left starting to move from\,\w{0} all the way
% to state\,\w{4} (thus violating mutual exclusion). %
% From the set of processes which have left state\,\w{0} (and
% which are now in state\,\w{1} or\,\w{2}), the leftmost
% process has the highest priority and it is encoded in the global
% condition: a process may move from \w{2} to\,\w{3}
% only if all processes on its left are in state\,\w{0}.

% \begin{figure}[!hb]
%   \centering
%   \tikzinput{experiments/szymanski-diagram}
%   \caption{Szymanski's protocol transition system}
%   \label{figure:szymanski:transitions}
% \end{figure}

\bigskip%
% ---------------------------------
\index{Parameterized Verification}%
The task is to check that the protocol guarantees exclusive access to
the shared resource regardless of the number of processes, i.e.\ to
show that the bad configurations are not reachable from the initial
ones.
%
% ---------------------------------
Many techniques~%
\cite{AbHaHo:view:abstraction,AbDeRe:context:sensitive,Parosh:non-atomic,CTV06,Namjoshi:VMCAI07,AJNO:simple,BHV04}
have been used to verify automatically the safety property of
Szymanski’s mutual exclusion protocol but only in restricted
settings. They either assume atomicity of the global conditions and/or
only consider a~more compact variant of the protocol.
% (i.e.\ where the invariant can be expressed solely by
% a~downward-closed set).
%
\index{Challenge}%
The full and fine-grained version has been considered
a~challenge in the verification community.
%
\index{Automation}%
To the best of our knowledge, this thesis presents the first technique
to address the challenge of verifying the protocol fully automatically
without atomicity assumption.

% \begin{figure}[!hb]
%   \centering
%   \tikzinput{experiments/szymanski-diagram}
%   \caption{Szymanski's protocol transition system}
%   \label{figure:szymanski:transitions}
% \end{figure}

\begingroup%
\setlength\intextsep{\dazintextsep}
\begin{figure}[!b]%[!b]
  \centering
  \listinginput{experiments/code/szymanski.txt}
  \caption{Implementation of Szymanski's protocol (for process $i$)}
  \label{figure:szymanski:implementation}
\end{figure}
\endgroup

