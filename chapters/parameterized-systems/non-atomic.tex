%% ====================================================================
\section{Non-Atomic Global Conditions}
\label{section:non:atomic:global:conditions}
\label{section:paramsys:non:atomic}
\index{Global Conditions!non-atomic}%
%% ====================================================================
We extend our model to handle parameterized systems where global
conditions are \emph{not} checked atomically.
%
We replace both existentially and universally guarded transitions by
the following variant of a~for-loop rule%
% \\\parbox{\linewidth}{\centering
%   $\forrule{\sim}{\witnesses}{\src}{\dst}{e}$}\\%
$$
\forrule{\sim}{\witnesses}{\src}{\dst}{e}
$$
%
where $e\in Q$ is an \emph{escape} state and the other labels $\src$,
$\dst$, $\sim$ and $\witnesses$ are as in the previous section.
%~\ref{section:paramsys:definition}.
%
% For instance, line 3 of Szymanski's protocol is be replaced by
% $\forrule \neq {\neg\set{1,2}} 3 7 4$. 
%
Essentially, for a configuration with linear topology, a~process at
position~$i$ inspects the state of another process at position~$j$,
\emph{in-order}.
%
Without loss of generality, we will assume that the for-loops iterate
through process indices in increasing order.
%
If the state of the process at position~$j$ does not belong to
$\witnesses$, process~$i$ escapes to state~$e$. Otherwise, process~$i$
moves on to inspect the process at position~$j+1$, unless there is no
more process to inspect in which case process~$i$ completes its
transition.
%
This construction allows us to emulate both existential and universal
transitions by adjusting the set~$\witnesses$ and choosing the right
combination of $\src/\dst/e$ states.
%
% For universal transitions, the inspected process at position $j$ must
% all belong to the set~$\witnesses$. For existential transitions, we
% use the complement of~$\witnesses$.

%% -------------------------------------------------
%\myparagraph{Concrete semantics.}
%% -------------------------------------------------
%
% We extend the semantics of a~system with for-loop rules from
% transition systems of the previous section %~\ref{section:paramsys}
% in the following way: %
The transition systems of the previous
section %~\ref{section:paramsys}
are extended with for-loop rules in the following way: %
%
A configuration is now a pair $c = (q_1\cdots q_n,\tick)$ where
$q_1\cdots q_n\in Q^+$ is a~word as before and where $\tick:\range 1 n
\rightarrow \range 0 n$ is a total map which assigns to every
position~$i$ of~$c$ the last position which has been inspected by the
process~$i$.
%
Initially, $\tick(i)$ is assigned~$0$.

% --------------------------------
We fix a rule $\delta = \forrule \sim \witnesses {s} {t} {e}$ from
$\rules$, a configuration~$c$ with $\sizeof{c}=n$, and
$i\in\range{1}{n}$.
% 
We first define the position $\next(c,i)$ which the process at
position~$i$ is expected to inspect next. Formally, $\next(c,i) =
min\setcomp{j\in\range{1}{n}}{j>\tick(i), j\sim i}$ is the smallest
position larger than~$\tick(i)$ which satisfies $\next(c,i)\sim i$.
% 
Notice that if process~$i$ has already inspected the right-most
position~$j$ which satisfies $j\sim i$, then (and only then)
$\next(c,i)$ is undefined.

We distinguish three types of $\delta$-move on~$c$ by the process at
position~$i$:
(i)~$\delta^{it}(c,i)$ for a~loop iteration, %
(ii)~$\delta^{e}(c,i)$ for escaping and %
(iii)~$\delta^{t}(c,i)$ for termination. 
% 
Each type of move is defined only if $q_i = s$.
%
% The illustrations in Figure~\ref{figure:non:atomic} correspond to the
% transition \\%
% \parbox{\textwidth}{\centering $\forrule \neq {\neg\set{1,2}} 3 7 4$.}

\begingroup%
%\setlength\intextsep{\dazintextsep}
\setlength\intextsep{\the\medskipamount}
\begin{figure}[!b]%hb
  \centering
  \tikzinput{img/ticks-iteration}\hfill%
  \tikzinput{img/ticks-escape}\hfill%
  \tikzinput{img/ticks-terminal}
  %\caption{Consider for example the transition $\forrule \neq {\neg\set{1,2}} 3 7 4$.}
  \caption{$\forrule \neq {\neg\set{1,2}} 3 7 4$}
  \label{figure:non:atomic}
\end{figure}
\endgroup

\newpage
\begin{enumerate}[label={(\roman{*})}]%leftmargin=0cm,itemsep=0pt
\item %
  $\delta^{it}(c,i)$ is defined if $\next(c,i)$ is defined and
  $q_{\next(c,i)}\in\witnesses$. It is obtained from~$c$ by updating
  $\tick(i)$ to~$\next(c,i)$. Intuitively, process~$i$ is only
  \emph{ticking} position~$\next(c,i)$.
\item %
  $\delta^{e}(c,i)$ is defined if $\next(c,i)$ is defined and
  $q_{\next(c,i)}\not\in S$. It is obtained from~$c$ by changing the
  state of the process~$i$ to~$e$ and resetting $\tick(i)$
  to~$0$. Intuitively, process~$i$ has found a~reason to escape.
\item %
  $\delta^{t}(c,i)$ is defined if $\next(c,i)$ is undefined, and it is
  obtained from~$c$ by changing the state of the process~$i$ to~$t$
  and resetting $\tick(i)$ to~$0$. Intuitively, process~$i$ has
  reached the end of the iteration and terminates its transition
  (i.e.\ moves to its target state).
\end{enumerate}

% \begin{enumerate}[leftmargin=0cm,label={},itemsep=0pt]
% \item %
%   %\parbox{\linewidth}{
%   \marginpar{\raggedleft (i)}%
%   \parbox[c]{.7\linewidth}{\vspace{-1.2em}%
%     $\delta^{it}(c,i)$ is defined if $\next(i)$ is defined and
%     $q_{\next(i)}\in\witnesses$. It is obtained from~$c$ by only
%     updating $\tick(i)$ to~$\next(i)$. Intuitively, process~$i$ is
%     only \emph{ticking} position~$\next(i)$.
%   }\hfill\parbox{.29\linewidth}{\hfill\tikzinput{img/ticks-iteration}}
%   %}
% \item %
%   % \parbox{\linewidth}{
%   \marginpar{\raggedleft (ii)}%
%   \parbox{.29\linewidth}{\hfill\tikzinput{img/ticks-escape}}\hfill%
%     \parbox[c]{.7\linewidth}{\vspace{-.7em}$\delta^{e}(c,i)$ is defined if $\next(i)$
%       is defined and $q_{\next(i)}\not\in S$. It is obtained from~$c$
%       by changing the state of the process~$i$ to~$e$ and resetting
%       $\tick(i)$ to~$0$. Intuitively, process~$i$ has found a~reason
%       to escape.}
%   %}
% \item %
%   % \parbox{\linewidth}{
%   \marginpar{\raggedleft (iii)}%
%   \parbox[c]{.68\linewidth}{\vspace{-.1em}%
%     $\delta^{t}(c,i)$ is defined if $\next(i)$ is undefined, and it is
%     obtained from~$c$ by changing the state of the process~$i$ to~$t$
%     and resetting $\tick(i)$ to~$0$. Intuitively, process~$i$ has
%     reached the end of the iteration and terminates its transition
%     (i.e.\ moves to its target state).
%   }\hfill\parbox{.31\linewidth}{\hfill\tikzinput{img/ticks-terminal}}
%   %}
% \end{enumerate}
