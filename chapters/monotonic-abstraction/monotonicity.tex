%% ====================================================================
\section{Monotonicity}
%% ====================================================================
\KW{Monotonicity}%
\index{Monotonicity}%
%% ********************************************************************
Monotonicity is a mathematical notion that comes from calculus,%
\footnote{In calculus, a function $f$ defined on a subset \domain\ of
  the real numbers \reals\ is called monotonic if it is either
  entirely increasing \emph{or} decreasing. It is called increasing
  (resp.\ decreasing), if for all $x$ and $y$ in \reals\ such that $x
  \leq y$, it holds that $f(x)\leq f(y)$ (resp.\ $f(x)\geq f(y)$). So
  $f$ either preserves the order, or reverses the order, consistently
  over the domain \domain .} %
which deals with functions over sets that preserve a given preorder
$\preorder$. For a monotonic function $f$ over a domain \domain ,
whenever two elements $a$ and $b$ in \domain\ are ordered such that
$a\preorder b$, it holds that $f(a) \preorder f(b)$.

%% ********************************************************************
\KW{Monotonic Systems}%
%% ********************************************************************
%
\noindent%
\begin{wrapfigure}{r}[0pt]{0.45\linewidth}
  \centering
  \tikzinput{img/monotonicity}
  \caption{Monotonicity of the rule~$\delta$ with respect to the
    subword relation~$\subword$.}
  \label{figure:monotonicity}
\end{wrapfigure}
%
A parameterized system \mbox{${\parsys}=(\locs,\rules)$} is
\emph{monotonic} (with respect to preorder $\subword$),
% from Section~\ref{section:upward:closed})%
if for each rule $\delta\in\rules$ and configurations $c_1$, $c_2$ and
$c_3$, such that $c_1\transof{\delta} c_2$ and $c_1\subword c_3$, then
there exists a configuration $c_4$ such that $c_3\transof{\delta} c_4$
and $c_2\subword c_4$. That is to say, if we can fire a transition on
a configuration, we can also fire it on a larger configuration, and
the results are ordered accordingly.
%
This is illustrated in Figure~\ref{figure:monotonicity}.
% For a parameterized system \mbox{${\parsys}=(\locs,\rules)$}, if the
% rules from \rules\ are monotonic with respect to preorder $\subword$
% from Section~\ref{section:upward:closed}, it is possible to transform
% an upward-closed set of configurations into another upward-closed set.
% %
% %In other words, the upward-closedness is preserved and
% The parameterized system \parsys\ is called \emph{monotonic} w.r.t.\
% $\subword$.
% %
% Indeed, consider a monotonic rule $\delta\in\rules$ and the
% configurations $c_1$, $c_2$ and $c_3$, such that $c_1\transof{\delta}
% c_2$ and $c_1\subword c_3$, then there exists a configuration $c_4$
% such that $c_3\transof{\delta} c_4$ and $c_2\subword c_4$. That is to
% say, if we can fire a transition on a configuration, we can also fire
% it on a larger configuration, and the results are ordered accordingly.

In~\cite{Parosh:Bengt:Karlis:Tsay:general}, it is shown that for
monotonic systems, the upward-closedness is preserved when computing
predecessors, that is, if a set is upward-closed, so is its
pre-image. This is an important property that we use in the next
section.

%% ********************************************************************
\KW{Universal not monotonic}%
\index{Non-Monotonicity}%
%% ********************************************************************
However, systems are not always monotonic.
%
This is for example the case if \rules\ contains an universally
quantified global transition. For example, if $\delta\in\rules$ is of
the form $\quantrule{s}{d}{\forall}{\neq}{\set{w}}$, the subword
relation is not necessarily preserved by $\delta$ as depicted in
Figure~\ref{figure:universal:not:monotonic}.

\noindent%
\begin{wrapfigure}{l}[0pt]{0.6\linewidth}
  %\centering
  \tikzinput[\linewidth]{img/universal-not-monotonic}
  %\setlength{\abovecaptionskip}{-1em plus 0.3em}
  \caption{Universal transitions are not monotonic. Therefore, we add
    transitions and over-approximate the transition relation
    $\trans$.}
  \label{figure:universal:not:monotonic}
  \smallskip
\end{wrapfigure}
%
%% ********************************************************************
\KW{Reachability for MTS is decidable}%
\index{Reachability}%
\index{Decidability}%
%% ********************************************************************
It has been shown that, for monotonic systems, the problem of
determining if a set of bad configurations is reachable from some
initial configurations is decidable~\cite{MTSdecidable}, 
%despite an infinite state-space.
even though the state-space is infinite.

%% ********************************************************************
\KW{Monotonic Abstraction}%
\index{Abstraction}%
\index{Over-approximation}%
%% ********************************************************************
Therefore, the strategy is to make the system monotonic by introducing
an over-approximation.
% The new abstract system will contain more
% transitions. All transitions from the original system are preserved.
Existential and local transitions are preserved from the original
system as-is, since they are monotonic.
%
Universal transitions, however, are not and we over-approximate them
using the scenario depicted in
Figure~\ref{figure:universal:not:monotonic}.
%
Informally, for every rule $\delta\in\rules$, we define a new
successor function $\abstrans\delta$ such that $\abstrans\delta$
coincide with $\delta$ if $\delta$ is a local transition or an
existential global transition. In the case where $\delta$ is an
universal global transition, %
% say, of the form
%$\quantrule{\src}{\dst}{\forall}{\sim}{\witnesses}$
we define $\delta$ in the following manner (here only giving the
intuition):
%
We first remove the process states from the configuration, if they
violate the guard, i.e.\ if they don't respect the global condition
from the rule (which effectively disables the transition).
%
This creates a potentially smaller configuration on which we can apply
$\delta$ as usual. We say that we ``go down on the ordering'' before
applying~$\delta$.

The successor functions $\abstrans\delta$ form a new set
$\abstrans\rules$ and the abstract parameterized system %
$\abstrans\parsys = (\locs,\abstrans\rules)$ induces a transition
system $(\confs,\atrans)$ that is now monotonic and preserves the
order $\subword$ (as depicted in
Figure~\ref{figure:abstract:monotonic}).
%
It is decidable to prove that the bad configurations are not reachable
from some initial ones in the new abstract transition system
$\abstrans\parsys$. Since it is an over-approximation, it contains the
original transition system $(\confs,\trans)$ and proving the abstract
system safe implies that the original parameterized system $\parsys$
is also safe.

\begingroup%
\setlength\intextsep{\dazintextsep}
\begin{figure}[b]
  \smallskip
  \centering
  \tikzinput{img/abstract-monotonic}
  \caption{The abstract transition relation $\atrans$ is monotonic.}
  \label{figure:abstract:monotonic}
\end{figure}
\endgroup
