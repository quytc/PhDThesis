\begin{description}
\item[Upward-closed sets.] Monotonic abstraction relies on a preorder
  which allows us to symbolically and compactly represent sets of
  configurations, rather than individual configurations.
\item[Monotonic systems] enjoy the property that firing a transition
  on a configuration can always done on a bigger configuration
  (w.r.t.\ the preorder) and the images are ordered accordingly. %
  Moreover, reachability of some sets is decidable for such systems.
\item[Monotonic Over-Approximation] is used to transform non-monotonic
  systems into abstract monotonic systems. This allows us to define
  the pre-image that preserves the upward-closedness of sets.
\item[Bad configurations] often follow a simple pattern and the set of
  all bad configurations turn out to be upward-closed.
\item[Backward Reachability] is interesting to compute the
  predecessors of the bad set.
\item[Well-Quasi Orderings] imply that there is no infinite sequence
  of incomparable elements. Therefore, any upward-closed set can be
  fully-characterized by a finite set of minimal elements. This allows
  us to concretely implement a fully-automatic and efficient backward
  procedure that is guaranteed to terminate.
\item[Applications.] The technique derives over-approximation for
  systems arranged as linear topologies and we showed that it can
  easily be applied to multisets (with the important class of Petri
  Nets), sequential heap programs manipulating singly-linked lists and
  that it can be extended to tree topologies.
\end{description}
