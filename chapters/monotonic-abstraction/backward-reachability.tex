%% ====================================================================
\section{Backward Reachability}
\index{Reachability!Backward Analysis}%
%% ====================================================================


The procedure that will explore the state-space needs to work on
symbolic representations, rather than the configurations themselves,
in the spirit of Section~\ref{section:model:checking}. %
%
As we saw in Section~\ref{section:bad:states}, it is simple, using
upward-closure, to represent the set of all bad configurations.
%
Since we also know that a monotonic transition preserves the
upward-closedness when computing predecessors, the interesting idea is
then to use upward-closed sets as symbolic representations, define
their pre-image and run a backward analysis from the bad set (using
the abstract rules~$\abstrans\rules$). %

%\noindent%
\begin{wrapfigure}{r}[0pt]{0.35\textwidth}
  \hfill%
  \tikzinput{img/pre-image}
  %\caption{Pre-image}
  %\label{figure:pre:image}
\end{wrapfigure}
%
% There is a very important mathematical detail to catch here first. The
% computations regarding the pre-image of a configuration and the
% pre-image of an upward-closed set of configurations are different.
\index{Predecessors}%
\index{Pre-image operator}%
When computing predecessors, there is an important mathematical detail
to catch first: The pre-image of an upward-closed set is not
necessarily the same as taking the upward-closure of the pre-image of
its minimal elements.
%
For a configuration $c\in\confs$ and a transition
$\delta\in\abstrans\rules$, %
the set $\setcomp{u\in\confs}{u \atransof{\delta} c}$ of
configurations that can reach $c$ in one \mbox{$\delta$-step}, might
be empty (i.e.\ the pre-image of a given configuration does not
exist).
%
The configuration $c$ is potentially not attainable, but its
upward-closure could be.
%
Therefore the pre-image must be a computation over sets of
configurations as a whole rather than individual configurations only.
%
So we compute a slightly different set: the configurations that can
reach the \emph{upward-closure} of $c$ in one \mbox{$\delta$-step}
(not forcibly $c$ itself), that is, the set %
$\setcomp{u\in\confs}{\exists v\in\ucl c : u
  \atransof{\delta}v}$.

% For an upward-closed set $X\subseteq\confs$, we define its pre-image
% $\pre(X) = \union_{\delta\in\abstrans\rules}\inverse\delta (X)$ where
% %
% $$\inverse\delta (X)=\setcomp{u\in\confs}{\exists v\in X :
%   u\atransof\delta v}$$
More generally, for an upward-closed set $U\subseteq\confs$, we define
its $\delta$-pre-image as the set
$\inverse{\delta}(U)=\setcomp{u\in\confs}{\exists v\in U :
  u\atransof{\delta}v}$ and its pre-image as
$$\pre(U) = \union_{\delta\in\abstrans\rules}\inverse{\delta} (U)=\setcomp{u\in\confs}{\exists v\in U : u\atrans v}$$

By monotonicity, it follows that the pre-image of an upward-closed set
$U$ is also upward-closed. Notice though, that the minimal elements of
the pre-image ($\gen{\pre(U)}$) are not necessarily the inverse-image
of its minimal elements
($\union_{c\in\gen{U}}\setcomp{u\in\confs}{u\atransof{\delta}c}$).

%% ====================================================================
\subsection*{Scheme}
%\noindent\textbf{[Scheme]} %
%% ====================================================================
Given a upward-closed set of (bad) configurations, the procedure is
constructed to compute the fixpoint of the function $X\mapsto
X\cup\pre(X)$. %
%
Intuitively, the analysis computes the configurations that could reach
the bad set, by applying successively any rule from
$\abstrans\rules$. %

More precisely, the procedure computes a sequence of sets $U_0, U_1,
U_2, \ldots$ such that $U_0=\Bad=\ucl\minbad$ and for all $i\geq 0$,
$U_{i+1}= U_i \cup \pre(U_i)$. %
Every~$U_i$ represents the set of configurations that can reach \Bad\
in at most $i$ steps. By monotonicity, every~$U_i$ is upward-closed.

Observe that the sequence $(U_i)_{i\in\nat}$ is increasing, i.e.\ $U_i
\subseteq U_{i+1}$ for all $i\geq 0$. %
If the backward procedure reaches a point $n$ such that $U_n\supseteq
U_{n+1}$, it follows that $\forall m\ge n, U_m = U_n$ and the sequence
converges.
%
Consequently, %
$$\setcomp{c\in\confs}{\exists b\in \Bad :c\atransof{*} b}=U_n=\cup_{i\in\nat}U_i=\prestar(\ucl\minbad)$$

\index{Verification Scheme}%
\noindent%
\begin{wrapfigure}{r}[0pt]{0.53\linewidth}
\hfill%
\begin{algorithm}[H]
\DontPrintSemicolon
\caption{Backward Scheme}
\label{algo:backward:scheme}
$\worklist := \set{\Bad}$\;%
$\visited := \emptyset$\;%
\While{$\worklist \neq \emptyset$}{%
  remove some set $U$ from $\worklist$\;%
  \lIf{$U \cap \Inits \neq \emptyset$ \nllabel{scheme:backward:isinit}}{\Return Unsafe}%\;%
  \eIf{$U\subseteq\visited$\nllabel{scheme:backward:discard}}{%
    discard $U$ %
  }{%
    % $\worklist := \worklist \;\union\; \pre(U)$\nllabel{scheme:backward:pre}\;%
    % $\visited := U \union \visited$\nllabel{scheme:backward:minset}\;%
    add $\pre(U)$ to the worklist $\worklist$\nllabel{scheme:backward:pre}\;%
    add $U$ to the visited set $\visited$\nllabel{scheme:backward:minset}\;%
  }%
}%
\Return Safe%
\end{algorithm}
\end{wrapfigure}
%
Furthermore, if $U_n\cap\Inits=\emptyset$ holds, %
% the final result from
% the analysis does not contain any configuration that belongs to the
% set of initial configurations {\Inits}. %
% It is then obvious that 
the initial configurations cannot reach the set of bad configurations
\Bad\ and since $\abstrans\parsys$ is an over-approximation of
$\parsys$, the system $\parsys$ is safe.%
\footnote{%
  In the opposite case, it is necessary to examine whether it is a
  real error or if it is an artifact of the abstraction, and therefore
  a spurious example. %
  For spurious examples, it is possible to \emph{automatically} refine
  the abstraction and \mbox{re-run} the procedure, until it finds a
  suitable abstraction, that proves the system safe, or escapes with a
  real error~\cite{CEGAR}.%
}
%

The procedure is designed as a worklist algorithm, described in
Algorithm~\ref{algo:backward:scheme}, which manipulates upward-closed
sets. %
It takes as input an upward-closed set of bad configurations $\Bad$
and maintains two lists of %upward-closed sets: %
sets: %
(i) a list~$\worklist$ of sets of configurations that have not yet
been analyzed, initialized to $\set{\Bad}$ and %
(ii) a list~$\visited$, initially empty, that contains information
about the sets of configurations that have already been analyzed.
