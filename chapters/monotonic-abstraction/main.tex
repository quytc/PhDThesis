%% ====================================================================
\chapterwithtoc{Monotonic Abstraction}
\label{chapter:monotonic:abstraction}
\index{Abstraction}%
%% ====================================================================

We now present a technique to solve the reachability problem presented
in the previous chapter. This technique has been introduced by Abdulla
et al.~\cite{ACJT00} and is based on the Well-Quasi Ordering framework.

We first introduce the notion of ordering and monotonicity and how we
use them to derive an over-approximation. We then present the
algorithm to prove the safety property. Finally, we discuss
termination and show how we applied the method on a few protocols.
%
While the method has been applied to linear topologies~\cite{IJFCS09},
one of the contributions of this thesis is to apply it to multiset
topologies, to shape analysis and to extend it to tree topologies.

%% ====================================================================
\section{Upward-Closed Sets}
\label{section:upward:closed}
\label{section:bad:states}
%% ====================================================================
%
\index{Ordering}%
\index{Subword}%
We introduce a partial \emph{ordering} between the configurations, namely the
subword relation~$\subword$, which allows us to compare two
configurations (of potentially different size) and determine if one is
``smaller'' than the other.
%
% It is a partial ordering, which means that we sometimes cannot
% determine which one is smaller.

%% ********************************************************************
\KW{Ordering}%
%% ********************************************************************
\begin{wrapfigure}{r}{0.45\linewidth}
  \centering
  \tikzinput{img/subword}
  \caption{The subword relation}
  \label{figure:subword}
\end{wrapfigure}
%
Let $u$ and $v$ be two configurations of size $n$ and $m$
respectively, with $n\leq m$.
%
Intuitively, we would like to check if the configuration $u$ can be
``injected'' inside $v$ (or that $v$ can be ``reduced'' to $u$).
%
We first define $H_n^m$ as the set of strictly increasing injections
from $\range{1}{n}$ to $\range{1}{m}$, %
i.e.\ %such that %
for $h\in H_n^m$, $1 \leq i < j \leq n \implies 1 \leq h(i) < h(j)
\leq m$. %
(Recall $\range{a}{b} = [a,b]\cap\nat$ is the range of integers in the
interval between $a$ and~$b$).
%
Then, we write $u \subword v$ if there exists $h$ in $H_n^m$ such that
$u[i]=v[h(i)]$ for all $i\in\range{1}{n}$ (see
Figure~\ref{figure:subword}).
%
If there is no injection that can be found to compare $u$ and $v$,
they are simply incomparable.

%% ********************************************************************
\KW{Upward-closure}%
\index{Upward-closure}%
%% ********************************************************************
\begin{wrapfigure}{r}{0.18\linewidth}
  \hfill%
  \tikzinput{img/upward-closure}
  % \caption{Upward-closure}
  % \label{figure:upward:closure}
\end{wrapfigure}
%
For a configuration $c\in\confs$, we define its \emph{upward-closure}
as the set of all the configurations of any (larger) size, that at
least contain $c$ as a subword (depicted on the right).
%
$$\ucl{c}=\setcomp{u\in\confs}{c\subword u}$$
%
Notice that $c\subword c'$ implies $\ucl{c}\supseteq\ucl{c'}$, meaning
that the smaller a configuration is in the ordering, the larger set of
configurations it represents. %
We abuse the notation and for a set %$A\subseteq\confs$, we use %
$A$, we use %
$\ucl{A} = %
\union_{a{\in}A}\ucl{a} = %
\setcomp{u\in\confs}{\exists a\in A : a\subword u}$.

%\index{Upward-closed set}%
We say that a set $A\subseteq\confs$ is \emph{upward-closed} if $A =
\ucl A$.
%
%% ********************************************************************
\KW{Generators}%
\index{Generators}%
\index{Minimal elements}%
\index{Entailement}%
%% ********************************************************************
Moreover, for an upward-closed set $U$, we define the \emph{minimal
  elements} of $U$, that is, the set $M$ such that
\begin{enumerate}[label=(\roman*),leftmargin=0pt]
\item %
%(i) %
  \textbf{[Closure]} $\ucl{M}{=}U$, i.e.\ $U$ can be generated by taking
  the upward-closure of~$M$,
\item %
%and (ii) %
  \textbf{[Minimality]} $\forall a,b\in M, a\subword b \implies a=b$,
  i.e.\ all the elements of $M$ are incomparable with respect to the
  subword relation $\subword$.
\end{enumerate}
% 
The set $M$ is therefore uniquely defined for a given~$U$ and
denoted~$\gen{U}$.

% %% ********************************************************************
% \KW{Entailement}%
% %% ********************************************************************
% We define an \emph{entailement relation} $\entails$ over upward-closed
% sets of configurations such that $A\entails B$ if and only if $\forall
% b\in B : \exists a\in A : a\subword b$. Intuitively, all the
% configurations represented in the set $B$ are already represented in
% the (more ``general'') set $A$.

%% ********************************************************************
\KW{Minimal elements}%
\index{Bad configurations}%
%% ********************************************************************
Characterizing the bad configurations for the mutual exclusion
property of Figure~\ref{figure:szymanski:implementation}, is now an
easy task since they \emph{all} follow a simple pattern.
%
The smallest bad configurations for the protocol in
Figure~\ref{figure:szymanski:diagram} is the %\emph{finite}
set $\minbad = \set{\w[b]{9,9}, \w[b]{10,9}, \w[b]{9,10},
  \w[b]{10,10}}$.
%
Any bad configuration contains at least one of the elements of
\minbad\ as a subword. We can therefore craftily define the set \Bad\
of all the bad configurations as the upward-closure of {\minbad}.
%
% We say that \minbad\ are the \emph{minimal elements} of {\Bad},
% sometimes also called its \emph{generators}.
The set \minbad\ are the minimal elements of {\Bad}, i.e.\ %
$\gen\Bad=\minbad$.
%
\index{Infinite-state systems}%
We represent the whole set of bad configurations (albeit infinite)
using a compact and finite symbolic representation.
%
$$\Bad = \ucl{\minbad} = \setcomp{c\in\confs}{\exists b\in\minbad : b\subword c}$$



%% ====================================================================
\section{Monotonicity}
%% ====================================================================
\KW{Monotonicity}%
\index{Monotonicity}%
%% ********************************************************************
Monotonicity is a mathematical notion that comes from calculus,%
\footnote{In calculus, a function $f$ defined on a subset \domain\ of
  the real numbers \reals\ is called monotonic if it is either
  entirely increasing \emph{or} decreasing. It is called increasing
  (resp.\ decreasing), if for all $x$ and $y$ in \reals\ such that $x
  \leq y$, it holds that $f(x)\leq f(y)$ (resp.\ $f(x)\geq f(y)$). So
  $f$ either preserves the order, or reverses the order, consistently
  over the domain \domain .} %
which deals with functions over sets that preserve a given preorder
$\preorder$. For a monotonic function $f$ over a domain \domain ,
whenever two elements $a$ and $b$ in \domain\ are ordered such that
$a\preorder b$, it holds that $f(a) \preorder f(b)$.

%% ********************************************************************
\KW{Monotonic Systems}%
%% ********************************************************************
%
\noindent%
\begin{wrapfigure}{r}[0pt]{0.45\linewidth}
  \centering
  \tikzinput{img/monotonicity}
  \caption{Monotonicity of the rule~$\delta$ with respect to the
    subword relation~$\subword$.}
  \label{figure:monotonicity}
\end{wrapfigure}
%
A parameterized system \mbox{${\parsys}=(\locs,\rules)$} is
\emph{monotonic} (with respect to preorder $\subword$),
% from Section~\ref{section:upward:closed})%
if for each rule $\delta\in\rules$ and configurations $c_1$, $c_2$ and
$c_3$, such that $c_1\transof{\delta} c_2$ and $c_1\subword c_3$, then
there exists a configuration $c_4$ such that $c_3\transof{\delta} c_4$
and $c_2\subword c_4$. That is to say, if we can fire a transition on
a configuration, we can also fire it on a larger configuration, and
the results are ordered accordingly.
%
This is illustrated in Figure~\ref{figure:monotonicity}.
% For a parameterized system \mbox{${\parsys}=(\locs,\rules)$}, if the
% rules from \rules\ are monotonic with respect to preorder $\subword$
% from Section~\ref{section:upward:closed}, it is possible to transform
% an upward-closed set of configurations into another upward-closed set.
% %
% %In other words, the upward-closedness is preserved and
% The parameterized system \parsys\ is called \emph{monotonic} w.r.t.\
% $\subword$.
% %
% Indeed, consider a monotonic rule $\delta\in\rules$ and the
% configurations $c_1$, $c_2$ and $c_3$, such that $c_1\transof{\delta}
% c_2$ and $c_1\subword c_3$, then there exists a configuration $c_4$
% such that $c_3\transof{\delta} c_4$ and $c_2\subword c_4$. That is to
% say, if we can fire a transition on a configuration, we can also fire
% it on a larger configuration, and the results are ordered accordingly.

In~\cite{Parosh:Bengt:Karlis:Tsay:general}, it is shown that for
monotonic systems, the upward-closedness is preserved when computing
predecessors, that is, if a set is upward-closed, so is its
pre-image. This is an important property that we use in the next
section.

%% ********************************************************************
\KW{Universal not monotonic}%
\index{Non-Monotonicity}%
%% ********************************************************************
However, systems are not always monotonic.
%
This is for example the case if \rules\ contains an universally
quantified global transition. For example, if $\delta\in\rules$ is of
the form $\quantrule{s}{d}{\forall}{\neq}{\set{w}}$, the subword
relation is not necessarily preserved by $\delta$ as depicted in
Figure~\ref{figure:universal:not:monotonic}.

\noindent%
\begin{wrapfigure}{l}[0pt]{0.6\linewidth}
  %\centering
  \tikzinput[\linewidth]{img/universal-not-monotonic}
  %\setlength{\abovecaptionskip}{-1em plus 0.3em}
  \caption{Universal transitions are not monotonic. Therefore, we add
    transitions and over-approximate the transition relation
    $\trans$.}
  \label{figure:universal:not:monotonic}
  \smallskip
\end{wrapfigure}
%
%% ********************************************************************
\KW{Reachability for MTS is decidable}%
\index{Reachability}%
\index{Decidability}%
%% ********************************************************************
It has been shown that, for monotonic systems, the problem of
determining if a set of bad configurations is reachable from some
initial configurations is decidable~\cite{MTSdecidable}, 
%despite an infinite state-space.
even though the state-space is infinite.

%% ********************************************************************
\KW{Monotonic Abstraction}%
\index{Abstraction}%
\index{Over-approximation}%
%% ********************************************************************
Therefore, the strategy is to make the system monotonic by introducing
an over-approximation.
% The new abstract system will contain more
% transitions. All transitions from the original system are preserved.
Existential and local transitions are preserved from the original
system as-is, since they are monotonic.
%
Universal transitions, however, are not and we over-approximate them
using the scenario depicted in
Figure~\ref{figure:universal:not:monotonic}.
%
Informally, for every rule $\delta\in\rules$, we define a new
successor function $\abstrans\delta$ such that $\abstrans\delta$
coincide with $\delta$ if $\delta$ is a local transition or an
existential global transition. In the case where $\delta$ is an
universal global transition, %
% say, of the form
%$\quantrule{\src}{\dst}{\forall}{\sim}{\witnesses}$
we define $\delta$ in the following manner (here only giving the
intuition):
%
We first remove the process states from the configuration, if they
violate the guard, i.e.\ if they don't respect the global condition
from the rule (which effectively disables the transition).
%
This creates a potentially smaller configuration on which we can apply
$\delta$ as usual. We say that we ``go down on the ordering'' before
applying~$\delta$.

The successor functions $\abstrans\delta$ form a new set
$\abstrans\rules$ and the abstract parameterized system %
$\abstrans\parsys = (\locs,\abstrans\rules)$ induces a transition
system $(\confs,\atrans)$ that is now monotonic and preserves the
order $\subword$ (as depicted in
Figure~\ref{figure:abstract:monotonic}).
%
It is decidable to prove that the bad configurations are not reachable
from some initial ones in the new abstract transition system
$\abstrans\parsys$. Since it is an over-approximation, it contains the
original transition system $(\confs,\trans)$ and proving the abstract
system safe implies that the original parameterized system $\parsys$
is also safe.

\begingroup%
\setlength\intextsep{\dazintextsep}
\begin{figure}[b]
  \smallskip
  \centering
  \tikzinput{img/abstract-monotonic}
  \caption{The abstract transition relation $\atrans$ is monotonic.}
  \label{figure:abstract:monotonic}
\end{figure}
\endgroup

%% ====================================================================
\section{Backward Reachability}
\index{Reachability!Backward Analysis}%
%% ====================================================================


The procedure that will explore the state-space needs to work on
symbolic representations, rather than the configurations themselves,
in the spirit of Section~\ref{section:model:checking}. %
%
As we saw in Section~\ref{section:bad:states}, it is simple, using
upward-closure, to represent the set of all bad configurations.
%
Since we also know that a monotonic transition preserves the
upward-closedness when computing predecessors, the interesting idea is
then to use upward-closed sets as symbolic representations, define
their pre-image and run a backward analysis from the bad set (using
the abstract rules~$\abstrans\rules$). %

%\noindent%
\begin{wrapfigure}{r}[0pt]{0.35\textwidth}
  \hfill%
  \tikzinput{img/pre-image}
  %\caption{Pre-image}
  %\label{figure:pre:image}
\end{wrapfigure}
%
% There is a very important mathematical detail to catch here first. The
% computations regarding the pre-image of a configuration and the
% pre-image of an upward-closed set of configurations are different.
\index{Predecessors}%
\index{Pre-image operator}%
When computing predecessors, there is an important mathematical detail
to catch first: The pre-image of an upward-closed set is not
necessarily the same as taking the upward-closure of the pre-image of
its minimal elements.
%
For a configuration $c\in\confs$ and a transition
$\delta\in\abstrans\rules$, %
the set $\setcomp{u\in\confs}{u \atransof{\delta} c}$ of
configurations that can reach $c$ in one \mbox{$\delta$-step}, might
be empty (i.e.\ the pre-image of a given configuration does not
exist).
%
The configuration $c$ is potentially not attainable, but its
upward-closure could be.
%
Therefore the pre-image must be a computation over sets of
configurations as a whole rather than individual configurations only.
%
So we compute a slightly different set: the configurations that can
reach the \emph{upward-closure} of $c$ in one \mbox{$\delta$-step}
(not forcibly $c$ itself), that is, the set %
$\setcomp{u\in\confs}{\exists v\in\ucl c : u
  \atransof{\delta}v}$.

% For an upward-closed set $X\subseteq\confs$, we define its pre-image
% $\pre(X) = \union_{\delta\in\abstrans\rules}\inverse\delta (X)$ where
% %
% $$\inverse\delta (X)=\setcomp{u\in\confs}{\exists v\in X :
%   u\atransof\delta v}$$
More generally, for an upward-closed set $U\subseteq\confs$, we define
its $\delta$-pre-image as the set
$\inverse{\delta}(U)=\setcomp{u\in\confs}{\exists v\in U :
  u\atransof{\delta}v}$ and its pre-image as
$$\pre(U) = \union_{\delta\in\abstrans\rules}\inverse{\delta} (U)=\setcomp{u\in\confs}{\exists v\in U : u\atrans v}$$

By monotonicity, it follows that the pre-image of an upward-closed set
$U$ is also upward-closed. Notice though, that the minimal elements of
the pre-image ($\gen{\pre(U)}$) are not necessarily the inverse-image
of its minimal elements
($\union_{c\in\gen{U}}\setcomp{u\in\confs}{u\atransof{\delta}c}$).

%% ====================================================================
\subsection*{Scheme}
%\noindent\textbf{[Scheme]} %
%% ====================================================================
Given a upward-closed set of (bad) configurations, the procedure is
constructed to compute the fixpoint of the function $X\mapsto
X\cup\pre(X)$. %
%
Intuitively, the analysis computes the configurations that could reach
the bad set, by applying successively any rule from
$\abstrans\rules$. %

More precisely, the procedure computes a sequence of sets $U_0, U_1,
U_2, \ldots$ such that $U_0=\Bad=\ucl\minbad$ and for all $i\geq 0$,
$U_{i+1}= U_i \cup \pre(U_i)$. %
Every~$U_i$ represents the set of configurations that can reach \Bad\
in at most $i$ steps. By monotonicity, every~$U_i$ is upward-closed.

Observe that the sequence $(U_i)_{i\in\nat}$ is increasing, i.e.\ $U_i
\subseteq U_{i+1}$ for all $i\geq 0$. %
If the backward procedure reaches a point $n$ such that $U_n\supseteq
U_{n+1}$, it follows that $\forall m\ge n, U_m = U_n$ and the sequence
converges.
%
Consequently, %
$$\setcomp{c\in\confs}{\exists b\in \Bad :c\atransof{*} b}=U_n=\cup_{i\in\nat}U_i=\prestar(\ucl\minbad)$$

\index{Verification Scheme}%
\noindent%
\begin{wrapfigure}{r}[0pt]{0.53\linewidth}
\hfill%
\begin{algorithm}[H]
\DontPrintSemicolon
\caption{Backward Scheme}
\label{algo:backward:scheme}
$\worklist := \set{\Bad}$\;%
$\visited := \emptyset$\;%
\While{$\worklist \neq \emptyset$}{%
  remove some set $U$ from $\worklist$\;%
  \lIf{$U \cap \Inits \neq \emptyset$ \nllabel{scheme:backward:isinit}}{\Return Unsafe}%\;%
  \eIf{$U\subseteq\visited$\nllabel{scheme:backward:discard}}{%
    discard $U$ %
  }{%
    % $\worklist := \worklist \;\union\; \pre(U)$\nllabel{scheme:backward:pre}\;%
    % $\visited := U \union \visited$\nllabel{scheme:backward:minset}\;%
    add $\pre(U)$ to the worklist $\worklist$\nllabel{scheme:backward:pre}\;%
    add $U$ to the visited set $\visited$\nllabel{scheme:backward:minset}\;%
  }%
}%
\Return Safe%
\end{algorithm}
\end{wrapfigure}
%
Furthermore, if $U_n\cap\Inits=\emptyset$ holds, %
% the final result from
% the analysis does not contain any configuration that belongs to the
% set of initial configurations {\Inits}. %
% It is then obvious that 
the initial configurations cannot reach the set of bad configurations
\Bad\ and since $\abstrans\parsys$ is an over-approximation of
$\parsys$, the system $\parsys$ is safe.%
\footnote{%
  In the opposite case, it is necessary to examine whether it is a
  real error or if it is an artifact of the abstraction, and therefore
  a spurious example. %
  For spurious examples, it is possible to \emph{automatically} refine
  the abstraction and \mbox{re-run} the procedure, until it finds a
  suitable abstraction, that proves the system safe, or escapes with a
  real error~\cite{CEGAR}.%
}
%

The procedure is designed as a worklist algorithm, described in
Algorithm~\ref{algo:backward:scheme}, which manipulates upward-closed
sets. %
It takes as input an upward-closed set of bad configurations $\Bad$
and maintains two lists of %upward-closed sets: %
sets: %
(i) a list~$\worklist$ of sets of configurations that have not yet
been analyzed, initialized to $\set{\Bad}$ and %
(ii) a list~$\visited$, initially empty, that contains information
about the sets of configurations that have already been analyzed.

%% ====================================================================
\section{Procedure and Termination}
\label{section:backward:procedure}
\label{section:backward:termination}
%% ====================================================================

We are facing several issues to implement
Algorithm~\ref{algo:backward:scheme}.
%
The latter is only a scheme as it manipulates potentially infinite
upward-closed sets of configurations.
%
Even though upward-closed sets can be fully characterized using their
minimal elements, why would it guarantee that there are
\emph{finitely} many minimal elements for each upward-closed set?

%% ********************************************************************
\KW{WQO}%
\index{Ordering}%
\index{Well-Quasi Ordering (WQO)}%
%% ********************************************************************
The answer lies in a property of the subword relation: it is a
\emph{well-quasi ordering} (WQO for short). %
An ordering $\preorder$ over a set $A$ is said to be a WQO if for any
infinite sequence $a_0, a_1, a_2, \ldots$ of elements of $A$, there
exists $i$ and $j$ such that $i<j$ and $a_i\preorder a_j$. 
%
The definition implies that there is no infinite strictly decreasing
sequence of elements of $A$.
%
The subword relation $\subword$ is in fact a WQO %
(by Higman's lemma) % citation?
and in particular, every set of minimal elements is finite. (Recall
indeed that minimal elements are incomparable).
%
% Observe that, for two upward-closed sets $A$ and $B$, $\gen{A\union B}
% \subseteq \gen{A}\ \union\ \gen{B}$ (with equality if the minimal
% elements of both $A$ and $B$ are incomparable) and %
% and recall that for two configurations $u$ and $v$, $u\subword v$
% implies $\ucl{u} \supseteq \ucl{v}$ (and reciprocally).


% %% ********************************************************************
% \KW{Entailement}%
% %% ********************************************************************
% We define an \emph{entailement relation} $\entails$ over upward-closed
% sets of configurations such that $A\entails B$ if and only if $\forall
% b\in \gen{B} : \exists a\in \gen{A} : a\subword b$. Intuitively, all
% the configurations represented in the set $B$ are already represented
% in the (more ``general'') set $A$.

\index{Symbolic Representations}%
Since every upward-closed set is characterized by a finite number of
minimal elements, the scheme in algorithm~\ref{algo:backward:scheme}
can be adapted to manipulate individual configurations, that are used
as symbolic representations for upward-closed sets.
%
% The pre-image of an upward-closed set $U$ is now defined as: %
% $$
% \pre(U) =
% \gen{\union_{\delta\in\abstrans\rules}\inverse{\delta}(U)} =
% \gen{\setcomp{u\in\confs}{\exists v\in U : u\atrans v}}
% $$
%
\index{Pre-image operator}%
The pre-image of such a minimal configuration $m$ is now defined as:%
\footnote{Defining the pre-image as $\pre(m) =
  \gen{\union_{\delta\in\abstrans\rules}\inverse{\delta}(\ucl{m})}$,
  is only slightly more efficient theoretically since it avoids in the
  union to carry around comparable elements, but we have yet to see
  any real difference using the benchmarks from
  Chapter~\ref{chapter:experimentation}.} %
$$
\pre(m) = \union_{\delta\in\abstrans\rules}\gen{\inverse{\delta}(\ucl{m})}%
$$
%\gen{\setcomp{u\in\confs}{\exists v\in\confs: m\subword v \wedge u\atrans v}}

%
\index{Verification Algorithm}%
%\noindent%
\begin{wrapfigure}{r}[0pt]{0.55\textwidth}
\hfill%
\begin{minipage}{0.56\textwidth}
\begin{algorithm}[H]
\DontPrintSemicolon
\caption{Backward Procedure}
\label{algo:backward:procedure}
$\worklist := \minbad$\;%
$\visited := \emptyset$\;%
\While{$\worklist \neq \emptyset$}{%
  remove some $m$ from $\worklist$\;%
  \lIf{$\ucl{m} \cap \Inits \neq \emptyset$\nllabel{implementation:backward:isinit}}{\Return Unsafe}%\;%
  \eIf{$\exists s\in\visited: s\subword m$\nllabel{implementation:backward:discard}}{%
    discard $m$ %
  }{%
    $\worklist := \worklist \;\union\; \pre(m)$\;%
    $\visited := m \union \setcomp{s\in\visited}{m\not\subword s}$\nllabel{implementation:backward:minset}\;%
  }%
}%
\Return Safe%
\end{algorithm}
\end{minipage}%
\end{wrapfigure}
%
The implementation is presented in
Algorithm~\ref{algo:backward:procedure}.
%
It initializes the worklist to be the set of minimal bad
configurations~{\minbad}.
%
The list $\visited$~and~$\worklist$ now only contain configurations,
as symbolic representations of upward-closed sets.
%
Line~\ref{implementation:backward:minset} ensures that the
\emph{visited} configurations are incomparable with each other, i.e.\
$\visited$ is a set of minimal elements, while the configurations from
$\worklist$ are potentially comparable.
%
Line~\ref{implementation:backward:discard} tests whether a
configuration should be discarded, because the set it represents
already belongs to the visited set.\footnote{Recall that for two
  configurations $u$ and $v$, $u\subword v$ implies
  $\ucl{u}\supseteq\ucl{v}$ (and reciprocally).} %
%
The test on line~\ref{implementation:backward:isinit} is usually
carried out using automata theoretic constructs, but it is usually
simple and not a bottleneck.
%
An illustration of a backward run is depicted in
Figure~\ref{figure:backward:reachability}.

% Finally, we can sum up the few requirements in order to implement the
% procedure. 
% %
% \begin{strategy}
% \item The test $U \cap \Inits \neq \emptyset$ (line~\ref{line:isinit})
%   is decidable for any upward-closed set $U$.
% \item The computation of $\pre(U)$ (line~\ref{line:pre}) is decidable
%   for any upward-closed set~$U$. This allows us to derive the elements
%   of the sequence.
% % \item $U_i \subseteq U_j$ is decidable for any upward-closed set~$U_i$
% %   and $U_j$ (line~\ref{line:discard} and~\ref{line:minset}). %
% %   Using $\subword$ on the generators.
% \item The set of bad configurations is the upward-closure of a
%   \emph{finite} set of minimal configurations.
% \item \label{implementation:backward:termination} Is the algorithm guaranteed to terminate?
% \end{strategy}

\begingroup%
\setlength\intextsep{\dazintextsep}
\begin{figure}[htb]
  \centering
  \tikzinput{img/sbr}
  \caption{Backward Reachability}
  \label{figure:backward:reachability}
\end{figure}
\endgroup

%% ********************************************************************
\KW{Termination}%
\index{Verification Algorithm!Termination}%
%% ********************************************************************
The procedure computes a sequence $V_0, V_1, V_2, \ldots$ of (visited)
sets of minimal configurations such that $V_0=\minbad$ and for all
$i\geq 0$, $\ucl{V_{i+1}}\supseteq \ucl{V_i}$.
%
Assume that the increasing sequence $(\ucl{V_i})_{i\in\nat}$ is not
converging. There is then no such point $m$ for which $\ucl{V_{m}} =
\ucl{V_{m+1}}$, and the sequence is \emph{strictly} increasing. We can
pick an element $c_{m+1}$ in $V_{m+1}$ that is not in $V_{m}$, and
therefore, extract an infinite sequence of configurations
$(c_i)_{i\in\nat}$ such that those configurations are all incomparable
with each other, with respect to the subword relation~$\subword$.
%
This is a contradiction with the fact that $\subword$ is a WQO.
% 
Consequently, the sequence $(\ucl{V_i})_{i\in\nat}$ must converge, say
at point $n$, and then %we then have, %
$$
\setcomp{c\in\confs}{\exists b\in \Bad : c \atransof{*} b} =
\ucl{V_n} = \union_{i\in\nat}\ucl{V_i}
$$
The procedure is then guaranteed to terminate if the ordering in use
is a WQO. If it furthermore holds that
$\ucl{V_n}\cap\Inits=\emptyset$, then the bad configurations are not
reachable from the inital ones and the safety property is proven.


%% ====================================================================
\section{Applications}
\index{Experiments}%
%% ====================================================================

\index{Cache Coherence Protocols}%
\index{Distributed Protocols}%
The method has been applied on a variety of protocols such as cache
coherence protocol or distributed protocols, presented in
Chapter~\ref{chapter:experimentation}.
%
In~\cite{ParamVerif:global:conditions,Parosh:non-atomic}, Abdulla et
al.\ apply the method to parameterized systems, in particular with
linear topologies and atomic global conditions. We have applied the
method to protocols modeled as a Petri Net (i.e.\ a multiset) and
extended it to tree topologies.

% -------------------------
\makeatletter
\if@UU@margnum
\subsection*{Multisets \ldots and in particular Petri Nets}
\else
\subsection{Multisets \ldots and in particular Petri Nets}
\fi
\makeatother
\label{section:monotonic:abstraction:applications:pn}
% -------------------------
%
In this section, we apply the technique of monotonic abstraction and
symbolic backward reachability to the problem of \emph{Race
  Detection}.
%
In particular, we focus on \emph{race-freeness}, that is, the absence
of race conditions (also known as data races) in shared-variable
pthreaded programs.
%
Such programs can be modelled as parameterized systems that consist of
an arbitrary number of identical finite-state processes, competing for
a global resource and communicating through a finite set of variables.
%
The systems must satisfy mutual exclusion, that is, the safety
property stating that at most one process may hold the global resource
at a~time.

% -------------------------
\KW{Difficult bugs}%
% -------------------------
Race conditions can lead to devious bugs that are hard to track, due
to non-determinism and limited reproducibility. Errors caused by race
conditions are very subtle and often manifest themselves in the form
of corrupted or incorrect variable data. Unfortunately, this often
means that the error will not harm the system immediately, but only
when some other code is executed, which relies on the data to be
correct. This makes the process of locating the original race
condition even more difficult.%

% -------------------------
\KW{read/write}%
\KW{write/write}%
% -------------------------
Detecting a race condition, for a particular shared variable in the
code, is akin to finding places where two threads (or more) are
accessing the variable and one of them is performing a~write
operation. The timing of the read/write or write/write operations is
critical to determining the value of the variable that the threads read.
A program cannot depend on such imprecision as it could compute wrong
results only due to ``unfortunate'' interleavings of the different
thread operations.

\noindent%
\begin{wrapfigure}{r}{0.6\linewidth}
  %\centering
  \tikzinput[\linewidth]{img/pn-example}
  %\caption{Modeling the critical section problem with Petri Nets.}
  \caption{Critical section as Petri Net.}
  \label{figure:pn:example}
\end{wrapfigure}
%
% -------------------------
\KW{locking}%
% -------------------------
Most programmers guard themselves against such situations by
synchronizing the threads, which should allow only some interleavings
to happen, in a controlled manner.
%
They introduce locks around the shared resource they want to protect
and make sure that all threads follow the locking discipline they
impose.
%
Many race detection tools (e.g.~\cite{Eraser}) investigate therefore,
in a dynamic fashion, program executions to detect violations of the
locking discipline. They however do not guarantee a full coverage (see
Chapter~\ref{chapter:verification}).
%
% -------------------------
\KW{annotations}%
% -------------------------
Moreover, some programs require extra annotations along the code, to
state some form of specification (or contract) that code should follow
in order to behave as the programmer intended.
%
Given that industrial-size concurrent programs is becoming
increasingly important, this approach reaches its limit and the
analysis should, as much as possible, be automatic.

% -------------------------
The technique presented in the section detects the race conditions
themselves for a particular class of pthreaded C~programs. There is no
need for annotations and works also in the absence of explicit locking
mechanisms.
%
% -------------------------
\KW{model}%
% -------------------------
We first extract a model from the code in the form of a Petri Net
(see, e.g.\ Figure~\ref{figure:pn:example}).
%
The extraction is already a challenge and an over-approximation is
introduced to cope with many C~intricacies.
%
We do not cover it in this section, but the details of the model and
how it is extracted are presented in paper~\ref{paper:MCC08}.
%
%See an example in Figure~\ref{figure:pn:example}.

% -------------------------
\KW{verification}%
% -------------------------
%\noindent%
\begin{wrapfigure}{r}[0pt]{0.55\linewidth}
  \hfill%
  \tikzinput{img/pn-entailement}
  \caption{Multiset ordering.}
  \label{figure:multiset:ordering}
  \smallskip
\end{wrapfigure}
%
We then verify the Petri Net model.
%
Transition rules are characterized by the usual Petri Net flow
relation, with \emph{input} and \emph{output places}.
%
Informally, each place in the Petri Net (or a set of places)
represents an instruction from the program code and the number of
\emph{tokens} in a place represent the number of threads currently
executing that instruction.
%
Firing transitions moves tokens from the input places to the output
places of the transition only if there were ``enough'' tokens in the
input places.

The key aspect of the model is that a shared variable~$v$ is
associated with two places, $\LD{v}$ and $\ST{v}$, in the Petri Net
(in green in the figures). A process places a token in $\LD{v}$
(resp.\ $\ST{v}$) if it is currently accessing $v$ for reading (resp.\
writing).
%
That way, we can distinguish situations where a \emph{read} and a
\emph{write} assignment on some shared variable can happen
simultaneously, effectively detecting a race condition.

A configuration in the system is a multiset, i.e\ a valuation of the
tokens in each place of the Petri Net, often called a~\emph{marking}.
%
The subword relation is replaced by a relation on multisets such that,
for a Petri Net with markings $PN_1$ and $PN_2$, we write $PN_1
\subword PN_2$ if and only if, intuitively, tokens can be removed from
$PN_2$ to obtain $PN_1$ (see Figure~\ref{figure:multiset:ordering}).
% In particular, $PN_1$ doesn't have more tokens than $PN_2$ in each
% corresponding places.
%

\noindent%
\begin{wrapfigure}{r}{0.4\linewidth}
  \hfill%
  \tikzinput{img/pn-bad}
\end{wrapfigure}
%
The set of bad configurations is characterized by the upward-closure
of the (minimal) elements depicted on the right.
%
The multiset ordering is a WQO, so
procedure~\ref{algo:backward:procedure} is guaranteed to terminate and
proves the race-freeness of the system.


% -------------------------
\makeatletter
\if@UU@margnum
\subsection*{Shape analysis}
\else
\subsection{Shape analysis}
\fi
\makeatother
% -------------------------

This section focuses on the verification of sequential iterative
programs manipulating dynamic memory heaps, only briefly since we
dedicate a full chapter (Chapter~\ref{chapter:shape:analysis} on
page~\pageref{chapter:shape:analysis}) to shape analysis, using yet
another abstraction technique, which handles the concurrent case.
%
% The issue of verifying such programs automatically has received a
% lot of attention in the last few years, and many approaches and
% techniques have been developed including static-analysis and
% abstraction-based frameworks (see, e.g., \cite{Sagiv02}),
% logic-based frameworks (see, e.g., \cite{Reynolds02,OHearn06}),
% automata-based frameworks (see, e.g., \cite{pale97,BHRV06}),
% etc\ldots %
%
More precisely, heap structures are built using heap cells which
contain \emph{one} next-selector, i.e.\ programs are manipulating
(possibly circular and shared) singly-linked lists.

% Here, we introduce a framework based on symbolic (backward)
% reachability analysis using upward-closed sets of heap graphs
% (w.r.t. some appropriate preorder).  As a first step toward this
% framework
%
We model heaps by labeled graphs, where labels correspond to positions
of program variables.
%
In fact, heap graphs are symbolic representations to characterize sets
of heaps instead of a single one.
%
The main issue is to define a~preorder~$\subword$ on heap graphs. We
introduce the following preorder: Given two graphs $g_1$ and $g_2$, we
have $g_1 \subword g_2$ if $g_1$ can be obtained from $g_2$ by a
sequence of transformations consisting of either deleting an edge, a
variable, or an isolated vertex, or of contracting segments (i.e.,
sequence of vertices) without sharing in the graph.
%
% Actually, the graphs can be used as symbolic representations in
% general to characterize sets of heaps instead of a single one. 
%
% Those graphs can be seen as minimal elements of upward-closed sets of
% heap graphs w.r.t.\ $\subword$.
%
The transition system, abstraction and the procedure to check the
entailement relation~$\subword$ are described in details in
paper~\ref{paper:CAV08}.

The analysis allows to check properties such as absence of null
dereferencing as well as absence of garbage creation. Moreover, it
allows to check ``shape'' properties over heaps, such as
``well-formedness'', where, for instance, the output is always a list
without sharing. We show that these kinds of verification problems can
be reduced to the problem of reaching sets of bad configurations
corresponding to the existence in the heap graph of some \emph{minimal
  bad patterns}.
%
For instance, the set of configuration with garbage can be
represented by minimal graphs containing all programs variables plus
one isolated vertex.

We applied the monotonic abstraction method to verify such sequential
programs fully-automatically (see Paper~\ref{paper:CAV08}).
%
Furthermore, the graph relation $\subword$ is proven to be a WQO so
procedure~\ref{algo:backward:procedure} is guaranteed to terminate.

% -------------------------
\makeatletter
\if@UU@margnum
\subsection*{Tree topologies}
\else
\subsection{Tree topologies}
\fi
\makeatother
% -------------------------

Finally, we extend the method to parameterized systems, here organized
according to a~tree topology.
%
The topology and the communication primitives define the behaviour of
the system, which is modeled similarly as in
Section~\ref{section:paramsys:definition}.
%
In the linear case, global conditions could mention the state of
processes on the right or left of a given process.
%
With tree topologies, it is no longer the case, but we see a closely
related notion: \emph{pattern matching}.

A configuration of the system is represented by a tree over a finite
set of local process states~$\locs$.
%
The behaviour of the system is induced by a set of
transitions~$\rules$ conditioned by the local states of neighboring
processes, i.e.\ the parent and children processes.
%
In this topology, a transition is a \emph{rewriting rule} which may
change the states of all involved processes. %
The arity or the order of the children might be relevant.
%
For example, in Figure~\ref{figure:tree:transition}, a process in
state $a_1$ can change state if the parent is in state $p_1$ and the
children are in state $b_1$ and $c_1$.
%
In such a case, the process changes state to $a_2$ and the parent and
the two children change state to $p_2,b_2,c_2$, respectively.
%
In other words, process $a_1$ fits the pattern of the rewriting rule,
when $p_1,b_1,c_1$ are its neighbors.

\begingroup%
\setlength\intextsep{\dazintextsep}
\begin{figure}[ht]
  \centering
  \tikzinput{img/tree-transition}
  \caption{A tree transition~$\delta$: If the pattern
    $p_1,a_1,b_1,c_1$ is found in the tree, the rule is applied and
    the nodes change their local state to $p_2,a_2,b_2,c_2$
    respectively.}
  \label{figure:tree:transition}
\end{figure}
\endgroup

The subword relation is replaced by a tree embedding
relation. Intuitively, a tree $t$ is embedded in tree $t'$ if it is
possible to obtain $t$ by removing nodes from $t$. Removing nodes is a
complex operation that \mbox{re-attaches} the subtrees of a node to
its parent.  The embedding relation is presented in details in
paper~\ref{paper:FORTE08}.

We over-approximate the behaviour of the system by modifying the
semantics of the transitions, such that a rule is applied to
a node and two nodes in its left and right subtrees (rather than its
immediate left and right children).
%
Nodes ``violating'' the pattern are removed, i.e.\ we ``go down in the
ordering'' before we apply the rule (see
Figure~\ref{figure:tree:transition:abstract}).
%
The resulting abstract transition system is then monotonic with
respect to the tree embedding relation on configurations --- larger
configurations are able to perform the same transitions as smaller
ones with results ordered accordingly.
%

\begingroup%
\setlength\intextsep{\dazintextsep}
\begin{figure}[hb]
  \centering
  \tikzinput{img/tree-abstract-transition}
  \caption{An abstract tree transition: the pattern to apply the
    rule~$\abstrans\delta$ is searched in the subtree, rather than the
    immediate children or parent.}
  \label{figure:tree:transition:abstract}
\end{figure}
\endgroup

Since the abstract transition relation is monotonic and the tree
embedding relation is a WQO (by Kruskal's theorem~\cite{kruskal}), it
follows that we can apply the backward reachability analysis of
Algorithm~\ref{algo:backward:procedure}, with the guaranty that it
terminates.

Upward-closed sets of configurations are symbolically (and finitely)
represented with trees, which allows the reachability analysis to be
performed by computing predecessors of trees, simply and efficiently
--- more than applying transducer relations on general tree regular
languages~\cite{rmc:wo:transducers}.
%
Based on the method, we have implemented a prototype which works well
on several tree-based protocols such as the percolate, leader
election, tree-arbiter, and the IEEE 1394 Tree identity protocols (see
Paper~\ref{paper:FORTE08}).



%% ====================================================================
\whatwelearned{monotonic-abstraction}
%% ====================================================================
