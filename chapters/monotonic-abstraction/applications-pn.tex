% -------------------------
\makeatletter
\if@UU@margnum
\subsection*{Multisets \ldots and in particular Petri Nets}
\else
\subsection{Multisets \ldots and in particular Petri Nets}
\fi
\makeatother
\label{section:monotonic:abstraction:applications:pn}
% -------------------------
%
In this section, we apply the technique of monotonic abstraction and
symbolic backward reachability to the problem of \emph{Race
  Detection}.
%
In particular, we focus on \emph{race-freeness}, that is, the absence
of race conditions (also known as data races) in shared-variable
pthreaded programs.
%
Such programs can be modelled as parameterized systems that consist of
an arbitrary number of identical finite-state processes, competing for
a global resource and communicating through a finite set of variables.
%
The systems must satisfy mutual exclusion, that is, the safety
property stating that at most one process may hold the global resource
at a~time.

% -------------------------
\KW{Difficult bugs}%
% -------------------------
Race conditions can lead to devious bugs that are hard to track, due
to non-determinism and limited reproducibility. Errors caused by race
conditions are very subtle and often manifest themselves in the form
of corrupted or incorrect variable data. Unfortunately, this often
means that the error will not harm the system immediately, but only
when some other code is executed, which relies on the data to be
correct. This makes the process of locating the original race
condition even more difficult.%

% -------------------------
\KW{read/write}%
\KW{write/write}%
% -------------------------
Detecting a race condition, for a particular shared variable in the
code, is akin to finding places where two threads (or more) are
accessing the variable and one of them is performing a~write
operation. The timing of the read/write or write/write operations is
critical to determining the value of the variable that the threads read.
A program cannot depend on such imprecision as it could compute wrong
results only due to ``unfortunate'' interleavings of the different
thread operations.

\noindent%
\begin{wrapfigure}{r}{0.6\linewidth}
  %\centering
  \tikzinput[\linewidth]{img/pn-example}
  %\caption{Modeling the critical section problem with Petri Nets.}
  \caption{Critical section as Petri Net.}
  \label{figure:pn:example}
\end{wrapfigure}
%
% -------------------------
\KW{locking}%
% -------------------------
Most programmers guard themselves against such situations by
synchronizing the threads, which should allow only some interleavings
to happen, in a controlled manner.
%
They introduce locks around the shared resource they want to protect
and make sure that all threads follow the locking discipline they
impose.
%
Many race detection tools (e.g.~\cite{Eraser}) investigate therefore,
in a dynamic fashion, program executions to detect violations of the
locking discipline. They however do not guarantee a full coverage (see
Chapter~\ref{chapter:verification}).
%
% -------------------------
\KW{annotations}%
% -------------------------
Moreover, some programs require extra annotations along the code, to
state some form of specification (or contract) that code should follow
in order to behave as the programmer intended.
%
Given that industrial-size concurrent programs is becoming
increasingly important, this approach reaches its limit and the
analysis should, as much as possible, be automatic.

% -------------------------
The technique presented in the section detects the race conditions
themselves for a particular class of pthreaded C~programs. There is no
need for annotations and works also in the absence of explicit locking
mechanisms.
%
% -------------------------
\KW{model}%
% -------------------------
We first extract a model from the code in the form of a Petri Net
(see, e.g.\ Figure~\ref{figure:pn:example}).
%
The extraction is already a challenge and an over-approximation is
introduced to cope with many C~intricacies.
%
We do not cover it in this section, but the details of the model and
how it is extracted are presented in paper~\ref{paper:MCC08}.
%
%See an example in Figure~\ref{figure:pn:example}.

% -------------------------
\KW{verification}%
% -------------------------
%\noindent%
\begin{wrapfigure}{r}[0pt]{0.55\linewidth}
  \hfill%
  \tikzinput{img/pn-entailement}
  \caption{Multiset ordering.}
  \label{figure:multiset:ordering}
  \smallskip
\end{wrapfigure}
%
We then verify the Petri Net model.
%
Transition rules are characterized by the usual Petri Net flow
relation, with \emph{input} and \emph{output places}.
%
Informally, each place in the Petri Net (or a set of places)
represents an instruction from the program code and the number of
\emph{tokens} in a place represent the number of threads currently
executing that instruction.
%
Firing transitions moves tokens from the input places to the output
places of the transition only if there were ``enough'' tokens in the
input places.

The key aspect of the model is that a shared variable~$v$ is
associated with two places, $\LD{v}$ and $\ST{v}$, in the Petri Net
(in green in the figures). A process places a token in $\LD{v}$
(resp.\ $\ST{v}$) if it is currently accessing $v$ for reading (resp.\
writing).
%
That way, we can distinguish situations where a \emph{read} and a
\emph{write} assignment on some shared variable can happen
simultaneously, effectively detecting a race condition.

A configuration in the system is a multiset, i.e\ a valuation of the
tokens in each place of the Petri Net, often called a~\emph{marking}.
%
The subword relation is replaced by a relation on multisets such that,
for a Petri Net with markings $PN_1$ and $PN_2$, we write $PN_1
\subword PN_2$ if and only if, intuitively, tokens can be removed from
$PN_2$ to obtain $PN_1$ (see Figure~\ref{figure:multiset:ordering}).
% In particular, $PN_1$ doesn't have more tokens than $PN_2$ in each
% corresponding places.
%

\noindent%
\begin{wrapfigure}{r}{0.4\linewidth}
  \hfill%
  \tikzinput{img/pn-bad}
\end{wrapfigure}
%
The set of bad configurations is characterized by the upward-closure
of the (minimal) elements depicted on the right.
%
The multiset ordering is a WQO, so
procedure~\ref{algo:backward:procedure} is guaranteed to terminate and
proves the race-freeness of the system.

