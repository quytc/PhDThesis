% -------------------------
\makeatletter
\if@UU@margnum
\subsection*{Tree topologies}
\else
\subsection{Tree topologies}
\fi
\makeatother
% -------------------------

Finally, we extend the method to parameterized systems, here organized
according to a~tree topology.
%
The topology and the communication primitives define the behaviour of
the system, which is modeled similarly as in
Section~\ref{section:paramsys:definition}.
%
In the linear case, global conditions could mention the state of
processes on the right or left of a given process.
%
With tree topologies, it is no longer the case, but we see a closely
related notion: \emph{pattern matching}.

A configuration of the system is represented by a tree over a finite
set of local process states~$\locs$.
%
The behaviour of the system is induced by a set of
transitions~$\rules$ conditioned by the local states of neighboring
processes, i.e.\ the parent and children processes.
%
In this topology, a transition is a \emph{rewriting rule} which may
change the states of all involved processes. %
The arity or the order of the children might be relevant.
%
For example, in Figure~\ref{figure:tree:transition}, a process in
state $a_1$ can change state if the parent is in state $p_1$ and the
children are in state $b_1$ and $c_1$.
%
In such a case, the process changes state to $a_2$ and the parent and
the two children change state to $p_2,b_2,c_2$, respectively.
%
In other words, process $a_1$ fits the pattern of the rewriting rule,
when $p_1,b_1,c_1$ are its neighbors.

\begingroup%
\setlength\intextsep{\dazintextsep}
\begin{figure}[ht]
  \centering
  \tikzinput{img/tree-transition}
  \caption{A tree transition~$\delta$: If the pattern
    $p_1,a_1,b_1,c_1$ is found in the tree, the rule is applied and
    the nodes change their local state to $p_2,a_2,b_2,c_2$
    respectively.}
  \label{figure:tree:transition}
\end{figure}
\endgroup

The subword relation is replaced by a tree embedding
relation. Intuitively, a tree $t$ is embedded in tree $t'$ if it is
possible to obtain $t$ by removing nodes from $t$. Removing nodes is a
complex operation that \mbox{re-attaches} the subtrees of a node to
its parent.  The embedding relation is presented in details in
paper~\ref{paper:FORTE08}.

We over-approximate the behaviour of the system by modifying the
semantics of the transitions, such that a rule is applied to
a node and two nodes in its left and right subtrees (rather than its
immediate left and right children).
%
Nodes ``violating'' the pattern are removed, i.e.\ we ``go down in the
ordering'' before we apply the rule (see
Figure~\ref{figure:tree:transition:abstract}).
%
The resulting abstract transition system is then monotonic with
respect to the tree embedding relation on configurations --- larger
configurations are able to perform the same transitions as smaller
ones with results ordered accordingly.
%

\begingroup%
\setlength\intextsep{\dazintextsep}
\begin{figure}[hb]
  \centering
  \tikzinput{img/tree-abstract-transition}
  \caption{An abstract tree transition: the pattern to apply the
    rule~$\abstrans\delta$ is searched in the subtree, rather than the
    immediate children or parent.}
  \label{figure:tree:transition:abstract}
\end{figure}
\endgroup

Since the abstract transition relation is monotonic and the tree
embedding relation is a WQO (by Kruskal's theorem~\cite{kruskal}), it
follows that we can apply the backward reachability analysis of
Algorithm~\ref{algo:backward:procedure}, with the guaranty that it
terminates.

Upward-closed sets of configurations are symbolically (and finitely)
represented with trees, which allows the reachability analysis to be
performed by computing predecessors of trees, simply and efficiently
--- more than applying transducer relations on general tree regular
languages~\cite{rmc:wo:transducers}.
%
Based on the method, we have implemented a prototype which works well
on several tree-based protocols such as the percolate, leader
election, tree-arbiter, and the IEEE 1394 Tree identity protocols (see
Paper~\ref{paper:FORTE08}).
