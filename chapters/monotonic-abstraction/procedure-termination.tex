%% ====================================================================
\section{Procedure and Termination}
\label{section:backward:procedure}
\label{section:backward:termination}
%% ====================================================================

We are facing several issues to implement
Algorithm~\ref{algo:backward:scheme}.
%
The latter is only a scheme as it manipulates potentially infinite
upward-closed sets of configurations.
%
Even though upward-closed sets can be fully characterized using their
minimal elements, why would it guarantee that there are
\emph{finitely} many minimal elements for each upward-closed set?

%% ********************************************************************
\KW{WQO}%
\index{Ordering}%
\index{Well-Quasi Ordering (WQO)}%
%% ********************************************************************
The answer lies in a property of the subword relation: it is a
\emph{well-quasi ordering} (WQO for short). %
An ordering $\preorder$ over a set $A$ is said to be a WQO if for any
infinite sequence $a_0, a_1, a_2, \ldots$ of elements of $A$, there
exists $i$ and $j$ such that $i<j$ and $a_i\preorder a_j$. 
%
The definition implies that there is no infinite strictly decreasing
sequence of elements of $A$.
%
The subword relation $\subword$ is in fact a WQO %
(by Higman's lemma) % citation?
and in particular, every set of minimal elements is finite. (Recall
indeed that minimal elements are incomparable).
%
% Observe that, for two upward-closed sets $A$ and $B$, $\gen{A\union B}
% \subseteq \gen{A}\ \union\ \gen{B}$ (with equality if the minimal
% elements of both $A$ and $B$ are incomparable) and %
% and recall that for two configurations $u$ and $v$, $u\subword v$
% implies $\ucl{u} \supseteq \ucl{v}$ (and reciprocally).


% %% ********************************************************************
% \KW{Entailement}%
% %% ********************************************************************
% We define an \emph{entailement relation} $\entails$ over upward-closed
% sets of configurations such that $A\entails B$ if and only if $\forall
% b\in \gen{B} : \exists a\in \gen{A} : a\subword b$. Intuitively, all
% the configurations represented in the set $B$ are already represented
% in the (more ``general'') set $A$.

\index{Symbolic Representations}%
Since every upward-closed set is characterized by a finite number of
minimal elements, the scheme in algorithm~\ref{algo:backward:scheme}
can be adapted to manipulate individual configurations, that are used
as symbolic representations for upward-closed sets.
%
% The pre-image of an upward-closed set $U$ is now defined as: %
% $$
% \pre(U) =
% \gen{\union_{\delta\in\abstrans\rules}\inverse{\delta}(U)} =
% \gen{\setcomp{u\in\confs}{\exists v\in U : u\atrans v}}
% $$
%
\index{Pre-image operator}%
The pre-image of such a minimal configuration $m$ is now defined as:%
\footnote{Defining the pre-image as $\pre(m) =
  \gen{\union_{\delta\in\abstrans\rules}\inverse{\delta}(\ucl{m})}$,
  is only slightly more efficient theoretically since it avoids in the
  union to carry around comparable elements, but we have yet to see
  any real difference using the benchmarks from
  Chapter~\ref{chapter:experimentation}.} %
$$
\pre(m) = \union_{\delta\in\abstrans\rules}\gen{\inverse{\delta}(\ucl{m})}%
$$
%\gen{\setcomp{u\in\confs}{\exists v\in\confs: m\subword v \wedge u\atrans v}}

%
\index{Verification Algorithm}%
%\noindent%
\begin{wrapfigure}{r}[0pt]{0.55\textwidth}
\hfill%
\begin{minipage}{0.56\textwidth}
\begin{algorithm}[H]
\DontPrintSemicolon
\caption{Backward Procedure}
\label{algo:backward:procedure}
$\worklist := \minbad$\;%
$\visited := \emptyset$\;%
\While{$\worklist \neq \emptyset$}{%
  remove some $m$ from $\worklist$\;%
  \lIf{$\ucl{m} \cap \Inits \neq \emptyset$\nllabel{implementation:backward:isinit}}{\Return Unsafe}%\;%
  \eIf{$\exists s\in\visited: s\subword m$\nllabel{implementation:backward:discard}}{%
    discard $m$ %
  }{%
    $\worklist := \worklist \;\union\; \pre(m)$\;%
    $\visited := m \union \setcomp{s\in\visited}{m\not\subword s}$\nllabel{implementation:backward:minset}\;%
  }%
}%
\Return Safe%
\end{algorithm}
\end{minipage}%
\end{wrapfigure}
%
The implementation is presented in
Algorithm~\ref{algo:backward:procedure}.
%
It initializes the worklist to be the set of minimal bad
configurations~{\minbad}.
%
The list $\visited$~and~$\worklist$ now only contain configurations,
as symbolic representations of upward-closed sets.
%
Line~\ref{implementation:backward:minset} ensures that the
\emph{visited} configurations are incomparable with each other, i.e.\
$\visited$ is a set of minimal elements, while the configurations from
$\worklist$ are potentially comparable.
%
Line~\ref{implementation:backward:discard} tests whether a
configuration should be discarded, because the set it represents
already belongs to the visited set.\footnote{Recall that for two
  configurations $u$ and $v$, $u\subword v$ implies
  $\ucl{u}\supseteq\ucl{v}$ (and reciprocally).} %
%
The test on line~\ref{implementation:backward:isinit} is usually
carried out using automata theoretic constructs, but it is usually
simple and not a bottleneck.
%
An illustration of a backward run is depicted in
Figure~\ref{figure:backward:reachability}.

% Finally, we can sum up the few requirements in order to implement the
% procedure. 
% %
% \begin{strategy}
% \item The test $U \cap \Inits \neq \emptyset$ (line~\ref{line:isinit})
%   is decidable for any upward-closed set $U$.
% \item The computation of $\pre(U)$ (line~\ref{line:pre}) is decidable
%   for any upward-closed set~$U$. This allows us to derive the elements
%   of the sequence.
% % \item $U_i \subseteq U_j$ is decidable for any upward-closed set~$U_i$
% %   and $U_j$ (line~\ref{line:discard} and~\ref{line:minset}). %
% %   Using $\subword$ on the generators.
% \item The set of bad configurations is the upward-closure of a
%   \emph{finite} set of minimal configurations.
% \item \label{implementation:backward:termination} Is the algorithm guaranteed to terminate?
% \end{strategy}

\begingroup%
\setlength\intextsep{\dazintextsep}
\begin{figure}[htb]
  \centering
  \tikzinput{img/sbr}
  \caption{Backward Reachability}
  \label{figure:backward:reachability}
\end{figure}
\endgroup

%% ********************************************************************
\KW{Termination}%
\index{Verification Algorithm!Termination}%
%% ********************************************************************
The procedure computes a sequence $V_0, V_1, V_2, \ldots$ of (visited)
sets of minimal configurations such that $V_0=\minbad$ and for all
$i\geq 0$, $\ucl{V_{i+1}}\supseteq \ucl{V_i}$.
%
Assume that the increasing sequence $(\ucl{V_i})_{i\in\nat}$ is not
converging. There is then no such point $m$ for which $\ucl{V_{m}} =
\ucl{V_{m+1}}$, and the sequence is \emph{strictly} increasing. We can
pick an element $c_{m+1}$ in $V_{m+1}$ that is not in $V_{m}$, and
therefore, extract an infinite sequence of configurations
$(c_i)_{i\in\nat}$ such that those configurations are all incomparable
with each other, with respect to the subword relation~$\subword$.
%
This is a contradiction with the fact that $\subword$ is a WQO.
% 
Consequently, the sequence $(\ucl{V_i})_{i\in\nat}$ must converge, say
at point $n$, and then %we then have, %
$$
\setcomp{c\in\confs}{\exists b\in \Bad : c \atransof{*} b} =
\ucl{V_n} = \union_{i\in\nat}\ucl{V_i}
$$
The procedure is then guaranteed to terminate if the ordering in use
is a WQO. If it furthermore holds that
$\ucl{V_n}\cap\Inits=\emptyset$, then the bad configurations are not
reachable from the inital ones and the safety property is proven.
