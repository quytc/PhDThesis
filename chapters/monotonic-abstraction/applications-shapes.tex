% -------------------------
\makeatletter
\if@UU@margnum
\subsection*{Shape analysis}
\else
\subsection{Shape analysis}
\fi
\makeatother
% -------------------------

This section focuses on the verification of sequential iterative
programs manipulating dynamic memory heaps, only briefly since we
dedicate a full chapter (Chapter~\ref{chapter:shape:analysis} on
page~\pageref{chapter:shape:analysis}) to shape analysis, using yet
another abstraction technique, which handles the concurrent case.
%
% The issue of verifying such programs automatically has received a
% lot of attention in the last few years, and many approaches and
% techniques have been developed including static-analysis and
% abstraction-based frameworks (see, e.g., \cite{Sagiv02}),
% logic-based frameworks (see, e.g., \cite{Reynolds02,OHearn06}),
% automata-based frameworks (see, e.g., \cite{pale97,BHRV06}),
% etc\ldots %
%
More precisely, heap structures are built using heap cells which
contain \emph{one} next-selector, i.e.\ programs are manipulating
(possibly circular and shared) singly-linked lists.

% Here, we introduce a framework based on symbolic (backward)
% reachability analysis using upward-closed sets of heap graphs
% (w.r.t. some appropriate preorder).  As a first step toward this
% framework
%
We model heaps by labeled graphs, where labels correspond to positions
of program variables.
%
In fact, heap graphs are symbolic representations to characterize sets
of heaps instead of a single one.
%
The main issue is to define a~preorder~$\subword$ on heap graphs. We
introduce the following preorder: Given two graphs $g_1$ and $g_2$, we
have $g_1 \subword g_2$ if $g_1$ can be obtained from $g_2$ by a
sequence of transformations consisting of either deleting an edge, a
variable, or an isolated vertex, or of contracting segments (i.e.,
sequence of vertices) without sharing in the graph.
%
% Actually, the graphs can be used as symbolic representations in
% general to characterize sets of heaps instead of a single one. 
%
% Those graphs can be seen as minimal elements of upward-closed sets of
% heap graphs w.r.t.\ $\subword$.
%
The transition system, abstraction and the procedure to check the
entailement relation~$\subword$ are described in details in
paper~\ref{paper:CAV08}.

The analysis allows to check properties such as absence of null
dereferencing as well as absence of garbage creation. Moreover, it
allows to check ``shape'' properties over heaps, such as
``well-formedness'', where, for instance, the output is always a list
without sharing. We show that these kinds of verification problems can
be reduced to the problem of reaching sets of bad configurations
corresponding to the existence in the heap graph of some \emph{minimal
  bad patterns}.
%
For instance, the set of configuration with garbage can be
represented by minimal graphs containing all programs variables plus
one isolated vertex.

We applied the monotonic abstraction method to verify such sequential
programs fully-automatically (see Paper~\ref{paper:CAV08}).
%
Furthermore, the graph relation $\subword$ is proven to be a WQO so
procedure~\ref{algo:backward:procedure} is guaranteed to terminate.
