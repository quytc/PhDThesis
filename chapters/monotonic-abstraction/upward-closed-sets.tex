%% ====================================================================
\section{Upward-Closed Sets}
\label{section:upward:closed}
\label{section:bad:states}
%% ====================================================================
%
\index{Ordering}%
\index{Subword}%
We introduce a partial \emph{ordering} between the configurations, namely the
subword relation~$\subword$, which allows us to compare two
configurations (of potentially different size) and determine if one is
``smaller'' than the other.
%
% It is a partial ordering, which means that we sometimes cannot
% determine which one is smaller.

%% ********************************************************************
\KW{Ordering}%
%% ********************************************************************
\begin{wrapfigure}{r}{0.45\linewidth}
  \centering
  \tikzinput{img/subword}
  \caption{The subword relation}
  \label{figure:subword}
\end{wrapfigure}
%
Let $u$ and $v$ be two configurations of size $n$ and $m$
respectively, with $n\leq m$.
%
Intuitively, we would like to check if the configuration $u$ can be
``injected'' inside $v$ (or that $v$ can be ``reduced'' to $u$).
%
We first define $H_n^m$ as the set of strictly increasing injections
from $\range{1}{n}$ to $\range{1}{m}$, %
i.e.\ %such that %
for $h\in H_n^m$, $1 \leq i < j \leq n \implies 1 \leq h(i) < h(j)
\leq m$. %
(Recall $\range{a}{b} = [a,b]\cap\nat$ is the range of integers in the
interval between $a$ and~$b$).
%
Then, we write $u \subword v$ if there exists $h$ in $H_n^m$ such that
$u[i]=v[h(i)]$ for all $i\in\range{1}{n}$ (see
Figure~\ref{figure:subword}).
%
If there is no injection that can be found to compare $u$ and $v$,
they are simply incomparable.

%% ********************************************************************
\KW{Upward-closure}%
\index{Upward-closure}%
%% ********************************************************************
\begin{wrapfigure}{r}{0.18\linewidth}
  \hfill%
  \tikzinput{img/upward-closure}
  % \caption{Upward-closure}
  % \label{figure:upward:closure}
\end{wrapfigure}
%
For a configuration $c\in\confs$, we define its \emph{upward-closure}
as the set of all the configurations of any (larger) size, that at
least contain $c$ as a subword (depicted on the right).
%
$$\ucl{c}=\setcomp{u\in\confs}{c\subword u}$$
%
Notice that $c\subword c'$ implies $\ucl{c}\supseteq\ucl{c'}$, meaning
that the smaller a configuration is in the ordering, the larger set of
configurations it represents. %
We abuse the notation and for a set %$A\subseteq\confs$, we use %
$A$, we use %
$\ucl{A} = %
\union_{a{\in}A}\ucl{a} = %
\setcomp{u\in\confs}{\exists a\in A : a\subword u}$.

%\index{Upward-closed set}%
We say that a set $A\subseteq\confs$ is \emph{upward-closed} if $A =
\ucl A$.
%
%% ********************************************************************
\KW{Generators}%
\index{Generators}%
\index{Minimal elements}%
\index{Entailement}%
%% ********************************************************************
Moreover, for an upward-closed set $U$, we define the \emph{minimal
  elements} of $U$, that is, the set $M$ such that
\begin{enumerate}[label=(\roman*),leftmargin=0pt]
\item %
%(i) %
  \textbf{[Closure]} $\ucl{M}{=}U$, i.e.\ $U$ can be generated by taking
  the upward-closure of~$M$,
\item %
%and (ii) %
  \textbf{[Minimality]} $\forall a,b\in M, a\subword b \implies a=b$,
  i.e.\ all the elements of $M$ are incomparable with respect to the
  subword relation $\subword$.
\end{enumerate}
% 
The set $M$ is therefore uniquely defined for a given~$U$ and
denoted~$\gen{U}$.

% %% ********************************************************************
% \KW{Entailement}%
% %% ********************************************************************
% We define an \emph{entailement relation} $\entails$ over upward-closed
% sets of configurations such that $A\entails B$ if and only if $\forall
% b\in B : \exists a\in A : a\subword b$. Intuitively, all the
% configurations represented in the set $B$ are already represented in
% the (more ``general'') set $A$.

%% ********************************************************************
\KW{Minimal elements}%
\index{Bad configurations}%
%% ********************************************************************
Characterizing the bad configurations for the mutual exclusion
property of Figure~\ref{figure:szymanski:implementation}, is now an
easy task since they \emph{all} follow a simple pattern.
%
The smallest bad configurations for the protocol in
Figure~\ref{figure:szymanski:diagram} is the %\emph{finite}
set $\minbad = \set{\w[b]{9,9}, \w[b]{10,9}, \w[b]{9,10},
  \w[b]{10,10}}$.
%
Any bad configuration contains at least one of the elements of
\minbad\ as a subword. We can therefore craftily define the set \Bad\
of all the bad configurations as the upward-closure of {\minbad}.
%
% We say that \minbad\ are the \emph{minimal elements} of {\Bad},
% sometimes also called its \emph{generators}.
The set \minbad\ are the minimal elements of {\Bad}, i.e.\ %
$\gen\Bad=\minbad$.
%
\index{Infinite-state systems}%
We represent the whole set of bad configurations (albeit infinite)
using a compact and finite symbolic representation.
%
$$\Bad = \ucl{\minbad} = \setcomp{c\in\confs}{\exists b\in\minbad : b\subword c}$$


