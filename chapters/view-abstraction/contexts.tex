%% ====================================================================
\section{Contexts}
\label{section:view:contexts}
%% ====================================================================
%
\index{Downward-closed sets}%
The simple abstraction presented in
Section~\ref{section:view:abstraction} allows us to compute
invariants\index{Invariant} that are \emph{downward-closed} with
respect to the subword relation~$\subword$. Several methods~%
\cite{%KMMPS2001,DLS01,BLW03,Tou01,
  CTV06,%
  PRZ-tacas01,%
  BHV04,%
  AJNO:simple,%
  APRXZ01,%
  Namjoshi:VMCAI07,%
  rmc:wo:transducers,%
  GerSis:many,%
  PXZ02}
have been devised to take advantage from such invariants, but there
are several classes of systems that are beyond their applicability.
%
The reason is that such systems do not allow good downward-closed
invariants, and hence over-approximating the set of reachable
configurations by downward-closed sets will give false
counter-examples.
%


Let us briefly explain where the problem comes from. A deeper
explanation can be found in Paper~\ref{paper:SAS14}.
%
Consider for example the view \w{a,b}. It emerges from some
configurations that contain the former view as a~subword.
%
It might be the case %is possible
that \emph{all} those configurations do also contain $\w{d}$ on the
right of $\w{b}$ (i.e.\ all the configurations contain in fact the
subword $\w{a,b,d}$).
%
In such a case, the transition
$\quantrule{a}{c}{\forall}{>}{\neg\set{d}}$ will be enabled on the
view $\w{a,b}$, but it would not have been enabled on any of the
configurations that contain that view as a~subword.
%
The problem is now that the new view $\w{c,b}$ is potentially harmful
in the sense that it can lead to a counter-example.
%
Notice moreover that no extension to views of larger size will always
disable the transition on those views (i.e.\ increasing the precision
will not eliminate the problem).

In this section, we introduce a new type of view and adapt the method
from Section~\ref{section:view:method} to target a class of invariants
which are needed in many practical cases and cannot be expressed as
downward-closed sets.

%% -----------------------------------------------------
\subsection{Context-Sensitive Views}
\label{section:view:contexts:abstract:domain}
%% -----------------------------------------------------
\index{View!Context-sensitive}%
A \emph{context-sensitive view} (henceforth only called \emph{view})
is a pair $(b_1\ldots b_k,R_0\ldots R_k)$, often written as
$R_0b_1R_1\ldots b_kR_k$, where $b_1\ldots b_k$ is a configuration and
$R_0\ldots R_k$ is a \emph{context}, such that $R_i\subseteq \locs$
for all $i\in\range 0 k$.
%
\index{Configuration}%
We call the configuration $b_1\ldots b_k$ the \emph{base} of the view
where $k$ is its \emph{size} and we call the set $R_i$ the $i^{th}$
\emph{context}. %
We use $\viewsof k$ to denote the set of views of size up to $k$.
%
For $k,n\in\nat, k\leq n$, let $H_n^k$ be the set of strictly
increasing injections $h \colon \range{0}{k+1} \rightarrow
\range{0}{n+1}$, i.e.\ \mbox{$1\leq i<j\leq
  k$} $\Rightarrow$ \mbox{$1\leq h(i)<h(j)\leq n$}.
%
Moreover, we require that $h(0) = 0$ and $h(k+1) = n+1$.

%% -----------------------------------------------------
\KW{Projections}%
\index{Projection}%
%% -----------------------------------------------------
%\noindent%
\begin{wrapfigure}{r}{0.48\linewidth}
  \hfill%
  \tikzinput{img/contexts-projection}
  %\caption{Projection}
  %\label{figure:projection}
\end{wrapfigure}
%
We define the projection of a configuration.
%
For $h\in H_n^k$ and a configuration \mbox{$c=q_1\ldots q_n$}, we use
$\proj{h}{c}$ to denote the view %
\mbox{$v=R_0b_1R_1\ldots b_kR_k$}, obtained in the following way %
%(see Figure~\ref{figure:projection}): %
(depicted on the right): %
\begin{enumerate}[label={(\roman{*})}]
\item $b_i=q_{h(i)}$ for $i\in\range 1 k$, %
\item $R_i=\setcomp{q_j}{h(i)<j<h(i+1)}$ for $i\in\range{0}{k}$. %
\end{enumerate}
%
\smallskip%
Intuitively, respecting the order, $k$ elements of~$c$ are retained as
the base of $v$, while all other elements are collected
into %an appropriate
\emph{contexts} as sets in the appropriate positions.

%% -----------------------------------------------------
%% Projection of a view
%% -----------------------------------------------------
%\noindent%
\begin{wrapfigure}[5]{r}{0.36\textwidth}
  \vspace{-1em}
  \hfill%
  \tikzinput{img/contexts-projection-view}
  %\caption{View Projection}
  %\label{figure:projection-view}
\end{wrapfigure}
%
We also define projections of views (depicted on the right).
% 
For a view \mbox{$v=R_0b_1R_1\ldots b_nR_n$} and %an injection
\mbox{$h\in H_n^k$}, we overload the notation from the projection of
configurations and use $\proj{h}{v}$ to denote the view %
\mbox{$v'=R'_0b'_1R'_1\ldots b'_kR'_k$}, such that: %
%(see Figure~\ref{figure:projection-view})%
\begin{enumerate}[label={(\roman{*})}]
\item $b'_i=b_{h(i)}$ for $i\in\range 1 k$ and 
\item $R'_i=\setcomp{b_j}{h(i)<j<h(i+1)}\union(\bigcup_{h(i)\leq
    j<h(i+1)}R_j)$ for all $i\in\range{0}{k}$.
\end{enumerate}

\smallskip%
%% -----------------------------------------------------
%% Entailement on views
\index{Entailement}%
%% -----------------------------------------------------
We define an \emph{entailment relation} on views of the same size. %
%
Let $u=R_0b_1R_1,\ldots,b_nR_n$ and $v=R'_0b'_1R'_1,\ldots,b'_nR'_n$
be views of the same size~$n$. %
We say that $v$ \emph{entails} $u$ or that $u$ is \emph{weaker}
than~$v$, denoted $u\entails v$, if $b_1\cdots b_n =b'_1\cdots b_n'$
and $R_i\subseteq R'_i$ for all $i\in\range{0}{n}$. %
%
Views of different sizes are not comparable.
% 
For two sets $V$ and~$W$ of views, we write $V\entails W$ if every
$w\in W$ entails some $v\in V$. Formally, $V\entails W \Leftrightarrow
\forall w\in W, \exists v\in V, v\entails w$.
%
We use $\minsetof V$ to denote the set of views in $V$ that are
\emph{weakest}, i.e.\ minimal w.r.t.~$\entails$.
%
We use $V\minunion W$ to denote the set $\minsetof{V\union W}$.

%% -----------------------------------------------------
\KW{Abstraction and Concretization}%
%% -----------------------------------------------------
We are now ready to define the abstraction and concretization
functions using context-sensitive views as symbolic encoding.
%
Let $k\in\nat$. %
\index{Abstraction}%
The \emph{abstraction function}~$\Absof{k}$ maps~$x$, a~view or
a~configuration, into the set of its projections of size~$k$ or
smaller:
%
$$
\Absof{k}(x) = \setcomp{\proj{h}{x}}{h\in H_{\sizeof x}^\ell,\ell\leq
  \min(k,\sizeof x)}
$$
% 
For a set $X$ of views or of configurations, we define $\Absof{k}(X)$
as the set of its weakest projections %
$\minsetof{\cup_{x\in X}\Absof{k}(x)}$.
%
\index{Concretization}%
The \emph{concretization} function $\Concof{k}$ maps a~set of views
$V\subseteq\viewsof k$ into the set of configurations %
$$\Concof{k}(V) = \setcomp{c\in\confs}{V\entails\Absof{k}(c)}$$

%% -----------------------------------------------------
\subsection{Verification Procedure and Approximation}
%% -----------------------------------------------------
%
The verification procedure from Algorithm~\ref{algo:view:procedure}
must be adapted to cope with contexts.
%
The procedure will still take advantage of extensions, but the latter
are no longer configurations of smaller size. They are now equipped
with a context, and the concrete post from
Section~\ref{section:view:abstraction} cannot directly be used on
them. Moreover, since $\Concof{k}(V)$ is in general infinite, we need
to compute the abstract post-image symbolically.
\index{Symbolic Representations}%

Although it is possible to compute the abstract transformer precisely
(see Paper~\ref{paper:SAS14}), we introduce an over-approximation for
efficiency reasons and show that the resulting procedure is sound and
complete.

%% -----------------------------------------------------
\subsubsection{Symbolic Post Operator}
\label{section:view:contexts:symbolic:post:operator}
\index{Post-image operator}%
%% -----------------------------------------------------
To define the symbolic post operator, we first define a~transition
relation on views.
%
For a view~$v=(\base,\ctx)$, $i\leq\sizeof\base$, and a~transition
$\delta\in\Delta$, we define the symbolic immediate successor of $v$
under a $\delta$-move of the $i^{th}$~process from $\base$, denoted
$\sdelta(v,i)$.
%
Informally, the moving process checks the other processes from the
base. In addition, if $\delta$ is a~universal transition, the moving
process checks as well the processes in the contexts.
%, to determine if it can move.
%
If the transition is enabled, the moving process from $\base$ changes
its state according to the $\delta$-transition, otherwise it is
blocked.
%
The contexts do not change.
%
% Applying a symbolic move is similar to first ``materializing'' the
% processes in the context $\ctx$ into a~configuration~$c$, performing
% the $\delta$-transition on $c$ (which necessarily contains $\base$)
% and projecting back the materialized processes return into their
% respective context.
%
In fact, we can here observe the role played by a context: it retains
enough information in a~view to \emph{disable} (or \emph{block})
universal transitions, which would have been otherwise enabled without
the presence of contexts. %
%
This reduces the risk of running a~too~coarse over-approximation.

% -----------------------------
Formally, for a view~$v=R_0b_1R_1{\ldots}b_nR_n$ and $i\leq n$, we
have that $\sdelta(v,i) = R_0b_1'R_1{\ldots}b_n'R_n$ %
iff %
\mbox{$b_i = \src$}, $b_i' = \dst$, $b_j=b_j'$ for all $j:j\neq i$ and
either
\begin{enumerate}[label={(\roman{*})}]
\item $\delta$~is a local rule $\src\trans\dst$, or   
\item $\delta$~is a global rule of the form
  $\quantrule\src{\dst}{\quantify}{\sim}{S}$, and one of the following
  two conditions is satisfied:
  % 
  \begin{enumerate}[label={(\alph{*})}]
  \item %
    $\quantify=\forall$ and it holds both that $b_j\in\witnesses$ for
    all $j\in\range{1}{n}$ such that $j\sim i$ and that
    $R_j\subseteq\witnesses$ for all $j\in\range{0}{n}$ such that
    $j\sim i$, or
  \item %
    $\quantify=\exists$ and there exists $j\in\range{1}{n}$ such that
    $j\sim i$ and $b_j\in\witnesses$.
  \end{enumerate}
\end{enumerate}
%
Note that we do not need to check the contexts in the latter
case. Indeed, this is supported by the fact that the views work
collectively. If there is a view where a process appears in a context,
then there is always another view where it appears in the base, while
the others are in a context.
%
% This is a nice place to recall the strength of the abstraction, and
% say that the set $\concretizeoflim{k}{k+1}{V}$ is
% super-precise. Moreover, if a context contains some crap process, it
% does not necessarily create a problem.
%
%% -------------------------
Finally, for a~set of views~$V$, we define $\spost(V) =
\setcomp{\sdelta(v,i)}{ v \in V, i \leq \sizeof{v}, \delta\in\rules}$.
%

%% -------------------------
We now explain how we define the \emph{symbolic post} operating on
views.
%
It is based on the observation that a process needs \emph{at most one}
other process as a witness in order to perform its transition %
%
(cf. an existential transition as illustrated in
Figure~\ref{figure:view:small:preconditions}).
% ; Other processes do not help
% to enable local or universal transitions at all).
%
A~moving process can appear either (i)~in the base of a~view, or
(ii)~in a~context.
%
\index{View!Extensions}%
\emph{Extending adequately} the view with one extra process is enough
to determine whether the moving process, in case~(i), can perform its
transition.
%
However, in case~(ii), since $\spost$ only updates processes of the
base, a~first extension %with one process
``materializes'' a moving process into the base and a~second
extension %by one process
considers its witness.
%
Therefore, it is sufficient to extend the views with two extra
processes to determine if a~transition is \emph{enabled}, whether the
moving process belongs to the base or a~context of a~view.
%
Formally, for a~set $V$ of views of size~$k$ and for $\ell>k$, we
define the \emph{extensions} of~$V$ of size~$\ell$ as the set of views
% $\Concoflim{k}{\ell}(V)=\setcomp{v\in\viewsof{\ell}}{\abstrof{k}(v)\entailedby{}V}$.
$\Concoflim{k}{\ell}(V)=\Absof{l}(\Concof{k}(V))$.
%
Finally, we define the \emph{symbolic post} as %
$\Absof{k}\circ\spost\circ\Concoflim{k}{k+2}(V)$.
%
Similarly to the equation on page~\pageref{view:abstraction:equation},
we have shown in Paper~\ref{paper:SAS14} that the symbolic post is the
best abstract transformer, i.e.\ for any~$k$ and set of views $V$ of
size up to $k$, it holds that %
$$V \minunion
\Absof{k}\circ\post\circ\Concof{k}(V) = V \minunion \Absof{k}\circ
\spost \circ \Concoflim{k}{k+2}(V) $$

%% -----------------------------------------------------
\subsubsection{Approximation}
\index{Concretization}%
%% -----------------------------------------------------
The computation of the above exact symbolic post comes at some cost,
mostly due to the computation of $\Concoflim{k}{\ell}(V)$ .
%
We therefore introduce an over-approximation and compute the set
$$\aConcoflim{k}{\ell}(V)=\setcomp{v\in\views}{\Absof{k}(v)\entailedby{}V,\sizeof{v}\leq\ell}$$
i.e.\ the set of views of size~$\ell$ that can be generated from~$V$,
without inspecting its concretization first.
%
Views in $\aConcoflim{k}{\ell}(V)$ have (at least) the same bases as
the views in $\Concoflim{k}{\ell}(V)$, but they might have smaller
contexts (and are therefore weaker). As a~consequence, they might
enable more (universal) transitions than their counterparts in
$\Concoflim{k}{\ell}(V)$.
%
Consider for example the case where $k=2$, $\ell=3$ and the set of
views $V = \set{ab, bc, ac[e], ce[f], ae, be, af, bf, cf, ef}$ --- we
write contexts in brackets, and ignore the empty contexts for brevity.
%
The set $\aConcoflim{2}{3}(V)$ contains the view~$abc[e]$ but
$\Concoflim{k}{\ell}(V)$ contains the view~$abc[e,f]$ because the
smallest configuration in $\Concof{2}(V)$ that has $abc$ as a subword
is $abcef$ (this is due to the view $ce[f]$ which enforces the
presence of~$f$).
%
Another example is $V = \{ab,bc,ac[e],a[c]e,[a]ce\}$.
%
Here, $\aConcoflim 2 3 (V)$ contains $abc[e]$, however, there is
no view with the base $abc$ in $\Concoflim{2}{3}(V)$ since there
is no configuration with the subword $abc$ in $\Concof 2 (V)$.

We show in Paper~\ref{paper:SAS14}, that it is an
over-approximation, i.e.\ for any $\ell\geq k$ and $V\subseteq\views$,
it holds that %
$$\aConcoflim{k}{\ell}(V)\entails \Concoflim{k}{\ell}(V)$$

%% -----------------------------------------------------
\subsubsection{Sound and Complete algorithm}
%% -----------------------------------------------------
%
We combine the fixpoint computation of the symbolic post with a
systematic state-space exploration in order to find a %concrete
bad configuration.

\noindent%
\begin{wrapfigure}{r}[0pt]{0.62\linewidth}
  \hfill%
  \begin{algorithm}[H]
    \DontPrintSemicolon%
    \For{$k:=1$ \bf{to} $\infty$\nllabel{implementation:outerloop}}{%
      \lIf{$\isbad(\Reach_{k})$\nllabel{implementation:exact}}{\Return Unsafe}%\;%
      $V := \mu X\,.\,\Absof{k}(\Inits) \minunion \Absof{k}\circ\spost\circ\aConcoflim{k}{k+2}(X)$\nllabel{implementation:fixpoint}\;%
      \lIf{$\neg\isbad(V)$\nllabel{implementation:test}}{\Return Safe}%
    }
    \caption{Verification Procedure}
    \label{algo:view:procedure:contexts}
  \end{algorithm}
\end{wrapfigure}
%
The algorithm (described succintly in
Algorithm~\ref{algo:view:procedure:contexts}) proceeds by iteration
over configurations and views of size up to~$k$, starting from $k=1$
and increasing~$k$ after each iteration.
%
%\newpage%====================================
Every iteration consists in two computations in parallel:
\begin{enumerate}[label={(\alph{*})}]
\item %
  Using the exact post-image, we compute the set $\Reach_k$ of
  configurations reachable from the initial configurations, involving
  only configurations of size~$k$ % up to~$k$
  (line~\ref{implementation:exact}).
  %
  Note that there are only finitely many such configurations and that
  we consider %, in this chapter,
  length-preserving transitions,
  % \footnote{Refer to Paper~\ref{paper:SAS14} for the case of non
  %   length-preserving transitions.} %
  so this step terminates and
\item %
  the fixpoint computation of the symbolic post over views of size up
  to~$k$.
\end{enumerate}
%
%
% Moreover, a reachable bad configuration is reachable by a path via
% configurations of some maximal size, hence it will be eventually
% discovered.
%
A~reachable bad configuration of some size must be reachable through
a~sequence of transitions involving configurations of %
some maximal size, %the same size,
so it will be eventually discovered.
%
In the fixpoint computation of the symbolic post
(line~\ref{implementation:fixpoint}), it is sound to replace
$\Concoflim k {k+2}$ with the
over-approximation~$\aConcoflim{k}{k+2}$.


\index{Verification Algorithm!Termination}%
Finally, the termination criteria on line~\ref{implementation:exact}
and~\ref{implementation:test} require the use of the
function~$\isbad$, which returns whether a set of configurations
contains a~bad configuration or whether a set of views characterizes
a~bad configuration, depending on the type of its input parameter.
%
If its input parameter is a set $X$ of configurations, the
function~$\isbad$ is implemented by checking whether any configuration
from~$\minbad$ can be a~subword of some configuration in $X$.
%
If its input parameter is a set $V$ of views, the function~$\isbad$ is
implemented by checking whether an element of~$\minbad$ can appear
within the base of a~view from $V$.
%
We do not inspect any context, because the views work collectively and
there is always another view in the set which contains this context
element in its base.

\index{Soundness}%
The resulting verification algorithm is sound and terminates for
some~$k$ if and only if there is a reachable bad configuration or if
there is a~good invariant\index{Invariant} %
(of a~specific form which expresses more than downward-closed
properties --- see Paper~\ref{paper:SAS14}).
%
\index{Small Model Property}%
It uses the property of small models, that is, most behaviors are
captured with small instances of the systems, either in the form of
configurations and views.

%% ------------------------------------------------------------------
\subsubsection{Acceleration}
\index{Entailement}%
\index{Verification Algorithm!Acceleration}%
%% ------------------------------------------------------------------
In a similar manner to the procedure from \ref{section:view:method},
the fixpoint computation on line~\ref{implementation:fixpoint} can be
accelerated by leveraging the entailment relation. %
It is based on the observation that $\Reach_k$ contains configurations
of size up to~$k$, which can be used as initial input for the fixpoint
computation (rather than~$\Inits$).
%
All views of size~$k$ in $\Absof{k}(\Reach_k)$ have empty contexts
(i.e.\ they are weakest). %
They avoid the computations of the symbolic post on any stronger
views.
%
A~similar argument can be used to see that it is not necessary to
apply $\spost$ on the views in $\aConcoflim{k}{k+2}(X)$ that are
stronger than the views in $\Absof{k+2}(\Reach_{k+2})$.
%
We therefore seed the fixpoint computation with a larger set than
$\Absof{k}(\Inits)$, namely
$\Absof{k}(\Reach_k\union\Reach_{k+1}\union\Reach_{k+2})$, and cache
the set of views $\Absof{k+2}(\Reach_{k+2})$.
