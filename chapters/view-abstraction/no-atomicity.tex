%% ====================================================================
\section[No Atomicity]{Dropping the Atomicity Assumption}
\label{section:view:non:atomic}
%% ====================================================================

\index{Global Conditions!non-atomic}%
We present an extension of our method to handle parameterized systems
where global conditions are \emph{not} checked atomically.
%
It is necessary for the model to keep track of intermediate
configurations when a~non-atomic global condition is evaluated at the
same time as another transition, potentially also guarded by a~global
condition.
%
We use the model presented in
Section~\ref{section:paramsys:non:atomic}: %
both existentially and universally guarded transitions are replaced
with a~variant of a~for-loop rule of the form:
%
\hfill%
$\forrule \sim \witnesses \src {\dst} e$

\noindent%
where $e\in Q$ is the \emph{escape} state.
%
We recall that, for a configuration with linear topology, in that
model, a~process inspects the states of the other processes
\emph{in-order}.
%
Therefore, it is sufficient to only keep track of the last position
that each process has inspected (using the total map
$\tick:\range{1}{n}\rightarrow\range{0}{n}$, initially assigned the
value~$0$). It follows that, for every process $i$, lower positions
than~$\tick(i)$ (in the available range $\sim$) are then also
inspected, while higher positions are not yet inspected.%
\footnote{%
  In the case processes do not loop in-order, $\tick$~is replaced with
  a~binary relation $R\subseteq\range{1}{n}\times\range{1}{n}$ on
  positions, initially empty. When process $i$ inspects process $j$,
  the pair $(i,j)$ is added to the relation. We say that $i$
  \emph{ticks} $j$. This can be implemented with a~matrix of size
  $n{\times}n$ of boolean values and allows us to cover the case where
  processes inspect each other in a~random order.
  % 
  % However, for simplicity in this section, we present the case where
  % processes are inspected in-order.%
}%
% Furthermore, we present at once the non-atomic extension in the case
% of context-sensitive views.

% ----------------------------------
%\subsubsection{Abstract Domain in the Non-Atomic Setting}
%\subsubsection{Abstract Domain}
%
\index{Abstraction}%
In order to instantiate an abstract domain, we need to handle the fact
that a context is a set and does not reflect which processes have been
inspected by another given process.
%
\index{Context-sensitive view}%
We extend the (context-sensitive) views
% from Section~\ref{section:view:contexts:abstract:domain}.
with some extra information.
%
% They will be analogies of the original ones, customized to the new
% notion of configuration and successor function.
%
A view is now of the form %
$(R_0q_1R_1\ldots q_nR_{n},\tick,\unticked)$, %
where $(q_1\cdots q_n,\tick)$ is a configuration called the
\emph{base}, and $(R_0,\cdots,R_n,\unticked)$ is a \emph{context},
such that $R_0,\ldots,R_{n}\subseteq Q$ and $\unticked:\range 1
n\rightarrow 2^\locs$ is a total map which assigns a subset of
$\locs$ %$R_{\tick(i)}$ % a subset of $\importantof i \cap R_{\tick(i)}$
to every $i\in\range{1}{n}$.
%
Intuitively, the role of $\unticked(i)$ is to keep track of the
processes that process~$i$ has not yet inspected in case they get
mixed up in a context with other already inspected processes.
%
This will be the case, as depicted in
Figure~\ref{figure:projection-na}, for one context only, say $R_\ell$
(in fact, $R_\ell$ is the context where $\tick(i)$ is projected~to). %
It is obvious that contexts of higher (resp.\ lower) indices than
$\ell$ contain processes that are not (resp.\ are) inspected by
process~$i$.
% Intuitivelly, $\loc \in \unticked(i)$ marks the presence of a~process
% in state $\loc$ in the context $R_{\tick(i)}$ that the
% $i^{th}$~process has not yet ticked.

\begingroup
%\setlength{\intextsep}{\dazintextsep}
\setlength{\intextsep}{\the\bigskipamount}
\begin{figure}[ht]
  \begin{minipage}{0.55\linewidth}
    \noindent\tikzinput[\linewidth]{img/na-projection}
  \end{minipage}%
  \begin{minipage}{0.45\linewidth}
    \caption{Projection with non-atomicity. The blue states have been
      inspected by process~$i$, the green states have not. The
      abstraction needs to distinguish for that context which states
      have been inspected from those that have not.}
  \label{figure:projection-na}
  \end{minipage}
\end{figure}
\endgroup

%% -------------------------------------------------
In order to use Algorithm~\ref{algo:view:procedure:contexts} with the
new abstract domain (i.e.\ abstraction and concretization), we need in
fact only to adjust the notion of projections, the entailement
relation and the symbolic post on views.
%
The procedure will then run as in the previous section.

%% -------------------------------------------------
\index{Projection}%
The projection of a~configuration into a~view is defined in a similar
manner as in
Section~\ref{section:view:method}. % provided that the right-most ticked
% position and the set of not-yet ticked states are now properly
% updated.
%
For $h\in H_n^k, k\leq n$, and a~configuration
$c=(q_1{\cdots}q_n,\tick)$, $\proj h c = (\proj h
{q_1{\cdots}q_n},\tick',\unticked')$ where $\tick'$ and $\unticked'$
are defined as follows. %
For all $i\in \range{1}{k}$, there exists $\ell$ such that
$h(\ell)\leq \tick(i) < h(\ell+1)$. Then, $\tick'(i) = \ell$ and
$\unticked'(i) = \setcomp{q_j}{\tick(i) < j < h(\ell+1)}$.
%
% Informally, the sets of ticked and unticked processes in the contexts
% are updated accordingly as in Fig.~\ref{figure:projection}.
%
The projection of views is defined analogously.
%
Note that this definition also implies that the concretization of a
set of views is precise enough and reconstructs configurations with
in-order ticks.

% ----------------------------------
\index{Entailement}%
The entailment relation between the views $v = (R_0q_1R_1\ldots
q_nR_{n},\tick,\unticked)$ and $v' = (R_0'q_1'R_1'\ldots
q_n'R_{n}',\tick',\unticked')$ (of the same size) is defined such that
$v\entails v'$ iff
\begin{enumerate}[label={(\roman{*})}]
\item both have the same base, i.e.\ %
  $(q_1\cdots q_n,\tick) = (q_1'\cdots q_n',\tick')$,
\item $R_i\subseteq R_i'$ for all $i\in\range{0}{n}$, and
%$R_0\subseteq R_0', \cdots, R_n\subseteq R_n'$
\item $\unticked(i)\subseteq \unticked'(i)$ for all $i\in\range{1}{n}$.
%$\unticked(1)\subseteq \unticked'(1), \cdots, \unticked(n)\subseteq \unticked'(n)$
\end{enumerate}
%
This intuitivelly reflects that the more unticked states within a
context the likelier it is for a transition to be blocked, and the
larger contexts are the likelier they retain non-ticked
states. %in the future.
%
%Both impose more restrictions on configurations.

% Finally, the abstraction and concretization functions are used as-is
% using the new definition of projection and entailment.

% ----------------------------------
%\subsubsection{Method}
Since we do not maintain any order in the contexts, we
cannot %have no possibilities to
make a~particular view reflect an intermediate state of an in-order
for-loop rule. However, recall that views work collectively, so there
will be another view distinguishing that intermediate state.
%
Therefore, the symbolic post $\spost$ is adapted from the previous
post operators, with the particularity that it ``ticks'' each context
at~once as a~block and handles the extra $\unticked$ information
adequately.
%
% The only component to adapt in the function $\spost$, which handles
% views similarly as in
% Section~\ref{section:view:contexts:symbolic:post:operator}.
% %
% In fact, the symbolic post is close to the concrete post operator
% defined in Section~\ref{section:paramsys:non:atomic}, with the
% particularity that it ``ticks'' each context at~once as a~block.
Intuitively, process $i$ inspects its ``next'' context by ``ticking''
the elements of~$\unticked(i)$ unless it has to escape.
%
If it then cannot %does not need to
escape, it moves on to the following position and marks the following
context as unticked (the new value of~$\unticked(i)$).
%
Notice that the inspected context might contain process states that
would make the transition escape (but not~$\unticked(i)$). It means
that those processes potentially changed state \emph{after} they got
ticked.
%
Finally, process~$i$ terminates if there is no more context or base
element to inspect.

Formally, we fix a~view $v=(R_0b_1R_1{\ldots}b_nR_n,\tick,\unticked)$
of size~$n$, a~position~$i\in\range{1}{n}$ and a~global transition
$\delta = \forrule{\sim}{\witnesses}{\src}{\dst}{e}$ from $\rules$ %
(since the case of a~local transition is trivial).
%
As before, we distinguish three types of symbolic $\sdelta$-move
on~$v$ by the process in the base at position~$i$ %
(See Figure~\ref{figure:view:non:atomic:ticking}):
%
(a)~$\sdelta_i(v,i)$ for a~loop iteration, %
(b)~$\sdelta_e(v,i)$ for escaping and %
(c)~$\sdelta_t(v,i)$ for termination.
% 
Each type of move is defined only if $b_i = \src$. 

\begingroup
%\setlength{\intextsep}{\dazintextsep}
\setlength{\intextsep}{\the\bigskipamount}
\begin{figure}[!h]
  \centering
  \tikzinput[0.32\linewidth]{img/na-ticks-iteration}\hfill%
  \tikzinput[0.32\linewidth]{img/na-ticks-escape}\hfill%
  \tikzinput[0.32\linewidth]{img/na-ticks-terminal}
  \caption{Non-atomic ``ticking'' for the global transition
    $\forrule{\neq}{\neg\set{1,2}}{3}{7}{4}$. Note that contexts are
    ticked at once.}
  \label{figure:view:non:atomic:ticking}
\end{figure}
\endgroup

%===========================
Recall that $\tick(i)$ represents the position that process $i$ has
last inspected, which is either $0$ at the start or always points to
an element of the base.
%
% We illlustrate the non-atomic ``ticking'' using the global transition
% $\forrule{\neq}{\neg\set{1,2}}{3}{7}{4}$ (Note that contexts are
% ticked at once).
%

% \noindent%
% {%
%   \tikzinput[0.32\linewidth]{img/na-ticks-iteration}\hfill%
%   \tikzinput[0.32\linewidth]{img/na-ticks-escape}\hfill%
%   \tikzinput[0.32\linewidth]{img/na-ticks-terminal}%
% }
%
\begin{enumerate}[label={(\alph{*})},leftmargin=0pt]
\item %
  Carefully looking at the indices in the view, process $i$ now
  inspects the processes of the \emph{first} context that it has not
  inspected, i.e.\ the context~$R_{\tick(i)}$.
  %
  $\sdelta_i(c,i)$ is defined if both the following two properties are
  satisfied: %
  (i) $\unticked(i)\subseteq\witnesses$ and %
  (ii) $\tick(i)+1\sim i$, $\tick(i)+1 \leq n$ and
  $b_{\tick(i)+1}\in\witnesses$.
  %
  It is then obtained from~$v$ by only
  updating $\tick(i)$ to~$\tick(i)+1$ and resetting $\unticked(i)$ to
  $R_{\tick(i)+1}$.
  %
  % Intuitively, process~$i$ is only \emph{ticking}
  % position~$\tick(i)+1$ and the context in between as a~block. Note
  % that it is possible that $R_{\tick(i)+1}$ contains elements from
  % $\witnesses$, but if they are not members of $\unticked(i)$, then
  % they might have been already inspected and do not disable the
  % transition or causes it to escape.
\item %
  $\sdelta_e(c,i)$ is defined if one of the following two properties
  is satisfied: %
  (i) $\unticked(i)\not\subseteq\witnesses$ or
  (ii) $\tick(i)+1\sim i$, $\tick(i)+1 \leq n$ and $b_{\tick(i)+1}\not\in\witnesses$.
  % 
  It is obtained from~$v$ by changing the state of the process~$i$
  to~$e$ and resetting $\tick(i)$ to~$0$ and $\unticked(i)$
  to~$\emptyset$. Intuitively, process~$i$ has found a~reason to
  escape.
\item %
  $\sdelta_t(c,i)$ is defined if $\tick(i)+1\not\sim i$ or $\tick(i)+1>n$.
  %
  It is obtained from~$v$ by changing the state of the process~$i$
  to~$\dst$ and resetting $\tick(i)$ to~$0$ and $\unticked(i)$
  to~$\emptyset$. Intuitively, process~$i$ has reached the end of the
  iteration and terminates its transition (i.e.\ moves to its target
  state).
\end{enumerate}

%\index{Verification Algorithm}%
\smallskip%
This concludes how we adapted
Algorithm~\ref{algo:view:procedure:contexts} to cope with
non-atomically checked global conditions in the presence of contexts.
%
% See also Paper~\ref{paper:VMCAI13},~\ref{paper:SAS14}
% and~\ref{paper:STTT15}.
For further details, refer to
Paper~\ref{paper:VMCAI13},~\ref{paper:SAS14} and~\ref{paper:STTT15}.
