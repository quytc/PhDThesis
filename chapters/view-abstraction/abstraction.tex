%% ====================================================================
\section{Abstract Domain}
\label{section:view:abstraction}
\index{Abstraction}%
%% ====================================================================


The \emph{view abstraction} considers the configurations of the system
from the perspective of a few \emph{fixed} number of processes.
%
The abstraction is parameterized by a constant $k$, and any
configuration is intuitively ``broken down into pieces'' of size~(at
most)~$k$, called \emph{views}.
%
A view retains the information about $k$ processes from a
configuration and abstracts away the other processes. %
There is a certain freedom in what information to retain and what to
abstract away. %
For the $k$ remaining processes, the information could be partial
or %stay
intact, while the information about the abstracted processes could be
fully or partially discarded. %
This choice defines the level of abstraction that transforms
configurations into views. %
% We will see in Section~\ref{section:view:contexts} that there is a
% fine and simple balance to strike in what information to retain and
% what to abstract away.

%\smallskip%
%% ********************************************************************
\KW{Abstraction}%
%% ********************************************************************
We first define a simple abstraction: for every $k\in\nat$, the $k$
chosen processes are retained intact, while the other abstracted
processes are ignored. %
\index{View}%
A view is then a subword of a configuration, using the subword
relation~$\subword$ from Section~\ref{section:upward:closed}. %
% We present another variant in Section~\ref{section:view:contexts}.
% 
We use $\views$ to denote the set of all views (and $\viewsof k$ for
the set of views of size up to~$k$). %
% (i.e.\ $\confsof k=\setcomp{c\in\confs}{\sizeof{c}\leq k}, \viewsof
% k=\confs_{\leq k}$).
Since views are here configurations of smaller sizes, we have hence
that %
$\viewsof{k} = \setcomp{v\in\views}{\sizeof{v}\leq k} \subseteq
\setcomp{c\in\confs}{\sizeof{c}\leq k}$.
%
The \emph{abstraction function} \index{Abstraction}%
$\Absof k : \confs\mapsto 2^{\viewsof k}$ maps a~configuration $c$
into the set of
all its views (subwords) of size up to $k$:
%
$$\Absof k (c) = \setcomp{v\in\viewsof k}{v\subword c}$$
% 
We lift $\Absof k$ to sets of configurations as usual.
% 
% Recall that we restrain the presentation in this chapter to linear
% topology. %
% % Formally, 
% We define the downward-closure of a~configuration~$c$, using the
% subword relation~$\subword$ from Section~\ref{section:upward:closed},
% as the set $\setcomp{d\in\confs}{d\subword c}$ of all smaller (in
% size) configurations that are subword of~$c$. %
% We use $\dcl{c}$ to denote it.
% % 
Observe that views ressemble configurations of smaller sizes but this
does not always need to be the case: they can be complex abstract
entities (c.f.\ Section~\ref{section:view:contexts}). %

%% ********************************************************************
\KW{Concretization}%
%% ********************************************************************
The \emph{concretization function} \index{Concretization}%
$\Concof k : 2^{\viewsof k}\mapsto 2^\confs$ takes as input a~set of
views $V\subseteq\viewsof k$, and returns the set of configurations
that can be reconstructed from the views in $V$.
%
In other words,\footnote{In the field of 
  abstract interpretation, $(\Absof k, \Concof k)$ forms a
  Galois connection (see Paper~\ref{paper:SAS14} and~\ref{paper:STTT15}).%
  \index{Galois Connection}%
}
$$\Concof{k}(V) =
\setcomp{c\in\confs}{\Absof{k}(c)\subseteq V}$$

\begingroup%
\setlength\intextsep{\dazintextsep}
\begin{figure}[ht]
  \centering
  \tikzinput{img/view-example-abstraction}
  \caption{Consider the configurations on the left, theirs views in
    the middle and their concretization on the right. %
    The concretization contains the original set of configurations,
    but also the extra configuration \protect\w{1,2,4}, %
    i.e.\ this abstraction is an over-approximation.}
  \label{figure:view:abstraction:example}
\end{figure}
\endgroup

%\smallskip%
%% ********************************************************************
\KW{View work collectively}%
%% ********************************************************************
In general, the set of views \emph{collectively} represent a set of
reachable configurations.
%
There is an important subtlety here to catch: No information about
configurations is necessarily destorted or ignored in this
abstraction, the configurations are merely scattered across views. If
the information about a given process is ignored in one view, it will
necessarily appear in another view.
%
It is an over-approximation because the views can reconstruct the
configurations they emerge from, but they could also be recombined to
form new configurations that were not part of the original system, for
any size, as illustrated in the example from
Figure~\ref{figure:view:abstraction:example}.

%% ********************************************************************
%\paragraph{Precision.}
\KW{Precision}%
\index{Precision of View Abstraction}%
%% ********************************************************************
It is important to observe the precision of the set
$\Concof{k}(V)$. As we just mentioned, the views work collectively to
represent the set of configurations. If a view, say \w{1,2}, is
recombined into the configuration \w{1,2,3} in $\Concof{2}(V)$, it
must be the case that the views \w{1,3} and \w{2,3} were \emph{also}
present in $V$.
%
\index{Process correlation}%
The correlation between the $k$ processes appearing in a~view
increases the precision of the abstraction. In other words, we get
more precise recombinations with larger $k$, i.e.\ formally, for a set
of configurations $X\in\confs$,
$$
\Concof{1}(\Absof{1}(X))\supseteq 
\Concof{2}(\Absof{2}(X))\supseteq
\Concof{3}(\Absof{3}(X))\supseteq
% \cdots\supseteq
% \Concof{k}(\Absof{k}(X))\supseteq
\cdots\supseteq X$$

%% ********************************************************************
%\paragraph{Abstract post-image.}
\KW{Concrete post-image}%
\index{Post-image operator}%
%% ********************************************************************
For a~set of configurations $X\subseteq\confs$, we define the
\emph{post-image} of $X$ as the set %
$\post(X) = \setcomp{\delta(c,i)}{c\in X,
  i\leq\sizeof{c},\delta\in\Delta} = \setcomp{c'}{c\in X, c\trans
  c'}$.
%% ********************************************************************
%\paragraph{Abstract post-image.}
\KW{Abstract post-image}%
\index{Abstract Post-image operator}%
%% ********************************************************************
%
The \emph{abstract post-image} of a~set of views
$V\subseteq\viewsof{k}$ is defined, as usual, as the composition
$\apost{k}(V) = \Absof{k}(\post(\Concof{k}(V)))$.



