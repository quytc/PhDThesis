%% ====================================================================
\section{Method}
\label{section:view:method}
%% ====================================================================

Now that we have defined the abstraction function and how to jump
between sets of configurations and sets of views, we are ready to
describe the procedure that manipulates views.
\index{Verification Scheme}%
%
The procedure is a forward analysis, composed of two nested loops.
%
The inner-loop explores if the initial configurations can reach the
set~\Bad\ of bad configurations, in the abstract domain using views of
size up to~$k$.
%
\index{Cut-off}%
The outer-loop determines a~\emph{\mbox{cut-off}} point $K$ such that
the views $V\subseteq\viewsof{K}$ of size~$K$, computed by the
inner-loop, satisfy the following properties:
%
\begin{enumerate}[label=(\roman*)]
\item \index{Invariant}%
  \label{view:abstraction:fixpoint}$V$ is an invariant for all instances, %
  i.e.\ $\Reach\subseteq\Concof{K}(V)$ and $\apost{K}(V)\subseteq{V}$
\item \index{Safety}%
  \label{view:abstraction:safety}$V$ is sufficient to prove safety, %
  i.e.\ $\Concof{K}(V) \cap \Bad = \emptyset$.
\end{enumerate}
%
%
% Recall that \Reach\ is the set of the reachable configurations (of any
% size). %

\noindent%
\begin{wrapfigure}{r}[0pt]{0.55\linewidth}
  \begin{algorithm}[H]
    \DontPrintSemicolon
    \For{$k:=1$ \bf{to} $\infty$\nllabel{scheme:view:outerloop}}{%
      \lIf{$\Reach_{k}\cap\Bad\neq\emptyset$\nllabel{scheme:view:exact}}{\Return Unsafe}%\;
      $V := \mu X\,.\,\Absof{k}(\Inits) \cup \apost{k}(X)$\nllabel{scheme:view:fixpoint}\;
      \lIf{$\Concof{k}(V) \cap \Bad = \emptyset$\nllabel{scheme:view:test}}{\Return Safe}
    }
    \caption{Verification Scheme}
    \label{algo:view:scheme}
  \end{algorithm}
\end{wrapfigure}
%
Point~\ref{view:abstraction:fixpoint} expresses that we have reached a
set of views that collectively over-approximates the set~$\Reach$ of
all reachable configurations, and that we cannot get new views by
applying the abstract post-image. So the set $V$ is ``stable'' and
point~\ref{view:abstraction:safety} states that none of the
configurations represented by~$V$ are bad, so the system is safe.
%
% We denote $\Reach_n$ the set of reachable configurations of the
% system~$S_n$, i.e.\ $\Reach_n = \post^{*}(\Inits_n)$ where $\Inits_n$
% is the set of initial configurations of size~$n$ and $\post^{*}$ is
% the repeated application of the (concrete) post-image.%
% \footnote{We consider here transitions that do not change the size of
%   a configuration. Paper~\ref{paper:VMCAI13} presents some extensions
%   to deal with non size-preserving transitions.} %
%

%% ********************************************************************
\KW{Nested loops}%
%% ********************************************************************
\noindent%
\begin{wrapfigure}{l}[0pt]{0.65\linewidth}
  %\centering
  \tikzinput[\linewidth]{img/view-method}
  \caption{Algorithm~\ref{algo:view:scheme} as a diagram.}
  \label{algo:view:diagram}
  \smallskip
\end{wrapfigure}
%
The outer-loop on line~\ref{scheme:view:outerloop} searches for a
suitable~$k$.
%
The procedure starts by computing (on line~\ref{scheme:view:exact})
the reachable configurations of size~$k$, concretely (i.e.\ without
abstraction), denoted $\Reach_k$.
%
Formally, $\Reach_k = \post^{*}(\Inits_k)$ where $\Inits_k$
is the set of initial configurations of size~$k$ and $\post^{*}$ is
the repeated application of the (concrete) post-image.%
\footnote{We consider here transitions that do not change the size of
  a configuration. Paper~\ref{paper:VMCAI13} presents some extensions
  to deal with non size-preserving transitions.} %
%
If a bad configuration is discovered during that step, the system is
surely not safe, and we can even pinpoint a~counter-example.
%
\index{Fixpoint}%
Otherwise, the procedure continues to the inner-loop, a fixpoint
computation (line~\ref{scheme:view:fixpoint}) of a set of views that
at least covers the views of size~$k$ generated from the initial
configurations.
%
The inner-loop consists of computing the recombinations from the
so-far collected set of views, and detecting if the abstract
post-image generates new views. In such a case, the procedure loops
and starts again from the new set of views. If not, the fixpoint is
reached and no new views will be detected.

Assume that the inner-loop on line~\ref{scheme:view:fixpoint} reaches
a fixpoint. The set $V$, by construction at the end of the inner-loop,
contains the views of~$\Inits$ and is stable under the abstract
post-image, i.e.\ for some $k$, we have both
$\Absof{k}(\Inits)\subseteq V$ and $\apost{k}(V)\subseteq V$.
%
It is not hard to derive that this set covers in fact all the views of
$\Reach$, that is $\Absof{k}(\Reach)\subseteq V$ and the set $V$
fulfills point~\ref{view:abstraction:fixpoint}.
%
The proof is in paper~\ref{paper:STTT15}.


The \mbox{cut-off} condition \index{Cut-off}%
is tested on line~\ref{scheme:view:test}. %
There are two outcomes: %
\begin{enumerate}[label=(\alph*)]
\item if the test fails, we don't know whether the system is indeed
  unsafe or whether the abstraction introduced a~spurious
  behaviour. The procedure increases $k$, hence the precision of the
  abstraction, and reiterates the outer-loop
  (line~\ref{scheme:view:outerloop}),
\item the test succeeds so the computed set $V$ fulfills
  point~\ref{view:abstraction:safety} %
  (and point~\ref{view:abstraction:fixpoint} by the fixpoint
  computation) so the system is then safe. %
\end{enumerate}

%% ====================================================================
%\vspace{-\baselineskip} % Cheating
%\subsection*{Implementation}
\subsubsection*{Implementation}
\KW{Heart of the method}%
\index{Verification Algorithm}%
%% ====================================================================
%
Algorithm~\ref{algo:view:scheme} is only a scheme and computing
$\apost{k}(V)$ is a~central component of the verification procedure.
%
It cannot be computed in a~straightforward manner because the set
$\Concof{k}(V)$ is in general infinite and therefore hardly
representable.
%
However, we can easily see that applying the abstraction function on
the post-image of the configurations, which are the (potentially
infinite) re-constructions~$\Concof{k}(V)$ from some set of views~$V$,
will mostly return the same views that were already in~$V$, and
generate a~few new ones. %
Indeed, for each configuration, only a ``small'' part changes, the
other parts will be abstracted into the same views.
%
This is why it is interesting to enable all transitions on views
directly, as if they were performed on configurations, effectively
removing the need to re-construct the full~configurations.

Going in that direction, we must notice that the global condition of
a~transition can mention some process states that have been abstracted
away, making the transition disabled.
%
\index{View!Extensions}%
In order to enable those transitions, we try to \emph{extend} the
views such that they would encompass the missing witness processes,
necessary to perform the transitions.
%
In fact, we show that it is sufficient to consider only the
configurations in $\Concof{k}(V)$ with size up to~$k+1$ (that is,
extensions with only \emph{one} extra witness).
%
There are finitely many such configurations, and hence their
post-image can be computed.
%
Formally, for $\ell\geq 0$, we define %
$$
\Concoflim{k}{\ell}(V)\coloneqq
\setcomp{c\in\confs}{\Absof{k}(c)\subseteq X,\,\sizeof{c}\leq\ell}
$$%
and we have proven that the set of views
$\Absof{k}(\post(\Concof{k}(V)))\union V$ can be instead computed
using the following (finite) set: %
\label{view:abstraction:equation}%
$$\Absof{k}(\post(\Concof{k}(V)))\,\union\,V\;=\;\Absof{k}(\post(\Concoflim{k}{k+1}(V)))\,\cup\,V$$

Using the previous equation, we can alleviate the problem of
reconstructing the (potentially infinite) set of configurations
$\Concof{k}(V)$ in the abstract post computations.
%
Instead, it is enough to extend the views $V$ of size $k$ in a
consistent way, yielding a finite set, and apply the transitions on
those extended views.
%
The union on both side of the equation handles the case where a
transition is potentially disabled.
%
The full proof can be found in paper~\ref{paper:VMCAI13}.



%% ********************************************************************
\KW{Small preconditions}%
\index{Small Preconditions}%
%% ********************************************************************
Moreover, it is the case that the transitions have \emph{small
  preconditions}, that is, the views don't need to be extended by much
in order for transitions to be enabled.
%
Depending on the transition, we can hence get away with a
configuration that is only a slightly extended version of some view
(here, extended with one extra \emph{witness} process, see
Figure~\ref{figure:view:small:preconditions}).

\begingroup%
\setlength\intextsep{\dazintextsep}
\begin{figure}[ht]
  \centering
  \tikzinput[\linewidth]{img/small-preconditions}
  \caption{Consider the configuration on the top and a~view~$v$
    (marked with a~black border). The transition
    {\protect$\quantrule{s}{t}{\exists}{>}{\set{w}}$} is disabled on
    the view as-is, but it is enabled in case the view is extended
    with the state \w[witness]{w} and gives then the view~$v'$.}
  \label{figure:view:small:preconditions}
\end{figure}
\endgroup

%% ********************************************************************
\KW{Requirements}%
\index{Verification Algorithm!Requirements}%
%% ********************************************************************
To implement an effective procedure, the following steps are required
(marked with a red wave in diagram~\ref{algo:view:diagram}):
%
\begin{strategy}
\item {\bf Computing the abstraction $\Absof{k}(\Inits)$ of initial
    configurations.} %
  This is not a difficult step, since $\Inits$ is usually a~(very
  simple) regular set, and $\Absof{k}(\Inits)$ is easily computed
  using a~straightforward automata construction.
  % 
  For instance, in the case of Szymanski's protocol from
  Section~\ref{section:paramsys:example}, all processes are initially
  in state~\w[i]{0}, hence $\Absof{k}(\Inits)$ contains only the words
  of size $l\leq k$ consisting of only the letter~$\w{0}$. %
%
\item {\bf Computing the abstract post-image.} %
  We circumvent the potential infinite set $\Concof{k}(V)$ and only
  consider the abstract post-image of the configurations from the
  finite set $\Concoflim{k}{k+1}(V)$.
  %
  This is the main reason that makes the method so efficient, together
  with the small precondition property from the transitions.
%
\item {\bf Evaluating the test $\Concof{k}(V)\cap\Bad = \emptyset$.} %
  Since $\Bad$ is usually the upward-closure of a~finite set
  $\minbad$, the test can be carried out by testing whether there is
  $b\in\minbad$ such that $\Absof{k}(b)\subseteq V$.
  %
  For instance, in the case of Szymanski's protocol from
  Section~\ref{section:paramsys:example} (for $k=2$), all the bad
  configurations follow the same patterns: %
  {\badpattern{9}{9}}, %
  {\badpattern{9}{10}}, %
  {\badpattern{10}{9}} or %
  {\badpattern{10}{10}}. %
  This means that they must contain the views \w[b]{9,9}, \w[b]{10,9},
  \w[b]{9,10} or \w[b]{10,10}. We check whether any latter view is
  included in the set of views from the fixpoint. %
%
\item {\bf Exact reachability analysis.}  Line~\ref{scheme:view:exact}
  requires the computation of $\Reach_k$. %
  Since $\Reach_k$ is finite, it can be computed with any (fast)
  procedure for exact exploration of finite-state systems.
\end{strategy}
%
% Since the problem is generally undecidable, existence of $k$ for which
% the test on line~\ref{scheme:view:test} succeeds for a~safe system
% cannot be guaranteed and the algorithm may not terminate. However, as
% discussed in Section~\ref{section:view:completeness}, such a guarantee
% can be given under the additional requirement of monotonicity of
% transition relation w.r.t. a well-quasi ordering.
% %
% The method terminates otherwise for all our examples discussed in
% Section~\ref{section:experiments}, many of which are \emph{not} well
% quasi-ordered.

\begin{wrapfigure}{r}{0.68\linewidth}
  \hfill%
  \begin{algorithm}[H]
    \DontPrintSemicolon%
    \For{$k:=1$ \bf{to} $\infty$\nllabel{implementation:view:outerloop}}{%
      \lIf{$\Absof{k}(\Reach_{k})\cap\minbad\neq\emptyset$\nllabel{implementation:view:exact}}{\Return Unsafe}%\;%
      $V := \mu X\,.\,\Absof{k}(\Reach_{k}) \cup \Absof{k}\circ\post\circ\Concoflim{k}{k+1}(X)$\nllabel{implementation:view:fixpoint}\;%
      \lIf{$V\cap\minbad=\emptyset$\nllabel{implementation:view:test}}{\Return Safe}%
    }
    \caption{Verification Procedure}
    \label{algo:view:procedure}
  \end{algorithm}
\end{wrapfigure}
%
%% ********************************************************************
\KW{Acceleration}%
\index{Verification Algorithm!Acceleration}%
%% ********************************************************************
The implementation presented on the right %in Algorithm~\ref{algo:view:procedure}
fulfills the above requirements. %
%
Notice a simple acceleration in that procedure, based on the
observation that we can seed the fixpoint computation (on
line~\ref{implementation:view:fixpoint}) with a larger set than
$\Absof{k}(\Inits)$, namely $\Absof{k}(\Reach_k)$.
%
The effect of this acceleration is that most experiments happen to be
(almost already) at fixpoint, using $\Absof{k}(\Reach_k)$ as initial
input.
%
This demonstrates the efficiency of the method and that most
behaviours are captured by small instances of the system.
%
The test performed on line~\ref{implementation:view:exact} is carried
out using the minimal elements \minbad\ of the set \Bad\ of bad
configurations, such that we still compute $\Reach_k$ but only check
if $\Absof{k}(\Reach_k)$ contains a bad ``view'' from {\minbad}.



%% ====================================================================
%\subsection*{Termination and Completness}
\subsubsection*{Termination and Completness}
% \label{section:view:completeness}
%% ====================================================================
We have shown so far that the method is in fact sound, i.e.\ when it
returns that the system is safe, it is indeed the case.
\index{Soundness}%
%
We show in Paper~\ref{paper:VMCAI13} and Paper~\ref{paper:STTT15} that
the method is also complete for a~large class of systems, namely those
that are monotonic with respect to a well-quasi ordering (see
Chapter~\ref{chapter:monotonic:abstraction}).
% equipped with a~well-quasi ordering and a~monotonic transition
% relation (i.e.\ WQO systems from
% Chapter~\ref{chapter:monotonic:abstraction}).
%
\index{Completeness}%
\index{Well-Quasi Ordering (WQO)}%
Completeness states that if the system is safe, the method will show
that it is. In our case, if the system is WQO and safe, there exists
$k$ such that the test on line~\ref{implementation:view:test}
succeeds. If the system is not safe, there is surely a~$k$ such that
$\Reach_k$ exhibits the error. Consequently, for WQO systems, the
method is guaranteed to terminate.
\index{Verification Algorithm!Termination}%

\begin{comment}
Graph with downward-closure of R (=all reachable confs).
In case of Szymanski, no version of downward-closed R will not intersect with Bad.
Must be expressed in other ways (it is not downward-closed and hence: contexts)
\end{comment}
