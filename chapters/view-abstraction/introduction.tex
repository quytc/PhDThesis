%% ------------
\KW{Parameterized systems}%
\index{Parameterized Systems}%
%% ------------
We recall the definitions introduced in
Chapter~\ref{chapter:parameterized:systems}: a~parameterized system
$\parsys=(\locs,\rules)$ represents the parallel composition of
several processes, whose local states are in \locs,
%, where \locs\ is a set of local states and \rules\ is a set of rules,
and gives rise to a transition system $(\confs,\trans)$, where \confs\ is
a set of configurations and \trans\ is the post-operator which
transforms a configuration into another.
%
The configurations have different sizes and are organized according to
some predefined topology, such as arrays, trees, rings and multisets.
%
%% ------------
\KW{Reachability}%
\index{Reachability}%
%% ------------
The reachability problem is to determine, for each size of
configurations, whether the set \Bad\ of bad configurations is
reachable from the set \Inits\ of initial configurations.
%
We use $\Reach$ to denote the set of all reachable configurations
from~$\Inits$ (and $\Reach_i$ for the set of configurations in \Reach\
of size~$i$). %, i.e.\ $\cup_{i\in\nat}\Reach_i = \Reach$.
%
The system is hence characterized by an infinite family of reachable
configurations and is proven safe when $\Reach\cap\Bad = \emptyset$.
%
Here, the parameter of the verification problem is the size of the
configurations.

%% ------------
\KW{Forward Analysis}%
\index{Reachability!Forward Analysis}%
%% ------------
We present, in this chapter, another technique to solve the
reachability problem.
%
The technique is a forward reachability analysis using yet another
abstraction.
%
It has been introduced originally to solve the problem of Shape
Analysis (presented in Chapter~\ref{chapter:shape:analysis}), but it
applies surprisingly well to the settings of parameterized systems.
%
%
%% ------------
\KW{Simplification}%
\index{Size-preserving transitions}%
%% ------------
To simplify the presentation, we focus, in this chapter only, on
parameterized systems with a linear topology. The set of states \locs\
is finite and the transitions from \rules\ are size-preserving.
%
Extensions and other topologies can be found in
Paper~\ref{paper:VMCAI13}.

%% ------------
\KW{Key insight}%
%% ------------
%
\begin{wrapfigure}{r}[0pt]{0.5\linewidth}
  \vspace{-\baselineskip}
  %\centering
  \hfill
  \tikzinput[\linewidth]{img/small-model-property}
  \caption{Repeated patterns.}
  \label{figure:view:abstraction:intuition}
\end{wrapfigure}
%
The key insight of the method is to take advantage of the fact that
the first instances of the system (i.e.\ for configurations of small
sizes) give enough information to derive the behaviour of the system
in general.
%
We can see the small configurations as \emph{patterns} that will only
be repeated and intertwined in configurations of larger sizes, as
illustrated in Figure~\ref{figure:view:abstraction:intuition}.
%
Moreover, bad patterns, if existing, are often detected already when
only a few processes are involved.
%
Consider, for example, the configurations with six processes, where
one process is passive in, say, its initial state. The remaining five
processes of such configurations often (but not necessarily) ``cover''
the configurations based only on five processes.
%


%% ------------
\KW{Main Idea}%
\index{Small Model Property}%
%% ------------
The main idea of our method is to exploit this %
\emph{small model property} and perform parameterized verification by
only exploring a small number of \emph{fixed} instances %of the system
to prove the system safe -- for all sizes of configurations. In
practice, it is often the case that we only need to compute the finite
sets $\Reach_1$,$\Reach_2$ and $\Reach_3$ to determine whether the
whole set \Reach\ contains a bad configuration.

\index{Cut-off}%
The method \emph{automatically} detects a \emph{\mbox{cut-off}} point
beyond which the verification procedure need not continue.
%
Intuitively, it means that the information already collected during
the exploration of the state-space until the \mbox{cut-off} point
allows us to conclude safely that no bad configuration will occur in
the larger instances.
%
The cut-off detection is performed on-the-fly during the verification
procedure itself (and illustrated in
Figure~\ref{figure:small:model:property}).

\begingroup%
\setlength\intextsep{\dazintextsep}
\begin{figure}[ht]
  \centering
  \tikzinput[0.9\linewidth]{img/cut-off}
  \caption{Small Model Property: The method detects a cut-off point
    beyond which the verification procedure need not continue.}
  \label{figure:small:model:property}
\end{figure}
\endgroup

\index{Over-approximation}%
The configurations from the first instances of the system are
abstracted but retain enough information to ``reconstruct'' the sets
of reachable configurations of larger sizes, as we create larger
configurations by combining small patterns. In fact, the collected
patterns allow to characterize an \emph{over-approximation} of the
reachable configurations. It is possible that a recombination of
patterns creates a configuration that the original system would not
have computed.
%
% The key is to find a balance between the information to be potentially
% discarded and the information to be necessarily retained in order to
% avoid a too coarse over-approximation.
The key is to abstract the small configurations while retaining enough
information in order to not over-approximate the set of reachable
configurations too much.
%
Nevertheless, inspecting only small (and finite) instances of the
system allows for an efficient method.
%
As usual, if the over-approximation does not contain any bad
configuration, we can safely conclude that the system is safe.
% as a whole.

% ------------------------
The chapter is structured as follows. 
%
We first introduce the abstraction at the heart of the method,
focusing solely on \emph{atomically} checked global conditions.
% in Section~\ref{section:view:abstraction}.
%
We then show the method, its soundness, and discuss its
completeness.
% in Section~\ref{section:view:method}.
%
Finally, we %further
present how the method can be extended to cope with non-atomic global
conditions and how to handle Szymanski's mutual exclusion protocol
(i.e.\ the challenging problem of
Section~\ref{section:paramsys:example}).
