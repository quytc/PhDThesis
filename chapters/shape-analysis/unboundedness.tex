%% ====================================================================
%\section[Abstracting Three Degrees of Unboundedness]%
\section[Bounded Abstract Domain | No Garbage Collection]%
%{Handling 3 Degrees of Unboundedness {\footnotesize and No Garbage Collection}}
{Three Degrees of Unboundedness {\footnotesize and No Garbage Collection}}
\label{section:unboundedness}
\label{section:nogc}
%% ====================================================================

We can now specify if a property gets violated using the finite
observers on an abstract stack using data independence.
%
Thus, we can handle
Challenge~\ref{unbounded:data:domain:specification} and
~\ref{unbounded:data:domain}.
%
However, the memory layout of heap cells (called \emph{shapes}) that
the program can create and manipulate is not bounded in size.
%
To tackle Challenge~\ref{unbounded:heap:size}, it is necessary to
handle the size of those shapes.

Each thread has a finite set of variables and can access the heap
cells that are pointed to by one of its variables or a global
variable. The heap cell that are not pointed by any variable can be
potentially accessed through a succession of \prgcode{next}
pointers. Otherwise, they are considered garbage.
%
The idea is to only keep track of \emph{important} cells. Those are
the ones pointed by some variable and the ones containing some
\emph{important} data, i.e.\ \prgcode{red} and \prgcode{blue} in the
case of Treiber's stack.
%
Intuitively, the cells not pointed by any variable play a~secondary
role. There are merely ``relays'' between the cells that the threads
can currently access. We do not need to keep how many there are, we
only need to keep the memory layout they form.

\index{Transitive Closure}%
We thus adapt a variant of the transitive closure logic by Bingham and
Rakamari\'c~\cite{BiRa:vmcai06} for reasoning about heaps with single
selectors, to our framework. This formalism, called here \emph{shape
  analysis}, %
\index{Shape Analysis}\index{Heap Analysis}%
tracks reachability properties \index{Reachability}%
between pairs of pointer variables.
%
Moreover, we have developed a novel optimization, based on the
observation that it suffices to track the possible relations between
each pair of pointer variables \emph{separately}.
% , analogously to the
% use of DBMs used in reasoning about timed automata~\cite{Dill:DBM}.

% , and we adapt it to our analysis, in which pairs of
% threads are correlated.
%


%% -------------------------------------------
We use shape analysis to cope with challenge~\ref{unbounded:heap:size}
but it is still not enough to bound the shapes, since we have an
unbounded number of threads. Each thread could have one~variable
pointing to a~different cell and that creates an unbounded shape.
%
\index{Thread-Modular}%
To cope with challenge~\ref{unbounded:threads}, we try to adapt the
successful thread-modular approach~\cite{BLMRS:cav08}, which verifies
a concurrent program by generating an invariant\index{Invariant} %
that correlates the global state with the local state of an arbitrary
thread.
%
In other words, it keeps track of the shape that \emph{one} thread can
see, %realize
abstracting away (and ignoring) all the other threads.
%
The resulting shape is necessarily of bounded size using the above
transitive closure.
%% -------------------------------------------
We can draw a~parallel with the view abstraction method from
Chapter~\ref{chapter:view:abstraction}, where the configurations are
abstracted into views of size 1 ($k=1$).
%
\index{Interference}%
The thread-modular approach includes a step where it takes the
information about one thread and couples it with the information from
another thread, in order to take into account %capture
the interference %operations
of all the other threads on the first thread.
%
This step is akin to the concretization from the view abstraction,
where we extend the views which additional information from other
views, in a consistent manner.

%% -------------------------------------------
However, this leaves us with a problem when $k=1$: the interfering
step would create shapes where threads possibly share some particular
cell, even though the concrete system would never produce those
shapes.
%
This is a too coarse approximation that leads to erroneous behaviors.
%
\index{Garbage-Collection}%
In fact, in the case of garbage collection, even though the
thread-modular approach only keeps track of one thread, it is possible
to get around the problem by including additional information on the
shape cells, to reflect whether a cell is shared or not. Thanks to the
guarantee that garbage collection provides, when a cell is created,
the creating thread is the only one accessing that cell. The cell is
fresh and we mark it as such. This flag is lost as soon as the cell
becomes accessible by another thread or a global variable.
\index{Fresh cell}%
%
This information allows the interference step to be more precise, and
separate the situations where it would otherwise merge together two
fresh cells from different threads, making the resulting cell
erroneously shared. The thread-modular approach, along with the
freshness information, is often enough to carry on the verification.

%% -------------------------------------------
\index{Non Garbage-Collection}%
Nevertheless, it is not sufficient to cope with
challenge~\ref{no:garbage:collection} and the ABA problem.
%
The generated invariant must be able to express that \emph{at most}
one~thread accesses some cell on the global heap.
%
Since this cannot be expressed in general by the thread-modular
approach, we need to extend it %the approach
to generate invariants\index{Invariant} %
that correlate the global state with the local states of an arbitrary
\emph{pair} of threads.
%
\index{Thread Corrolation}%
\index{Precision of View Abstraction}%
By correlating two threads, we increase the precision of the
abstraction, alleviating the above problem. This is similar to the
view abstraction when $k=2$ and the concretization function extends
views in a consistent manner.

\begingroup%
\setlength\intextsep{\dazintextsep}
\begin{figure}[ht]
  %\centering
  \tikzinput[\linewidth]{img/shape-abstraction}
  \caption{From a concrete shape into one of the shape abstractions}
  \label{figure:shape:abstraction}
\end{figure}
\endgroup

In Figure~\ref{figure:shape:abstraction}, we show how a concrete shape
with three threads can be abstracted into a bounded abstract
shape. The figure depicts only one of the shape abstraction that we
obtain. There is a combinatorial factor in the choice of which two
threads to pick out of, here, the three threads.
%
Indeed, the above abstraction techniques bring the needed precision
for verification, at the price of significant state-space explosion,
which mainly arises from the fact that, reusing the terminology from
Chapter~\ref{chapter:view:abstraction}, the dynamically allocated
heaps are scattered across several views which describe the
correlation between pairs of threads.
%
We have therefore developed a novel optimization, where we
\emph{merge} abstracted shapes and manipulate the compound instead of
every individual abstracted shapes.
\index{Symbolic Representation}%
%
Intuitively, merging does not bring a~penalizing %dreadful
imprecision due to the structure of the retry-loops.
\index{Retry-Loop}%
%
When the thread's local variables are initially placed, but the global
shape changed through the actions of other threads, the current thread
will detect it and restart its initialization.
%
Hence, recording precisely where those local variables are is often
superfluous information that can be discarded.
%
We touch more on that optimization in the next section.
