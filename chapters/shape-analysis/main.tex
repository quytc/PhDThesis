 %%% ====================================================================
%\chapterwithtoc{Shape Analysis}
%\label{chapter:shape:analysis}
%%% ====================================================================
%
%In this chapter, we consider a difficult challenge \index{Challenge}
%in software verification, namely to automate its application to
%algorithms with an unbounded number of threads that concurrently
%access a~dynamically allocated shared heap.
%\index{Heap Analysis}%
%%
%Such algorithms are notoriously difficult to get correct and verify,
%since they often employ fine-grained synchronization and avoid locking
%wherever possible.
%%
%This chapter presents an efficient technique to verify that
%a~concurrent implementation of a~common data type abstraction, such as
%a~queue or a~stack, conforms to a~simple abstract specification of its
%(sequential) functionality.
%%
%There are several combined challenges to be addressed.
%%
%\begin{challenges}
%\item \label{unbounded:data:domain:specification}%
%  \index{Challenge}%
%  \index{Infinite-state}%
%  The abstract specification is infinite-state, because the
%  implemented data structure may contain an unbounded number of data
%  values from an infinite domain.
%\item The program is infinite-state in several dimensions:
%  \index{Challenge}%
%  \index{Unbounded}%
%  \begin{subchallenges}
%  \item \label{unbounded:threads}%
%    it consists of an unbounded number of concurrent threads,
%  \item \label{unbounded:heap:size}%
%    it uses unbounded dynamically allocated memory, and
%  \item \label{unbounded:data:domain}%
%    the domain of data values is unbounded.
%  \end{subchallenges}
%\item \label{no:garbage:collection}%
%  \index{Challenge}%
%  \index{Non Garbage-Collection}%
%  The program does not rely on automatic garbage collection.\\%
%  It manages memory explicitly.
%\end{challenges}
%
%%% -------------------------------------------
%We present, in the first section of this chapter, the type of programs
%we consider. In the following sections, we introduce in a stepwise
%manner how we cope with each of the above challenges. In the last
%section, we combine the different techniques and present the
%verification method.
%%
%%All in a succint manner.
%%
%%Details can be found in Paper~\ref{paper:TACAS13}.

%% ====================================================================

\chapter{Concurrent Data Structures}
\label{section:shape:programs}
\index{Program Model}%
%% ====================================================================
\index{Concurrency}%

%\begin{figure}
%\center
%\begin{tikzpicture}[]
%\node[rounded corners,draw = cyan,name=cell8,minimum width=24pt, minimum height=20pt,anchor=south]{};
%\node[minimum width=8pt, minimum height=10pt,anchor=north west,font=\tiny,inner sep=0pt] at (cell8.north west){};
%\node[ellipse callout, fill=yellow!20,anchor=east,inner sep=2pt, 
%callout absolute pointer={($(cell8.north west)+(4pt,-5pt)$)},draw,,font=\tiny,align=center]  
%at ($(cell8.north west)+(-4pt,-2pt)$){abstract\\ value};
%%
%\node[minimum width=8pt, minimum height=10pt,anchor=south west,font=\tiny,inner sep=0pt] at (cell8.south west){};
%\node[ellipse callout, fill=yellow!20,anchor=east,inner sep=2pt, 
%callout absolute pointer={($(cell8.south west)+(4pt,5pt)$)},draw,,font=\tiny,align=center]  
%at ($(cell8.south west)+(-4pt,2pt)$){concrete\\ value};
%%
%%
%\node[draw = blue!50,minimum width=8pt, minimum height=10pt,anchor=north,font=\tiny,inner sep=0pt]at (cell8.north){};
%\node[ellipse callout, fill=yellow!20,inner sep=2pt, 
%callout absolute pointer={($(cell8.north)+(0pt,-5pt)$)},draw,,font=\tiny,align=center]  
%at ($(cell8.north)+(0pt,7pt)$){{\tt mark}};
%%
%\node[draw = cyan,minimum width=8pt, minimum height=10pt,anchor=south,font=\tiny,inner sep=0pt]at (cell8.south){};
%\node[ellipse callout, fill=yellow!20,inner sep=2pt, 
%callout absolute pointer={($(cell8.south)+(0pt,5pt)$)},draw,,font=\tiny,align=center]  
%at ($(cell8.south)+(0pt,-7pt)$){{\tt lock}};
%%
%\node[ellipse callout, fill=yellow!20,inner sep=2pt, 
%callout absolute pointer={($(cell8.east)+(-4pt,0pt)$)},draw,,font=\tiny,align=center]  
%at ($(cell8.east)+(13pt,0pt)$){{\tt next}};
%%
%%
%\node[rounded corners,draw = cyan,name=cell1,minimum width=24pt, minimum height=20pt,anchor=west] at ($(cell8.east)+(-50pt,-40pt)$){};
%\node[font=\tiny,inner sep=0pt,scale=0.8,anchor=west] at ($(cell1.west)+(0.5pt,1pt)$){{--$\infty$}};
%\node[draw = blue!50,minimum width=8pt, minimum height=10pt,anchor=north,font=\tiny,inner sep=0pt]at (cell1.north){};
%\node[draw = cyan,minimum width=8pt, minimum height=10pt,anchor=south,font=\tiny,inner sep=0pt]at (cell1.south){\cross};
%\node[name=succ1,circle,fill,minimum size=3pt,inner sep=0pt,outer sep=0pt] at ($(cell1.east)+(-4pt,0pt)$) {};
%\node[anchor=north,font=\tiny,align=center] at ($(cell1.south)+(0pt,2pt)$) {{\tt head}};
%%
%\node[rounded corners,draw = cyan,name=cell6,minimum width=24pt, minimum height=20pt,anchor=west,name=cell6]
%at ($(cell1.east)+(10pt,0pt)$){};
%\node[minimum width=8pt, minimum height=10pt,anchor=west,font=\tiny,inner sep=0pt,scale=0.8,name=d6a] at ($(cell6.west)+(1pt,1pt)$){$\infty$};
%\node[draw = blue!50,minimum width=8pt, minimum height=10pt,anchor=north,font=\tiny,inner sep=0pt]at (cell6.north){\cross};
%\node[draw = cyan,minimum width=8pt, minimum height=10pt,anchor=south,font=\tiny,inner sep=0pt]at (cell6.south){\cross};
%\draw ($(cell6.north east)+(-1pt,-2pt)$) -- ($(cell6.south east)+(-8pt,0pt)$);
%\draw ($(cell6.north east)+(-8pt,0pt)$) -- ($(cell6.south east)+(-1pt,2pt)$);
%\draw[->] (succ1) -- (cell6);
%\node[anchor=south,font=\tiny,align=center] at ($(cell6.north)+(0pt,-2pt)$) {{\tt tail}};
%\end{tikzpicture}
%\caption{A cell and the initial heap in the {\tt Lazy Set} Algorithm.}
%\label{lazy:list:heap:cell:fig}
%\end{figure}

A concurrent data structure is a way of storing and organizing data for access and manipulated by multiple computing threads (or processes) on a single computer. The implementation of a concurrent data structure usually requires writing a set of operations that access and manipulate instances of that structure concurrently. Processes are sequential, each process applies a sequence of its operations to a share structures. A process can halt or have various speed. Actually, we do not tell whether a process is halted or is running fast or slowly.
Each data structure has a type which defines a set of possible values and a set of operations such as queue, stack or set. Each object has a sequential
specification that defines how the object behaves when its operations are
invoked one at a time by a single process. For example, the behavior of a
queue object can be specified by requiring that enqueue insert an item in the
queue, and that dequeue remove the oldest item present in the queue. In a
concurrent system, however, an object’s operations can be invoked by concurrent
processes, and it is necessary to give a meaning to interleaved operation
executions.

An object is linearizable if each operation appears
to take effect instantaneously at some point between the operation’s invocation
and response. Linearizability implies that processes appear to be interleaved
at the granularity of complete operations, and that the order of
nonoverlapping operations is preserved. The notion of linearizability generalizes and uniles a number of ad hoc correctness conditions in the literature, and it is related to (but not identical with) correctness criteria such as sequential consistency and strict serializability.
 
 There are four main techniques to construct concurrent data structures namely coarse-grained locking, fine-grained locking lazy, synchronization and lock-free programming. In the coarse-grained locking technique, a single lock is used to synchronize every access to an object. Coarse-grained synchronization is easy to reason about, however it works well when levels of concurrency are low, but if too many threads try to access the object at the same time, then the object becomes a sequential bottleneck, forcing threads to wait in line for access. In the second technique, they split the object into independently synchronized components, ensuring that method calls interfere only when trying to access the same component at the same time. In the third technique, the task of removing a component from a data structure can be split into two phases: the component is logically removed simply by setting a tag bit, and later, the component can be physically removed by unlinking it from the rest of the data structure. The lock-free technique help us to eliminate locks entirely, it relies on built-in atomic operations such as {\tt compareAndSet()} for synchronization.
 
 Each of these techniques can be applied (with appropriate customization) to a
variety of common data structures (queues, stacks, sets) implemented by different linked data structures such as singly linked lists, skiplists, trees, or lists of lists. 
%We consider systems consisting of an arbitrary number of concurrently executing threads. Each thread may at any time invoke one of a finite set of methods. Each method corresponding to one operation on the data structure.
%Each method declares local variables and a method body.
%We assume that the local variables include the
%program counter $\tt pc$ and also potentially
%include an input parameter of the method.
%%
%The body is built in the standard way
%from atomic commands using standard control
%flow constructs (sequential composition, selection, and loop constructs).
%%
%Each run of the program consists of an arbitrary (but finite) number of
%concurrently executing threads. 
%%
%Each thread invokes one of the methods.
%%
%Thread execution is terminated by executing a {\tt return} command,
%which may return a value.
%%
%The shared variables can be
%accessed by all threads, whereas local variables can be accessed only
%by the thread which is invoking the corresponding method.
%%
%We assume that the global variables and the heap are initialized by
%an initialization thread, which is executed once at the beginning
%of program execution.
%%
%Furthermore, we assume that the local variables  have 
%arbitrary initial values.
%%
%In this thesis, we assume that variables are either pointer variables
%(to heap cells) or data variables.
%%
%The data variables assume values 
%from an unbounded or infinite (ordered) domain,
%or from some finite set $\mathbb F$.
%%
%We assume w.l.o.g. that the infinite set is given by the set $\mathbb Z$ of integers.
%
%%
%%As usual, we will also use arbitrary finite domains that we built from $\boolset$.
%%
%A parameter of a method may be instantiated by any value in $\mathbb Z$.
%%
%Heap cells can have a number of data fields that contain data either from
%$\mathbb Z$ or $\mathbb F$.
%%
%A cell has only one pointer field, denoted {\tt next}.
%%
%Atomic commands include assignments between data variables, 
%pointer variables, or fields of cells pointed to by a pointer variable.
%%
%The command {\tt new Node()} allocates a new structure of type
%{\tt Node} on the heap, and returns a reference to it.
%%
%The compare-and-swap command {\tt CAS(\&a,b,c)} atomically
%compares the values of {\tt a} and {\tt b}.
%If  equal, it assigns the value of
%{\tt c} to {\tt a}  and returns {\tt true}, 
%otherwise, it leaves {\tt a} unchanged and returns {\tt false}. 
%%% 
%We assume that each statement in  a method
%has a unique label.
%
%
%
%%\endgroup
%
%
%\input heap-lazy-list
\vspace{1cm}
\input img/lazy-list



%
%
%Each method declares local variables (including the input parameters
%of the method) and a method body.
%%
%In this chapter, we assume that variables are either pointer variables
%(to heap cells), or data variables (assuming values from an unbounded
%or infinite domain%, which will be denoted
%~$\dset$).
%%
%The body is built in the standard way from atomic commands using
%standard control flow constructs (sequential composition, branching,
%and loop constructs).
%%
%Method execution is terminated by executing a \prgcode{return}
%command, which may return a value.
%%
%The global variables can be accessed by all threads, whereas local
%variables can be accessed only by the thread which is invoking the
%corresponding method.
%%
%We assume that the global variables and the heap are initialized by an
%initialization method, which is executed once at the beginning of
%program execution.
%
%Programs manipulate heap cells of type \prgcode{node}, each consisting
%of a~\prgcode{val} field and a~\prgcode{next} field, which carry
%respectively a data value and the address to another heap cell.
%%
%Atomic commands include assignments between data variables, pointer
%variables, or fields of cells pointed to by a~pointer variable.
%%
%The command \prgcode{new node()} allocates a new structure of
%type \prgcode{node} on the heap, and returns a reference to it.
%%
%\index{Compare-and-Swap (CAS)}%
%The compare-and-swap command \prgcode{CAS(a,b,c)} compares the memory
%locations \prgcode{a} and \prgcode{b}. If equal, it also atomically
%assigns the value~\prgcode{c} to \prgcode{a} and returns
%\prgcode{TRUE}. Otherwise, it leaves \prgcode{a} unchanged and returns
%\prgcode{FALSE}.
%%
%Note that \prgcode{a}, \prgcode{b} and \prgcode{c} can be pointers or
%variables using the referencing and dereferencing C constructs
%\prgcode{\&}~and~\prgcode{*}.

%\endgroup

As an example, Fig.~\ref{figure:lazy-list} depicts a program
{\tt Lazy Set} \cite{Lazyset}
that implements a concurrent set containing integer
elements.
%
The set is implemented as an ordered singly linked list.
%
The program contains three methods, namely {\tt add}, {\tt rmv},
and {\tt ctn},  corresponding to operations
that respectively add, remove, and check the existence
of an element in the set.
%
%
Each method takes an argument which is the value
of the element, and returns a value which indicates whether
the operation has been successful or not.
%
For instance, the operation {\tt add}(e) returns the value
{\it true} if  $e$ is not already a member of the set.
%
In such a case a new cell with data value $e$ is added to its
appropriate position in the list.
%
If $e$ is already present, then the list is not changed and 
the value {\tt false} is returned.
% 
The program also contains the subroutine {\tt locate}
that is called by the three methods.
%
A cell in the list has  three fields ${\tt mark}$, ${\tt lock}$, and
${\tt val}$.
%
The {\tt rmv} method first logically removes the node
from the list by setting the {\tt mark} field, before 
physically removing the node.
%
The {\tt ctn} method is wait-free and traverses the list ignoring the locks
inside the cells. The algorithm uses two global pointers, {\tt head} that points to  the first cell of the heap, and {\tt tail} that points to the last cell.  
These two cells contain two values that are smaller 
and larger respectively than all keys that may be                     
inserted in the set.

%\begin{figure}[]
%  \begin{tikzpicture}
%  %[
%   % property/.append style={left,rotate=-10,scale=0.7,shift={(-0.5,-0.7)}},
%   % ]
%    \node[codeblock] (struct) {\begingroup\scriptsize\VerbatimInput[numbers=none]{experiments/code/treiber/struct-gc}\endgroup};
%   % \node[codeblock] (init) at (struct.north east) {\begingroup\scriptsize\VerbatimInput[numbers=none]{experiments/code/treiber/init-gc}\endgroup};
%    
%
%    
%    \node[codeblock,below right] (push) at (struct.south west) {\begingroup\scriptsize\VerbatimInput{experiments/code/treiber/push-gc}\endgroup};%locate
%    
%        \node[codeblock,below right] (pop) at (struct.south east) {\begingroup\scriptsize\VerbatimInput[firstnumber=10]{experiments/code/treiber/pop-gc}\endgroup};
%        
%            \node[codeblock, below] (cnt) at (push.south) {\begingroup\scriptsize\VerbatimInput[firstnumber=10]{experiments/code/treiber/cnt}\endgroup};
%    
%                \node[codeblock, below] (rmv) at (pop.south) {\begingroup\scriptsize\VerbatimInput[firstnumber=10]{experiments/code/treiber/rmv}\endgroup};
% %   \node[property] at (init.north east) {\sc Init};
%  %  \node[property] at (push.north east) {\sc Push};
%  %  \node[property] at (pop.north east) {\sc Pop};
%    
%  \end{tikzpicture}
%  \caption{Lazy set.}
%  \label{figure:lazy-list}
%\end{figure}

%\begingroup%
%\setlength\intextsep{\dazintextsep}
%\begin{figure}[ht]
%  \centering
%  \tikzinput{img/shape-heap}
%  \caption{Memory layout of a Treiber stack (with $\dset=\nat$),
%    showing only two threads.}% out of all
%  \label{figure:shape:heap}
%\end{figure}
%\endgroup

%% ====================================================================
\section*{Linearizability}      
\label{section:specification:concurrent:data:structure}     
\index{Specification}%         
%% ====================================================================
\begingroup%     
In a concurrent program, the methods of the different executing threads can overlap in time. Thus, a {\tt rmv} method that
executes in parallel with an {\tt add} method for the same key may or may not
find the element in the set, depending on how the individual method statements
overlap in time. For a user of the data structure, it is important to know precisely
what can happen when several methods access a data structure concurrently
without inspecting the code of each method. Such a user would want to have
a criterion for how operations take effect, which considers only the points in
time of method calls and returns. The most widely accepted such condition is
linearizability.  
%The program statements 
%in each method are totally ordered. Whereas, statements from different methods in different executing threads might form a partial order. This partial order raise the difficult of reasoning about program execution. One of the main correctness criterion of a concurrent program is linearizability, 
Linearizability defines consistency for the history
of call and response events generated by an execution of the program at hand \cite{HeWi:linearizability}. Intuitively, linearizability requires every method to take effect
at some point ({\emph {linearization point}}) between it's call and return events. A linearization point is intuitively a moment where the effect of the method
becomes visible to other threads. An execution of a (concurrent) system is modeled by a (concurrent) history, which is a finite
sequence of method invocation and response events. A (concurrent) history is linearizable if and only if there is some order for
the effects of the actions that corresponds to a valid sequential history. The valid sequence history can be generated by an execution of the sequential specification object. A concurrent object is linearizable iff each of its histories
is linearizable.

\setlength\intextsep{\dazintextsep}
\begin{figure}[ht]
  \centering
  \tikzinput[\linewidth]{img/linearizability} 
  \caption{Linearizability, where the commit points are marked with \protect\commitpoint{}.}
  \label{figure:shape:linearizability}  
\end{figure}          
  

\index{Linearizability}%
Figure~\ref{figure:shape:linearizability} provides a examples of trace of methods of concurrent program implementing sets.
In the trace, each method takes effect
instantaneously at its (called the \emph{linearization point})
between call and return events~\cite{HeWi:linearizability}. When we order methods according to its linearization point, we get a total ordered sequence that respect the sequential specification of the set.
%A linearization point normally stays inside the code of the
%method.  However, in some cases, it it is located in
%the code of another method depending on the execution path.



\endgroup
%%% ====================================================================
\section{Data Independence}
\label{section:data:independence}
%% ====================================================================

We can notice that data structures such as stacks and queues are
merely containers and do not look at data: their execution does not
branch in some particular part of the code depending on the data
values. All data values are treated equally. %
%
% For instance, if $\dset=\nat$ and if the system accepts a~trace that,
% say, inputs the values 1,2,3 and outputs 3,2,1 (i.e.\ a stack), we can
% easily see that there is an equivalent trace that inputs the data
% values 4,5,6 and outputs 6,5,4 since the system can execute in the
% same way regardless of the data values that it manipulates.
For instance, if $\dset=\nat$ and if the system can input the values
1,2,3 and output 3,2,1 (i.e.\ a stack), we can easily see that there
is an equivalent behaviour where the system inputs the data values
6,5,4 and outputs 4,5,6. The system can execute in the same way
regardless of the data values that it manipulates.
%
Intuitively, data values do not matter, so we might as well rename
them. In that previous simple example, we could rename 6 into 1, 5
into 2, and 4 into 3. We would then have two equivalent traces.

\index{Traces}%
Even though we can rename the traces adequately, % judiciously
the set of all traces includes traces where input events can be
repeated using the same data value, and is therefore still potentially
infinite.
%
However, we can observe that it is possible for any trace to ``count''
the repeated input values and enumerate them while renaming them,
effectively creating a trace where the data values of input events are
all distinct. In such a case, we call the latter
a~\emph{differenciated trace}.
%
For example, if the system takes 1,2,3,1,2 as input, we could rename
the second 1 into 4 and the second 2 into 5. This creates an
equivalent trace where all the data values are distinct.

\begingroup%
\setlength\intextsep{1.25\baselineskip plus 3pt minus 2 pt}
\begin{figure}[ht]
  %\centering
  \tikzinput[\linewidth]{img/data-independence}
  \caption{Stack do not look at data}
  \label{figure:data:independence}
\end{figure}
\endgroup

\index{Data Independence}%
We can now introduce the desired definition: we say that a~set of
traces (or the program it characterizes) is \emph{data independent} if
and only if (i) it is closed under renaming and (ii) it can be
generated by a set of differentiated traces.
%
Detecting data-independence is easily done by syntactically checking
that data values are not compared and that uninitialized variables are
not used in the source code.
% These are straightforward extensions of [28].

If a data independent system accepts a bad trace, then it will also
accept a bad differentiated trace (and vice-versa).
%
In consequence, for a data independent program, to determine whether
the safety property is satisfied, it is sufficient to only consider
differentiated traces, i.e.\ executions in which any data value is
inserted at most once.
%
Furthermore, by adapting an argument from
Wolper~\cite{Wolper:dataindependence}, we can abstract away the data
values, by picking any two of them and flattening all the others
values to a third one.
%
In Figure~\ref{figure:data:independence}, the top part represents a
system which inputs a sequence of \emph{distinct} data values and
outputs them in reverse order, i.e\ a~stack.
%
In the bottom part, on the other hand, we abstract the data away and
choose \prgcode{blue} and \prgcode{red} to be the ``important'' values
while another ``neutral'' value \prgcode{white} is not.
%
If the bottom system is safe, so is the top one, and so is the system
in general (c.f.\ Paper~\ref{paper:TACAS13} for further details).
%
% The specification has been reduced to traces involving three abstract
% data values, where two of them are not repeated.
%

%% ====================================================================
\chapter{Verification of Linearizabilities}
\bjcom{A better title is ``Specifying Linearizability''}
In the previous sections, we describe data structures and linearization properties \bjfix{linearization properties of concurrent data structures}{the correctness criterion of linearizability for concurrent data structures} of concurrent data structures. In this section, we describe our approach to verify these linearization properties \bjcom{strange term}.
\bjcom{Here, you should motivate that the goal is to specify linearizability in a way that is suitable for automated verification}
Firstly, we explain the definition of \emph{observer} which is used to specify sequential specification of data structures. Thereafter, we describe how to verify linearization properties \bjcom{strange term} by using \bjcom{You should rather speak about ``linearization policies''}
our \emph{controller} which is used to specify how a data structure is linearizable.

\bjcom{Here, you can give a paragraph with overview of the chapter. You can say that you separate the problem of specifying linearizability into several ones.
  1. To specify the sequential semantics of a data structure in a way that is suitable for automated verification.
  2. To specify placement of linearization points in executions of a concurrent data structure.
  For the first, you use the techniques of (* TACAS 13 *) of observers. For the second, you present a new technique, in which methods are equipped with
  controllers. Controllers specify so-called ``linearization policies'', which prescribe how LPs are placed in executions.
  You can also add a section, where you survey the technique of observers that synchronize at call and return events
  }

\section*{Observers}
\label{section:observers}
\index{Observers}
%% ====================================================================
\bjcom{This introduction does not work. It is not an introduction to observers, but rather says how they are used in linearization policies. Some of the
material can be moved to the survey section before the ``Observers'' section (see above)}

In order to derive the totally ordered execution from a concurrent
execution, each method is instrumented to generate a so-called
abstract event whenever a linearization point is passed. Right after the program pass the linearization point, the abstract event is
communicated to an external \emph{observer}, which records the sequences of
abstract events from the code execution. In the next paragraph, we introduce the notion of observer, which essentially separates good traces of events from bad ones. Several observers shall be used to specify the safety property

\bjcom{It would be nice to have an extra introductory paragraph. It can say something like:
  \begin{inparaenum}[1.]
  \item Data structures are, by nature, infinite-state objects, since they are intended to carry an unbounded number of data elements.
  \item For automated verification, it is desirable with specifications that are constructed without explicitly mentioning such infinite objects.
  \item This problem is addressed by observers [2]. Observers specify allowed sequences of operations by constraining their projection on a small
    number of data elements. For instance, a stack data structure is specified by (* continue here *)
  \end{inparaenum}
  After this, you can continue as below.
}
    

We specify the sequencial \bjcom{please spell check!!!!} semantics of data structures by observers, as introduced
in \cite{AHHR:integrated:rep}. Observers are finite automata extended with a finite set of observer registers \bjcom{put in italics} that
assume values in $\mathbb Z$. \bjcom{You must introduce this domain earlier} At initialization, the registers are nondeterministically assigned arbitrary
values, which never change during a run of the observer. 
\vspace{1cm}
\begin{figure}[h]
%\begin{wrapfigure}{r}{0.5\textwidth} 
  \centering
  \tikzinput{img/set-observer} 
  \vspace{0.3cm}
  \caption{A stack observer}
  \label{figure:shape:set:observers}
%\end{wrapfigure}
\end{figure} \vspace{1cm} Transitions are labeled
by linearization events \bjcom{I suggest you say ``operations''. Later you will say that linearization events are operations}
that may be parameterized on registers. Observers are used as
acceptors of sequences of linearization events. The observer processes such sequences
one event at a time. If there is a transition, whose label, after replacing registers by their values, matches the event, such a transition is performed. If there is no such transition,
the observer remains in its current state. The observer accepts a sequence if it can be
processed in such a way that an accepting state is reached. We use observers to give exact specifications of the behaviors of data structures such as sets, queues, and stacks. The observer is defined in such a way that it accepts
precisely those sequences of abstract events that are not allowed by the semantics of
the data structure. This is best illustrated by an example. Fig. 3 depicts an observer that
accepts the set of method invocations that are not allowed by behavior of a set 
\bjcom{Continue and explain in 5-10 lines how it works}

\section*{Linearization Policies}
\label{controllers:subsection}
In order to prove linearizability, the most intuitive approach is to find a linearization point (LP) in the code of the implementation, and show that it is the single point where the effect of the operation takes place \cite{AHHR:integrated:rep,BLMRS:cav08,Vafeiadis}.
\bjcom{Insert explanation of how this works for add and rm methods of the lazy set}
However, for a large class of linearizable implementations,
the LPs are not fixed in the code of their methods, but depend on actions
of other threads in each particular execution. This happens, e.g., for algorithms that
employ various forms of helping mechanisms, in which the execution of a particular
statement in one thread defines the LP for one or several other threads.
\bjcom{A problem here is that the lazy set does not use ``helping''. PLease adapt the text to avoid such a misunderstanding}
For example, in the the {\tt Lazy Set} algorithm, the linearization point of unsuccessful {\tt ctn} method is not fixed in the code of the method.
\bjcom{The previous sentence is sloppy. You probably mean that ``it is not possible to assign a fixed LP in the code of the ctn method''. Your current text
  may give the impression that there is an LP, but it is not fixed: it is unclear what this means}
It stays in the code of the {\tt add} method.
\bjcom{I suspect that the previous statement is incorrect. As I understand it, there is no fixed LP for ctn in the add method, since the placement depends on
  what happens in the execution}
\bjcom{Here, you must insert a detailed explanation of why this is the case, and how linearization points can be assigned in each execution. This will easily be
10-20 lines. After that, end the paragraph}

 There have been several previous works dealing with the problems of non-fixed linearization points \cite{Poling,Colvin:Lazy-List,CGLM:cav06,SWD:cav12,Derrick:fm14,SDW:tcl14,Vafeiadis:cav10,Vafeiadis:Aspect}. However, they are either manual approaches without tool implementation or not strong \bjfix{strong}{general} enough to cover various types of concurrent programs. \bjcom{You must comment in more detail about these works in a ``related works'' section of the intro}
 In this thesis we handle non-fixed linearization points by providing semantic for specifying linearization policies.providing semantic for specifying linearization policies.
\bjfix{providing semantic for specifying linearization policies}
{a mechanism for assigning LPs to executions, which we call \emph{linearization policies}}
The linearization point of a thread may be defined in two ways: \bjcom{Again, do not say that there are EXACTLY two ways}
(i) The thread may define its own linearization point,
and in that case may also help other threads define their 
own linearization points.
(ii) The thread may be helped by other threads.
\bjcom{The points (i) and (ii) are almost impossible to understand, since you do not explain anything about what happens}
\bjcom{I suggest you replace the preceding (i) and (ii) by giving a couple of EXAMPLES of how methods may linearize. You already have explained on
  in the preceding paragraph (for lazy set). You can informally present a helping mechanism, which you can borrow from some other algorithm that
  you verify. In all cases, you must explain concretely what happens in the concurrent algorithm, not just be abstract}
The helping mechanism may contain complicated patterns.
%
For instance, the helping thread may broadcast a message to the other threads
(e.g., the {\tt Lazy Set} algorithm).
\bjcom{I do not see any broadcasting in the code...}
%
Both the helping and the helped threads will then interact with the observer to
communicate their parameters and return values.
%
In such a case, the helping thread may be able to help an unbounded number
of threads (all those who can be helped in the current configuration).
%
In other cases, 
the helping thread may explicitly 
linearize for the helped thread, which means that 
the helped thread itself need not communicate with the observer.
%
%
Furthermore, a given algorithm may use several of these patterns to 
define its linearization points.
\bjcom{The preceding paragraph has the fundamental problem that you do not describe how the ALGORITHM works, but
  rather you describe what the CONTROLLERS do. The controllers are not part of the algorithm (they will be introduced only
  later, when you describe how they work}
\bjcom{To be more precise on ``helping''. My understanding is that in concurrent algorithms, ``helping'' means that one method
  actively helps another method to perform an operation, e.g., to remove an element. In your above text, you use ``helping''
  in another sense: namly that the LP of one method can be placed in the operation of another. This is different, and has nothing
  to do with the working of the algorithm, it is only relevant for reasoning about linearizability. So, if you discuss ``helping'', it
  should be in an example where the methods actually help each other}

%
\input img/policy 

To specify the linearization patterns \bjcom{new undefined term, you can use ``various ways in which LPs can be assigned}, we equip each method with a
{\it controller} whose behavior is defined by a set of rules which are described detail in paper II.
%
\bjcom{You need a couple of more sentences here to explain what is the role of controllers}
The controller is occasionally activated by the thread,
and helps organize the interaction  
of the thread with other threads as well as with the observer.
%
More precisely, some 
statements in a  method are declared to be {\it triggering}. If a triggering statement is executed then 
the controller of the thread will also be executed simultaneously.  

%\input lazy-list-rrules
In the {\tt Lazy Set} algorithm, a successful {\tt rmv} method has its LP at line 3, and an unsuccessful {\tt rmv} has its LP at line 2 when the test $\tt c.val = e$ evaluates to \false. A successful {\tt add} method has its LP at line 4 and an unsuccessful {\tt add} has its LP at line 2 when the test $\tt c.val <> e$ evaluates to \false\;. The successful {\tt ctn} is linearized at line 4 then the value of $\tt b$ is \true. However linearization point of unsuccessful {\tt ctn} is not fixed in the code of the method.
\bjcom{The preceding text in this paragraph can be moved and adapted to the earlier paragraphs of this chapter, where you explaine fixed and non-fixed LPs}
\bjcom{The rest of this paragraph fits into the text as it is}
 Let us describe an example of how the {\tt controller} handles non-fixed linearization point of {\tt ctn} method. Figure \ref{fig:policy} give an example of {\tt Lazy Set} with three threads \threada\;, \threadb\;, and \threadc. The \threada \; is executing the {\tt add} to add the element {\tt e} into the set, whereas \threadb\; and \threadc\; are executing {\tt ctn} to lock for the element {\tt e} in the set. When the thread \threada\; reaches the triggering statement at line 4 of the {\tt add} method at the step \;\stepa\;. The controller rule 
%the linearization point of the {\tt add} operation
%is defined statically in the code of the method
%(lines {\tt 2} and {\tt 4}).
%%
%There are two possible linearization points
%corresponding to whether the 
%operation is {\it unsuccessful} (the element to be added is already in the 
%list), or {\it successful} (the element is not in the list.)
%%
%In the first case, the thread reaches the triggering statement
%at line {\tt 2} and the condition ``{\tt c.val = e}''  holds,
%i.e., we have found the element {\tt e} in the list. Thereafter, it informs the observer that an {\tt add} operation 
%with argument {\tt e} has been performed, and that the outcome
%of the operation is {\tt false} (the operation was unsuccessful.)
%%
Before a successful {\tt add} operation is communicated to the observer 
to informs the observer that an {\tt add} operation 
with argument {\tt e} has been performed, and that the outcome
of the operation is {\tt true} (the operation was successful) in the step \stepf\;. The controller will help other threads
by to linearize. This is done by {\it broadcasting} a message to the threads \threadb, \threadc\; which is executing {\tt ctn} in steps \stepb, \stepc. These threads then inform the observer that the element e is not in the list in \stepc, \stepe\; respectively. Note that \threadb, \threadc\; can get message in any order. The detail of controller is described in paper II.
\bjcom{The description of linearization policy in the preceding paragraph is in principle OK. But the grammar and language is not acceptable. There are even unfinished sentences. Please make an effort to write nice english, explaining in a way that colleagues can easily understand how things work.}

\bjcom{Write more about linearization policies. You should by example explain syntax of controllers, intuitively what they can do, etc.
  For instance, how they reduce checking for linearizability to checking (non)reachability of bad states. This includes all the stuff that is
currently done by monitors.
  You shoudl also survey results in the paper: e.g., later you write ``In paper II, we show that ... reduces to reachability ...''}

\chapter{Shape Analysis}
Pointers and heap-allocated storage are features of all modern imperative programming languages.
%they are ignored in most formal treatments of the semantics of imperative programming languages
%because their inclusion complicates the semantics of assignment statements: 
They are the most complicated features of imperative programming language: an assignment through a
pointer variable (or through a pointer-valued component of a record) may have large side effects on programs. For example, dereferencing a pointer that has been freed will lead to segmentation faults in a C++ program.
There exist several works that have treated the semantics of pointers such as works in [5, 42, 43, 45]. 
These side effects also make program dependence analysis harder, because they make it
difficult to compute the aliasing relationships among different pointer expressions in a program. For instance, consider the instruction $\tt x.f := y$ written in an imperative language with mutable records. Its effect is to assign the value of the pointer $\tt y$ to the  field f of the record pointed to by $\tt x$. We have to require information about all possible aliases of $\tt x$ in order to propagate the structural modification induced by the assignment. Having
less precise program dependence information decreases the opportunities for automatic parallelization
and for instruction scheduling. The usage of pointers is error prone. Dereferencing NULL pointers and accessing previously deallocated storage are two common programming mistakes. The usage of pointers in programs is thus an obstacle for
program understanding, debugging, and optimization. These activities need answers to many questions
about the structure of the heap contents and the pointer variables pointing into the heap. 

By shapes, which mean descriptors of heap contents. Shape analysis is a generic term denoting static
program-analysis techniques that attempt to discover and verify properties of the heap contents in (usually imperative) computer programs. The shape analysis problem becomes more interesting in concurrent programs that manipulate pointers and dynamically allocated objects. The area of verifying these programs has been a subject of intense research for quite some time. Currently, there are several competing approaches for symbolic heap abstraction. The first approach is based on the use of logics to present heap configurations. The logics can be separation logic \cite{John:SL, Stephen:SL,JoshCris:SL,Hongseok:SL,Kamil:SL,Chin:SL,Quang:SL, Ruzica:SL, Constrantin:SL}, 3-valued logic \cite{SagivRW02}, monadic second- order logic \cite{Ander:ML, Jakob:ML,Madhusudan:ML} or other \cite{Shmuel:Shape, Karen:Shape}. Another approach is based on the use of automata. In this approach, elements of languages of the automata describe configurations of the heap \cite{Ahmed:TreeAutomata, Ahmed:TreeAutomata2}. The last approach that we will mention is based on graph grammars describing heap graphs \cite{Jonathan:Shape, Jonathan:Grammars}. The presented approaches differ in their degree of specialisation for a particular class of data structures, their efficiency, and their level of dependence on user assistance (such as definition of loop invariants or inductive predicates for the considered data structures). 
  
Among the works based on separation logic, the work, such as \cite{JoshCris:SL,Hongseok:SL, Quang:SL} proposed more efficient approaches. The reason for that is that their approaches effectively decomposes the heap into disjoint components and process them independently). However, most of the techniques based on separation logic are either specialised for some particular data structure, or they need to be provided inductive definitions of the data structures. In addition, their entailment checking procedures are either for specific class of data structures or based on folding/unfolding inductive predicates in the formulae and trying to obtain a syntactic proof of the entailment. 

This issue can be fixed by automata techniques using the generality of the automata-based representation such as techniques using tree automata. Finite tree automata, for instance, have been shown to provide a good balance between efficiency and expressiveness. The work \cite{Ahmed:TreeAutomata} uses a finite tree automaton to describe a set of heaps on a tree structure,
and represent non-tree edges by using regular “routing” expressions. These expressions
describe how the target can be reached from the source using tree edges. Finite tree transducers
are used to compute set of reachable configurations, and symbolic configuration is abstracted
collapsing certain states of the automat. The refinement technique called counterexample-guided
abstraction refinement (CEGAR) technique is used during the run of the analysis. This technique
is fully automatically and can handle complex data structures such as binary trees with linked
leaves. However, it suffers from the inefficiency and it also can not handle concurrent programs.

TVLA (Three-Valued Logic Analyzer) \cite{SagivRW02} is the first and one of the most popular shape analysis
method. It is based on a three-valued first-order predicate logic with transitive closure. Intuitively,
concrete heap structure is represented by a finite set of abstract summary nodes, each of them
representing a set of concrete nodes. The shape of the heap is characterized by a set of usersupplied
predicates. The method is not fully automatic, its the synthesis of appropriate predicates
that are able to express the invariants in the program. This problem is even more difficult with
complicated heap structures such as skiplist, trees, or lists of lists.  
  
\section*{Our Approaches}
In this thesis, we proposed three approaches for heap abstractions. In paper I, we proposed a novel approach of representing sets of heaps via tree automata (TA). In our representation, a heap is split in a canonical way into several tree components whose roots are the so-called cut-points. Cut-points are nodes pointed to by program variables or having several incoming edges. The tree components can refer to the roots of each other, and hence they are “separated” much like heaps described by formulae joined by the separating conjunction in separation logic [15]. Using this decomposition, sets of heaps with a bounded number of cut-points are then represented by the so called forest automata (FA) that are basically tuples of TA accepting tuples of trees whose leaves can refer back to the roots of the trees. Moreover, we allow alphabets of FA to contain nested FA, leading to a hierarchical encoding of heaps, allowing us to represent even sets of heaps with an unbounded number of cut-points (e.g., sets of DLL, skiplist). \input TA
In addition, we express relationships between data elements associated with nodes of the heap graph by two classes of constraints. Local data constraints are associated with transitions of TA and capture relationships between data of neighboring nodes in a heap graph; they can be used, e.g., to represent ordering internal to some structure such as a binary search tree. Global data constraints are associated with states of TA and capture relationships between data in distant parts of the heap. This approach was applied to verification of sequential heap manipulation programs. This approach is general and fully automatic, it can handle many types of sequential programs without any manual step. However, due to the complexity of tree automata operations, this approach is not suitable to handle concurrent programs where a large number of states and computation are needed. Figure \ref{figure:forest} shows an example of how to represent a heap by a set of tree automata. Figure \ref{figure:forest}(a) shows an example of a heap where nodes whose values are \nodea, \nodeb, \nodec, \noded, \nodee \; are cut-points, and $\tt x$, $\tt y$, $\tt z$ are local pointer variables and $\tt g$ is global pointer variable. Figure \ref{figure:forest}(b) shows its forest representation. In the forest representation, there are five TAs in which the TAs \taa \; and \tac \; refer to the root of the last TA \tae\;, and both TAs \tab \; and \tad \; refer to the root of TA \tac \;. The local data constraints are located along the solid arrows between nodes, whereas global constraints are located along the dashed arrows. In this figure, the global constraints $\tt \prec_{aa}$ means that all nodes in the left hand side are smaller than all nodes in the right hand side. We just show here small examples of data constraints, the detail about different types of constraints can be found in paper I.   

In paper II, we provide a symbolic encoding of the heap structure, that is less precise than the approach in paper I. However it is precise enough to allow the verification of the concurrent algorithms, and efficient enough to make the verification procedure feasible in practice. The main idea of the abstraction is to have a more precise description of the parts of the heap that are visible (reachable) from global variables, and to make a succinct representation of the parts that are local to the threads. More concretely, we will extract a set of heap segments, where the end points of a segment is pointed to by a cut-point which is reachable from global variables. A cut-point in this approach is a reachable node from global variable, and pointed by a global variables or having more than two incoming pointers. For each segment, we will store a summary of the content of the heap along the segment. This summary consists of two parts, each part contains different pieces of information, including the values of the cell variables if they have finite values, and the ordering among them if they are integer variables. The first part summaries information between the end point and its predecessor, whereas the second part summaries information between the start point and the predecessor node. For each given program, the set of possible abstract shapes insight and hence the verification procedure is guaranteed to terminate. This approach is very efficient but it is not optimal for complicated concurrent data structures like trees, lists of lists or skiplists. Figure \ref{heapsummary} gives our summary abstraction of the heap in figure \ref{figure:forest}(a). In this approach, \nodea, \nodeb, \nodec, \nodee \; are cut-points. The node \noded \; is not a cut-point like the approach in paper I because its not reachable from the global variable $\tt g$. In each heap segment, the fist part is described by the white box, and the second part is described by the gray box.   
\input SL
In paper III, we present an approach which can handle concurrent programs implemented from simple to complex data structures. In our fragment abstraction, we represent the part of the heap that is accessible to a thread by a set of fragments. A fragment represents a pair of heap cells (accessible to $\thread$)
that are connected by a pointer field, under the applied data abstraction. The fragment contains both
(i) {\em local} information about the cell's fields and variables that
  point to it, as well as
(ii) {\em global} information, representing how
  each cell in the pair can reach to and be reached from
  (by following a chain of pointers) a small set of globally significant
  heap cells.
 A set of fragments represents the set of heap
structures in which each pair of pointer-connected nodes is represented by some
fragment in the set.
Put differently, a set of fragments describes the set of heaps that can be formed by
``piecing together'' pairs of pointer-connected nodes that are represented
by some fragment in the set. This ``piecing together'' must
be both locally consistent (appending only fragments that agree on their
common node), and globally consistent (respecting the global reachability
information).
\input fragment

Let us illustrate how pairs of heap nodes can be represented by fragments. Figure \ref{fragment} shows the set of fragments abstracted from the heap in \ref{figure:forest}(a). In each fragment, the ordering between two keys of two nodes is shown as a label on the arrow between two tags. Above each tag is pointer variables. The first brown row under each tag is $\tt reachfrom$ information, whereas the second green row is $\tt reachto$ information.

%To verify linearizability of the algorithm in Figure~\ref{figure:lazy-list},
%we must represent several key invariants of the heap. These include (among others)
%\begin{numberedlist}
%	\item The list is strictly sorted in $\tt key$ order, two unmarked nodes cannot have the same $\tt key$.
%\item All nodes which are unreachable from the head of the list are marked.
%\item The variable $\tt p$ points to a cells whose $\tt key$ field is never
%  larger than the input parameter of its $\tt add$,$\tt rmv$ and $\tt cnt$ methods.
%\end{numberedlist}
%Let us illustrate how such invariants are captured by our fragment abstraction. 1) All fragments are strictly sorted, implying that the list is strictly sorted. 2) This is verified by inspecting each tag: $\frag_{6}$ contains the only unreachable tag, and it is also marked. 3) The fragments express this property in the case where the value of $\tt key$ is the same as the value of the observer register $\tt x$. Since the invariant holds for any value of $\tt x$, this property is sufficiently represented for purposes of verification.   







%%% ====================================================================
%\section[Abstracting Three Degrees of Unboundedness]%
\section[Bounded Abstract Domain | No Garbage Collection]%
%{Handling 3 Degrees of Unboundedness {\footnotesize and No Garbage Collection}}
{Three Degrees of Unboundedness {\footnotesize and No Garbage Collection}}
\label{section:unboundedness}
\label{section:nogc}
%% ====================================================================

We can now specify if a property gets violated using the finite
observers on an abstract stack using data independence.
%
Thus, we can handle
Challenge~\ref{unbounded:data:domain:specification} and
~\ref{unbounded:data:domain}.
%
However, the memory layout of heap cells (called \emph{shapes}) that
the program can create and manipulate is not bounded in size.
%
To tackle Challenge~\ref{unbounded:heap:size}, it is necessary to
handle the size of those shapes.

Each thread has a finite set of variables and can access the heap
cells that are pointed to by one of its variables or a global
variable. The heap cell that are not pointed by any variable can be
potentially accessed through a succession of \prgcode{next}
pointers. Otherwise, they are considered garbage.
%
The idea is to only keep track of \emph{important} cells. Those are
the ones pointed by some variable and the ones containing some
\emph{important} data, i.e.\ \prgcode{red} and \prgcode{blue} in the
case of Treiber's stack.
%
Intuitively, the cells not pointed by any variable play a~secondary
role. There are merely ``relays'' between the cells that the threads
can currently access. We do not need to keep how many there are, we
only need to keep the memory layout they form.

\index{Transitive Closure}%
We thus adapt a variant of the transitive closure logic by Bingham and
Rakamari\'c~\cite{BiRa:vmcai06} for reasoning about heaps with single
selectors, to our framework. This formalism, called here \emph{shape
  analysis}, %
\index{Shape Analysis}\index{Heap Analysis}%
tracks reachability properties \index{Reachability}%
between pairs of pointer variables.
%
Moreover, we have developed a novel optimization, based on the
observation that it suffices to track the possible relations between
each pair of pointer variables \emph{separately}.
% , analogously to the
% use of DBMs used in reasoning about timed automata~\cite{Dill:DBM}.

% , and we adapt it to our analysis, in which pairs of
% threads are correlated.
%


%% -------------------------------------------
We use shape analysis to cope with challenge~\ref{unbounded:heap:size}
but it is still not enough to bound the shapes, since we have an
unbounded number of threads. Each thread could have one~variable
pointing to a~different cell and that creates an unbounded shape.
%
\index{Thread-Modular}%
To cope with challenge~\ref{unbounded:threads}, we try to adapt the
successful thread-modular approach~\cite{BLMRS:cav08}, which verifies
a concurrent program by generating an invariant\index{Invariant} %
that correlates the global state with the local state of an arbitrary
thread.
%
In other words, it keeps track of the shape that \emph{one} thread can
see, %realize
abstracting away (and ignoring) all the other threads.
%
The resulting shape is necessarily of bounded size using the above
transitive closure.
%% -------------------------------------------
We can draw a~parallel with the view abstraction method from
Chapter~\ref{chapter:view:abstraction}, where the configurations are
abstracted into views of size 1 ($k=1$).
%
\index{Interference}%
The thread-modular approach includes a step where it takes the
information about one thread and couples it with the information from
another thread, in order to take into account %capture
the interference %operations
of all the other threads on the first thread.
%
This step is akin to the concretization from the view abstraction,
where we extend the views which additional information from other
views, in a consistent manner.

%% -------------------------------------------
However, this leaves us with a problem when $k=1$: the interfering
step would create shapes where threads possibly share some particular
cell, even though the concrete system would never produce those
shapes.
%
This is a too coarse approximation that leads to erroneous behaviors.
%
\index{Garbage-Collection}%
In fact, in the case of garbage collection, even though the
thread-modular approach only keeps track of one thread, it is possible
to get around the problem by including additional information on the
shape cells, to reflect whether a cell is shared or not. Thanks to the
guarantee that garbage collection provides, when a cell is created,
the creating thread is the only one accessing that cell. The cell is
fresh and we mark it as such. This flag is lost as soon as the cell
becomes accessible by another thread or a global variable.
\index{Fresh cell}%
%
This information allows the interference step to be more precise, and
separate the situations where it would otherwise merge together two
fresh cells from different threads, making the resulting cell
erroneously shared. The thread-modular approach, along with the
freshness information, is often enough to carry on the verification.

%% -------------------------------------------
\index{Non Garbage-Collection}%
Nevertheless, it is not sufficient to cope with
challenge~\ref{no:garbage:collection} and the ABA problem.
%
The generated invariant must be able to express that \emph{at most}
one~thread accesses some cell on the global heap.
%
Since this cannot be expressed in general by the thread-modular
approach, we need to extend it %the approach
to generate invariants\index{Invariant} %
that correlate the global state with the local states of an arbitrary
\emph{pair} of threads.
%
\index{Thread Corrolation}%
\index{Precision of View Abstraction}%
By correlating two threads, we increase the precision of the
abstraction, alleviating the above problem. This is similar to the
view abstraction when $k=2$ and the concretization function extends
views in a consistent manner.

\begingroup%
\setlength\intextsep{\dazintextsep}
\begin{figure}[ht]
  %\centering
  \tikzinput[\linewidth]{img/shape-abstraction}
  \caption{From a concrete shape into one of the shape abstractions}
  \label{figure:shape:abstraction}
\end{figure}
\endgroup

In Figure~\ref{figure:shape:abstraction}, we show how a concrete shape
with three threads can be abstracted into a bounded abstract
shape. The figure depicts only one of the shape abstraction that we
obtain. There is a combinatorial factor in the choice of which two
threads to pick out of, here, the three threads.
%
Indeed, the above abstraction techniques bring the needed precision
for verification, at the price of significant state-space explosion,
which mainly arises from the fact that, reusing the terminology from
Chapter~\ref{chapter:view:abstraction}, the dynamically allocated
heaps are scattered across several views which describe the
correlation between pairs of threads.
%
We have therefore developed a novel optimization, where we
\emph{merge} abstracted shapes and manipulate the compound instead of
every individual abstracted shapes.
\index{Symbolic Representation}%
%
Intuitively, merging does not bring a~penalizing %dreadful
imprecision due to the structure of the retry-loops.
\index{Retry-Loop}%
%
When the thread's local variables are initially placed, but the global
shape changed through the actions of other threads, the current thread
will detect it and restart its initialization.
%
Hence, recording precisely where those local variables are is often
superfluous information that can be discarded.
%
We touch more on that optimization in the next section.

%%% ====================================================================
\section{Verification procedure}
\label{section:shape:verification:procedure}
%% ====================================================================

To verify that no trace of the program is accepted by an observer, we
form, as in the automata-theoretic approach~\cite{VW:modelchecking},
the cross-product of the program and the observer, synchronizing on
abstract \emph{events}, and check that this cross-product cannot reach
a~configuration where the observer is in an accepting state.
%
Synchronizing on abstract events means that we instrument the program
to communicate with the observer every time it passes a~commit point
successfully. The observer will then advance state depending on the
event that has been communicated.

Using data independence, shape analysis and thread bounding, we
characterize \emph{all} the reachable configurations of the program,
from the point of view of two distinct executing threads, with
a~symbolic encoding.
\index{Symbolic Representations}%
%
The encoding consists in a~combination of several layers of
conjunctions and disjunctions, to record \emph{pairwise} the
relationships of the local configurations of the two threads with each
other, the relationships of the local variables of a thread with
global variables, the observer configuration, and finally the
assertions about the heap.

\begingroup%
\setlength\intextsep{\dazintextsep}
\begin{figure}[ht]
  %\centering
  \tikzinput[1.08\linewidth]{img/shape-analysis}
  \caption{Encoding during the analysis and a corresponding shape}
  \label{figure:shape:encoding}
\end{figure}
\endgroup

%
For example, in Figure~\ref{figure:shape:encoding}, we represent the
encoding in two parts. The first part is some bookkeeping information,
denoted $\sigma$, consisting of the \prgcode{pc} of each thread and
the observer state.
%
The second part is a~matrix representing the relationships between the
variables pairwise.
%
A local variable $v$ from thread $n$ is denoted $v_n$.
%
In the running stack example, the variables in concern are the global
variable~\prgcode{Top}, the special term~\nullconst\ (i.e.\ the
\prgcode{null} constant), the local variables of the two current
threads \prgcode{$t_1,x_1,t_2,n_2$}, %
and the cells that contain important data, dictated by the observer,
(here \prgcode{red} and \prgcode{blue}).
%
The precise mathematical definitions can be found in
Paper~\ref{paper:TACAS13}.

\newpage%
Here, we only give the intuition behind the relationships:
%succintly and intuitively.
%
\begin{itemize}[leftmargin=*]
\item $t_a = t_b$: the variables $t_a$ and $t_b$ point to the same cell,
\item $t_a \pointsto t_b$: the \prgcode{next} field of the cell $t_a$
  points to, and $t_b$ point to the same cell,
\item $t_a\reaches t_b$: $t_b$ points to a cell that can be reached by
  following a chain of two or more \prgcode{next} fields from the cell
  that $t_a$ points to,
\item
  $t_a \unrelated t_b$: none of 
  $t_a = t_b$, %
  $t_a \pointsto t_b$, %
  $t_b \pointsto t_a$, %
  $t_a \reaches t_b$, or %
  $t_b \reaches t_a$ is true.
  %% \item $\isfree{\term}$ means that the cell denoted by $\term$ is not allocated.
\end{itemize}
%
% We use $\Pred$ to denote the set
% $\set{=,\pointsto,\pointedby,\reaches,\reachedby,\unrelated}$ of all
% shape relational symbols. %
%Naturally, 
For any term $t$, we let $t=\nullconst$ denotes that $t$ is
null. % and $t\pointsto\undefconst$ denote that $t$ is undefined.
%
In Figure~\ref{figure:shape:encoding}, the relationship between $t_2$
and $n_2$, denoted $\pi[t_2,n_2]$, is that they are unrelated.

This encoding is precise enough to represent an~abstract
shape. However, to combat an obvious state-space explosion problem, we
\emph{merge} these matrix representations into yet another matrix
where each cell is a disjunction of relations from the set
$\set{=,\pointsto,\pointedby,\reaches,\reachedby,\unrelated}$.
%
We depict for example in Figure~\ref{figure:shape:merging}, two
abstract shapes, where the threads are (of course) at the same
\prgcode{pc}, which are merged into \emph{one} matrix
representation. Notice that the matrix representation now also
characterizes two other shapes that were not considered in the
merge. In other words, this new matrix representation is an
over-approximation of the set of concrete shapes it characterizes.

\begingroup%
\setlength\intextsep{\dazintextsep}
\begin{figure}[ht]
  %\centering
  \tikzinput[\linewidth]{img/shape-merging}
  \caption{Merging shapes: Each cell is a disjunction over
    $\set{=,\pointsto,\pointedby,\reaches,\reachedby,\unrelated}$.}
  \label{figure:shape:merging}
\end{figure}
\endgroup

% \paragraph{Joined shape constraint.}
% A joined shape constraint, for thread $i_1$ and
% $i_2$, denoted as $\matrixrep(i_1,i_2)$, is a (typically large)
% conjunction %
% $\mathop\bigwedge_{\term_1,\term_2\in \cellterms(i_1,i_2)}
% \entryof{\term_1}{\term_2}$ %
% where $\entryof{\term_1}{\term_2}$ is a non-empty
% disjunction of atomic heap constraints. %
% % for the pair
% % ($\term_1$,$\term_2$) of cell terms.
% % (i.e.\ of the form $\term_1\somerel \term_2$ for ${\somerel}\in\Pred$).
% % 
% Intuitively, it is a matrix representing the heap parts accessible by
% the two threads (along with the cell data). Such a representation can
% be (exponentially) more concise than using a large disjunction of
% conjunctions of atomic heap constraints, at the cost of some loss of
% precision.

%% ========================================================
%\paragraph{Postcondition computation.}
\KW{Postcondition}%
\index{Post-image operator}%
%% ========================================================
We can now describe how those matrix representations will be
manipulated and cover the different cases which arise in a concurrent
setting.
%
We need to consider two cases: %
On one hand, given a matrix, one of the two represented and currently
executing threads performs a~step.
%
On the other hand, we must consider the case where another third and
distinct thread interferes and changes the matrix, even when the two
represented threads do not perform any step.

In the first scenario, where one of the two represented threads
performs a step, we can compute the resulting matrix usually in a
straightforward manner.
%
We (i) remove all disjuncts that must be falsified by the
step %
(ii) add all disjuncts that may become true by the step, %
(iii) saturate the result.
%
The details can be found in Paper~\ref{paper:TACAS13}.
%
For example, when the action is a~variable assignment from a given
thread, the row and column relating that variable are ``canceled'' and
the new information is derived from what it is assigned to. A
saturation procedure determines whether there are new disjuncts to be
added or if some old ones are to be voided.

% Consider for instance the program statement {\code{x:=y.next}}. %
% Since only the value of $x$ is changing, the transformer updates only
% conjuncts $\entryof{\term}{x}$ and $\entryof{x}{\term}$ where
% $\term\in\cellterms(i_1,i_2)$. %
% All assertions about $x$ are reset by setting every conjunct
% $\entryof{x}{\term}$ and $\entryof{\term}{x}$ to $\Pred$, for all
% $\term\in\cellterms(i_1,i_2)$. %
% (The disjunction over all elements of $\Pred$ is the assertion
% $true$). %
% We then set $\entryof{x}{y}$ to $x\pointedby y$, $\entryof{y}{x}$ to
% $y\pointsto x$ and derive all predicates that may follow by
% transitivity. %
% Finally, we saturate the formula. It prunes the (newly added)
% predicates that are inconsistent with the rest of the shape formula.
% %Third, we infer possible relationships that the other cell terms can have with $x$ assuming $y\pointsto x$.
% %We iterate through all $\term\in\cellterms(i_1,i_2,i_3)$, $x\neq \term \neq y$ and assign to $\entryof{x}{\term}$ the disjunction of all relationships between $x$ and $\term$ consistent with $x\pointsto y$ and $\entryof{y}{\term}$.

% For {\code{x.next:=y}}, it is important to know the reachabilities
% that depend on the pointer $x\code{.next}$. %
% The representation might potentially contain imprecision %
% (it might for instance state that, for a term $\term$,
% $\entryof{\term}{x}$ contains $t\reachedby x$ and $t\reaches x$, %
% even if we know, via a simpler analysis, that no cycles are generated). %
% Hence, we first split the formula into stronger formulas in such a way
% that we disambiguate the part of the reachability relation involving
% $x$. %
% %
% On each resulting formula, we then remove reachability predicates
% between cell terms that depend on $x\code{.next}$ %
% (e.g., we remove $u\reaches v$ if $u\reaches x$ and $x\reaches v$). %
% %(e.g., we replace $u\reaches v$ in $\entryof{u}{v}$ by $u\unrelated v$
% %if $u\reaches x$ and $x\reaches v$). %
% We then set $\entryof{x}{y}$ to $x\pointsto y$ and derive all
% predicates that may follow by transitivity (e.g., if $u\reaches x$ and
% $y\reaches v$, we add $u\reaches v$), and we saturate the result.

\begingroup%
\setlength\intextsep{\dazintextsep}
\begin{figure}[ht]
  %\centering
  \tikzinput[\linewidth]{img/shape-interference}
  \caption{Interference}
  \label{figure:shape:interference}
\end{figure}
\endgroup

%% ==================================================
%\paragraph{Interference.}
\KW{Interference}%
\index{Interference}%
%% ==================================================
%
% In the case where we need to account for possible interference on the
% matrix by another thread, (distinct from the two it already
% represents), we proceed as follows.
In the case where we need to account for possible interference on the
matrix by a third thread, we proceed as follows.
%
We %
(i) extend the matrix with the interfering thread, %
(ii) apply the first scenario to compute new matrices and %
(iii) project away the interfering thread from the resulting matrices.
%
The steps are described in Figure~\ref{figure:shape:interference}.
%
On the top left corner, we show the abstract shape to be considered,
(but we hide and replace the content of the matrices with dots).
%
%
In step (i), if a~third interfering thread $i_3$ (here in light
purple) is to exist in the presence of the two other passive threads
$i_1$ and $i_2$ (here in yellow and light pink), then it must be the
case that \emph{both} the shape correlating $i_1$ and $i_3$, and the
shape correlating $i_2$ and $i_3$, must exist among the shapes that
have been created by the program!
%
If so, we extend the first shape accordingly with the third thread,
which might strenghten the information in the larger matrix.
%
Naturally, the $\sigma$ information is extend accordingly too.
%
We then apply a~post operation (step (ii)) as in the first scenario,
which creates new matrices where the content related to thread $i_1$
and $i_2$ might have changed (shown with stars).
%
Finally, in step (iii), we project away the third thread. The
resulting matrices correlate thread $i_1$ and $i_2$ but are the result
of a~third interfering thread.
%
Notice the similarity with the view extensions from
Chapter~\ref{chapter:view:abstraction}.

The main idea of the method is then to collect \emph{all} possible
abstract shapes without sending the observer onto an accepting state.
%
\index{Fixpoint}%
It is implemented using %a~straightforward
a~fixpoint procedure, starting from the shapes that characterize the
set of initial configurations of the program.
%
Upon termination, we obtain an invariant of the program which
characterizes the configurations of the program from the point of view
of two distinct executing threads $i_1$ and $i_2$.
%
The main advantage of the method is that it is a~direct approach for
verifying that a~concurrent program is a linearizable implementation
of, for example here, a stack. It consists in checking a~few small
properties of the algorithm, and is thus suitable for automated
verification.
% 
Previous approaches typically verified linearizability separately from
conformance to a~simple abstraction, most often using simulation-based
arguments, which are harder to automate than simple property-checking.
%
Moreover, the method can automatically verify concurrent programs that
use explicit memory management. This was previously beyond the reach
of automatic methods.
%
% In addition, on examples that have been verified automatically by
% previous approaches, our implementation is in many cases significantly faster.
%% Experimental evaluation, which shows that our technique can fully
%% automatically verify a range of 
%% concurrent implementations of common data
%% structure implementations, such as queues, stacks, etc.
%% \end{itemize}


%% ====================================================================
%\whatwelearned{shape-analysis}
%% ====================================================================
