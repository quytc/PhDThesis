%% ====================================================================
\section*{Linearizability}      
\label{section:specification:concurrent:data:structure}     
\index{Specification}%         
%% ====================================================================
\begingroup%     
In a concurrent program, the methods of the different executing threads can overlap in time. Therefore, the order in which they take effect is ambiguous. The program statements 
in each method are totally ordered. Whereas, statements from different methods in different executing threads might form a partial order. This partial order raise the difficult of reasoning about program execution. One of the main correctness criterion of a concurrent program is linearizability, which defines consistency for the history
of call and response events generated by an execution of the program at hand \cite{HeWi:linearizability}. Intuitively, linearizability requires every method to take effect
at some point ({\emph {linearization point}}) between it's call and return events. A linearization point is often a moment where the effect of the method
becomes visible to other threads. A (concurrent) history is linearizable if and only if there is some order for
the effects of the actions that corresponds to a valid sequential history. The valid sequence history can be generated by an execution of the sequential specification object. A concurrent object is linearizable iff each of its histories
is linearizable.

\setlength\intextsep{\dazintextsep}
\begin{figure}[ht]
  \centering
  \tikzinput[\linewidth]{img/linearizability} 
  \caption{Linearizability, where the commit points are marked with \protect\commitpoint{}.}
  \label{figure:shape:linearizability}  
\end{figure}          
  

\index{Linearizability}%
Figure~\ref{figure:shape:linearizability} provides a examples of trace of methods of concurrent program implementing sets.
In the trace, each method takes effect
instantaneously at its (called the \emph{linearization point})
between call and return events~\cite{HeWi:linearizability}. When we order methods according to its linearization point, we get a total ordered sequence that respect the behavior of the set.
%A linearization point normally stays inside the code of the
%method.  However, in some cases, it it is located in
%the code of another method depending on the execution path.



\endgroup