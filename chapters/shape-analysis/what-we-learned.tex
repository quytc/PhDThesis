\begin{description}
\item[Data Independence] is an argument used when verifying programs
  that treat data equally. It allows us to rename the data and
  typically handle the verification task using 3 key values.
\item[Observers] are automata that allow us to capture concisely the
  bad behaviours of the data-structures they are paired with.
\item[Heap Abstraction] is necessary to bound the size of the
  different shapes that the program can create, without introducing
  too much imprecision. It uses a shape analysis based on the
  transitive closure of the reachability relation. We showed that it
  is enough to record pairwise the reachabilities of program
  variables.
\item[Joined Heaps] help combat the state-space explosion
  problem % similarly to \ref{combat:state:space:explosion} on page \pageref{combat:state:space:explosion}.
  (from page~\pageref{combat:state:space:explosion}).
\item[Verification] is performed through a~simple forward analysis,
  with a fixpoint computation. The task consists in checking that the
  cross-product of the program and an observer never sends the latter
  onto an accepting state.
\item[Garbage Collection.] In the absence of garbage collection,
  programs may suffer from the ABA problem. The method can handle that
  case, which is one of the contribution of this thesis.
\item[Linearizability] offers the illusion that methods happen at
  once. The method does not prove linearizability per say, but it shows
  directly the conformance of a~concurrent program to its sequential
  specification.
\item[Combined Challenges.] The method addresses several challenges.
  \begin{itemize}[leftmargin=0pt]
  \item The specification is infinite-state, because the data domain
    is unbounded.
  \item The program is infinite-state in several dimensions since it
    consists of an unbounded numbers of current threads, it uses
    unbounded dynamically allocated memory and the data domain is also
    unbounded.
  \item It handles the case of explicit memory management.
  \end{itemize}
  To the best of our knowledge, no other method can handle those
  challenges at the same time.
\end{description}
