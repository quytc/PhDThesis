\begin{description}
\item[Bugs] make computers crash and some of them are vital to fix.
\item[Testing] is a classical method to ensure software quality and
  requires to run the program. However, it does not cover all possible
  executions and therefore might leave errors undetected.
\item[Formal Verification] uses, on the other hand, mathematical
  proofs to derive the absence of errors.
\item[Properties] that are interesting to verify, usually fall into
  two categories: safety or liveness properties. We focus on safety
  properties, which state that the program never encounters a bad
  situation.
\item[Model-Checking] is a fully automatic method where a model of the
  program is extracted and checked against a property.
  %
  States and transitions characterize how every step of the program
  takes place in the model.
  % 
  The number of states can be finite or infinite depending on the
  complexity of the program and its input parameters.
  % 
  This approach has the advantage that it does not require us to run
  the program but can suffer from state-space explosion.
  % 
\item[State-space explosion] is handled by grouping states together
  using compact representations and by using approximations.
\item[Over-approximations] aim %help
  to derive an abstract model in which it is easier %to handle
  to determine whether the original program is correct.
\item[Parameterized systems] consist in an arbitrary number of
  processes in parallel, communicating with each other at any
  time. The number of processes is the parameter of the verification
  problem. Parameterized systems are therefore considered
  infinite-state --- and are the focus of this thesis.
\item[Parameterized verification.] Many systems can be characterized
  by a collection of finite-state systems with one (unbounded)
  parameter. The task is to prove correctness, regardless of the value
  of the parameter.
\item[Practical algorithms.] This thesis focuses on making the
  algorithms practical and not suffer too much from state-space
  explosion, rather than theoretical results on the decidability of
  classes of systems.
\end{description}
