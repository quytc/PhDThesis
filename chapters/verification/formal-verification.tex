%% ====================================================================
\section*{Formal Verification} Computer programs are written based on human intuition, which is probably leads to programming errors. Current practice is to test programs on various sample inputs in the hope of finding any possibility of incorrect program behavior. There exist many approaches like testing, simulation, static analysis and simple debugging techniques, such as inserting assertions and print statements in the source code, which show the presence of software errors. 

Formal verification uses mathematical methods to prove that a program is correct. Formally, formal verification is the process of checking whether a software satisfies its predefined properties. There is a wide variety of properties to be checked for software programs, these properties can be either safety or liveness properties. Liveness properties state that program execution eventually reaches several desirable states at some point of execution, for example liveness properties can be "the postman delivers the letter to the recipient", "A sent message is eventually received". In contract, verifying safety property of a program is satisfied is reduced to checking that something bad will never happen in the execution of the program [48]. 
%\input model
%In order to specify liveness properties, it is needed to  describe traces of events by using temporal logics, statistics, and probabilities. Checking aliveness property is done by repeatedly checking reachability of good situations in program executions. 

There are several state-of-the-art approaches for formal verification namely model checking, theorem proving and equivalence checking. Model checking approach exhaustively explorer all possible states of the model which can be finite or infinite where infinite sets of states can be represented finitely by using abstraction techniques. Equivalence checking method decides whether system is equivalent to its specification with respect to some notation of behavioral equivalence. Theorem proving is a technique where both the system and its desired properties are expressed in mathematical logic. Then, theorem proving will try to prove these properties. In this thesis though, we consider programs where the specification describes the bad behaviors. We concentrate on safety properties and try to design abstraction techniques to verify that a program including both sequential and concurrent program respects its specifications. 

\section*{Verification of Data Structures}
Most modern programming languages provide libraries of data structures such as the C++ Standard Template Library, the Java Collections Framework. A data structure is a particular way of organizing and storing data in a computer so that it can be accessed and modified efficiently. More precisely, a data structure is a collection of data values, the relationships among them, and operations that can be applied to the data. Each data structure has it own specification which describes the behaviour of the data structure. A data structure can be both sequential or concurrent, concurrent data structures can be accessed and manipulated concurrently by many parallel threads are a central component of many parallel software applications. They should allow a large degree of parallelism among accessing threads to minimize serialization bottlenecks, while maintaining the appearance of atomic operations. Many modern programming languages provide libraries of concurrent data structures (e.g., the java.util.concurrent package and Intel Threading Building Blocks library) that are widely used. 

To ensure that a concurrent data structure is correct, we have to ensure that it respects to its sequential specification.  Ideally, the correctness is captured by linearizability. Linearizability is generally accepted as the standard correctness criterion for such concurrent data structure implementations. It states that each operation on the concurrent data structure can be viewed as being performed atomically at some point (called linearization point (LP)) between its invocation and return. The linearizability guarantee relieves the programmer from complex reasoning about possible interference among data-structure methods and removes the need to add explicit synchronization. Concurrent implementations of abstract data structures (stacks, queues, sets, etc.) are becoming more and more complex as implementations that increase the degree of concurrency are identified. This in turn is making linearizability verification harder. Existing approaches lack generality as they are limited to specific classes of concurrent data structures so far no technique (manual or automatic) for proving linearizability has been proposed that is both sound and generic. In this thesis, we focus on verifying safety properties including linearizability of both sequential and concurrent data structures.
\section*{Research Challenges}
%In this thesis, we consider two challenges in software verification, the first challenge is to automate its application to sequential programs that manipulate complex dynamic linked data structures. The problem becomes even more challenging when program correctness depends on relationships between data values that are stored in the dynamically allocated structures. Such ordering relations on data are central for the operation of many data structures such as search trees, priority queues (based, e.g., on skip lists), key-value stores, or for the correctness of programs that perform sorting and searching, etc. The challenge for automated verification of such programs is to handle both 
%\begin{challenges}
%\item infinite sets of reachable heap configurations and
%\item relationships between data values embedded in such graphs, 	
%\end{challenges}
%e.g., to establish sortedness properties, there exist many automated verification techniques, based on different kinds of logics, automata, graphs, or grammars, that handle these pointer structures. Most of these approaches abstract from properties of data stored in dynamically allocated memory cells. The few approaches that can automatically reason about data properties are often limited to specific classes of structures, mostly singly-linked lists (SLLs), and/or are not fully
%automated.
%
%We present a general framework for verifying programs with complex dynamic linked data structures whose correctness depends on relations between the stored data values. Our framework is based on the notion of forest automata (FA) which has previously been developed for representing sets of reachable configurations of programs
%with complex dynamic linked data structures [?]
In this thesis, we consider challenges in developing techniques for automated verification of both sequential and concurrent data structure using heaps. 
%Our challenges are to automate its application to both sequential and concurrent programs that manipulate complex dynamic linked data structures. We have to deal with concurrent programs with an unbounded number of threads that concurrently access and manipulate a dynamically allocated shared heap where data stored in each heap cell can be in unbound domain. Such programs and algorithms are difficult to get correct and verify, since their shapes are complicated to represent and they typically employ fine-grained synchronization, replacing locks by atomic operations such as compare-and-swap, and are therefore notoriously difficult to get correct, witnessed. It is therefore important to develop efficient techniques for automatically verifying their correctness. This requires overcoming several challenges. This thesis presents simple and efficient techniques to verify that a concurrent implementation of a common data type abstraction, namely queue, stack, set, conforms to a simple abstract specification of its (sequential) functionality. The data structures we consider for these programs can be singly-linked lists, sets of linked lists or skip-lists. In order to deal with this problem, we have to deal with several combined challenges as follow.
We have to deal with several sub-challenges as following: 
\newpage
\begin{challenges}
\item Heaps which are used by data structures are dynamically heap allocated memory. Therefore, we have to deal with unbounded number of heap cells. In this thesis, propose three heap abstraction techniques including forest automata, summary abstraction and view abstraction. The detail about these approaches are described in the next section.
\item In each cell of a heap, the domain of data values can be unbounded. We use the combination of shape analysis and data abstraction to deal with this challenge. 
\item We have to verify that the data structures are correct with any number of threads that access and manipulate the structures. We use the thread modular technique to dead with this challenge.
\item To specify the correctness of concurrent data structures, we prove the technique to verify linearization properties of the data structures.
\item In some data strucute, the number of pointers can be unbounded. For example, skiplist or arrays of lists. This problem is solved by our fragment abstraction technique.
\end{challenges}

We present, in the next chapters, the general background about model checking, concurrent data structures. Thereafter, in the following chapter, we introduce in a stepwise manner how
we cope with  above challenges. In the last chapter, we summary and give future plans for our work.