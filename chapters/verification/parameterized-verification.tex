%%% ====================================================================
%\section{Parameterized Verification}
%\label{section:paramsys}
%%% ====================================================================
%\KW{Parameter-ized systems are infinite-state}%
%\index{Parameterized Systems}%
%%% ********************************************************************
%Many systems can actually be characterized as a family of finite-state
%systems with one (or more) parameter ranging over an unbounded domain.
%%
%For each fixed value of the parameter, the system is finite-state.
%%
%However, a system that contains an \emph{a priori} unknown parameter
%must behave properly regardless of the value of the parameter. It is
%therefore considered infinite-state.
%
%%% ********************************************************************
%\KW{Parameter-ized verification}%
%\index{Parameterized Verification}%
%%% ********************************************************************
%The \emph{parameterized verification} problem is to prove
%correctness %\footnote{\ldots and namely here safety only}
%of the system, against some specification, for all values of the
%parameter.
%%
%For example, in the case of a program manipulating a list, the
%parameter could be the length of the list. The system as a whole is a
%set of finite-state systems, one for each length of the list.
%%
%For a network protocol, the parameter could be the topology of the
%network and the system is to be proven safe regardless of how the
%network nodes are arranged.
%
%%% ********************************************************************
%\KW{Parameter-ized systems}%
%%% ********************************************************************
%We concentrate in this thesis on a specific class of
%\emph{parameterized systems}, namely, systems consisting of an
%arbitrary number of processes (see
%Chapter~\ref{chapter:parameterized:systems}). %
%The size of the system is the parameter of the verification problem.
%% 
%For each value $n$ of the parameter, the system $S_n$ is the parallel
%composition of $n$ processes, which can interact with each other at
%any time.
%%
%\index{Safety}%
%Given a specification, we prove safety for the family $(S_n)_{n\ge 0}$
%--- as~a~whole --- by showing the non-reachability of some
%(potentially infinite) set of \emph{bad states}. %
%%
%We use the following strategy:
%\begin{strategy}
%\item \index{Model}%
%  We extract a model from each process in the system. This in turn
%  defines the operational semantics with states and transitions for
%  the entire family.
%\item \index{Over-approximation}%
%  We use an over-approximation to derive an abstract model from the
%  original (infinite-state) model.  This defines a new state-space and
%  set of transitions.
%\item \index{Abstract Model}%
%  We determine two sets in the abstract model:
%  \begin{itemize}[itemsep=0pt, leftmargin=0cm]
%    % \item the initial states --- usually mirroring the initial settings
%    %   of the program,
%  \item \index{Initial states}%
%    the initial states, usually mirroring the initial settings of the
%    program,
%    % \item the bad states --- usually along the lines of the targeted
%    %   property.
%  \item \index{Bad states}%
%    the bad states, usually along the lines of the targeted property.
%  \end{itemize}
%\item \index{Reachability}%
%  We finally check whether the bad states are reachable from the
%  initial states in the abstract model.
%  % 
%  % \item Apply model-checking techniques (from
%  %   Chapter~\ref{chapter:monotonic:abstraction}~or~\ref{chapter:view:abstraction})
%  %   to determine wether the bad states are reachable from the initial
%  %   states, by following a sequence of transitions --- also known as the
%  %   reachability problem.
%\end{strategy}%
%%
%\index{Soundness}%
%By construction of the over-approximation, showing that the abstract
%model is safe, will imply that the original system is also correct
%with respect to its specification, regardless of the number of
%processes in the system.
%%
%We present a formal definition and some examples in the next chapter.
%
%%% ********************************************************************
%\newpage
%\subsection*{Heap analysis}
%\KW{Heaps}%
%\index{Heap Analysis}%
%%% ********************************************************************
%An important extension in this thesis is the application of
%parameterized verification to the complex problem of shape analysis.
%%
%We consider programs that implement data-structures that can be
%concurrently accessed by an arbitrary number of threads.
%%
%The data-structures, e.g.\ stacks and queues, are constructed using
%singly-linked lists.
%%
%By following the chains of pointers, we observe that programs leave
%memory footprints that we coin \emph{shapes} or \emph{heaps}.
%
%The number of threads is one dimension of the parametrization. 
%%
%Shape analysis can also be parameterized along other dimensions, such
%as the size of the data-structure or the data domain.
%%
%In Chapter~\ref{chapter:shape:analysis}, we address the combined
%challenges of an arbitrary number of threads, an unbounded size of
%heaps and an unbounded data domain. Moreover, we do not assume the
%presence of a garbage collector which makes the task even more
%complex.
%
%The program specification itself is infinite since the programs can
%manipulate data values from a domain of unbounded size.
%%
%It is not enough to inspect each shape and conclude that it belongs to
%a particular set of bad shapes. We need to observe \emph{traces} of
%events. We therefore pair each shape with a special \emph{observer}
%whose role is to determine whether a sequence of events might not
%occur for a particular data-structure.
%%
%We still apply the above-described strategy and conclude that the
%program reaches a bad configuration when the observer is in a bad
%state.
%%
%We dedicate Chapter~\ref{chapter:shape:analysis} to explain this
%complex problem and the verification method in further details.
