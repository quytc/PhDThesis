%%% ====================================================================
%\section[Theoretical Limitations \ldots\ \emph{not}!]{Theoretical Limitations}
%\KW{Decidability}%
%\KW{Theoretical limitations}%
%\index{Theoretical limitations}%
%%% ====================================================================
%It is important to understand the type of programs that the methods
%can handle and with which limitations.
%%
%\index{Theoretical limitations!Decidability}%
%A problem is %said
%\emph{decidable} if there exists an algorithm to solve \emph{every}
%instance of the problem.
%% Otherwise, it is \emph{undecidable},
%% which is also the case when some instances can be solved but not all.
%%
%% Even though undecidable problems are hard ones to solve, some
%% instances can potentially still be solved.
%\index{Theoretical limitations!Intractability}%
%Decidable problems can still be \emph{intractable}, that is, even if a
%solution exists, it is difficult to compute because it requires too
%much time or space resources.
%%
%On the contrary, in the case of undecidable problems, it is sometimes
%possible to design algorithms that can work well in practice for
%\emph{some} instances but do not guarantee to compute a solution for
%all input combinations or even terminate in general.
%%
%Research challenges regarding the theoretical limitations of
%algorithms consist of identifying %
%(i) undecidable problems and %
%(ii) classes of systems and specifications for which the verification
%problem is decidable and designing efficient algorithms for these.
%%
%
%% The verification problem is known to be undecidable in general, but
%% decidable for the subclass of parameterized systems which are well
%% quasi-ordered
%% (WQO)~\cite{Parosh:Bengt:Karlis:Tsay:general,abdulla:well}.
%The problem of parameterized verification is well-known to be
%undecidable in general~\cite{Parosh:Bengt:Karlis:Tsay:general} %
%but for some subclass of systems (see
%Chapters~\ref{chapter:monotonic:abstraction}
%to~\ref{chapter:shape:analysis}), the verification problem becomes
%decidable, yet complex, and we present efficient algorithms to solve
%it.
%% chapter~\ref{chapter:monotonic:abstraction},
%% \ref{chapter:view:abstraction} and~\ref{chapter:shape:analysis}.
%%
%
%We would like to pinpoint that this thesis is \emph{not} about
%theoretical results on decidability of some class of programs.
%%
%Rather than providing computational bounds on the time and/or amount
%of space in memory the verification algorithms require or comparing
%which algorithm is best, the thesis focuses %instead
%%on whether the algorithms are practical. %
%on finding the suitable abstractions that make the algorithms
%\emph{practical}, i.e.\ terminate in a reasonable amount of time.
%% In other words, do they compute the answers in a reasonable amount of
%% time?
%We shall not bother much about the amount of space they require,
%because we can assume that %, as long as it stays within reason,
%it is always possible to extend the capacity of the machine and re-run
%the methods. %
%
%In the coming chapters, we will describe the different techniques to
%make the algorithms usable in practice. This was the challenge and we
%present the results in this thesis.
%%
%\begin{statement}
%  \it%
%  This focus of this thesis is on designing efficient algorithms\\%
%  for the intractable problem of parameterized verification\\%
%  for a specific class of programs.
%\end{statement}
%
%% There are numerous methods targeting primarily WQO systems and proven
%% to be complete for them. %
%% Universally quantified transitions are not monotonic, and systems with
%% such transitions are not WQO. %
%% Some of these methods are however still sound for such systems, and
%% were successfully used to verify many of them, mostly because these
%% systems have an inductive and upward-closed invariant which is strong
%% enough to imply safety. %
%% The WQO specialized methods are bound to fail, if they cannot derive
%% such upward-closed invariants.
%% %
%% Refer to Section~\ref{section:related_work} for more details on
%% related works.
