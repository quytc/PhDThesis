\label{MS:lock:free:queue}
%
We show a version of the concurrent queue by Michael and
Scott~\cite{MS:QueueAlgorithms}.
%
The program represents a queue as a linked list from the node pointed
to by \prgcode{Head} to a node that is either pointed by
\prgcode{Tail} or by \prgcode{Tail}'s successor.
%
The global variable \prgcode{Head} always points to a dummy cell whose
successor, if any, stores the head of the queue.
%
In the absence of garbage collection, the program must handle the ABA
problem~\cite{ABA:1983,ABA:Wikipedia} where a thread mistakenly
assumes that a globally accessible pointer has not been changed since
it previously accessed that pointer.
%
Each pointer is therefore equipped with an additional \prgcode{age}
field, which is incremented whenever the pointer is assigned a new
value.

The queue can be accessed by an arbitrary number of threads, either by
an enqueue method \prgcode{enq(d)}, which inserts a cell containing
the data value \prgcode{d} at the tail, or by a dequeue method
\prgcode{deq(d)} which returns \prgcode{empty} if the queue is empty,
and otherwise advances \prgcode{Head}, deallocates the previous dummy
cell and returns the data value stored in the new dummy cell.
%
The algorithm uses the atomic compare-and-swap (CAS) operation.
%
For example, the command %
\prgcode{CAS(\&Head, head, $\langle$next.ptr,head.age+1$\rangle$)} at
line~\ref{ms:code:line:deq} of the \prgcode{deq} method checks whether
the extended pointer \prgcode{Head} equals the extended pointer
\prgcode{head} (meaning that both fields \prgcode{ptr} and
\prgcode{age} must agree).
%
If not, it returns \prgcode{FALSE}. Otherwise it returns
\prgcode{TRUE} after assigning %
\prgcode{$\langle$next.ptr, head.age+1$\rangle$} to \prgcode{Head}.

The linearization points~\commitpoint{} %
are at line~\ref{ms:code:line:enq}, \ref{ms:code:line:empty} and \ref{ms:code:line:deq}.
%
For instance, line~\ref{ms:code:line:enq} of the \prgcode{enq} method
called with data value \prgcode{d} is instrumented to generate the
abstract event \s{\prgcode{enq(d)}} when the \prgcode{CAS} command
succeeds;
%
no abstract event is generated when the \prgcode{CAS} fails.
%
Generation of abstract events can be conditional.
%
For instance, line~\ref{ms:code:line:empty} of the \prgcode{deq}
method is instrumented to generate \s{\prgcode{deq(empty)}} when the
value assigned to \prgcode{next} satisfies \prgcode{next.ptr = NULL}
(i.e.\ it will cause the method to return \prgcode{empty} at
line~\ref{ms:code:line:emptyreturn}).

{\noindent\centering
\begin{tikzpicture}
        
  \node[codeblock] (init) at (current bounding box.north west) {\begingroup\scriptsize\VerbatimInput[numbers=none]{experiments/code/ms/init}\endgroup};
  \node[codeblock] (enq) at (init.south west) {\begingroup\scriptsize\VerbatimInput{experiments/code/ms/enq}\endgroup};
  \node[codeblock] (struct) at (init.north east) {\begingroup\scriptsize\VerbatimInput[numbers=none]{experiments/code/ms/struct}\endgroup};
  \node[codeblock] (deq) at (struct.south west) {\begingroup\scriptsize\VerbatimInput[firstnumber=17]{experiments/code/ms/deq}\endgroup};

  \node[property,rotate=-10,scale=0.7] at ([shift={(-0.5,-0.2)}]init.north east) {\sc Init};
  \node[property,rotate=-10,scale=0.7] at ([shift={(-0.5,-0.2)}]enq.north east) {\sc Enq};
  \node[property,rotate=-10,scale=0.7] at ([shift={(-0.5,-0.2)}]deq.north east) {\sc Deq};

\end{tikzpicture}%
\par%
}
