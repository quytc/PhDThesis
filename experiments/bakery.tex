%
\begin{wrapfigure}{r}[0pt]{0.3\linewidth}
\hfill%
\begin{tikzpicture}[%
  state/.append style={minimum width=6mm}%
  ]

  \node[state,state-i] (n1) at (0,0) {1};
  \node[state,state-n] (n2) at (1,1) {2};
  \node[state,state-b] (n3) at (2,0) {3};

   \draw [->,myedge] (n1) to[out=90,in=180] node[trlabel,above,sloped]{$\forall j<i : \set{1}$} (n2);
   \draw [->,myedge] (n2) to[out=0,in=90] node[trlabel,above,sloped]{$\forall j>i : \set{1}$} (n3);
   \draw [->,myedge] (n3) -- (n1);

\end{tikzpicture}
%\def\initstate{\tikz[baseline=(n.base)]\node[state,fill=green!20,scale=0.7](n){1};}
%\caption{A pseudocode of the $i$th process of
%  Bakery's protocol and the corresponding transition rules (in the form of a transition diagram). 
%  Initially, all processes are in state {\protect\initstate}.
%}
%\caption{Transition rules of Bakery's protocol.}
\end{wrapfigure}
%
This case study describes a simplified version of the original Bakery
algorithm~\cite{Lamport:Bakery}.
%
In this version~\cite{Marcus:thesis}, processes have states that range
over~$\set{\w[w]{1},\w[w]{2},\w[w]{3}}$, where \w[i]{1} is the initial
state.
%
A process gets a ticket with a value strictly higher than the ticket
value of any process in the queue (transition
\w[w]{1}$\rightarrow$\w[w]{2}).
%
A process accesses the critical section if it has a ticket with the
lowest value among the existing tickets (transition
\w[w]{2}$\rightarrow$\w[w]{3}).
%
Finally, a process leaves the critical section, freeing its ticket
(transition \w[w]{3}$\rightarrow$\w[w]{1}).
%
Mutual exclusion violation corresponds to configurations where more
than one process is in state~\w[b]{3}.
