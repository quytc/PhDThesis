%
\subsection{Percolate Protocol}

The protocol~\cite{KMMPS2001} operates on binary trees of processes
and evaluates the disjunction of the values in the leaves up to the
root. Initially, the leaves contain either a $0$ or a $1$ and all
other nodes are labeled as undefined $u$. An still undefined inner
node will be labeled as $1$ if at least one of its children contains a
$1$, and $0$ otherwise. As an example:

\begin{center}
\begin{tikzpicture}[level distance=1cm]
\foreach \x in {0,2.5,5}{
\draw[shift={(\x,0)},color=white,fill=treebg,rounded corners=8pt]
      (-1.24,0.3) rectangle (1.24,-2.3);
}
\tikzstyle{level 1}=[sibling distance=12mm]
\tikzstyle{level 2}=[sibling distance=6mm]
\node {$u$}
  child {node {$u$}
         child {node {$0$}}
         child {node {$0$}}}
  child {node {$u$}
         child {node {$1$}}
         child {node {$1$}}
};
\node[shift={(2.5,0)}] {$u$}
  child {node {$0$}
         child {node {$0$}}
         child {node {$0$}}}
  child {node {$1$}
         child {node {$1$}}
         child {node {$1$}}
};
\node[shift={(5,0)}] {$1$}
  child {node {$0$}
         child {node {$0$}}
         child {node {$0$}}}
  child {node {$1$}
         child {node {$1$}}
         child {node {$1$}}
};
\foreach \x in {0,2.5}{
\draw[shift={(\x,0)},->] (1.1,0) ..controls (1.3,0.1) .. (1.5,0);
}
\end{tikzpicture}
\end{center}

The corresponding rules are:

\begin{center}
\begin{tikzpicture}[show background rectangle]
\tikzstyle{level 1}=[sibling distance=10mm,level distance=1cm]
\tikzstyle{every node}=[draw,rounded corners]
\node {$u$ / $0$} child {node {$0$}} child {node{$0$}};
\end{tikzpicture}
\begin{tikzpicture}[show background rectangle]  
\tikzstyle{level 1}=[sibling distance=10mm,level distance=1cm]
\tikzstyle{every node}=[draw,rounded corners]
\node {$u$ / $1$} child {node {$1$}} child {node{$0$}};
\end{tikzpicture}
\begin{tikzpicture}[show background rectangle]  
\tikzstyle{level 1}=[sibling distance=10mm,level distance=1cm]
\tikzstyle{every node}=[draw,rounded corners]
\node {$u$ / $1$} child {node {$0$}} child {node{$1$}};
\end{tikzpicture}
\begin{tikzpicture}[show background rectangle]  
\tikzstyle{level 1}=[sibling distance=10mm,level distance=1cm]
\tikzstyle{every node}=[draw,rounded corners]
\node {$u$ / $1$} child {node {$1$}} child {node{$1$}};
\end{tikzpicture}
\end{center}

The set of bad constraints $\fconfs$ is represented by trees where a
$0$ get propagated upwards while there is a $1$ below.

\begin{center}
\begin{tikzpicture}[show background rectangle]
\tikzstyle{level 1}=[sibling distance=10mm,level distance=1cm]
\tikzstyle{every node}=[draw,rounded corners]
\node {$0$} child {node {$1$}} child {edge from parent[draw=none]};
\end{tikzpicture}
\begin{tikzpicture}[show background rectangle]  
\tikzstyle{level 1}=[sibling distance=10mm,level distance=1cm]
\tikzstyle{every node}=[draw,rounded corners]   
\node {$0$} child {edge from parent[draw=none]} child {node {$1$}};
\end{tikzpicture}
\end{center}

%We have been able to prove $\prestar{\fconfs} \intersect \init = \emptyset$.
We have been able to prove $\init$ is not backward reachable from
$\final$ (i.e.\ that $\init\transrel{*}{}\final$ does not hold).
