A barrier is a tool to ensure synchronization between threads and
allows a programmer to impose restriction for an asynchronous
multi-threaded program.
%
A process arriving at the barrier must stop at this point and cannot
proceed until all other processes reach this barrier.

\begin{wrapfigure}{r}[0pt]{0.30\linewidth}
  \hfill%
  \begin{tikzpicture}
    \node[draw,shade,rotate=90,anchor=center,inner xsep=15mm](barrier) at (0,0){Barrier};
    \foreach \d/\a/\n in {25mm/10/A,13mm/30/B,20mm/150/C,10mm/170/D}{
      \node[circle,draw,shade,left=\d of barrier.\a](\n-1){\n};
      \draw[-latex,dashed,gray] (\n-1) -- (barrier.\a);
    }
    \draw[latex-] (0,-2.3) -- +(-25mm,0) node[fill=white,midway]{time};      
  \end{tikzpicture}
\end{wrapfigure}
%
The following model represents a barrier with a \emph{pivot}. The
first thread at the barrier will take the role of a pivot and all
other processes wait as long as there is a pivot. When all processes
have arrived at the barrier, the pivot can then proceed, which in turn
releases all the waiting processes.

A process can be in state \emph{before}~\w[w]{B},
\emph{wait}~\w[w]{W}, \emph{pivot}~\w[w]{P} or
\emph{after}~\w[w]{A}. %
%
Initially, all processes are in state~\w[i]{B} and a bad configuration
is detected when there is a process in state~\w[w]{A}, while there is
still a process in state~\w[w]{B}.
%
The transition diagram and inhibitor net are as follows.

\begin{wrapfigure}{l}{0.55\linewidth}
\begin{tikzpicture}[%
  show background rectangle,
  transition/.append style={minimum width=6mm},
  mylabel/.append style={scale=0.7},
  every place/.append style={scale=0.7}
  ]

  \draw[white,thick] (-1.1,0) -- +(0,-2);

  %% Diagram
  \begin{scope}[shift={(-2.5,0)}]

    \node[state,state-i](before) at (0,0) {B};
    \node[state,state-n](wait) at (-1,-1) {W};
    \node[state,state-n](pivot) at (1,-1) {P};
    \node[state,state-n](after) at (0,-2) {A};

    \draw[myedge] (before) -- node[mylabel,midway,above left]{$\exists~P$} (wait);
    \draw[myedge] (before) -- node[mylabel,midway,above right]{$\forall~B$} (pivot);

    \draw[myedge] (wait) -- node[mylabel,midway,below left]{$\not\exists~P$} (after);
    \draw[myedge] (pivot) -- node[mylabel,midway,below right]{$\forall~W$} (after);

    %\node[caption] at (0,-1.3) {Arriving at the dock};
  \end{scope}

  %% Petri Net
  \begin{scope}%[xshift=4]

    \node(b)[state,place] at (0,0)  {};
    \node(w)[state,place] at (0,-2) {};
    \node(p)[state,place] at (1,-1) {};
    \node(a)[state,place] at (2,-3) {};

    \node[scale=0.5,left=1pt of b]{B};
    \node[scale=0.5,left=1pt of w] {W};
    \node[scale=0.5,right=1pt of p]{P};
    \node[scale=0.5,right=1pt of a]{A};

    \node[transition](t1) at (0,-1){};
    \draw[myedge,<-] (t1.135) to[out=90,in=-135] (b);
    \draw[myedge,->] (t1.-135) to[out=-90,in=90] (w);
    \draw[myedge,<-] (t1.35) ..controls +(up:6mm) and +(135:6mm).. (p);
    \draw[myedge,->] (t1.-35) ..controls +(down:6mm) and +(-135:6mm).. (p);

    \node[transition](t2) at (1,-2.5){};
    \draw[myedge,<-] (t2.150) ..controls +(up:5mm) and +(right:6mm).. (w);
    \draw[myedge,*-] (t2.30) to[out=90,in=-90] (p);
    \draw[myedge,->] (t2.-90) to[out=-90,in=180] (a);

    \node[transition,rotate=90](t3) at (1.5,0.5){};
    \draw[myedge,<-] (t3.90) to[out=180,in=45] (b);
    \draw[myedge,*-] (t3.160) to[out=180,in=90] (p);
    \draw[myedge,*-] (t3.20) ..controls +(left:25mm) and +(120:25mm).. (w);
    \draw[myedge,->] (t3.-90) to[out=0,in=45] (p);

    \node[transition](t4) at (2,-2){};
    \draw[myedge,*-] (t4.30) to[out=90,in=-10] (b);
    \draw[myedge,<-] (t4.150) to[out=90,in=-45] (p);
    \draw[myedge,->] (t4) to[out=-90,in=90] (a);

    %\node[caption] at (0,-3.3) {Boarding};
  \end{scope}

\end{tikzpicture}
\end{wrapfigure}
%
The following invariant can be derived: %
(i) a pivot~$P$ can coexist with a process in state~\w[w]{B}
and/or \w[w]{W}, or is alone, i.e.\ the set of configurations
$\set{\w[w]{P}\wedge\w[w]{B}^*\wedge\w[w]{W}^*}$, %
(ii) a process after the barrier can only coexist with other process
after the barrier or still waiting, i.e.\ the set of configurations
$\set{\w[w]{W}^*\wedge\w[w]{A}^+}$, %
and (iii) initially, all processes were before the barrier, i.e.\
$\set{\w[w]{B}^*}$.

It is also possible to model it with a linear topology (as
in~\ref{paper:SAS14}).
% Section~\ref{section:introduction}~and~\ref{chapter:parameterized:systems}. %
In that case, we can consider a non-atomic version, which would not
check the state of the other processes all at once.

A typically implementation uses an atomic counter as follows.
\lstinputlisting[style=custom]{experiments/code/barrier.txt}
