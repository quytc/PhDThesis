%
Interacting peers are organized in a circular pipeline and are given a
number. The protocol in place is to ensure that every participant in
the ring knows which number in the maximum among the values of the
ring members. We model this protocol as an instance of the framework
on rings. Each participant in the ring can communicate with its
adjacent neighbors. We assume the ring oriented and a member can send
messages to its (immediate) ``successor'' and receive messages from
its ``predecessor''. The reception is usually \emph{blocking} while
the sending is not. However, we will model this communication as a
rendez-vous.

\noindent%
\begin{wrapfigure}{r}[0pt]{0.25\linewidth}
  \hfill%
  \begin{tikzpicture}
    [scale=0.8,every node/.style={scale=0.8,circle,inner sep=1pt,fill=red!20}, show background rectangle] 

    \draw (0,0) .. controls +(up:15mm) and +(up:15mm) .. (3,0)
    \foreach \n/\pos in {1/0.2,2/0.5,3/0.9}{node[state,pos=\pos](p\n){P$_{\n}$}};
    \draw (0,0) .. controls +(down:15mm) and +(down:15mm) .. (3,0)
    \foreach \n/\pos in {0/0.1,5/0.5,4/0.8}{node[state,pos=\pos](p\n){P$_{\n}$}};
    \draw[->,myedge,thick] ([xshift=1mm,yshift=1mm]p2.east) to[out=-5,in=120] ([xshift=1mm,yshift=1mm]p3.north west);
  \end{tikzpicture}
\end{wrapfigure}
%
Initially, all process are in a \emph{dormant} state. One of the
process will wake up first. This is the one which initializes the
protocol (denoted in the pseudocode as $P_0$). Every process sends the
max between its own value and the value its received from its
predecessor (the biggest value it has seen so far). Note that the
protocol doesn't not terminate when $P_0$ receives the biggest value
in the ring. It must indeed communicate this value to the others. Each
will receive it from its predecessor and only pass it along to its
successor (after recording it). The protocol terminates when the
predecessor of $P_0$ receives the biggest value and avoids the
re-sending. A bad configuration is detected if one of the participant
is in its final state (\s{6} or \s{17}) but has not seen the
biggest value go by. %
We depict the protocol using the following pseudocode. The channel
between $P_{i-1}$ and $P_i$ is called {\tt values[n]}. A message in
the channel will contain the number the sender sent.

\bigskip
\lstinputlisting[style=custom]{experiments/code/agreement.txt}
