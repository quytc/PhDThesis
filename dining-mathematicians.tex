\paragraph{Example:}

\begin{center}
\setlength{\unitlength}{0.5cm}
\begin{picture}(26,17)
%
\put(0,15.6){\makebox(0,0)[bl]{{\bf Action System} {\it Dining Mathematicians}}}
\put(1,14.3){\makebox(0,0)[bl]{{\bf declare} $n : \it integer$}}
\put(1,13){\makebox(0,0)[bl]{{\bf initially} $n \geq 1$}}
\put(0,1){\makebox(0,0)[tl]{\bf end}}
%
\put(2,1.5){
\begin{picture}(26,11)
%
\put(-2,0){
\begin{picture}(10,11)
\put(6,2){\circle{3}}
\put(6,2){\makebox(0,0){$eat_1$}}
\put(6,9){\circle{3}}
\put(6,9){\makebox(0,0){$think_1$}}
%
\put(4.94,3.06){\vector(0,1){4.88}}
\put(7.06,7.94){\vector(0,-1){4.88}}
\put(4.8,5){\makebox(0,0)[r]{$n := 3n + 1$}}
\put(7.2,5){\makebox(0,0)[l]{$odd(n)$}}
\put(4,11){\vector(1,-1){0.94}}
\end{picture}
}
\put(11.1,0){\line(0,1){8}}
\put(10.9,0){\line(0,1){8}}
%
%
\put(11,0){
\begin{picture}(9,11)
\put(6,2){\circle{3}}
\put(6,2){\makebox(0,0){$eat_2$}}
\put(6,9){\circle{3}}
\put(6,9){\makebox(0,0){$think_2$}}
%
\put(4.94,3.06){\vector(0,1){4.88}}
\put(7.06,7.94){\vector(0,-1){4.88}}
\put(4.8,5){\makebox(0,0)[r]{$n := n/2$}}
\put(7.2,5){\makebox(0,0)[l]{$even(n)$}}
\put(4,11){\vector(1,-1){0.94}}
\end{picture}
}
\end{picture}
}
\end{picture}
\end{center}

The above example consists of two processes, which share an integer variable
$n$. The system has an infinite number of initial states, and the number of
states is therefore infinite. Suppose we want to check that the two
mathematicians are not eating at the same time, i.e., we want to
check whether
\[
\neg(\at{eat_1} \land \at{eat_2})
\]
is invariant. Simple inspection shows that this invariant indeed holds
for the system, for the simple reason that
when the left process is in $eat_1$, then $n$ is odd, and
when the right process is in $eat_2$, then $n$ is even.
However, this simple fact cannot be detected by naive reachability analysis,
since the system has infinitely many states.
