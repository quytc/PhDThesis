
This doctoral thesis considers the automatic verification of
\emph{parameterized systems}, i.e.\ systems with an arbitrary number
of communicating components, such as mutual exclusion protocols, cache
%coherence protocols or heap manipulating programs. %
%%
%The components may be organized in various topologies such as words,
%multisets, rings, or trees.
%
%The task is to show correctness regardless of the size of the system
%and we consider two methods to prove safety:
%%
%(i) a backward reachability analysis, using the well-quasi ordered
%framework and monotonic abstraction, and
%%
%(ii) a forward analysis which only needs to inspect a small number of
%components in order to show correctness of the whole system. The
%latter relies on an abstraction function that views the system from
%the perspective of a fixed number of components. The abstraction is
%used during the verification procedure in order to dynamically detect
%cut-off points beyond which the search of the state-space need not
%continue.
% 
%Our experimentation on a variety of benchmarks demonstrate that the
%method is highly efficient and that it works well even for classes of
%systems with undecidable property.
%%
%It has been, for example, successfully applied to verify a
%fine-grained model of Szymanski's mutual exclusion protocol.
%%
%Finally, we applied the methods to solve the complex problem of
%verifying highly concurrent data-structures, in a challenging setting:
%We do not a priori bound the number of threads, the size of the
%data-structure, the domain of the data to store nor do we require the
%presence of a garbage collector.
%%
%We successfully verified the concurrent Treiber's stack and
%Michael\&Scott's queue, in the aforementioned setting.
%
%To the best of our knowledge, these verification problems have been
%considered challenging in the parameterized verification community and
%could not be carried out automatically by other existing methods.
