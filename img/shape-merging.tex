\begin{tikzpicture}[framed]

  %% ============ Shape 1 ======================
  \begin{scope}[scale=0.7,every node/.append style={scale=0.7}]

    \foreach \n/\x/\y/\c in {
      1/0.8/0/white,
      2/1.3/-2.5/red,
      3/2.3/-1/white}{
      
      \node(m\n)[minimum width=12mm,minimum height=6mm,inner sep=0pt,
      fill=blue!20, draw=blue!50, very thick, rounded corners=4pt] at (\x,\y) {};
      \node[data=\c,left=1mm of m\n.center]{}; 
      \coordinate[right=2mm of m\n.center](m\n-n){};
      \draw[blue!50, very thick] (m\n.north) -- (m\n.south);
    }
    
    \node[globals](null) at (1.5,-4.5){\#};

    \draw[pointsto,shorten <=-2pt,dashed] (m1-n) .. controls +(-60:20mm) and +(120:20mm) .. (m2);
    \draw[pointsto,shorten <=-2pt] (m2-n) .. controls +(-60:10mm) and +(120:10mm) .. (null);
    \draw[pointsto,shorten <=-2pt] (m3-n) .. controls +(-80:10mm) and +(60:10mm) .. (m2);
 
    %% ============ Variables ======================

    \node[globals,left=1pt of m1.west] (T) {Top};
    \node[thread1,right=1pt of m1.east] (t1) {t};
    \node[thread1,left=1pt of m2.west] (x1) {x};
    \node[thread2,right=1pt of m2.east] (t2) {t};
    \node[thread2,right=1pt of m3.east] (n2) {n};

    %% ============ Threads info ======================
    \node[fill=t1color,rectangle,inner xsep=2pt] (thread1) at (0,1) {Thread 1 {\tiny (pc:~\ref{treiber:thread:pop})}};
    \node[fill=t2color,rectangle,inner xsep=2pt,right=2pt of thread1] (thread2) {Thread 2 {\tiny (pc:~\ref{treiber:thread:push})}};
    
  \end{scope} %% End shape 1

  %% ============ Shape 2 ======================
  \begin{scope}[shift={(3.8,0)},scale=0.7,every node/.append style={scale=0.7}]


    \foreach \n/\x/\y/\c in {
      1/0.8/0/white,
      2/1.3/-2.5/red,
      3/2.3/-1/white}{
      
      \node(m\n)[minimum width=12mm,minimum height=6mm,inner sep=0pt,
      fill=blue!20, draw=blue!50, very thick, rounded corners=4pt] at (\x,\y) {};
      \node[data=\c,left=1mm of m\n.center]{}; 
      \coordinate[right=2mm of m\n.center](m\n-n){};
      \draw[blue!50, very thick] (m\n.north) -- (m\n.south);
    }
    
    \node[globals](null) at (2,-4.5){\#};

    \draw[pointsto,shorten <=-2pt] (m1-n) .. controls +(-60:20mm) and +(120:20mm) .. (m2);
    \draw[pointsto,shorten <=-2pt,dashed] (m2-n) .. controls +(-60:10mm) and +(120:10mm) .. (null);
    \draw[pointsto,shorten <=-2pt] (m3-n) .. controls +(-80:10mm) and +(60:10mm) .. (m2);
 
    %% ============ Variables ======================

    \node[globals,left=1pt of m1.west] (T) {Top};
    \node[thread1,right=1pt of m1.east] (t1) {t};
    \node[thread1,left=1pt of m2.west] (x1) {x};
    \node[thread2,right=1pt of m2.east] (t2) {t};
    \node[thread2,right=1pt of m3.east] (n2) {n};

    %% ============ Threads info ======================
    \node[fill=t1color,rectangle,inner xsep=2pt] (thread1) at (0,1) {Thread 1 {\tiny (pc:~\ref{treiber:thread:pop})}};
    \node[fill=t2color,rectangle,inner xsep=2pt,right=2pt of thread1] (thread2) {Thread 2 {\tiny (pc:~\ref{treiber:thread:push})}};

      
  \end{scope} %% End shape 2


  %% ============== Merging ===========================
  \begin{scope}[shift={(6.6,0.7)},scale=0.7,every node/.append style={scale=0.7}]

    \arraycolsep=0pt%\def\arraystretch{0}
    \newcommand\mix[2]{\begin{array}{c}{#1}\\[-6pt]{#2}\end{array}}

    \matrix[matrix of math nodes, below right, inner sep=0pt, nodes={var,inner sep=0pt,anchor=center,minimum size=20pt}, row sep=0pt,column sep=0pt] (encoding) {
      & |[globals]|\text{Top}
      & |[globals]|\text{\#}
      & |[thread1]|\text{t}
      & |[thread1]|\text{x}
      & |[thread2]|\text{t}
      & |[thread2]|\text{n}
      & |[data=red,minimum size=1em]|
      \\
      |[globals]|\text{Top}
      & =          & \reaches  & =           & \mix{\pointsto}{\reaches}    & \mix{\pointsto}{\reaches}    & \unrelated & \mix{\pointsto}{\reaches}    \\
      |[globals]|\text{\#} 
      & \reachedby & =         & \reachedby  & \mix{\reachedby}{\pointedby} & \mix{\reachedby}{\pointedby} & \reachedby & \mix{\reachedby}{\pointedby} \\
      |[thread1]|\text{t} 
      & =          & \reaches  & =           & \mix{\pointsto}{\reaches}    & \mix{\pointsto}{\reaches}    & \unrelated & \mix{\pointsto}{\reaches}    \\
      |[thread1]|\text{x} 
      & \mix{\reachedby}{\pointedby} & \mix{\pointsto}{\reaches}  & \mix{\reachedby}{\pointedby}  & =          & =          & \pointedby & =          \\
      |[thread2]|\text{t} 
      & \mix{\reachedby}{\pointedby} & \mix{\pointsto}{\reaches}  & \mix{\reachedby}{\pointedby}  & =          & =          & \pointedby & =          \\
      |[thread2]|\text{n}
      & \unrelated & \reaches  & \unrelated  & \pointsto  & \pointsto  & =          & \pointsto  \\
      |[data=red,minimum size=1em]|
      & \mix{\reachedby}{\pointedby} & \mix{\pointsto}{\reaches}  & \mix{\reachedby}{\pointedby}  & =          & =          & \pointedby & =          \\
    };

  \draw (encoding-2-2.north west) -- (encoding-2-8.north east);
  \draw (encoding-2-2.north west) -- (encoding-8-2.south west);
  \foreach \n in {2,...,8}{
    \draw (encoding-\n-2.south west) -- (encoding-\n-8.south east);
    \draw (encoding-2-\n.north east) -- (encoding-8-\n.south east);
  }

  %% ============ Pi ======================
  \node[inner sep=0pt,below=8pt of encoding.south,scale=0.8,xshift=14pt] (pi){$\pi[t_2,t_1]$};
  \node[right=0pt of pi.east]{: disjuntion};
  \begin{pgfonlayer}{my background}
    \begin{scope}
      [bubble/.style={fill=purple!30!white}]%opacity=.3,transparency group % Note: transparency doesn't work with xetex
      \coordinate(c1) at ([shift={(-2pt,-2pt)}]encoding-6-4.center); 
      \coordinate(c2) at ([shift={(2pt,2pt)}]encoding-6-4.center); 
      \node [bubble,circle,fit=(c1)(c2)] (pit2t1){};
      \node [bubble,ellipse,fit=(pi)] (pit2t1'){};
      
      \fill[bubble]
      (pit2t1.east) .. controls +(down:3mm) and +(190:3mm) .. (pit2t1'.100) --
      (pit2t1'.150) .. controls +(60:3mm) and +(right:2mm) ..  (pit2t1.south) -- cycle;
    \end{scope}
  \end{pgfonlayer}


  \end{scope} %% End abstraction

  %% ============ White separations ======================
  %\draw[white,ultra thick] (3.2,1) -- ++(0,-4.7) -- +(-3,0) -- +(3,0);
  \draw[white,ultra thick] (2.8,1) -- +(0,-4.7);
  \draw[white,ultra thick] (6.5,1) -- +(0,-4.7);


\end{tikzpicture}
