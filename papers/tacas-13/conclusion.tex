%--------------------------------------------------------
\section{Conclusions and Future Work}
\label{sec:conclusion}
%--------------------------------------------------------

We have presented  a technique for automated verification of temporal
properties of concurrent programs, which can handle the challenges
of infinite-state specifications, an unbounded number of threads, and
an unbounded heap managed by explicit memory allocation.
We showed how such a technique
can be based naturally on the automata-theoretic approach to verification,
by nontrivial combinations and extensions that handle unbounded
data domains, unbounded number of threads, and heaps of arbitrary size. 
The result is a simple and direct method for verifying
correctness of concurrent programs.
%% For instance,
%% our approach to value abstraction is significantly simpler than that
%% employed by Vafeiadis~\cite{Vafeiadis:vmcai09}.
The power of our specification
formalism is enhanced by showing how
the data-independence argument by Wolper~\cite{Wolper:dataindependence} can
be introduced into standard program analysis. 
%
Our method can be parameterized by different shape analyses.
Although we concentrate on heaps with single selectors in the current
paper, we expect that our method can be adapted to deal with multiple
selectors, by integrating recent approaches such as~\cite{habermehl:forest}. %
Morever, our experminatation deals with the specification of stacks and queues.
Other data structures, such as deques, can be handled in an analogous way.
%% The presented analysis uses a straightforward DNF representation of constraints.
%% There is room for optimization, e.g., by defining a condensed form where
%% constraints are joined to decrease the size of formulas.
%% Finally, we expect that our approach could be employed analogously
%% to other classes
%% of programs, by a suitable adaption of the analysis method.

