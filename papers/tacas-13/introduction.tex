\section{Introduction}
\label{sec:introduction}
%
In this paper, we consider one of the most difficult current challenges in
software verification, namely to automate its
application to algorithms with an unbounded number of
threads that concurrently access a dynamically allocated shared state.
Such algorithms are of central importance in concurrent programs\TODO{F: Since they aim at removing unnecessary waiting time and lock contention.}. They
are widely used in libraries,
\TODO{\cite{Cambridge:lock-free}~BJ: What is this reference doing here? AR: it might have been me that added it a long time ago. Now, I also think it is better to remove it. }
such as the Intel Threading Building Blocks or
the \url{java.util.concurrent}  package,
to provide efficient concurrent realizations of
simple interface abstractions.
They are notoriously difficult to get correct and verify, since they
often employ fine-grained synchronization and avoid locking wherever
possible. A number of bugs  in published
algorithms have been reported~\cite{DDGJLMMSS:dcas,MiSc:correction}.
It is therefore important to develop efficient techniques for
verifying conformance to simple abstract specifications of overall
functionality, a concurrent implementation of a common
data type abstraction, such as a queue, should be verified to conform
to a simple abstract specification of a (sequential) queue.

%% It is well understood how to
%% accomplish this for finite-state programs (as embodied, e.g., in
%% the SPIN tool~\cite{Holzmann:spin}),
%% but we lack approaches for handling unbounded data domains in specifications
%% and implementations in connection with an unbounded number of threads and
%% dynamically allocated data structures.

We present an integrated technique for specifying and automatically
verifying that a concurrent program conforms to an abstract
specification of its functionality.
%% We can, for instance, specify and verify that a concurrent program is a
%% linearizable implementation of a FIFO queue.
Our starting point is the
automata-theoretic approach to model checking~\cite{VW:modelchecking}, in which
programs are specified by automata that accept precisely those
executions that violate the intended specification, and
verified by showing that these automata never accept when they are
composed with the program.
This approach is one of the most successful approaches to automated
verification of finite-state programs, but is still insufficiently developed
for infinite-state programs. In order to use this approach for our purposes,
we must address a number of challenges.
%
\begin{enumerate}
\item
The abstract specification is infinite-state, 
because the implemented data structure may 
contain an unbounded number of data values from
an infinite domain.
\item
The program is infinite-state in several dimensions: it
(i) consists of an unbounded number of concurrent threads, 
(ii) uses unbounded dynamically allocated memory, and 
(iii) the domain of data values is unbounded.
%% \item
%% the program uses dynamically allocated memory, and
\item
The program does not rely on automatic garbage collection, but manages
memory explicitly. This requires additional mechanisms to avoid
the ABA problem, i.e., that a thread mistakenly confuses an outdated
pointer with a valid one.
\end{enumerate}
%
Each of these challenges requires a significant advancement over current
specification and verification techniques.

We cope with challenge 1 by combining two ideas. First,
we present a novel technique for specifying
programs by a class of automata, called {\em observers}.
They extend automata, as used by~\cite{VW:modelchecking},
by being parameterized on
a finite set of variables that assume values from an unbounded domain.
This allows to specify properties that should hold for an infinite number of
data values.
%% They observe the execution of a program,
%% and are designed to accept precisely those executions that violate the
%% intended specification, similarly as in 
%% the automata-theoretic approach to model checking~\cite{VW:modelchecking}.
In order to use our observers to specify queues, stacks, etc., where one
must ``count'' the number of copies of a data value that have been inserted but
not removed, we must extend the power of observers by a second idea.
This is a data independence argument, adapted  from
Wolper~\cite{Wolper:dataindependence}, which 
implies that it is sufficient to consider executions in which
any data value is inserted at most once.
This allows us to succinctly specify data structures such as queues and 
stacks, using observers with typically less than $3$ variables.
%% For safety properties, this reduces to checking that the automaton cannot
%% reach an accepting state. This approach is implemented, e.g., in SPIN,
%% where the automata are called never-claims.
%% To specify a program, one first instruments it by the generation of
%% {\em abstract events} to determine what an observing automaton will see from
%% a program execution. The automaton (we call it an {\em observer}) then
%% observes the sequence of abstract events and can move to an
%% accepting location whenever the specified property is violated (in this
%% paper, we consider only safety properties).
%% The variables of the observer can initially assume any values in the data
%% domain. This allows observers to specify properties that are universally
%% quantified over data values: by checking whether there are some
%% values of the variables that violate the property.

To cope with challenge 2(i), we would like to adapt the successful
thread-modular approach~\cite{BLMRS:cav08},
which verifies a concurrent program by generating an invariant
that correlates the global state with the local state of an arbitrary thread.
However, to cope with challenge 3, the generated invariant must be able
to express that {\em at most} one thread accesses some
cell on the global heap. Since this cannot be expressed in the
thread-modular approach, we therefore extend it to
generate
invariants that correlate the global state with the local states of an
arbitrary {\em pair} of threads.

To cope with challenge 2(ii) we need to use shape analysis. %
We adapt a variant of the transitive closure logic by Bingham and
Rakamari\'c~\cite{BiRa:vmcai06} for reasoning about heaps with single
selectors, to our framework. This formalism tracks reachability
properties between pairs of pointer variables, and we adapt it to our
analysis, in which pairs of threads are correlated. %
On top of this, we have developed a novel optimization, based on the
observation that it suffices to track the possible relations between
each pair of pointer variables separately, analogously to the use of
DBMs used in reasoning about timed automata~\cite{Dill:DBM}.
%% This optimization makes our analysis efficient without sacrificing
%% essential information.
Finally, we cope with challenge 2(iii) by first observing that data values
are compared only by equalities or inequalities, and then employing
suitable standard abstractions on the concerned data domains.

%% The above techniques bring the needed precision for verification,
%% at the price of significant state-space explosion, which mainly arises
%% from reasoning about the dynamically allocated heap when pairs
%% of threads must be correlated.
%% We have therefore developed a novel optimization, built on the observation that
%% it is enough to consider pairs of pointer variables separately from each other.
%% This optimization tracks the possible relations between each pair of
%% pointer variables separately, vaguely analogous to the use of
%% DBMs used in reasoning about timed automata~\cite{Dill:DBM}.
%% It makes our analysis efficient without sacrificing essential information.

%% reasoning over an abstract domain which 
%% able to 
%% In this paper, we further present a novel verification technique that is
%% able to handle the combined challenges of  
%% unbounded data domains, an unbounded number of concurrent threads,
%% dynamically allocated memory, and explicit memory management.
%% The unbounded data domain is handled by the observer technique just
%% described. 
%% To cope with the unbounded number of concurrent threads, a natural
%% attempt would be to use a thread-modular analysis~\cite{BLMRS:cav08}, which
%% correlates the local variables of an arbitrary
%% single thread with the global variables, but does not correlate the
%% variables of different threads to each other.
%% However, a thread-modular approaches is not precise enough 
%% to verify programs with explicit memory management, not relying on a
%% garbage collector. 
%% The basis of our shape analysis is
%% The formalism is capable of capturing acyclic shape graphs where heap cells have
%% one pointer field (i.e., we consider singly linked list structures).%, similarly as \cite{Vafeiadis:RGSep}. 
%% \\
%% {\bf (* What does this limitation to ``acyclic'' really mean? *)}

We have implemented our technique, and
applied it to
specify and automatically
verify that a number of concurrent programs are linearizable
implementation of stacks and queues~\cite{HeWi:linearizability}.
This shows that our new contributions result in an integrated
technique that addresses the challenges 1 -- 3,
and can fully automatically verify a range of 
concurrent implementations of common data structures.
In particular, our approach advances the power of automated verification in
the following ways.
%
\begin{itemize}
\item
We present a direct approach for verifying that a concurrent program is a linearizable implementation of, e.g., a queue, which consists in checking a few
small properties of the algorithm, and is thus
suitable for automated verification.
Previous approaches typically verified linearizability separately from
conformance to a simple abstraction, most often
using simulation-based arguments, which are harder
to automate than simple property-checking.
\item
We can automatically verify concurrent programs that use
explicit memory management. This was previously beyond
the reach of automatic methods.
\end{itemize}
%
In addition, on examples that have been verified automatically by
previous approaches, our implementation is in many cases significantly faster.
%% The only manual intervention is the instrumentation by abstract events,
%% which should be done at linearization points.
%\\ {\bf (* Carefully check how our runtimes compare with various other papers.
%Maybe remove the last stuff about ``significantly faster''? *)}\\
%% In summary, the contributions of the paper include
%% \vspace*{-6pt}
%% \begin{itemize}
%% \item
%% A novel technique for automatically verifying
%% that a program conforms to an infinite-state specification, which can be used
%% to prove conformance to common data structure specifications.
%% \item
%% A verification technique, which is
%% able to handle the combined challenges of  
%% unbounded data domains, unbounded number of concurrent threads,
%% dynamically allocated memory, and explicit memory management.
%% \item
%% Experimental evaluation, which shows that our technique can fully
%% automatically verify a range of 
%% concurrent implementations of common data
%% structure implementations, such as queues, stacks, etc.
%% \end{itemize}

\paragraph{Overview} We give an overview of how our technique can be used
to show that a concurrent program is a linearizable
implementation of a data structure.
As described in Section~\ref{sec:concdata}, we consider concurrent programs
consisting of an arbitrary number of
sequential threads that access shared global variables and a shared heap using
a finite set of methods.
Linearizability provides the illusion that each
method invocation takes effect instantaneously at some point (called the
linearization point) between method invocation and
return~\cite{HeWi:linearizability}.
In Section~\ref{sec:observers}, we show how to specify this
correctness condition by first instrumenting each method to generate a
so-called abstract event whenever a linearization point is passed.
We also introduce {\em observers}, and show how to use them for specifying
properties of sequences of abstract events.
In Section~\ref{sec:wolper}, we introduce the data independence argument that
allows observers to specify queues, stacks, and other
unbounded data structures.
In Section~\ref{sec:annotations}, we describe our analysis for checking that
the cross-product of the program and the observer cannot reach an accepting
location of the observer. The analysis is based on a shape analysis, which
generates an invariant that correlates
the global state with the local states of an arbitrary pair of threads.
We also introduce our optimization which tracks
the possible relations between each pair of pointer variables separately.
We report on experimental results in Section \ref{section:experiments}. Section \ref{sec:conclusion} contains conclusions and
directions for future work. 
%


%% After the comparison with related work, we present
%% the class of programs considered in Section~\ref{sec:concdata}. 
%% %% In Section \ref{sec:concdata} we present the class of programs that we considr.
%% In Section \ref{sec:observers}, we define observers, and how to use them for
%% specification. 
%% % In the appendix, we provide observers along with proofs of precisely
%% % capturing standard natural specifications of data structures.
%% % added by AR


\paragraph{Related work.}
Much previous work on verification of concurrent programs has concerned the
detection of generic concurrency problems, such as race conditions,
atomicity violations, or deadlocks~\cite{FlFr:atomizer:journal,NAW:pldi06,NPSG:icse09}.
Verification of conformance to a simple abstract specification has
been performed using
refinement techniques, which establish simulation relations %between
between the implementation and specification, using partly manual
techniques~\cite{Doherty:lockfree,CGLM:cav06,EQSST:tacas10,WaSt:ppopp05}.
%% often using separate techniques for proving atomicity of sequences of
%% statements~\cite{

Amit et al~\cite{Amit:comparisonAbstraction} verify linearizability
by verifying conformance to an abstract specification, which is
the same as the implementation, but restricted to serialized executions.
They build a specialized abstract domain that correlates the state (including
the heap cells) of a concrete thread and the state of the serialized version, and a sequential reference data structure.
The approach can handle a bounded number of threads.
Berdine et al~\cite{BLMRS:cav08} generalize
the approach to an unbounded number of threads by making the shape analysis
thread-modular.
In our approach, we need not keep track of heaps emanating from
sequential reference executions, and so we can use a simpler shape
analysis.
Plain thread-modular analysis is also not powerful enough
to analyze e.g. algorithms with explicit memory management.
%
\cite{BLMRS:cav08} thus improves the precision by correlating local states of
different threads. This causes however a severe state-space explosion
which limits the applicability of the method.

%% We have adapted the approach for achieving thread-modularity of ~\cite{BLMRS:cav08}.

Vafeiadis~\cite{Vafeiadis:vmcai09} formulates the specification using
an unbounded sequence of data values that represent, e.g., a queue or
a stack. He verifies conformance using a specialized abstraction to
track values in the queue and correlate them with values in the implementation.
Like \cite{Shacham:thesis}\TODO{AR: added the citation without explaining that
In addition, data independence as introduced in 
\cite{Wolper:dataindependence} can justify to only 
consider a small number of values during verification.
For instance, \cite{Shacham:thesis} makes use of such an argument
in order to bound the number of values when verifying
compositions of linearizable operations.}, 
our technique for handling values in queues need only consider a
small number of data values (not an unbounded one), for which it is
sufficient to track equalities.
%% In the analysis, Vafeiadis
%% relies on manual simplification of the effect of actions when analyzing
%% interference, which we do automatically.
The approach is extended in~\cite{Vafeiadis:cav10} to automatically infer
the position of linearization points: these have to be supplied in our approach.

Our use of data variables in observers for specifying properties that
hold for all data values in some domain is related in spirit to the 
identification of arbitrary but fixed objects or resources
by Emmi et al.~\cite{EJKM:reference} and Kidd et
al.~\cite{KRDV:vmcai09:journal}.
In the  framework of regular model checking,
universally quantified temporal logic properties
can be compiled into automata with
data variables that are assigned arbitrary initial
values~\cite{AJNOS:logic:journal}.
%
%% In addition, data independence as introduced in 
%% \cite{Wolper:dataindependence} can justify to only 
%% consider a small number of values during verification.
%% For instance, \cite{Shacham:thesis} makes use of such an argument
%% in order to bound the number of values when verifying
%% compositions of linearizable operations.

% The extension of thread-modular analysis to track correlations between
% pairs of threads has been used by Segalov et al.~\cite{Sagiv:correlation}.
%% They record dependencies of local states of different threads 
%% by keeping track of possible states of pairs of processes.
%They note (as we do) that the approach brings a severe explosion.
Segalov et al.~\cite{Sagiv:correlation} continue the work of \cite{BLMRS:cav08}
by also considering an analysis that keeps track of correlations between threads.
They strive to counter the state-space explosion that \cite{BLMRS:cav08} suffers from,
and propose optimizations that are based on the
assumption that inter-process relationships that need to be recorded
are relatively loose, allowing a rather crude abstraction over the
state of one of the correlated threads. 
%These optimizations do not work well 
%when relationships between threads are tight.
These optimizations do not work well when thread correlations are tight.
Our experimental evaluation in
Section~\ref{section:experiments} shows that our optimizations of the
thread correlation approach achieve significantly better analysis
times than~\cite{Sagiv:correlation}.

There are several works that apply different verification
techniques to programs with a bounded number of threads, including
the use of TVLA~\cite{YaSa:queue}.
Several approaches produce decidability
results under limited conditions~\cite{CernyRZCA:CAV10},
or techniques based on non-exhaustive testing~\cite{Burckhardt:PLDI10} or
state-space exploration~\cite{Vechev:spin09} for a bounded number of
threads.

%% Thread-modular analysis~\cite{FlQa:spin03} has been adapted to
%% heap-manipulating programs and used for verifying
%% generic concurrency bugs, such as race conditions by Gotsman et al.
%% ~\cite{GBCS:pldi07}. Our shape analysis is extended from the
%% transitive closure logic by~\cite{BiRa:vmcai06},
%% where it is used to verify properties of sequential programs.



%We obtain a very simple automated technique which is
%able to directly specify and verify conformance to specifications of
%behavioral properties over both control and data, for programs that
%use  an arbitrary
%number of threads and an arbitrary large data domain.

