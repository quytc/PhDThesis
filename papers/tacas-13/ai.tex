\section{Verification by Shape Analysis}
%\enlargethispage{-3mm}
\label{sec:annotations}

To verify that no trace of the program is accepted by an observer, we
form, as in the automata-theoretic approach~\cite{VW:modelchecking},
the cross-product of the program and the observer, synchronizing on
abstract events, and check that this cross-product cannot reach a
configuration where the observer is in an accepting state.
%
% In this section, we describe how to do this by
% generating a formula that holds in the reachable configurations of
% this cross-product, but not in any accepting observer locations.

The analysis needs to deal with the challenges of an unbounded data
domain, an unbounded number of concurrently executing threads, an
unbounded heap, and an explicit memory management.
%
As indicated in Section~\ref{sec:introduction}, the explicit memory
management implies that the assertions generated by our analysis must
be able to track correlations between pairs of threads.
% For instance, an analysis that only gives properties of single threads
% (i.e., a thread-modular analysis~\cite{BLMRS:cav08})
% cannot express that, for the program in Figure~\ref{fig:MS},
% at most one thread is at any point in time at
% line~\ref{ms:code:line:free} and about to deallocate the first node in the queue, and
% would spuriously report a memory error. Furthermore, it
% cannot express that at most one thread is holding some lock.
%
%Our analysis must, thus, correlate different threads.
%
We present our shape analysis in two steps. We first describe a
symbolic encoding of the configurations of the program and then
present the verification procedure.

\subsection{Representation of Symbolic Encodings}

% We therefore manipulate formulas of the form %
% \(%
% \forall i_1, i_2. \ (i_1 \neq i_2 \IMPLIES \Theta[i_1,i_2])%
% \) %
% (abbreviated to %
% \(%
% \forall i_1 \neq i_2. \ \Theta[i_1,i_2] \)) %
% where $i_1$ and $i_2$ range over thread identifiers,
% which characterizes the configurations of the program 
% for {\em any} two distinct executing threads. %, with identifier $i_1$ and $i_2$
% %

A symbolic encoding characterizes {\em all} the reachable
configurations of the program from the point of view of two distinct
executing threads. %
This is done by recording the relationships of the local
configurations of the two threads with each other, the relationships
of the local variables of a thread with global variables, the observer
configuration, and the assertions about the heap.
%
It is a combination of several layers of conjunctions and
disjunctions. %
In this section, let us fix two thread identifiers $i_1$ and $i_2$ and
let us first introduce some necessary definitions in a bottom-up
manner. %

\paragraph{Cell terms.}
Let a {\em cell term} be one of the following: %
(i) a global pointer variable $\yvar$, %
which denotes the cell pointed to by the global variable $\yvar$, %
(ii) a term of the form $\xvar[i_j]$ (where $j=1$ or $j=2$) for a
local pointer variable $\xvar$ of thread $i_j$, which denotes the cell
pointed to by the thread-$i_j$-local-copy of $\xvar$, %
(iii) a special term {\nullconst}, {\undefconst}, or {\freeconst}, %
or (iv) a cell variable, which denotes a cell whose data value is
equal to the current value of an observer variable. %
(Note that the value of an observer variable is fixed during a run of
the observer). %
The latter allows us to keep track of the data in the heap cells, even
in the case where a heap cell is not denoted by any pointer variable
(in order to verify, e.g., the {\sc fifo} property of a queue). %
We use $\cellterms(i_1,i_2)$ to denote the set of all cell terms (of
thread $i_1$ and $i_2$). %

\paragraph{Atomic heap constraint.}
In order to obtain an efficient and practical analysis, which does not
lead to a severe explosion of formulas, we have developed a novel
representation, adapted from the transitive closure logic
of~\cite{BiRa:vmcai06}. %
The representation is motivated by the observation that relationships
between pairs of pointer variables are typically independent.
%
The key aspect of the representation is that it is sufficient to
consider only pairs of variables rather than correlating all variables.
%
An atomic heap constraint is of one of the following forms %
(where $t_1$ and $t_2$ are two cell terms): %
%\vspace{-2mm}
\begin{itemize}
\item $\term_1 \equals \term_2$: the cell terms $\term_1$
  and $\term_2$ denote the same cell,
\item $\nexttrans{\term_1}{\term_2}$: the $\code{next}$
  field of the cell denoted by $\term_1$ denotes the cell denoted by
  $\term_2$,
\item $\twotrans{\term_1}{\term_2}$: the cell denoted by
  $\term_2$ can be reached by following a chain of two or more
  $\code{next}$ fields from the cell denoted by $\term_1$,
\item
$\notrans{\term_1}{\term_2}$: none of 
$\term_1 \equals \term_2$, %
$\nexttrans{\term_1}{\term_2}$, %
$\nexttrans{\term_2}{\term_1}$, %
$\twotrans{\term_1}{\term_2}$, or %
$\twotrans{\term_2}{\term_1}$ is true.
%% \item $\isfree{\term}$ means that the cell denoted by $\term$ is not allocated.
\end{itemize}
%\vspace{-2mm}%
We use $\Pred$ to denote the set
$\{=,\pointsto,\pointedby,\reaches,\reachedby,\unrelated\}$ of all
shape relational symbols. %
We let ${\term}\equals{\mbox{\nullconst}}$ denote that $\term$ is
null, $\nexttrans{\term}{\mbox{\undefconst}}$ denote that $\term$ is
undefined, and $\nexttrans{\term}{\mbox{\freeconst}}$ denote that
$\term$ is unallocated.

%
% Our intention is to make a shape formula record essential
% relationships between the pointer variables of the program. %
%


% \begin{wrapfigure}[8]{r}{.5\textwidth}
%   \input{matrix}
% \end{wrapfigure}
\paragraph{Joined shape constraint.}
A joined shape constraint, for thread $i_1$ and
$i_2$, denoted as $\matrixrep(i_1,i_2)$, is a (typically large)
conjunction %
$\mathop\bigwedge_{\term_1,\term_2\in \cellterms(i_1,i_2)}
\entryof{\term_1}{\term_2}$ %
where $\entryof{\term_1}{\term_2}$ is a non-empty
disjunction of atomic heap constraints. %
%  for the pair
% ($\term_1$,$\term_2$) of cell terms.
% (i.e.\ of the form $\term_1\somerel \term_2$ for ${\somerel}\in\Pred$).
%
Intuitively, it is a matrix representing the heap parts accessible by
the two threads (along with the cell data). Such a representation can
be (exponentially) more concise than using a large disjunction of
conjunctions of atomic heap constraints, at the cost of some loss of
precision.

%% Saturation
We say that a joined shape constraint $\matrixrep(i_1,i_2)$ is {\em
  saturated} if for all terms $x$, $y$, and $z$ in
$\cellterms(i_1,i_2)$, every atomic heap constraint from the disjunction 
$\entryof{x}{z}$ implies the heap constraints that one can derive from
those found in $\entryof{x}{x}$, $\entryof{x}{y}$, $\entryof{y}{y}$,
$\entryof{y}{z}$, and $\entryof{z}{z}$. %
Any joined shape constraint can be saturated by a straightforward
fixpoint procedure, analogous to~\cite{BiRa:vmcai06} or the one for
DBMs~\cite{Dill:DBM}.
%
For instance, let $\entryof{x}{z}$ be $x\pointsto z$ and
$\entryof{y}{z}$ be $y\pointsto z \lor y\reaches z$ and let
$\entryof{x}{x}$ and $\entryof{y}{y}$ admit only equality (there is no
loop involving $x$ or $y$). %
Then $\entryof{x}{y}$ can contain the disjuncts $x = y$, $x\unrelated
y$, which are consistent with $x\pointsto z$ and $y\pointsto z$. It
can also contain $x\pointedby y$, $x\reachedby y$, and $x\unrelated
y$, that are consistent with $x\pointsto z$ and $y\reaches z$. %
In short, $x$ cannot reach $y$, thus when saturating, we remove
$x\pointsto y$ and $x\reaches y$ from $\entryof{x}{y}$.
%
%\\{\bf (*an example which works with the old version which does not handle loops, uses NULL trick *)}\\
%For instance, if $\entryof{x}{\nullconst} = x\pointsto \nullconst$ and $\entryof{y}{\nullconst} = y\pointsto \nullconst \lor y\reaches \nullconst$, then
%$\entryof{x}{y}$ can only contain the disjuncts
%$x = y$, $x\unrelated y$, that are consistent with $x\pointsto \nullconst$ and $y\pointsto \nullconst$, together with
%$x\pointedby y$, $x\reachedby y$, and $x\unrelated y$, that are consistent with $x\pointsto \nullconst$ and $x\reaches \nullconst$.
%In short, $y$ cannot reach $x$, thus when closing, we remove $x\pointedby y$ and $x\reachedby y$ from $\entryof{x}{y}$.

\paragraph{Symbolic Encoding.}
We can now define formally a symbolic encoding over two threads. %
%For two given thread identifiers $i_1$ and $i_2$, %
A symbolic encoding is a disjunction $\Theta[i_1,i_2]$ of formulas of
the form $(\symbstate[i_1,i_2] \land \shapeconstr[i_1,i_2])$ where
$\symbstate[i_1,i_2]$ is a {\em control formula} and
$\shapeconstr[i_1,i_2]$ is a {\em shape formula}.
%

A {\em control formula} $\symbstate[i_1,i_2]$ contains %
(i) the current control location of threads $i_1$ and $i_2$, and the
observer, and %
(ii) a conjunction encompassing the relations between the {\tt age}
fields of any pair of terms.
%
% for each pair of terms of form {\tt $\term_1$.age} and {\tt $\term_2$.age}, that
% may be compared in a \code{CAS} statement, a conjunct of the form
% $\mbox{\tt $\term_1$.age} \comprel \mbox{\tt $\term_2$.age}$, where
% $\comprel\ \in \set{<,=,>}$.
% %
For instance, when analyzing the program in Figure~\ref{fig:MS}, %
this conjunction includes among others, for a thread $i$, %
$\code{head[i].age}{\comprel}\code{Head.age}$ %
and %
$\code{tail[i].ptr{\to}next.age}{\comprel}\code{next[i].age}$, %
where ${\comprel}\in\set{<,=,>}$. %
% In fact, it is necessary for the examples of
% Section~\ref{section:experiments} to only keep track of those which may be
% compared in a \code{CAS} statement.

A {\em shape formula} $\shapeconstr[i_1,i_2]$ is a joined
shape constraint %$\matrixrep(i_1,i_2)$
conjoined with a formula $\psi[v_1 ,\ldots , v_m,\obvar_1, \ldots,
\obvar_n]$ which links the cell variables $\cvar_1,\ldots,\cvar_m$
with the observer variables $\obvar_1, \ldots, \obvar_n$ that are used
to keep track of heap cells with values equal to the observer
variables. %
% (Note that this allows us to represent different cells with the same
% data value). %
Formally, $\shapeconstr[i_1,i_2]$ is a formula of the form %
$$\exists
\cvar_1 ,\ldots , \cvar_m . \left[ %
  \psi[v_1 ,\ldots , v_m,\obvar_1, \ldots, \obvar_n]%
  \ \land \ %
  \matrixrep(i_1,i_2)%
\right]$$
%
%

%% =======================================================================
\subsection{Verification Procedure}
%% =======================================================================

We compute an invariant of the
program %quantified over two distinct threads, %
of the form %
$\forall i_1, i_2. \ (i_1 \neq i_2 \IMPLIES \Theta[i_1,i_2])$ %
% (abbreviated to %
% \(%
% \forall i_1 \neq i_2. \ \Theta[i_1,i_2] \)), %
which characterizes the configurations of the program from the point
of view of two distinct executing threads $i_1$ and $i_2$. %
We obtain the invariant by a standard fixpoint procedure, starting
from a formula that characterizes the set of initial configurations of
the program.
%
For two distinct threads $i_1$ and $i_2$, and for each control formula
$\symbstate[i_1,i_2]$, our analysis will generate one shape formula
$\shapeconstr[i_1,i_2]$.
%

The fixpoint analysis performs a postcondition computation that
results in a set of possible successor combinations of control and
shape formulas. %
The new shape formulas of which the control formula already appears in
the original $\Theta[i_1,i_2]$ will be used to weaken the
corresponding old shape formula. Otherwise, if
the control state is new, a new disjunct is added to
$\Theta[i_1,i_2]$.

%Different symbolic states are mutually exclusive, and hence
%the formula will contain (at most) one shape formula for each
%symbolic state.
%The fixpoint analysis will therefore consider each
%occurring symbolic state together with its corresponding shape formula,
%perform a postcondition computation that results in a set of possible
%successor combinations of symbolic states and shape formulas, and
%thereafter update the above formula by weakening the appropriate
%shape constraints if the corresponding symbolic state already occurs in
%$\Phi[i_1,i_2]$, otherwise a new disjunct is generated.

%We note that different symbolic states are mutually exclusive, and hence
%the formula will contain (at most) one shape formula for each
%symbolic state.
%The fixpoint analysis will therefore consider each
%occurring symbolic state together with its corresponding shape formula,
%perform a postcondition computation that results in a set of possible
%successor combinations of symbolic states and shape formulas, and
%thereafter update the above formula by weakening the appropriate
%shape constraints if the corresponding symbolic state already occurs in
%$\Phi[i_1,i_2]$, otherwise a new disjunct is generated.


%{\bf (* Provide intuition *)}
%% \paragraph{Saturation}
%% A formula of form 
%% \(
%% \wedge_{\term_1,\term_2\in \PVar(i_1,i_2)} \phi_{\term_1,\term_2}
%% \)
%% is called a {\em joined formula}. We define a normal form for joined formulas,
%% obtained by strengthening each conjunct $\phi_{\term_1,\term_2}$ under possible
%% consequences of pairs of conjuncts of form 
%% $\phi_{\term_1,\term_3}$ and $\phi_{\term_3,\term_2}$. More precisely, a joined
%% formula
%% \(
%% \wedge_{\term_1,\term_2\in \PVar(i_1,i_2)} \phi_{\term_1,\term_2}
%% \)
%% is {\em saturated} if for all terms $\term_1$, $\term_2$, and $\term_3$,
%% each disjunct in $\phi_{\term_1,\term_2}$ is consistent with some disjunct from
%% $\phi_{\term_1,\term_3}$ and some disjunct from  $\phi_{\term_3,\term_2}$. Saturation
%% consists in removing from
%% $\phi_{\term_1,\term_2}$ all disjuncts that that do not satisfy this criterion for
%% some term $\term_3$.

%% The generation of a formula $\forall i_1,i_2 . \Phi[i_1,i_2]$ which is satisfied
%% by all configurations, is performed by adapting a
%% standard fixpoint computation, which starts from a formula characterizing
%% initial configurations, and repeatedly extends it with postconditions
%% until convergence.
%% Since postconditions are disjunctive, the postcondition is computed separately
%% for each disjunct of form
%% \(
%% \psi[i_1,i_2] \land \phi_{\psi[i_1,i_2]}[i_1,i_2]
%% \)
%% that appears in $\Phi[i_1,i_2]$.

%% ========================================================
%% Postcondition computation
%% ========================================================
For two threads $i_1$ and $i_2$, we must consider two
scenarios: either $i_1$ or $i_2$ %
performs a step, or some other
(interfering) thread $i_3$, (distinct from $i_1$ and $i_2$), performs
a step.

\paragraph{Postcondition computation.}
In the first scenario, where one of the threads $i_1$ or $i_2$
performs a step, we can compute the postcondition of %
\(%
\left( \symbstate[i_1,i_2] \land \shapeconstr[i_1,i_2] \right)%
\) %
as follows. %
$\sigma[i_1,i_2]$ is first updated to a new control state
$\sigma'[i_1,i_2]$ in the standard way (by updating the possible
values of control locations and observer state). %
$\shapeconstr[i_1,i_2]$ is then updated to $\shapeconstr'[i_1,i_2]$ by
updating each conjunct $\entryof{\term_1}{\term_2}$ according to the
particular program statement that the thread is performing. %
In general, we (i) remove all disjuncts that must be falsified by the
step %
(this may require splitting the formula into several stronger formulas
whenever the falsification might be ambiguous), %
(ii) add all disjuncts that may become true by the step, %
(iii) saturate the result. %
%Closing under entailment may result in more than one solution,
%hence we may obtain multiple variants of $\shapeconstr'[i_1,i_2,i_3]$.

Consider for instance the program statement {\code{x:=y.next}}. %
Since only the value of $x$ is changing, the transformer updates only
conjuncts $\entryof{\term}{x}$ and $\entryof{x}{\term}$ where
$\term\in\cellterms(i_1,i_2)$. %
All assertions about $x$ are reset by setting every conjunct
$\entryof{x}{\term}$ and $\entryof{\term}{x}$ to $\Pred$, for all
$\term\in\cellterms(i_1,i_2)$. %
(The disjunction over all elements of $\Pred$ is the assertion
$true$). %
We then set $\entryof{x}{y}$ to $x\pointedby y$, $\entryof{y}{x}$ to
$y\pointsto x$ and derive all predicates that may follow by
transitivity. %
Finally, we saturate the formula. It prunes the (newly added)
predicates that are inconsistent with the rest of the shape formula.
%Third, we infer possible relationships that the other cell terms can have with $x$ assuming $y\pointsto x$.
%We iterate through all $\term\in\cellterms(i_1,i_2,i_3)$, $x\neq \term \neq y$ and assign to $\entryof{x}{\term}$ the disjunction of all relationships between $x$ and $\term$ consistent with $x\pointsto y$ and $\entryof{y}{\term}$.

For {\code{x.next:=y}}, it is important to know the reachabilities
that depend on the pointer $x\code{.next}$. %
The representation might potentially contain imprecision %
(it might for instance state that, for a term $\term$,
$\entryof{\term}{x}$ contains $t\reachedby x$ and $t\reaches x$, %
even if we know, via a simpler analysis, that no cycles are generated). %
Hence, we first split the formula into stronger formulas in such a way
that we disambiguate the part of the reachability relation involving
$x$. %
%
On each resulting formula, we then remove reachability predicates
between cell terms that depend on $x\code{.next}$ %
(e.g., we remove $u\reaches v$ if $u\reaches x$ and $x\reaches v$). %
%(e.g., we replace $u\reaches v$ in $\entryof{u}{v}$ by $u\unrelated v$
%if $u\reaches x$ and $x\reaches v$). %
We then set $\entryof{x}{y}$ to $x\pointsto y$ and derive all
predicates that may follow by transitivity (e.g., if $u\reaches x$ and
$y\reaches v$, we add $u\reaches v$), and we saturate the result.

%% ==================================================
%% Interference
%% ==================================================

\paragraph{Interference.}
In the case where we need to account for possible interference on the
formula \(%
\left( \symbstate[i_1,i_2] \land \shapeconstr[i_1,i_2] \right)%
\) %
by another thread, (distinct from $i_1$ or $i_2$), we proceed as
follows. We %
(i) extend the formula with the interfering thread, %
(ii) compute a postcondition as described in the first scenario %
and (iii) project away the interfering thread. %
%
%Let us introduce an additional thread identifier $i_3$.
%

Step (i) combines a given formula $(\symbstate[i_1,i_2] \land
\shapeconstr[i_1,i_2])$ with the information of an extra thread $i_3$.
%to characterize configurations involving 3 threads. %
%
In a similar manner to~\cite{AbHoHa:view:abstraction}, the resulting
formula is of the form \( \left( \symbstate[i_1,i_2,i_3] \land
  \shapeconstr[i_1,i_2,i_3] \right) \) such that any projection to two
threads is a formula compatible with some disjunct of $\Theta[i_1,i_2]$.
%
To generate all such formulas involving three threads, we must, besides
$(\symbstate[i_1,i_2] \land \shapeconstr[i_1,i_2])$ itself, consider
all pairs of disjuncts %
$(\sigma_\bullet[i_2,i_3] \land \shapeconstr_\bullet[i_2,i_3])$ and %
$(\sigma_\circ[i_1,i_3] \land \shapeconstr_\circ[i_1,i_3])$, %
such that $\sigma[i_1,i_2] \land \sigma_\bullet[i_2,i_3] \land
\sigma_\circ[i_1,i_3]$ is consistent. %
In this case, we generate the formula \(\sigma[i_1,i_2,i_3] \land
\phi[i_1,i_2,i_3]\) where 
% %\vspace{-1mm}
% \[
% \begin{array}{lcl}
% \sigma[i_1,i_2,i_3]
% & \equiv & 
% \sigma[i_1,i_2] \land \sigma_\bullet[i_2,i_3] \land \sigma_\circ[i_1,i_3] \\
% \phi[i_1,i_2,i_3]
% & \equiv & 
% \phi[i_1,i_2] \land \phi_\bullet[i_2,i_3] \land \phi_\circ[i_1,i_3]
% \end{array}
% %\vspace{-1mm}
%\]
$\sigma[i_1,i_2,i_3] \equiv \sigma[i_1,i_2] \land \sigma_\bullet[i_2,i_3] \land \sigma_\circ[i_1,i_3]$ and 
$\phi[i_1,i_2,i_3] \equiv \phi[i_1,i_2] \land \phi_\bullet[i_2,i_3] \land \phi_\circ[i_1,i_3]$. %
We then saturate \(\phi[i_1,i_2,i_3]\) (in the same way as for
joined shape formulas over two threads). %
% Notice that the original shape formula \(\phi[i_1,i_2]\) might become
% stronger. %
%
For each statement $\stmt$ of thread $i_3$ that can be executed when
$\sigma[i_1,i_2,i_3]$ holds, we compute (step ii) its postcondition
\(\sigma'[i_1,i_2,i_3] \land \shapeconstr'[i_1,i_2,i_3]\). %
Finally (step iii), \(\sigma'[i_1,i_2,i_3] \land \phi'[i_1,i_2,i_3]\)
is projected back onto \(\sigma'[i_1,i_2] \land
\shapeconstr'[i_1,i_2]\) by removing all information about the
variables of thread $i_3$.

Since the domain of control formulas and the domain of shape formulas
over a fixed number of cell terms are finite, the abstract domain of
formulas %
$\forall i_1, i_2. \ (i_1 \neq i_2 \IMPLIES \Theta[i_1,i_2])$ %
is finite as well. %
The iteration of postcondition computation is thus guaranteed to
terminate.
%verification procedure is guaranteed to terminate by reach a fixpoint.
%Since our abstract domain of shape formulas over $i_1,i_2$ is finite, the iteration of postconition computation reaches eventually a fixpoint.



