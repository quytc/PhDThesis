\section{Programs}
\label{sec:concdata}

We consider systems consisting of an arbitrary number of
concurrently executing threads. Each thread may at any time invoke one of a
finite set of methods.
%% We assume that the global variables and the shared heap have been initialized
%% before threads are created.
%% Each thread can access the
%% shared state concurrently, using a finite set of methods.
Each method declares local variables (including
the input parameters of the method) and a method body.
In this paper, we assume that variables are either pointer variables (to heap cells), 
or data variables (assuming values from an unbounded or infinite domain,
which will be denoted by $\dset$).
The body is built in the standard way
from atomic commands using standard control
flow constructs (sequential composition, selection, and loop constructs).
Method execution is terminated by executing a {\tt return} command,
which may return a value.
%% When representing the behavior of a concurrent object in our program model,
%% global variables are used to maintain the persistent state of the object.
%% Methods can be invoked by concurrently executing threads at arbitrary
%% points in time.
The global variables can be
accessed by all threads, whereas local variables can be accessed only
by the thread which is invoking the corresponding method.
We assume that the global variables and the heap are initialized by
an initialization method, which is executed once at the beginning
of program execution.

%% In this paper, we assume that (global and local) variables
%% are either {\em data variables}, which
%% assume values from an infinite domain, denoted by $\dset$, or
%% {\em pointer variables}, whose values are heap cells. We specialize the
%% presentation for the later examples, and assume that
%% each heap cell has two fields: a {\it data} field which contains a data
%% value in $\dset$, and a {\it next} field which contains a pointer
%% to a heap cell or $\nil$. In addition, fields can also contain the value
%% $\bot$ (undefined): this is their initial value.

Atomic commands include assignments between data variables, 
pointer variables, or fields of cells pointed to by a pointer variable.
The command {\tt {\bf new} node()} allocates a new structure of type
{\tt node} on the heap, and returns a
reference to it. The cell is deallocated by the command {\tt \bf free}.
%% We assume a memory management mechanism, which automatically collects
%% garbage, and also ensures that a new cell is fresh, i.e., has
%% not been used before by the program; this avoids the so-called
%% ABA problem (e.g.,~\cite{MS:QueueAlgorithms}).
%% The assume command $\assume{\gvar}$ blocks execution of the thread when the
%% guard $\gvar$ (a boolean expression) evaluates to false, and is
%% executed without any effect when $\gvar$ evaluates to true.
%% The command $\assert{\gvar}$ causes an error when the
%% guard $\gvar$ (a boolean expression) evaluates to false, and is
%% equivalent to a {\tt skip} statement when $\gvar$ evaluates to true.
The compare-and-swap command {\tt CAS(\&a,b,c)} atomically
compares the values of {\tt a} and {\tt b}.
If  equal, it assigns the value of
{\tt a} to {\tt c}  and returns {\tt TRUE}, 
otherwise, it leaves {\tt a} unchanged and returns {\tt FALSE}. 


As an example, Figure~\ref{fig:MS} shows %by Michael and Scott \cite{MS:QueueAlgorithms} (described in Fig.\ref{fig:MS}) 
%
%
a version of the concurrent queue by Michael and
Scott~\cite{MS:QueueAlgorithms}.
The program represents a queue as a linked list from the node
pointed to by {\tt Head} to a node that is either pointed by {\tt Tail} or by {\tt Tail}'s successor. 
%
The global variable {\tt Head} always points to a dummy cell whose successor, if any, stores the head of the queue. 
%
In the absence of garbage collection, the program must handle the ABA
problem where a thread mistakenly assumes that a globally accessible
pointer has not been changed since it previously accessed that
pointer. Each pointer is therefore equipped with an additional {\tt
  age} field, which is incremented whenever the pointer is assigned a
new value.

% Since the algorithm manages memory explicitly, each pointer has an additional
% {\tt age} field, which is incremented whenever the pointer
% is assigned a new value. In the absence of automatic garbage collection, the
% {\tt age} field avoids the ABA problem, i.e., that a thread mistakenly assumes
% that a globally accessible pointer has not been changed
% since the the previous access to that pointer by the thread.

%% It manipulates heap cells of type {\tt node}, each consisting of  
%% a field {\tt val}, which carries a data value, and an extended pointer field
%% {\tt next}. An extended pointer field (of type {\tt pointer\_t}) consists of
%% a pointer {\tt ptr} to a {\tt node} and an integer counter {\tt age}.
%% %
%% The {\tt age} field is necessary in order to avoid the ABA problem in the absence
%% of automated garbage collection.
%% We say that an extended pointer points to a node {\tt n} to mean that its {\tt ptr} field points to {\tt n}.

The queue can be accessed by an arbitrary number of threads, either by
an enqueue method $\enqueue{d}$, which inserts a cell containing the data value {\tt d} at the tail, 
or by a dequeue method $\dequeue{d}$ which returns  {\tt empty}
if the queue is empty, and otherwise advances {\tt Head},
deallocates the previous dummy cell
and returns the data value stored in the new dummy cell. 
The algorithm uses the atomic compare-and-swap (CAS) operation.
For example,
the command {\tt CAS(\&Head, head, $\langle$next.ptr,head.age+1$\rangle$)}
at line~\ref{ms:code:line:deq} of the {\tt deq}
method checks whether the extended pointer {\tt Head} equals the extended pointer {\tt head} (meaning that both fields must agree).
If not, it returns {\tt FALSE}. Otherwise it returns {\tt TRUE} after assigning
{\tt $\langle$next.ptr,head.age+1$\rangle$} to {\tt Head}.

%Initially, the stack is empty, i.e., the pointer field of variable {\tt S} points 
%to {\tt null}.
% In order for a {\tt dequeue} method executed by a thread 
% to remove an element from the queue, it needs to make
% local copies of {\tt Head}, the variable field of the cell pointed by it,
% and of {\tt Tail} and to perform a successful compare-and-swap
% opertation 
% %It returns {\tt empty}
% %if {\tt Head} and {\tt Tai} turned out to point to the same cell with 
% %{\tt null} successor.

%  and creates a new cell, pointed to
% by {\tt x}, to hold the new value. It copies the global variable
% {\tt S} to its local variable {\tt t}, and lets the {\tt next} field of
% the cell point to the top of the stack. It finally inserts the
% cell into the stack by an atomic 
% compare-and-swap operation.
% The atomic {\tt cas(\&Head,h,[nxt.ptr,(t.age)+1])} operation
% compares the values of {\tt S} and {\tt t} (both the {\tt ptr} and the {\tt age} fields).
% If they are equal, it performs the following three actions: i) assigns the value of
% {\tt x} to {\tt S.ptr}, ii) assigns the value of {\tt (t.age)+1} to {\tt S.age}, and iii)
% returns {\tt true}, 
% otherwise, it leaves {\tt S} unchanged and returns {\tt false}. 
% %% %


\begin{figure}[h]
\centering
\begin{tikzpicture}[%
  codeblock/.style={text width=0.4\linewidth,rounded corners,inner xsep=12pt, inner ysep=4pt, below right,fill=blue!5!white,draw=gray!3!white,fill=gray!6!white,double}%
  ]
  \node[codeblock] (init) at (current bounding box.north west) {\begingroup\scriptsize\VerbatimInput[numbers=none]{ms/init}\endgroup};
  \node[codeblock] (enq) at (init.south west) {\begingroup\scriptsize\VerbatimInput{ms/enq}\endgroup};
  \node[codeblock] (struct) at (init.north east) {\begingroup\scriptsize\VerbatimInput[numbers=none]{ms/struct}\endgroup};
  \node[codeblock] (deq) at (struct.south west) {\begingroup\scriptsize\VerbatimInput[firstnumber=17]{ms/deq}\endgroup};

  \node[property,rotate=-10,scale=0.7] at ([shift={(-0.5,-0.2)}]init.north east) {\sc Init};
  \node[property,rotate=-10,scale=0.7] at ([shift={(-0.5,-0.2)}]enq.north east) {\sc Enq};
  \node[property,rotate=-10,scale=0.7] at ([shift={(-0.5,-0.2)}]deq.north east) {\sc Deq};

%  \node[codeblock] (init) at (current bounding box.north west) {\lstinputlisting[style=code,escapechar=\%,numbers=none]{ms/init}};
%   \node[codeblock] (enq) at (init.south west) {\lstinputlisting[style=code,escapechar=\%,firstnumber=0]{ms/enq}};
%   \node[codeblock] (struct) at (init.north east) {\lstinputlisting[style=code,numbers=none]{ms/struct}};
%   \node[codeblock] (deq) at (struct.south west) {\lstinputlisting[style=code,escapechar=\%,firstnumber=14]{ms/deq}};
\end{tikzpicture}%
\caption{Michael \& Scott's non-blocking queue~\protect\cite{MS:QueueAlgorithms}.}
%% implementation. Integer counters (here  {\tt ages}) 
%%are associated to variables to avoid ABA-like problems; {\tt older(x)} returns
%% an integer that is strictly larger than {\tt x}.}
\label{fig:MS}
\end{figure}
%
%% Note that the values of two pointer variables are equal only if they point
%% to the same cell (i.e., contain the same ``address'').
%% $\xvar \assigned \yvar$,
%% $\nxt\xvar \assigned \yvar$, {\em {\em
%% $\xvar \assigned \nxt\yvar$, 
%% $\dvar \assigned \dvar'$, 
%% $\datafield\xvar \assigned \dvar$,
%% or $\dvar \assigned \datafield\yvar$ where $\xvar$ and $\yvar$ are pointer 
%% variables and $\dvar$ is a data variable), 
%
%% The  atomic construct $\atomic{\stmt}$ lets the compound statement 
%% $\stmt$ (which must be known to terminate)
%% %\TODO{$\stmt$ can contain loops}
%% be executed atomically.
%% $\xvar \assigned \yvar$,
%% $\nxt\xvar \assigned \yvar$, {\em {\em
%% $\xvar \assigned \nxt\yvar$, 
%% $\dvar \assigned \dvar'$, 
%% $\datafield\xvar \assigned \dvar$,
%% or $\dvar \assigned \datafield\yvar$ where $\xvar$ and $\yvar$ are pointer 
%% variables and $\dvar$ is a data variable), 
%
%% We represent the execution of a $\keyword{return}$
%% command by assigning the return value to the local variable
%% $\retval$, which is assumed to be an implicit local variable of each
%% method. In the initial local state, the value of $\retval$ is undefined.

%% {\bf (* Explain memory management? *)}

%% We concisely describe program
%% syntax by a set of control locations $\cstates$ and by
%% representing the method body by a set of
%% transitions $\loctrans{\cstate}{\stmt}{\cstate'}$, 
%% which denote that a thread can move from control location $\cstate$
%% to $\cstate'$ by executing the command or atomic statement $\stmt$.

%% We here briefly sketch how an operational semantics for programs can
%% be described in a natural way.
%% Define a {\em heap} as a triple $\tuple{M,Next,Val}$,
%% where $M$ is a finite set of {\em cells},
%% where $Next : M \mapsto (M \cup \set{\nil,\bot})$
%% maps each cell to either a cell in $M$, the value
%% $\nil$, or $\bot$ (undefined), and where
%% $Val : M \mapsto (\dset \cup \set{\bot})$
%% maps each memory cell to a data value in $\dset$ or to $\bot$.
%% Since we aim at programs without garbage collector,
%% memory cells are partitioned into \emph{allocated} and \emph{free} cells.

%% A {\em valuation}
%% maps each data variable to a data value in $\dset$ or to $\bot$,
%% and each pointer variable to either a cell in $M$, the value $\nil$, or $\bot$.
%% Define a {\em local configuration} as a pair 
%% $\tuple{\localcstate,\localval}$, where $\localcstate$ is a control location
%% in some method, and $\localval$ is a valuation of the local variables of that
%% method (including the input parameters and the variable $\retval$).
%% Define a {\em thread configuration} $\threadstate$ as a
%% mapping from a set $\domof\threadstate$ of thread identifiers (which represent
%% the currently active method invocations) to local configurations.
%% Define a {\em global configuration} as a triple
%% $\globconf = \tuple{\heap,\globalval,\threadstate}$, where $\heap$ is a heap,
%% $\globalval$ is a valuation of the global variables, and
%% $\threadstate$ is a thread configuration.
%% The {\em initial global configuration} $\initglobconfig$
%% is the configuration that results after
%% execution of the $\initmeth$ method from a configuration consisting
%% of an empty heap with
%% no cells, undefined initial values of global variables,
%% and an empty set of active threads.

%% The dynamic behavior of a program is represented by a set of
%% {\em labeled transitions}.
%% An {\em observable label} is a term of form $\invoke{\tid}{\mname}{d}$ or
%% $\respond{\tid}{\mname}{d}$,
%% where $\tid$ is a thread identifier,
%% where $\mname$ is a method name, and $d$ is the value of the input or return
%% parameter associated with the invocation or response.
%% We omit the obvious modifications to allow for constant or missing return values
%% (such as $\emptyconst$) and methods without input parameters.

%% Program behavior is formalized by a labeled transition
%% relation on the set of configurations.
%% We omit a detailed account of each command, and just
%% assume that the meaning of commands and atomic statements is given in terms
%% of a labeled transition relation. Each element in the
%% transition relation represents the
%% execution of a command or atomic statement, which changes
%% the local configuration of the thread, possibly updating
%% the heap and global variables.
%%  Elements of the transition
%% relation are either of form
%% $\transrel{\globconfig}{\observable{\location}{d}}{\globconfig'}$, denoting that
%% the configuration can change from $\globconfig$ to $\globconfig'$ while
%% emitting the abstract symbol $\observable{\location}{d}$, or of form
%% $\transrel{\globconfig}{\tau}{\globconfig'}$ in case no abstract symbol
%% is emitted.
%% %% An initial global configuration is of form
%% %% $\tuple{\heap,\globenv, \locconfig}$, where $\heap$ and $\globenv$ have
%% %% been initialized, and $\locconfig$ maps each thread identifier to an
%% %% initial control location and its local variables are undefined.
%% An {\em execution} of a program is a sequence of transitions
%% \(
%% \initglobconfig 
%% {\stackrel{l_1}{\rightarrow}}
%% \globconfig_1
%% {\stackrel{l_2}{\rightarrow}}
%% \cdots
%% {\stackrel{l_n}{\rightarrow}}
%% \globconfig_n
%% \).
%% A {\em trace} of a program is the sequence of non-$\tau$ labels on transitions in some
%% execution of the program.
%% Traces capture an abstract view of program executions. 
%% A program is specified in terms of properties of its
%% traces. We now present how to use observers, i.e., automata equipped
%% with variables that can take values from infinite domains, for
%% specifying safety properties of programs.

\section{Specification by Observers}
\label{sec:observers}
To specify a correctness property,
we  instrument each method to generate abstract events. 
An {\em abstract event} is a term of the form $\location(d_1, \ldots , d_n)$
where $\location$ is an event type, taken from a finite set of event types,
and  $d_1, \ldots , d_n$ are data values in $\dset$.
To specify linearizability, the abstract event $\location(d_1, \ldots , d_n)$
generated by a method should be such that $\location$ is the name of the
method, and $d_1, \ldots , d_n$ is the sequence of actual parameters and
return values in the current invocation of the method.
This can be established using standard sequential verification techniques.
\TODO{AR: removed ``This relationship between the behavior of the method, and the generated abstract events should be verified: it can be done by standard sequential verification techniques.''}
%% should To specify this correctness condition,
%% we instrument each method to generate an abstract event whenever a
%% linearization point is passed. 
%% The abstract event is simply the name of the method
%% including its data parameters and return values. For instance, 

%% A {\em parameterized event} is a term of  form $\location(p_1, \ldots , p_n)$, where 
%% $p_1, \ldots , p_n$ are formal parameters.
%% A command of a program can be instrumented to generate an
%% abstract event, in which the parameters are obtained by evaluating
%% expressions over the current program variables. The generation can be
%% conditional on some boolean expression.

We illustrate how to instrument
the program of Figure~\ref{fig:MS} in order to specify that it is a linearizable
implementation of a queue. 
%% Linearizability provides the illusion that each
%% method invocation takes effect instantaneously at some point (called the
%% {\em linearization point}) between method invocation and return.
%% This means that in any behavior, the sequence of executed linearization
%% points conforms to the behavior of a sequential queue.
%% To specify this correctness condition,
%% we instrument each method to generate an abstract event whenever a
%% linearization point is passed. 
%% The abstract event is simply the name of the method
%% including its data parameters and return values. 
The linearization points %
{\tikz{\node[shape=circle,inner sep=0, draw=blue,fill=red!50!blue,text width=1pt,scale=0.3]{};}} %
are at line~\ref{ms:code:line:enq}, \ref{ms:code:line:empty} and \ref{ms:code:line:deq}.
For instance, line~\ref{ms:code:line:enq} of the {\tt enq} method called with data value {\tt d} is instrumented to generate the
abstract event {$\enqueue{d}$} when the {\tt CAS} command succeeds; no abstract
event is generated when the {\tt CAS} fails.
Generation of abstract events can be conditional. For instance,
line~\ref{ms:code:line:empty} of the {\tt deq} method is instrumented to generate $\dequeue{empty}$
when the value assigned to {\tt next} satisfies {\tt next.ptr = NULL}
(i.e., it will cause 
the method to return {\tt empty} at line~\ref{ms:code:line:emptyreturn}).

%% When verifying linearizability, one must check that
%% the instrumentation reflects the actual behavior of methods, i.e., that
%% each method invocation generates exactly one abstract event which ``says''
%% what the method does. This can be done by standard sequential verification
%% techniques.

%% After instrumentation,
Each execution of the instrumented program will generate a sequence
of abstract events called a {\em trace}. A {\em correctness property} (or
simply a {\em property}) is a set of traces. We say that an instrumented program
{\em satisfies} a property if each trace of the program is in the property.
%
In contrast to the classical (finite-state) automata-theoretic approach~\cite{VW:modelchecking},
we specify properties by \emph{infinite-state} automata, called {\em observers}. An observer has
a finite set of control locations, and a finite set of data variables that
range over potentially infinite domains.
It observes the trace and can reach an accepting control location if the trace
is not in the property.

Formally, let a {\em parameterized event} be a term of the 
form $\location(p_1, \ldots , p_n)$, where 
$p_1, \ldots , p_n$ are formal parameters.
We will write $\pbar$ for $p_1, \ldots , p_n$, and
$\dbar$ for $d_1, \ldots , d_n$.
%
An {\em observer} consists of
%% $\auto=\tuple{\locs, \initloc, \final, \obsvars, \obsconstants, \obstransitions{\auto}}$
%% where  $\locs$ is 
a finite set of {\em observer locations}, one of which is
%% $\initloc\in\locs$ is
{\em initial} and some of which are 
%% $\final$ is the set of 
{\em accepting},
a finite set of {\em observer variables},
%% $\obsconstants$ is a finite set of {\em observer constants} in $\datadomain$,
and
%% $\obstransitions{\auto}$ is 
a finite set of {\em transitions}.
Each transition is of
form ${\loc} \stackrel{\location(\pbar); g}{\obstransitions{\auto}}{\loc'}$ where 
$\loc, \loc'$ are observer locations,
$\location(\pbar)$ is a parameterized event, and the guard
$g$ is a Boolean combination of equalities over formal parameters $\pbar$,
and observer variables. Intuitively, it denotes that the observer can move
from location $\loc$ to location $\loc'$ when an abstract event of form
$\location(\dbar)$ is generated such that $g[\dbar/\pbar]$ is true.
Note that the values of observer variables are not
updated in a transition. 
%% The observer variables assume values  in $\dset$: initially, they can assume
%% any value in $\dset$.
%% To simplify, % the presentation, and without loss of generality, 
%% we always take  the observer variables to be pairwaise different.
%
%% with     $\varsof{\initloc}$ being the empty tuple,
%$\obsvars$ is a finite set of {\em observer data variables}, and
%with a distinguished subset $\vmark{\obsvars}\subseteq\obsvars$, and
%% Observers are complete automata in that every unmatched abstract event
%% is sent to a sink observer location. We do not 
%% represent sink locations in the figures in order to improve readability.
%\TODO{Sink stuff removed by Bengt}
%
An {\em observer configuration} is a
pair $\tuple{\loc,\obsenv}$, where $\loc$ is an observer location,
and $\obsenv$ maps each observer variable to a value in the data domain $\dset$.
The configuration is initial if $\loc$ is initial; thus the variables can
assume any initial values.
An {\em observer step} is a triple
\(
\tuple{\loc,\obsenv}
\stackrel{\location(\dbar)}{\obstransitions{\auto}}
\tuple{\loc',\obsenv}
\)
such that there is a transition 
${\loc} \stackrel{\location(\pbar) ; g}{\obstransitions{\auto}}{\loc'}$ for which
$g[\dbar/\pbar]$ is true.
%
%
A \emph{run} of the observer on a trace
\(
\trace =
\location_1(\dbar_1)
\location_2(\dbar_2)
\cdots
\location_n(\dbar_n)
\)
is a sequence of observer steps 
\(
\tuple{\loc_0,\obsenv}
\stackrel{\observable{\location_1}{\dbar_1}}{\obstransitions{\auto}}
%\tuple{\loc_1,\obsenv}
%\stackrel{\tau}{\obstransitions{\auto}}
%\tuple{\loc_2,\obsenv}
\cdots
\stackrel{\observable{\location_n}{\dbar_n}}{\obstransitions{\auto}}
%\stackrel{\observable{\location}{\dvar_2}}{\obstransitions{\auto}}
\tuple{\loc_n,\obsenv}
\) where $\loc_0$ is the initial observer location.
The run is \emph{accepting} if $\loc_n$ is accepting.
A trace $\trace$ is \emph{accepted} by an observer $\auto$ if $\auto$ has an accepting run
on $\trace$.
The property specified by $\auto$ is the set of traces that are 
not accepted by $\auto$.



%% \begin{figure}[ht]
%% \begin{tikzpicture}[framed, node distance=2.8cm, sstate/.style={circle, draw=blue!50,fill=blue!20,thick, minimum size = 5mm}, initial/.style=initial by arrow ]


%% \draw (-2.2, 1.5) node (oname) {{\sc Double Insertion}:};
%% %\draw (-1,0) node (order) {${\bigvee}_{\dred\neq\dblue}$};
%% %
%% \draw (-1,0) node[circle, draw=green!50,fill=green!20,thick, minimum size = 5mm] (o_0)   {$s_0$};
%% \node[sstate] (o_1)    [right=of o_0]      {$s_1$};
%% \node[circle, accepting, draw=red!50,fill=red!20,thick, minimum size = 5mm] (o_2)   [right=of o_1]  
%% {$s_2$};

%% \path[->] (o_0) edge [loop above] node {\scriptsize{$\begin{array}{c}
%%                                               \aletter{insert(\pvar)}{\pvar}\\
%%                                               \aletter{delete(\pvar)}{\pvar}
%%                                             \end{array}$}} () 
%%                 (o_0) edge node [above] {\scriptsize{$\aletter{insert(\pvar)}{\pvar=\obvar}$}} (o_1)
%%                 (o_1) edge [bend left] node [below] {\scriptsize{$\aletter{delete(\pvar)}{\pvar=\obvar}$}} (o_0)
%%                 (o_1) edge [loop above] node {\scriptsize{$\begin{array}{c}
%%                                               \aletter{insert(\pvar)}{\pvar\neq\obvar}\\
%%                                               \aletter{delete(\pvar)}{\pvar\neq\obvar}
%%                                             \end{array}$}} ()
%%                 (o_1) edge node [above] {\scriptsize{$\aletter{insert(\pvar)}{\pvar=\obvar}$}} (o_2);
%% \end{tikzpicture}
%% \caption{An observer for checking that any data value can be inserted at most once 
%%   before being deleted (useful in e.g. a set specification).
%%   The  $\set{s_2}$ is the set of final locations, and $\obsvars=\set{\obvar}$.
%% }
%% \label{fig:paperdinsertion}
%% \end{figure}

%% Thus, a program satisfies the property represented by observer
%% $\auto$ if no trace of the program is accepted by $\auto$.
%% %
%% Intuitively, an observer observes a trace, and accepts whenever the
%% trace violates the intended specification.
Since the data variables can assume arbitrary initial values, observers
can specify properties that are universally quantified over all data values.
If a trace violates such a property for some data values, the
observer can non-deterministically choose these
as initial values of its variables, and thereafter detect the
violation when observing the 
\begin{wrapfigure}[8]{r}{.55\textwidth}
%\begin{figure}
\vspace{-25pt}
\begin{center}
\begin{tikzpicture}[%
  framed%
  ,initial/.style=initial by arrow%
  ]
%\draw (-2.2, 2) node (oname) {{\sc No Creation}:};
%
\node[obsstate=green] (o_0) {$s_0$};
\node[obsstate=red,accepting,right=2.5cm of o_0] (o_1) {$s_1$};

\path[->] (o_0) edge [loop left] node {\scriptsize{$\begin{array}{c}
                                              \aletter{insert{(\pvar)}}{\pvar\neq\obvar}\\
                                              \aletter{delete{(\pvar)}}{\pvar\neq\obvar}\\
                                              \aletter{isEmpty{()}}{true}
                                            \end{array}$}} () 
                (o_0) edge node [above] {\scriptsize{$\aletter{delete{(\pvar)}}{\pvar=\obvar}$}} (o_1);
\end{tikzpicture}
\end{center}
\vspace{-15pt}
\caption{An observer for checking that no data value can be deleted if it has not been first inserted.
  The variable $\obvar$ is an observer variable.
}
\label{fig:set:creation}
%\end{figure}
\end{wrapfigure}
%
trace. Several data structures can be specified
by a collection of properties, each of which is represented by an observer.
For instance, a set can be specified by a collection of properties, one of
which is that a data value can be deleted, only if it has been previously
inserted. 
The observer in
Figure~\ref{fig:set:creation} specifies this property: it accepts executions
in which for some data value $d$, 
a $\mbox{\tt delete}(d)$-event is observed even though no
earlier $\mbox{\tt insert}(d)$-event has been observed. 
%


\section{Data Independence}
\label{sec:wolper}
In the previous section, we showed how observers can 
specify some data structures, such as sets, in a straight-forward way.
%% domain, observers can specify properties that are universally
%% quantified over data values, by checking whether the property can be
%% violated for some values of the variables.
However, in order to specify some other
data structures, including queues and stacks, 
for which one must
be able to ``count'' the number of copies of a data value that have been
inserted but not removed, we must additionally employ an extension of a
data independence argument, originating from
Wolper~\cite{Wolper:dataindependence}, as follows.

The argument assumes that for each trace, there is a fixed subset of
all occurrences of data values in the trace, called the set of
{\em input occurrences}.
Formally, this subset can be arbitrary, but to make the argument work,
input occurrences should typically be the data values that are provided as
actual parameters of method invocations.
Thus, in the program of Figure~\ref{fig:MS},
the input occurrences are the parameters of
$\enqueue{d}$ events, whereas parameters of $\dequeue{d}$ events are
{\em not} input occurrences, since they are provided as return values.

Let us introduce some definitions.
A trace is {\em differentiated} if all its
input occurrences are pairwise different.
A {\em renaming} is any function $\renaming: \dset \mapsto \dset$
on the domain of data values.
A renaming $\renaming$ can be applied to trace $\trace$, resulting in the
trace $\renaming(\trace)$, where each
data value $\ditem$ in $\trace$ has been replaced by $\renaming(\ditem)$. 
%
%% \begin{definition}
%% \label{def:data-independence}
A set $\traceset$ of traces is {\em data independent} if for any trace $\trace\in\traceset$ the  following two conditions hold:
\begin{itemize}
\item
$\renaming(\trace)\in\traceset$ for any renaming $\renaming$, and 
\item
there exists a differentiated trace $\trace_d\in\traceset$ with $\renaming(\trace_d)=\trace$ for some renaming $\renaming$.
%% \qed
\end{itemize}
%% \end{definition}
We say that a program is {\em data independent} if the set of its
traces is data independent.
A program, like the one in Figure~\ref{fig:MS}, 
can typically be shown to be data independent by a simple syntactic 
analysis that checks that data values are not
manipulated or tested, but only copied.
%% check. Property (i) holds if
%% the only operations on data variables are copying an input value to a data variable, or copying
%% the value of a data variable to an output or to another data variable, without
%% reading an unitialized variable.
%% %
%% Property (ii) follows by observing that given a trace, pairwise different input values can be fed to the data structures by appending to the input data values the value of 
%% a counter that is systematically incremented. Property (i)
%% then ensures that the resulting trace can be renamed to the original trace.
In a similar manner, a correctness property % $\traceprop$ is data independent if 
% it is closed under renamings. In other words, $\trace\in\traceprop$ implies
% $\renaming(\trace)\in\traceprop$ for any renaming $\renaming$.
%
is {\em data independent} if the set of traces that it specifies
is data independent. 
From these definitions, the following theorem follows directly.

\begin{theorem}
\label{thm:differentiated}
A data independent program satisfies a data independent property
iff its differentiated traces satisfy the property.
\qed
\end{theorem}
%Thus, when checking that a data independent program satisfies a
%data independent property, it suffices to consider the
%differentiated  traces of the program. Hence,  
%an observer for a data independent property need only accept the
%differentiated traces that violate the property.
%This means that whenever a data value occurs in two input occurrences of
%a trace, the observer can stop checking (i.e., move to
%a non-accepting sink state), since the trace will anyway be ignored.
%
Thus, when checking that a data independent program satisfies a
data independent property, 
it suffices to check that all differentiated traces of the program 
belong to the property. 
Hence, an observer for a data independent property need only accept the
differentiated traces that violate the property. It should not accept any
(differentiated or non-differentiated) trace that satisfies it. 
%

Note that the set of traces of a set is {\em not}
data independent, e.g., since it contains a trace
where two different data values are inserted, but {\em not} its renaming
which inserts the same data value twice. This is not a problem,
since the set of {\em all} traces of a set can be specified by observers,
without using a data independence argument.

\begin{wrapfigure}{r}{.71\textwidth}
\vspace{-30pt}
\begin{center}
\begin{tikzpicture}[%
  remember picture%
  ,framed%
  ,initial/.style=initial by arrow%
  ]
%
%\draw (-1,0) node (order) {${\bigvee}_{\dred\neq\dblue}$};
%
\node[obsstate=green] (o_0)  {$s_0$};
\node[obsstate, right=2cm of o_0] (o_1) {$s_1$};
\node[obsstate, right=2cm of o_1] (o_2) {$s_2$};
\node[obsstate=red, right=2cm of o_2, accepting] (o_3) {$s_3$};
% \node[below=8mm of o_0, xshift=-3mm, anchor=west](guard) {\scriptsize{$guard=\left(\pvar\neq\obvar_1 \land \pvar\neq\obvar_2 \land \pvar\neq {\tt empty}\right)$}};
% \node[right=3mm of guard] {\scriptsize{$guard'=\left(\pvar\neq\obvar_1 \land \pvar\neq\obvar_2\right)$}};

\node[anchor=center](guard) at (4,-1) {\scriptsize{
    $\begin{array}{l} guard=\left(\pvar\neq\obvar_1 \land \pvar\neq\obvar_2 \land \pvar\neq {\tt empty}\right) \\
      guard'=\left(\pvar\neq\obvar_1 \land \pvar\neq\obvar_2\right)
    \end{array}$}};

\path[->] (o_0) edge [loop above] node[anchor=south west] {\mbox{\scriptsize $\begin{array}{c}
                                              \aletter{enq(\pvar)}{guard}\\
                                              \aletter{deq(\pvar)}{guard'}
                                            \end{array}$}} () 
                (o_0) edge node [below] {\mbox{\scriptsize $\aletter{enq(\pvar)}{\pvar=\obvar_1}$}} (o_1)
                (o_1) edge [loop above] node[anchor=south west] {\mbox{\scriptsize $\begin{array}{c}
                                              \aletter{enq(\pvar)}{guard}\\
                                              \aletter{deq(\pvar)}{guard}
                                            \end{array}$}} ()
                (o_1) edge node [below] {\mbox{\scriptsize $\aletter{enq(\pvar)}{\begin{array}{c} \pvar=\obvar_2 \\ \pvar\neq\obvar_1\end{array}}$}} (o_2)
                (o_2) edge [loop above] node[anchor=south west] {\mbox{\scriptsize $\begin{array}{c}
                                              \aletter{enq(\pvar)}{guard}\\
                                              \aletter{deq(\pvar)}{guard}
                                              \end{array}$}} ()
                (o_2) edge node [below] {\mbox{\scriptsize $\aletter{deq(\pvar)}{\pvar=\obvar_2}$}} (o_3);
\end{tikzpicture}
\tikz[remember picture,overlay]{\node[property,above=8mm of o_3,rotate=-10]{\sc Fifo};}
\end{center}
\vspace{-18pt}
\caption{An observer to check that {\sc Fifo} ordering is respected. 
All unmatched abstract events, for example $\aletter{deq(\pvar)}{\pvar=\obvar_1}$ at location $s_1$, send the observer to a sink state.
}
\label{fig:fifo:observer}
\vspace{-8mm}
\end{wrapfigure}

%% A program, like the one in Figure~\ref{fig:Treiber}
%% can typically be shown to be data independent by a simple sufficient
%% check. Property (i) holds if
%% the only operations on data variables are copying an input value to a data variable, or copying
%% the value of a data variable to an output or to another data variable, without
%% reading an uninitialized variable.
%% %
%% Property (ii) follows by observing that given a trace, pairwise different input values can be fed to the data structures by appending to the input data values the value of 
%% a counter that is systematically incremented. Property (i)
%% then ensures that the resulting trace can be renamed to the original trace.

The key observation is now that the differentiated traces of queues and
stacks can be specified succinctly by observers with a small
number of variables.
%% In the appendix, we provide a standard natural specification of a queue,
%%  that data independent data structures
%% are queues or stacks precisely when all differentiated traces verify 
%% properties captured by simple observers.
%
%% and prove, in Lemma~\ref{lem:observational}, 
In the case of a FIFO queue, its differentiated traces are precisely those
that satisfy the following four properties
%% For the case of queues, we show that
%% a data independent data structure is a queue
%% precisely when all differentiated traces satisfy
%% the following properties
for all data values $d_1$ and $d_2$.
%% \begin{figure}
%\begin{minipage}{.9\textwidth}
\begin{description}
\item[] {\sc No Creation:} $d_1$ must not be dequeued before it is enqueued % i.e., data cannot be created,
\item[] {\sc No Duplication:} $d_1$ must not be dequeued twice, %i.e., data cannot be duplicated,
\item[] {\sc No Loss:} {\tt empty} must not be returned if $d_1$ has been
enqueued but not dequeued, %i.e., data cannot be lost
\item[] {\sc Fifo:}   $d_2$ must not be dequeued before $d_1$
  if it is enqueued after $d_1$ is enqueued.
%i.e., FIFO order is respected, 
\end{description}
%% \caption{ Differentiated traces of a queue are exactly those verifying the four properties above 
%% for all pairs of different data values $d_1$ and $d_2$}
%% %\end{minipage}
%% \end{figure}
%
Each such property can be specified by an observer with 
one or two variables.
If the property is violated by some
specific data values $d_1$ and $d_2$, then there
is some run of the observer,
in which the initial values of the variables are $d_1$ and $d_2$,
which leads to an accepting state.
Figure~\ref{fig:fifo:observer} shows an observer for the {\sc Fifo} property.

We can also
%% Lemma~\ref{lem:observational} also
provide an analogous characterization of the
differentiated traces of a stack.


%% in an atomic statement. For example, the assignment
%% {\tt b:=CAS(\&S,t,x)} is instrumented to generate an
%% abstract event $\mbox{\tt foo}(\mbox{\tt d})$ for a local variable
%% {\tt d} by the 
%% $\atomic{\mbox{\tt b:=CAS(\&S,t,x)}; \mbox{\tt emit}(\mbox{\tt foo}(\mbox{\tt d}))}$
%% construct.
%% To verify that no trace of the program is accepted by $\auto$, we
%% form, as in the automata-theoretic approach~\cite{VW:modelchecking},
%% the cross-product of the program and the observer, synchronizing
%% on abstract events, and check that
%% this cross-product cannot reach a configuration where the observer
%% is in an accepting state.


%% Using observers, we can characterize universally quantified properties 
%% on program traces. If such a property is violated by some specific 
%% values on the data of the observable events,
%% then there will be some initial configuration of the observer that detects this violation. 
%% %
%% Using a small number of observers, depicted in \cite{technicalreport}, 
%% we characterize all traces allowed by sets data structures.
%% %
%% For example, Figure \ref{fig:dinsertion} captures all traces where
%% some arbitrary data value is inserted twice.
%
%
%
% In addition to the trace observers, we define a property 
% of data independence that makes the verification tractable. 
% Under data independance, it is sufficient
% to specify only traces of a specific form, namely traces in which all input data values
% are distinct. If both the program and the correctness property
% are data independent, then an observer need only 
% consider traces where each value assumed by the observer variables
% is inserted once into the data structure.
% %% if the correctness property is symmetric wrp.\ to different data values,
% %% then the trace observer need only consider a finite number of distinct
% %% data values.
% % why there is no need to check
% % trace properties over all (possibly infinite) data values by showing 
% % that it is sufficient to check generated traces over a finite number of data values.
% %
% The techniques are illustrated by proposing trace observers using finite
% sets of data variables
% to specify familiar data structures like queues, stacks, and sets.
%
%
%Trace observers introduce data variables from an unbounded domain. %, and thus complicate the analysis. 
%which allows to restrict the specification to traces in which all input values are distinct\TODO{this is the first time 
%the term input is used. Introduce it earlier?}.
