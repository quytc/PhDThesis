\section{Algorithm}
\label{sec:alg}
%
In this section, we instantiate the scheme of Section~\ref{sec:scheme} to derive 
an algorithm for solving \acovproblem.
%
We do that by introducing an effective and well quasi-ordered constraint system.
%

Throughout this section, we assume a parameterized tree system $\parsys=\parsystuple$
and the induced approximate transition system $\atsof{\parsys}=\tuple{\confs,\atransrel{}{}}$.
%
We define a constraint to be a tree in $\trees{\states}$.
%
Although we use the same syntax as for configurations, constraints are interpreted 
differently.
%
More precisely, given a constraint $\constr$, we let $\denotationof{\constr}=\setcomp{\conf
\in\confs}{\constr\preceq\conf}$. 
%

An aspect of our constraint system is that each constraint characterizes a set of 
configurations which is upward closed with respect to $\preceq$.
%
Conversely (by Higman's Lemma~\cite{Higman:divisibility}), any upward closed set 
$\fconfs$ of configurations can be characterized as $\denotationof{\constrs_\fconfs}$ 
where $\constrs_\fconfs$ is a finite set of constraints.
%
In this manner, \acovproblem{} is reduced to checking the reachability of a finite set of 
constraints.
%

Below we show effectiveness and well quasi-ordering of our constraint system, meaning that 
we obtain an algorithm for solving \acovproblem{}.
%
First, observe that the entailment relation can be computed in a straightforward manner since 
for any constraints $\constr,\constr'$, we have  $\constr\entailed\constr'$ iff $\constr\preceq\constr'$.
%

In order to check the initial condition, we rely on previous works on \emph{regular tree languages}~\cite{CoDa:book} 
and provide a sufficient condition on $\init$ which guarantees effectiveness of $\init\intersect\denotationof{
\constr}=\emptyset$ for any constraint $\constr$.
%
More precisely, we require that the set $\init$ can be characterized by a regular tree
language.
%

For the computation of Pre we rely on the following result.
%
\begin{lemma}
\label{lem:pre:comp}
For any constraint $\constr$, the set of constraints $\pre{\constr}$ is computable and finite.
% for an approximate transition system.
\end{lemma}
%

It was shown in~\cite{kruskal} that the embedding relation on trees $\preceq$ is a well quasi-order 
(Kruskal's theorem).
%
This combined with results in~\cite{Parosh:Bengt:Karlis:Tsay:general} guarantee termination of 
our scheme when instantiated on the constraints we have defined above.
%


