\section{Scheme}
\label{sec:scheme}
%
In this section, we recall a generic scheme from~\cite{Parosh:Bengt:Karlis:Tsay:general}
for performing symbolic backward reachability analysis.
%
The scheme in question is based on symbolic representations of infinite sets of 
configurations called \emph{constraints}.
%
Throughout this section, we fix a transition system $\tssystem=\gtstuple$ and a set 
$\init\subseteq\confs$ of initial configurations.
%

\paragraph{\bf Constraint Systems}
%
A \emph{constraint system} $\constrsys$ relative to the transition system $\tssystem$ is
a set whose elements are called constraints and can be finitely encoded, such that there 
is a function $\denotationof{\cdot}:\constrsys\to 2^\confs$.
%
For a finite set $\constrs$ of constraints, we let $\denotationof{\constrs}=\Union_{
\constr\in\constrs}\denotationof{\constr}$.
%
We say that a set $D\subseteq\confs$ is \emph{computable} or \emph{representable} 
(in the constraint system $\constrsys$) if it is possible to compute a finite set 
of constraints $\constrs\subseteq\constrsys$ such that $D=\denotationof{\constrs}$.
%

We define an \emph{entailment relation $\entailed$} on constraints, where $\constr_1
\entailed\constr_2$ iff $\denotationof{\constr_2}\subseteq \denotationof{\constr_1}$.
%
For sets $\constrs_1,\constrs_2$ of constraints, abusing notation, we let $\constrs_1
\entailed\constrs_2$ denote that for each $\constr_2\in\constrs_2$ there is a $\constr_1
\in\constrs_1$ with $\constr_1\entailed\constr_2$.
%
Notice that $\constrs_1\entailed\constrs_2$ implies that $\denotationof{\constrs_2}
\subseteq\denotationof{\constrs_1}$.
%

For a constraint $\constr$, we let $\pre{\constr}$ be the set of constraints, 
such that $\denotationof{\pre{\constr}}=\setcomp{\conf}{\exists\conf'\in\denotationof{
\constr}.\;\conf\gtransrel{}\conf'}$.
%
In other words, $\pre{\constr}$ characterizes the set of configurations from
which we can reach a configuration in $\constr$ through the application of a 
single rewrite rule.
%
Such a set does not necessarily exist, nevertheless, for our class of systems, we will 
show that such a set always exists and is in fact computable.
%
For a set $\constrs$ of constraints, we let $\pre{\constrs}=\Union_{\constr\in
\constrs}\pre{\constr}$.
%

\paragraph{\bf Symbolic Backward Reachability}
%
We present a scheme for a symbolic algorithm which, given a finite set $\constrs_\fconfs$ 
of constraints, checks whether $\init\gtransrel{*}\denotationof{\constrs_\fconfs}$.
%

In the scheme, we perform a backward reachability analysis, generating a sequence 
$\sequence{\constrs}{i}:\constrs_0\sqsupseteq\constrs_1\sqsupseteq\constrs_2\sqsupseteq
\cdots$ of finite sets of constraints such that $\constrs_0=\constrs_F$, and $\constrs_{i+1}=
\constrs_i\union\pre{\constrs_i}$.
%
Since $\denotationof{\constrs_0}\subseteq\denotationof{\constrs_1}\subseteq
\denotationof{\constrs_2}\subseteq\cdots$, the procedure terminates when we reach a 
point $j$ where $\constrs_j\entailed\constrs_{j+1}$.
%
Consequently, $\constrs_j$ characterizes the set of all predecessors of $\denotationof{
\constrs_\fconfs}$.
%
This means that $\init\gtransrel{*}\denotationof{\constrs_F}$ iff $\init\intersect
\denotationof{\constrs_j}\neq\emptyset$.
%

Observe that, in order to implement the scheme (i.e.\ transform it into an algorithm), we 
need to be able to (i) compute Pre; (ii) check for entailment between constraints; and 
(iii) check for emptiness of $\init\intersect\denotationof{\constr}$ for any constraint 
$\constr$.
%
A constraint system satisfying these three conditions is said to be \emph{effective}.
%
Moreover, in~\cite{Parosh:Bengt:Karlis:Tsay:general}, it is shown that termination is 
guaranteed in case the constraint system is \emph{well quasi-ordered (WQO)} with respect 
to $\entailed$, i.e.\ for each infinite sequence $\constr_0,\constr_1,\constr_2,\ldots$ 
of constraints, there are $i<j$ with $\constr_i\entailed\constr_j$.
%
