\section{Case Studies}
\label{sec:examples}
%

In this section, we provide descriptions of two tree protocols we have analyzed using our method.
%
For each protocol, we define the corresponding parameterized tree system model and we give the 
sets of unsafe ($\fconfs$) and initial ($\init$) configurations.
%

\ignore{
In the formal description of the protocols, we make use of an extension of the definition provided in 
Section~\ref{sec:parsys}.
%
More precisely, in the definition of a parameterized tree system, we assume that we are given a finite set 
of variables ranging over finite domains.
%

Formally, a parameterized system $\parsys$ is a tuple $\parsysextuple$, where $\vars$ is a finite set of 
\emph{variable definitions} and $\rwrules$ is finite set of rewrite rules.
%
A variable definition is a pair $\tuple{\var,\dom{\var}}$, where $\var$ is the variable name and $\dom{\var}$ 
is a finite set representing the possible values which can be taken by $\var$.
%
We redefine rewrite rules as follows.
%
A rule $\rwrule\in\rwrules$ is a tree in $\trees{\guards_\vars\times\assigns_\vars}$.
%
$\guards_\vars$ is the set of conjuctions of the form $\bigwedge_{i=1}^n\var_i=e_i$ ($n\in\nat$) where for each 
$i:1\le i\le n$, $\var_i$ is a varibale and $e_i\in\dom{\var_i}$ is a value in the domain of $\var_i$.
%
$\assigns_\vars$ is the set of \emph{operations on $\vars$} each of the the form $\var_1=e_1;\ldots;\var_m=e_m$ 
(for $m\in\nat$) where for any $j:1\le j\le m$, $\var_j$ is a variable in $\parsys$ and $e_j\in\dom{\var_j}$.
%

Observe that by finitness of $\vars$, the new definition is equivalent to the one introduced earlier 
(Section~\ref{sec:parsys}).
%
Therefore, the definitions of the induced transition system, the approximation, the constraint system and 
the computation of Pre can be modified accordingly in a straightforward manner.
%

In the description of the examples below, we use the extended definition in order to make the presentation more 
clear.
}

\subsection{The Tree-arbiter Protocol}

The protocol supervises the access to a shared resource of a set of processes arranged in 
a tree topology.
%
The processes competing for the resource reside in the leaves.
%

A process in the protocol can be in state \emph{idle} ($i$), \emph{requesting} ($r$),  
\emph{token} ($t$) or \emph{below} ($b$).
%
All the processes are initially in state $i$.
%
A node is in state $b$ whenever it has a descendant in state $t$.
%
When a leaf is in state $r$, the request is propagated upwards
until it encounters a node which is aware of the presence of the token (i.e. a
node in state $t$ or $b$).
%
A node that has the token (in state $t$) can choose to pass it upwards or pass it downwards 
to a requesting child (node in state $r$).
%

We model the tree-arbiter protocol with a parameterized tree system $\parsys=\parsystuple$ 
where $\states=\setcomp{\state_s^n}{s\in\set{i,r,t,b}\band n\in\set{leaf,inner,root}}$ and 
$\rwrules$ is as depicted in the figure below (figure~\ref{fig:arbiter:rules}).
%
Observe that in the definition of $\states$, we use the scripts $s$ and $n$ to model respectively 
the state and the nature (leaf, inner or root) of the nodes.
%
In the definition of the rules, we will drop the script(s) whenever we mean that it is arbitrary 
(it can take any value).
%

The rules to model this protocol are as follows: 
%
2 rules to propagate the
request upwards, 
%
2 rules to propagate the token downwards, 
%
2 rules to propagate the token upwards and one rule to initiate a request from a leaf. 
%

\begin{figure}[htb]
\centering
\begin{tikzpicture}[show background rectangle]
  \begin{scope}[scale=0.5]
    \tikzstyle{every node}=[draw,rounded corners,minimum width=1cm]
    \tikzstyle{level 1}=[sibling distance=2.5cm,level distance=2cm]
    %\tikzstyle{level 2}=[sibling distance=1.8cm,level distance=1.3cm]
    \node[shift={(0cm,0cm)}] {$\state_i$/$\state_r$} child{edge from parent[draw=none]} child {node {$\state_r$}};
    \node[shift={(3cm,0cm)}] {$\state_i$/$\state_r$} child{node {$\state_r$}} child{edge from parent[draw=none]};
    \node[shift={(0cm,-2.2cm)}] {$\state_t$/$\state_b$} child {edge from parent[draw=none]} child {node {$\state_r$/$\state_t$}};
    \node[shift={(3cm,-2.2cm)}] {$\state_t$/$\state_b$} child {node {$\state_r$/$\state_t$}} child {edge from parent[draw=none]};
    \node[shift={(0cm,-4.4cm)}] {$\state_b$/$\state_t$} child{edge from parent[draw=none]} child {node {$\state_t$/$\state_i$}};
    \node[shift={(3cm,-4.4cm)}] {$\state_b$/$\state_t$} child {node {$\state_t$/$\state_i$}} child {edge from parent[draw=none]};
    \node[shift={(1.5cm,-6.6cm)}] {$\state_i^{leaf}$/$\state_r$};
  \end{scope}
  \begin{scope}[color=white]
    \draw (0cm,-1.5cm) -- +(3cm,0cm);
    \draw (0cm,-3.7cm) -- +(3cm,0cm);
    \draw (0cm,-5.9cm) -- +(3cm,0cm);
    \draw (1.5cm,0cm) -- +(0cm,-5.9cm);
  \end{scope}
\end{tikzpicture}
\caption{The rewrite rules for the tree-arbiter protocol.
%
We mention here that there are more rules in the model we have verified.
%
For example, the rule in the top-left corner is represented in the concrete model by $2$ rules, each of which 
corresponds to a particular combination of the natures of the parent and child nodes:
%
For the parent there are $2$ possibilities ($\state_i^{inner}/\state_r^{inner}$ and $\state_i^{root}/\state_r^{root}$) 
while for the child, there are $2$ ($\state_r^{inner}$ and $\state_r^{leaf}$).}
\label{fig:arbiter:rules}
\end{figure}

The set of bad constraints $\fconfs$ is represented by trees where at least two
processes (i.e. two leaves) obtain the token (i.e. in state $\state_t^{leaf}$).
%
The set of initial configurations $\init$ contains all trees where the leaf nodes
are either idle or requesting, inner nodes are idle, and the root has the token.
%
\begin{center}
  \begin{tikzpicture}[show background rectangle]
    \begin{scope}[scale=0.5]
      \tikzstyle{every node}=[draw,rounded corners,minimum width= 1cm]
      \tikzstyle{level 1}=[sibling distance=3cm,level distance=2cm]
      \node {$\state$} child {node {$\state_t^{leaf}$}} child {node {$\state_t^{leaf}$}};
    \end{scope}
    \end{tikzpicture}
\end{center}
%



%
\subsection{The IEEE 1394 Tree Identification Protocol}
%

The 1394 High Performance serial bus~\cite{ieee:1394} is used to transport digitized video and
audio signals within a network of multimedia systems and devices. 
% 

The tree identification protocol is used in one of the phases implementing the IEEE 1394 protocol.
%
More precisely, it is run after a bus reset in the network and leads to the election of a unique 
leader node.
%

In this section, we consider a version working on tree topologies.
%
Furthermore, we assume that (i) each inner node is connected to 3 neighbors, (ii) the root is connected 
to 2 neighbors, and (ii) communication is atomic.
%

Initially, all nodes are in state \emph{undefined} ($u$). 
%
We identify two steps in the protocol depending on the number $n$ of neighbors which are 
still in state $u$.
%
If $n>1$, the node waits for (``be my parent'') requests from its neighbors. 
%
If $n=1$, the node sends a request to the remaining neighbor in state $u$. 
%
Observe that we implicitly assume that the leaf nodes are the first to communicate with their neighbors.
%

Formally, we derive a parameterized tree system model $\parsys=\parsystuple$ as follows.
%
We define the set of states by $\states=\setcomp{\state_s^n}{s\in\set{u,c,l}\band n\in\set{leaf,inner,root}}$ 
where the scripts $s$ and $n$ describe respectively the state and the nature of the node.
%
In the definition of the state ($s$), the letters $u$, $c$ and $l$ stand respectively for \emph{undefined}, 
\emph{child} and \emph{leader}.
%
In a similar manner to the previous section, we drop the script(s) whenever we mean that it can take any value 
(see caption of Figure~\ref{fig:arbiter:rules}).
% 

The rewrite rules $\rwrules$ are described below.
%
\begin{itemize}
\item The leaves initiate the communications:

\begin{center}
  \begin{tikzpicture}[show background rectangle]
    \tikzstyle{level 1}=[level distance=1cm,sibling distance=2cm]
    \tikzstyle{every node}=[draw,rounded corners,minimum width=1cm]
    \node {$\state_u$}
    child{node{$\state_u^{leaf}$/$\state_c$}}
    child[level distance=1cm,sibling distance=0cm]{node[draw=none]{} edge from parent[draw=none]};
  \end{tikzpicture}
%
\begin{tikzpicture}[show background rectangle]
    \tikzstyle{level 1}=[level distance=1cm,sibling distance=2cm]
    \tikzstyle{every node}=[draw,rounded corners,minimum width=1cm]
    \node {$\state_u$}
    child[level distance=1cm,sibling distance=0cm]{node[draw=none]{} edge from parent[draw=none]}
    child{node{$\state_u^{leaf}$/$\state_c$}};
\end{tikzpicture}
\end{center}

\item The inner nodes become children or wait for requests:

\begin{center}
  \begin{tikzpicture}[grow cyclic,show background rectangle]
    \tikzstyle{level 1}=[level distance=1cm,sibling angle=120]
    \tikzstyle{every node}=[draw,rounded corners,minimum width=1cm]
    \node {$\state_u$/$\state_c$}
    child {node {$\state_u$}}
    child[level distance=1.3cm] {node {$\state_c$}}
    child {node {$\state_c$}};
  \end{tikzpicture}
  % 
  \begin{tikzpicture}[grow cyclic,show background rectangle]
    \tikzstyle{level 1}=[level distance=1cm,sibling angle=120]
    \tikzstyle{every node}=[draw,rounded corners,minimum width=1cm]
    \node {$\state_u$/$\state_c$}
    child {node {$\state_c$}}
    child[level distance=1.3cm] {node {$\state_u$}}
    child {node {$\state_c$}};
  \end{tikzpicture}
  % 
  \begin{tikzpicture}[grow cyclic,show background rectangle]
    \tikzstyle{level 1}=[level distance=1cm,sibling angle=120]
    \tikzstyle{every node}=[draw,rounded corners,minimum width=1cm]
    \node {$\state_u$/$\state_c$}
    child {node {$\state_c$}}
    child[level distance=1.3cm] {node {$\state_c$}}
    child {node {$\state_u$}};
  \end{tikzpicture}
\end{center}

\item The leader is chosen:

\begin{center}
\begin{tikzpicture}[grow cyclic,show background rectangle]
\tikzstyle{every node}=[draw,rounded corners,minimum width=1cm]
\tikzstyle{level 1}=[level distance=1cm,sibling angle=120]
\node {$\state_u$/$\state_l$}
    child {node {$\state_c$}}
    child[level distance=1.6cm] {node {$\state_c$}}
    child {node {$\state_c$}};
\end{tikzpicture}
%
\begin{tikzpicture}[grow cyclic,show background rectangle]
\tikzstyle{every node}=[draw,rounded corners,minimum width=1cm]
\tikzstyle{level 1}=[level distance=1cm,sibling angle=120]
\node {$\state_c$}
    child{node {$\state_u^{leaf}$/$\state_l$}}
    child[level distance=0cm]{node[draw=none]{} edge from parent[draw=none]}
    child{node[draw=none]{} edge from parent[draw=none]};
\end{tikzpicture}
%
\begin{tikzpicture}[grow cyclic,show background rectangle]
\tikzstyle{every node}=[draw,rounded corners,minimum width=1cm]
\tikzstyle{level 1}=[level distance=1cm,sibling angle=120]
\begin{scope}[rotate=60]
  \node {$\state_c$}
    child{node {$\state_u^{leaf}$/$\state_l$}}
    child{node[draw=none]{} edge from parent[draw=none]}
    child[level distance=0cm]{node[draw=none]{} edge from parent[draw=none]};
\end{scope}
\end{tikzpicture}
\end{center}
\end{itemize}
%

The set of initial configurations $\init$ is represented by trees where all nodes
are in state undefined, and the set of bad constraints $\fconfs$ is represented by 
trees where at least 2 leaders are elected.
%
\begin{center}
\begin{tikzpicture}[level distance=1cm,show background rectangle]
\tikzstyle{every node}=[draw,rounded corners,minimum width=1cm]
\node {$\state_l$} child {node {$\state_l$}}
child[sibling distance=0cm]{node[draw=none]{} edge from parent[draw=none]};
\end{tikzpicture}
%
\begin{tikzpicture}[level distance=1cm,show background rectangle]
\tikzstyle{every node}=[draw,rounded corners,minimum width=1cm]
\node {$\state_l$}
child[sibling distance=0cm]{node[draw=none]{} edge from parent[draw=none]}
child {node {$\state_l$}};
\end{tikzpicture}
%
\begin{tikzpicture}[level distance=1cm,show background rectangle]
\tikzstyle{every node}=[draw,rounded corners,minimum width=1cm]
\node {$\state$} child {node {$\state_l$}} child {node {$\state_l$}};
\end{tikzpicture}
\end{center}
%
