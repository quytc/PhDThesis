\section{Approximation}
\label{sec:approx}
%
In this section, we introduce an over-approximation of the transition relation 
of a parameterized tree system.
%

In Section~\ref{sec:ts}, we mentioned that each parameterized tree system $\parsys=
\parsystuple$ induces a transition system $\tsof{\parsys}=\tstuple$.
%
A parameterized tree system $\parsys$ also induces an \emph{approximate} transition system
$\atsof{\parsys}=\tuple{\confs,\atransrel{}}$, where the set $\confs$ of configurations 
is identical to the one in $\tsof{\parsys}$ and the transition relation $\atransrel{}$ 
is defined below.
%

First, we define a special operation on trees needed in order to describe the semantics of 
$\atransrel{}$.
%

\paragraph{\bf Tree Subtraction}
%
In this paragraph, we fix two trees $\tree=(\nodes,\labeling),\tree'=(\nodes',\labeling')\in
\trees{\alphabet}$ such that $\tree'\preceq_f\tree$ for some embedding $f$.
%
We define $\tree\sminus_f\tree'$ to be the tree $\tree''$ obtained from $\tree$ by performing
a sequence of operations described below.
%
First, we enumerate the nodes of $\tree'$ in a bottom-up fashion.
%
Formally, let $\{\node_i\}_{1\le i\le\sizeof{\nodes'}}$ be an enumeration of the set $\nodes'$ of
nodes in $\tree'$
such that for any $i,j:1\le i\neq j\le \sizeof{\nodes'}$, $\node_i\spref\node_j$ implies that
$j < i$.
%
In other words, if $\node_j$ is a descendant of $\node_i$ in $\tree'$, then
$\node_j$ occurs earlier than $\node_i$ in the enumeration.
%
Based on the enumeration, we define a sequence of trees $\{\tree_i\}_{1\le i\le\sizeof{\nodes'}-1}$
as follows.
%
We let $\tree_1:=\tree$.
%
For any $i:1\le i\le\sizeof{\nodes'}-2$, we denote by $\node^p_i$ the parent of $\node_i$, i.e.\ $\node^p_i\concat b=
\node_i$ for some $b$; and we define
%
\[
\tree_{i+1}:=\tree_i\append\left(f(\node_i^p)\concat b,\tree(f(\node_i))\right)\text{.}
\]
%
Finally, we let $\tree'':=\tree_{\sizeof{\nodes'}-1}$.
%
In other words, we go through the nodes of $\tree'$ one by one in a bottom-up manner.
%
For each node $\node_i$ and its parent $\node^p_i$ in $\tree'$ (say $\node_i^p\concat b=\node_i$
for some $b$), we consider their images $f(\node_i^p)$ and $f(\node_i)$ in $\tree$.
%
We replace the subtree rooted in the child of the image
$f(\node_i^p)\concat b$ by the one rooted in the image $f(\node_i)$ (see Figure~\ref{fig:subtraction}).
%
Notice that the resulting tree $\tree''$ and the trees $\tree',\tree$ are related by
$\tree'\includedin\tree''\preceq\tree$.
%
In the sequel, we denote by $\minusinc{f}$ the inclusion of $\tree'$ in $\tree''$ such that 
$\minusinc{f}(\emptyword)=f(\emptyword)$ (such a function exists and is unique by 
the definition above).
%
\begin{figure}
\centering
  \begin{tikzpicture}[show background rectangle]
    \tikzstyle{level 1}=[sibling distance=2.5cm,level distance=1.5cm]
    \tikzstyle{level 2}=[sibling distance=1.8cm,level distance=1.3cm]
    \tikzstyle{level 3}=[sibling distance=1.6cm,level distance=1.1cm]
    \tikzstyle{level 4}=[sibling distance=1.4cm,level distance=0.9cm]
    \begin{scope}[scale=0.4,yshift=-8mm]
      \node[xshift=-7mm,shape=circle,fill=white,draw=none] {$\tree'$};
      \node (rectangleD2){}[fill] (-2mm,-2mm)rectangle +(4mm,4mm)
      child {node(circleD2){}[fill] circle(2mm)}
      child {
        node(crossD2){} [fill] ++(-2mm,2mm)--++(4mm,-4mm)--++(0mm,4mm)--++(-4mm,-4mm)--cycle
        child {
          node(triangleD2){}[fill] ++(-2mm,-2mm)--++(2mm,4mm)--++(2mm,-4mm)--cycle
        }
        child{node {} edge from parent[draw=none]}
      }
      ;
    \end{scope}
    \begin{scope}[xshift=1.3cm,yshift=-13mm]
      \node[draw=none,scale=2] {$\preceq_f$};
    \end{scope}
    \begin{scope}[scale=0.4,xshift=7cm]
      \node[xshift=-7mm,shape=circle,fill=white,draw=none] {$\tree$};
      \node (rectangleA2){}[fill] (-2mm,-2mm)rectangle +(4mm,4mm)
      child {node {.}
        child
        child {node(circleA2){}[fill] circle(2mm)}}
      child {
        node {.} child child {
          node(crossA2){} [fill] ++(-2mm,2mm)--++(4mm,-4mm)--++(0mm,4mm)--++(-4mm,-4mm)--cycle
          child {
            node {.}
            child {node {} edge from parent[draw=none]}
            child { node(triangleA2){}[fill] ++(-2mm,-2mm)--++(2mm,4mm)--++(2mm,-4mm)--cycle}
          }
          child
        }
      }
      ;
    \end{scope}
  \end{tikzpicture}
  %
  \vspace{2mm}
  
  \begin{tikzpicture}[show background rectangle]
    \begin{scope}[scale=0.4]
    \tikzstyle{level 1}=[sibling distance=2.5cm,level distance=1.5cm]
    \tikzstyle{level 2}=[sibling distance=1.8cm,level distance=1.3cm]
    \tikzstyle{level 3}=[sibling distance=1.6cm,level distance=1.1cm]
    \node[xshift=-7mm,shape=circle,fill=white,draw=none] {$\tree_1$};
    \node (rectangleA2){}[fill] (-2mm,-2mm)rectangle +(4mm,4mm)
    child {node (A2){.}
      child{node(A1) {}}
      child {node(circleA2){}[fill] circle(2mm)}}
    child {
      node {.} child child {
        node(crossA2){} [fill] ++(-2mm,2mm)--++(4mm,-4mm)--++(0mm,4mm)--++(-4mm,-4mm)--cycle
        child {
          node {.}
          child {node {} edge from parent[draw=none]}
          child { node(triangleA2){}[fill] ++(-2mm,-2mm)--++(2mm,4mm)--++(2mm,-4mm)--cycle}
        }
        child
      }
    }
    ;
    \draw (A1)--(A2) node[coordinate,midway] (midA){};
    \draw[dashed,help lines,line width=1pt,rotate=55] (midA) ellipse (10mm and 4mm);
    \draw[dotted,line width=1pt,->,shorten <=1pt, shorten >=1pt] (circleA2) .. controls +(70:15mm) .. (A2);
    \end{scope}
  \end{tikzpicture}
  %
  \begin{tikzpicture}[show background rectangle]
    \begin{scope}[scale=0.4]
    \tikzstyle{level 1}=[sibling distance=2.5cm,level distance=1.5cm]
    \tikzstyle{level 2}=[sibling distance=1.8cm,level distance=1.3cm]
    \tikzstyle{level 3}=[sibling distance=1.6cm,level distance=1.1cm]
    \node[xshift=-7mm,shape=circle,fill=white,draw=none] {$\tree_2$};
    \node (rectangleB2){}[fill] (-2mm,-2mm)rectangle +(4mm,4mm)
    child {node(circleB2){}[fill] circle(2mm)}
    child {
      node {.} child child {
        node(crossB2){} [fill] ++(-2mm,2mm)--++(4mm,-4mm)--++(0mm,4mm)--++(-4mm,-4mm)--cycle
        child {
          node(midB) {.}
          child {node {} edge from parent[draw=none]}
          child { node(triangleB2){}[fill] ++(-2mm,-2mm)--++(2mm,4mm)--++(2mm,-4mm)--cycle}
        }
        child
      }
    }
    ;
    \draw[dashed,help lines,line width=1pt] (midB) ellipse (6mm and 3mm);
    \draw[dotted,line width=1pt,->,shorten <=1pt, shorten >=1pt] (triangleB2) .. controls +(180:1.5cm) .. (midB);
    \end{scope}
  \end{tikzpicture}
  %
  \begin{tikzpicture}[show background rectangle]
    \begin{scope}[scale=0.4]
    \tikzstyle{level 1}=[sibling distance=2.5cm,level distance=1.5cm]
    \tikzstyle{level 2}=[sibling distance=1.8cm,level distance=1.3cm]
    \tikzstyle{level 3}=[sibling distance=1.6cm,level distance=1.1cm]
    \node[xshift=-7mm,shape=circle,fill=white,draw=none] {$\tree_3$};
    \node (rectangleC2){}[fill] (-2mm,-2mm)rectangle +(4mm,4mm)
    child {node(circleC2){}[fill] circle(2mm)}
    child {
      node (C1){.} child{node(C2){}} child {
        node(crossC2){} [fill] ++(-2mm,2mm)--++(4mm,-4mm)--++(0mm,4mm)--++(-4mm,-4mm)--cycle
        child {
          node(triangleC2){}[fill] ++(-2mm,-2mm)--++(2mm,4mm)--++(2mm,-4mm)--cycle
          child {node {}  edge from parent[draw=none]}
        }
        child
      }
    }
    ;
    \draw (C1)--(C2) node[coordinate,midway] (midC){};
    \draw[dashed,help lines,line width=1pt,rotate=55] (midC) ellipse (11mm and 4mm);
    \draw[dotted,line width=1pt,->,shorten <=1pt, shorten >=1pt] (crossC2) .. controls +(70:2cm) .. (C1);
    \end{scope}
  \end{tikzpicture}
  %
  \begin{tikzpicture}[show background rectangle]
    \begin{scope}[scale=0.4]
    \tikzstyle{level 1}=[sibling distance=2.5cm,level distance=1.5cm]
    \tikzstyle{level 2}=[sibling distance=1.8cm,level distance=1.3cm]
    \tikzstyle{level 3}=[sibling distance=1.6cm,level distance=1.1cm]
    \node[xshift=-7mm,shape=circle,fill=white,draw=none] {$\tree_4$};
    \node (rectangleD2){}[fill] (-2mm,-2mm)rectangle +(4mm,4mm)
    child {node(circleD2){}[fill] circle(2mm)}
    child {
      node(crossD2){} [fill] ++(-2mm,2mm)--++(4mm,-4mm)--++(0mm,4mm)--++(-4mm,-4mm)--cycle
      child {
        node(triangleD2){}[fill] ++(-2mm,-2mm)--++(2mm,4mm)--++(2mm,-4mm)--cycle
        child {node {} edge from parent[draw=none] child { node{} edge from parent[draw=none]}}
      }
      child
    }
    ;
    \end{scope}
  \end{tikzpicture}
  \caption{In the first row, we give an example of two trees $\tree,\tree'$ satisfying 
$\tree'\preceq_f\tree$ for some embedding $f$.
%
In the second row, we give the sequence of trees used in the definition of $\tree\sminus_f\tree'$.
%
In each of the trees, the arrow shows where subtrees are re-rooted, while the nodes surrounded by 
a dashed line are those which are removed.}
  \label{fig:subtraction}
\end{figure}
%


\paragraph{\bf The Approximate Transition Relation}
%
Consider two configurations $\conf_1,\conf_2$ and a rule $\rwrule\in\rwrules$.
%
We write $\conf_1\atransrel{\rwrule}\conf_2$ to denote that there is an $f$ such that 
%
%% Typo corrected
%(i) $\tright{\rwrule}\preceq_f\conf_1$, and 
(i) $\tleft{\rwrule}\preceq_f\conf_1$, and 
%
(ii) $\conf_2=(\conf_1\sminus_f\tleft{\rwrule})
\rename_{\minusinc{f}}\tright{\rwrule}$.
%
Intuitively, starting from $\conf_1$ and an embedding $f$ of $\tleft{\rwrule}$ in $\conf_1$, we 
first remove all nodes in $\conf_1$ such that $\tleft{\rwrule}$ is included in the resulting 
configuration.
%
This is done by taking $\tleft{\rwrule}\sminus_f\conf_1$ and the inclusion $\minusinc{f}$.
%
Then we apply the rule $\rwrule$ and obtain $\conf_2$ from $\tleft{\rwrule}
\sminus_f\conf_1$ in a similar manner to how it is described in the previous section, i.e.\ by 
renaming the labels of the nodes in $\image{\minusinc{f}}$ according to $\tright{\rwrule}$ 
(see Figure~\ref{fig:approximation}).
%
We use $\conf_1\atransrel{}_1\conf_2$ if $\conf_1\atransrel{\rwrule}\conf_2$ for some 
$\rwrule\in\rwrules$.
%

\begin{figure}
\centering
  \begin{tikzpicture}[show background rectangle]
    \begin{scope}[scale=0.5]
    \tikzstyle{level 1}=[sibling distance=2cm,level distance=1.5cm]
    \tikzstyle{level 2}=[sibling distance=1.8cm,level distance=1.3cm]
    \tikzstyle{level 3}=[sibling distance=1.6cm,level distance=1.1cm]
    \tikzstyle{every node}=[draw,rounded corners,minimum width=4mm]
    \node[xshift=-10mm,shape=circle,fill=white,draw=none] {$\tree_1$};
    \node {$\state_u$}
    child {
      node[fill=white] {$\state_u$}
      child {
        node(r1) {$\state_u$}
        child {node[fill=white] {$\state_1$}}
        child {node(r2) {$\state_u$}}
      }
      child {node[fill=white] {$\state_1$}}
    }
    child {
      node {$\state_u$}
      child {edge from parent[draw=none]}
      child {node {$\state_0$}}
    }
    ;
%     \draw[dashed,help lines, line width=1pt,rounded corners] plot coordinates {%
%       (r1.north west) (r1.north east) (r2.north east) (r2.south east) (r2.south west) (r1.south west) (r1.north west)%
%     }; 
    \draw[dashed,help lines, line width=1pt] 
    ([xshift=-1mm]r1.south west) -- ([xshift=-1mm]r1.north west) -- %
    ([yshift=1mm]r1.north west) -- ([yshift=1mm]r1.north east) -- %
    ([xshift=1mm]r2.north east) -- ([xshift=1mm]r2.south east) -- %
    ([yshift=-1mm]r2.south east) -- ([yshift=-1mm]r2.south west) -- cycle; 
    \end{scope}
  \end{tikzpicture}
  %
  \begin{tikzpicture}
    \node {$\atransrel{\rwrule_4}$};
    \node[yshift=-15mm,draw=none]{};
  \end{tikzpicture}
  % 
  \begin{tikzpicture}[show background rectangle]
    \begin{scope}[scale=0.5]
    \tikzstyle{level 1}=[sibling distance=2cm,level distance=1.5cm]
    \tikzstyle{level 2}=[sibling distance=1.8cm,level distance=1.3cm]
    \tikzstyle{level 3}=[sibling distance=1.6cm,level distance=1.1cm]
    \tikzstyle{every node}=[draw,rounded corners,minimum width=4mm]
    \node[xshift=-10mm,shape=circle,fill=white,draw=none] {$\tree_2$};
    \node[fill=white] {$\state_u$}
    child {
      node[fill=white] {$\state_1$}
      child {
        node {$\state_1$}
        child {node[draw=none]{} edge from parent[draw=none]}
        child {node[draw=none](bottom){} edge from parent[draw=none]}
      }
      child {node {$\state_1$}}
    }
    child {
      node(r1) {$\state_u$}
      child {edge from parent[draw=none]}
      child {node[fill=white] {$\state_0$}}
    }
    ;
    \draw[dashed,help lines, line width=1pt] 
    ([xshift=-1mm]r1.south west) -- ([xshift=-1mm]r1.north west) -- %
    ([yshift=1mm]r1.north west) -- ([yshift=1mm]r1.north east) -- %
    ([xshift=1mm]r1.north east) -- ([xshift=1mm]r1.south east) -- %
    ([yshift=-1mm]r1.south east) -- ([yshift=-1mm]r1.south west) -- cycle; 
    % cheating for the size of the grey posit
    \node[draw=none] at ([yshift=-1mm]bottom.south) {};
    \end{scope}
  \end{tikzpicture}
  %
  \begin{tikzpicture}
    \node {$\atransrel{\rwrule_2}$};
    \node[yshift=-15mm,draw=none]{};
  \end{tikzpicture}
  %
  \begin{tikzpicture}[show background rectangle]
    \begin{scope}[scale=0.5]
    \tikzstyle{level 1}=[sibling distance=2cm,level distance=1.5cm]
    \tikzstyle{level 2}=[sibling distance=1.8cm,level distance=1.3cm]
    \tikzstyle{level 3}=[sibling distance=1.6cm,level distance=1.1cm]
    \tikzstyle{every node}=[draw,rounded corners,minimum width=4mm]
    \node[xshift=-10mm,shape=circle,fill=white,draw=none] {$\tree_3$};
    \node{$\state_1$}
    child {
      node {$\state_1$}
      child {
        node {$\state_1$}
        child {node[draw=none]{} edge from parent[draw=none]}
        child {node[draw=none](bottom){} edge from parent[draw=none]}
      }
      child {node {$\state_1$}}
    }
    child {
      node {$\state_0$}
      child {edge from parent[draw=none]}
      child {node[draw=none]{} edge from parent[draw=none]}
    }
    ;
    % cheating for the size of the grey postit
    \node[draw=none] at ([yshift=-1mm]bottom.south) {};
    \end{scope}
  \end{tikzpicture}
  %
  \caption{A possible run of the approximate transition system induced by the percolate 
protocol (see Example~\ref{exp:percolate}).
%
The nodes with a white background represent those where the rule will apply while the dashed lines 
surround the nodes which are removed.
}
  \label{fig:approximation}
\end{figure}
%

Observe that the relation $\atransrel{}$ is an over-approximation of the transition relation 
defined in the previous section (i.e.\ $\atransrel{}\supseteq\transrel{}$) by the following 
argument.
%
Consider two configurations $\conf_1,\conf_2\in\confs$ with $\conf_1\transrel{}\conf_2$.
%
By definition, this implies the existence of a rule $\rwrule\in\rwrules$ and an inclusion $f$ of 
$\tleft{\rwrule}$ in $\conf_1$ such that $\conf_2=\conf_1\rename_f\tright{\rwrule}$.
%
Observe that, by definition of the $\sminus$ operation, since $f$ is an inclusion it follows that 
$\conf_1\sminus_f\tleft{\rwrule}=\conf_1$ and $\minusinc{f}=f$.
%
Therefore, we obtain $\conf_2=\conf_1\rename_f\tright{\rwrule}=(\conf_1\sminus_f\tleft{\rwrule})
\rename_{\minusinc{f}}\tright{\rwrule}$, and as a consequence, $\conf_1\atransrel{\rwrule}\conf_2$.
%

We are now ready to state a key property of the approximated transition system.
%
\begin{lemma}
\label{lem:approx:monotonic}
The approximate transition system $\tuple{\confs,\atransrel{}}$ is 
monotonic with respect to $\preceq$.
\end{lemma}
%

We define the coverability problem for the approximate system as follows.
%

\probbox{\acovproblem}
{%
  \item A parameterized tree system $\parsys=\parsystuple$
  \item An upward closed set $\final$ of configurations.
}{%
$\init\atransrel{*}{}\final$ ?
}

%
Since $\transrel{}\subseteq\atransrel{}$, a negative answer to \acovproblem{} implies
a negative answer to \covproblem.
%

