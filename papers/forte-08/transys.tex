\section{Operational Semantics}
\label{sec:ts}
%
The operational semantics of a parameterized tree system can be captured 
by a transition system.
%
In this section, we first describe the induced transition system.
%
Then we introduce the \emph{coverability problem}.
%

\paragraph{\bf Transition System}
%
A \emph{transition system} $\tssystem$ is a pair $\gtstuple$, where $\confs$ 
is an (infinite) set of \emph{configurations} and $\gtransrel{}$ is a binary 
relation on $\confs$.
%
We use $\gtransrel{*}$ to denote the reflexive transitive closure of $\gtransrel{}$.
%
Given an ordering $\order$ on $\confs$, we say that $\tssystem$ is \emph{monotonic with respect 
to $\order$} if the following holds:
%
For any configurations $\conf_1,\conf_2,\conf_3\in\confs$ with $\conf_1\gtransrel{}\conf_3$ 
and $\conf_1\order\conf_2$, there is a configuration $\conf_4\in\confs$ such that $\conf_2
\gtransrel{}\conf_4$ and $\conf_3\order\conf_4$.
%
We will consider several transition systems in this paper.
%

First, a parameterized system $\parsys=\parsystuple$ induces a transition system
$\tsof{\parsys}=\tuple{\confs,\transrel{}}$ where $\confs=\trees{\states}$.
%
Intuitively, a configuration $\conf=(\nodes,\labeling)\in\confs$ represents an instance 
of the system with $\sizeof{\nodes}$ processes.
%
These processes are arranged according to the tree structure $\nodes$ 
and their current local states are given by $\labeling$.
%
More precisely, each node $\node\in\nodes$ represents a process in the state $\labeling(
\node)$.
%

Next, we define the transition relation $\transrel{}$ on the set of configurations
as follows.
%
Let $\rwrule\in\rwrules$ be a rewrite rule.
%
Consider two configurations $\conf_1$ and $\conf_2$.
%
We write $\conf_1\transrel{\rwrule}\conf_2$ to denote that there is an $f$ such that 
the following conditions hold:
%
(i) $\tleft{\rwrule}\includedin_f\conf_1$, and
(ii) $\conf_2=\conf_1\rename_f\tright{\rwrule}$.
%
Intuitively, $\conf_2$ can be derived from $\conf_1$ by changing the labels of all the nodes in 
$\image{f}$ according to the labeling function of $\tright{\rwrule}$.
%
Below, we give informal explanations of the conditions.
%
First, in condition (i), we identify the ``active processes'' (those which participate
in the transition) by the inclusion $f$ ($\image{f}$).
%
Implicitly, we interpret $\tleft{\rwrule}$ as a guard and therefore require, through condition (i),
that the configuration $\conf_1$ contains a tree which is a copy of the left hand side
of the rule. 
%
Then, in condition (ii), we interpret $\tright{\rwrule}$ as an operation and require that,
in $\conf_2$, the processes in $\image{f}$ (the active ones) should all change state according to 
$\tright{\rwrule}$.
%
Observe that the local states of the ``passive processes'', i.e.\ those not participating in the transition, 
should remain unchanged through the transition, and also that the transition does not change the structure 
of the tree
%
\footnote{In fact, our method can also cope with non-structure preserving rules, such dynamic
creation and deletion of processes. However, for simplicity of presentation, we choose not 
to do so.} (see Figure~\ref{fig:percolate:execution}).
%

We use $\conf\transrel{}\conf'$ to denote that $\conf\transrel{\rwrule}\conf'$ 
for some rule $\rwrule\in\rwrules$.
%

\begin{figure}
\centering
\begin{tikzpicture}[scale=0.9,show background rectangle]
  \tikzstyle{level 1}=[level distance=0.8cm,sibling distance=10mm]
  \tikzstyle{level 2}=[level distance=0.8cm,sibling distance=6mm]
  \draw[fill=white,rounded corners,draw=none] (-10mm,-18mm) rectangle +(10mm,12mm);
  \node {$\state_u$}
  child {node{$\state_u$} child {node{$\state_0$}} child {node{$\state_0$}}}
  child {node{$\state_u$} child {node{$\state_1$}} child {node(bottom){$\state_1$}}}
  ;
  % cheating for the size of the grey postit
  \node[draw=none] at ([yshift=-1pt]bottom.south) {};
\end{tikzpicture}
\begin{tikzpicture}[scale=0.75]
\node {$\transrel{\rwrule_1}$} child{edge from parent[draw=none]};
\end{tikzpicture}
%
\begin{tikzpicture}[scale=0.9,show background rectangle]
  \tikzstyle{level 1}=[level distance=0.8cm,sibling distance=10mm]
  \tikzstyle{level 2}=[level distance=0.8cm,sibling distance=6mm]
  \draw[fill=white,rounded corners,draw=none] (0mm,-18mm) rectangle +(10mm,12mm);
  \node {$\state_u$}
  child {node{$\state_0$} child {node{$\state_0$}} child {node{$\state_0$}}}
  child {node{$\state_u$} child {node{$\state_1$}} child {node(bottom){$\state_1$}}}
  ;
  % cheating for the size of the grey postit
  \node[draw=none] at ([yshift=-1pt]bottom.south) {};
\end{tikzpicture}
\begin{tikzpicture}[scale=0.75]
\node {$\transrel{\rwrule_4}$} child{edge from parent[draw=none]};
\end{tikzpicture}
%
\begin{tikzpicture}[scale=0.9,show background rectangle]
  \tikzstyle{level 1}=[level distance=0.8cm,sibling distance=10mm]
  \tikzstyle{level 2}=[level distance=0.8cm,sibling distance=6mm]
  \draw[fill=white,rounded corners,draw=none] (-8mm,-10mm) rectangle +(16mm,12mm);
  \node {$\state_u$}
  child {node{$\state_0$} child {node{$\state_0$}} child {node{$\state_0$}}}
  child {node{$\state_1$} child {node{$\state_1$}} child {node(bottom){$\state_1$}}}
  ;
  % cheating for the size of the grey postit
  \node[draw=none] at ([yshift=-1pt]bottom.south) {};
\end{tikzpicture}
\begin{tikzpicture}[scale=0.75]
\node {$\transrel{\rwrule_3}$} child{edge from parent[draw=none]};
\end{tikzpicture}
%
\begin{tikzpicture}[scale=0.9,show background rectangle]
  \tikzstyle{level 1}=[level distance=0.8cm,sibling distance=10mm]
  \tikzstyle{level 2}=[level distance=0.8cm,sibling distance=6mm]
  \node {$\state_1$}
  child {node{$\state_0$} child {node{$\state_0$}} child {node{$\state_0$}}}
  child {node{$\state_1$} child {node{$\state_1$}} child {node(bottom){$\state_1$}}}
  ;
  % cheating for the size of the grey postit
  \node[draw=none] at ([yshift=-1pt]bottom.south) {};
\end{tikzpicture}
%
\caption{A possible run of the percolate protocol. 
%
We highlight in white the zone where the rule applies 
%
(see Example~\ref{exp:percolate}).}
\label{fig:percolate:execution}
\end{figure}
%

\paragraph{\bf Safety Properties}
%
In order to analyze safety properties, we study the \emph{coverability problem} defined 
below.
%
For a parameterized tree system $\parsys=\parsystuple$, we assume that we are given a set 
of initial configurations $\init$, each of which characterizes a possible state of the system 
prior to starting the execution.
%

We recall the definition of the relation $\preceq$ defined in Section~\ref{sec:prels}.
%
A set of configurations $D\subseteq\confs$ is said to be \emph{upward closed} 
(with respect to $\preceq$) if $\conf\in D$ and $\conf\preceq\conf'$ implies $\conf'\in D$.
%
For sets of configurations $D,D'\subseteq\confs$ we use $D\transrel{} D'$ to denote that 
there are $\conf\in D$ and $\conf'\in D'$ with $\conf\transrel{}\conf'$.
%
The \emph{coverability problem} for parameterized tree systems is defined as follows:
%

\probbox{\covproblem}
{%
  \item A parameterized tree system $\parsys=\parsystuple$.
  \item An upward closed set $\final$ of configurations.
}{%
$\init\transrel{*}{}\final$ ?
}
%

It can be shown, using standard techniques (see ~\cite{VW:modelchecking,GoWo:safety}),
that checking safety properties (expressed as regular languages) can be translated into instances 
of the coverability problem.
%
Therefore, checking safety properties amounts to solving \covproblem{} (i.e.\ to the reachability 
of upward closed sets).
%
