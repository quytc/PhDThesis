%
\subsection{The IEEE 1394 Tree Identification Protocol}
%

The 1394 High Performance serial bus~\cite{ieee:1394} is used to transport digitized video and
audio signals within a network of multimedia systems and devices. 
% 

The tree identification protocol is used in one of the phases implementing the IEEE 1394 protocol.
%
More precisely, it is run after a bus reset in the network and leads to the election of a unique 
leader node.
%

In this section, we consider a version working on tree topologies.
%
Furthermore, we assume that (i) each inner node is connected to 3 neighbors, (ii) the root is connected 
to 2 neighbors, and (ii) communication is atomic.
%

Initially, all nodes are in state \emph{undefined} ($u$). 
%
We identify two steps in the protocol depending on the number $n$ of neighbors which are 
still in state $u$.
%
If $n>1$, the node waits for (``be my parent'') requests from its neighbors. 
%
If $n=1$, the node sends a request to the remaining neighbor in state $u$. 
%
Observe that we implicitly assume that the leaf nodes are the first to communicate with their neighbors.
%

Formally, we derive a parameterized tree system model $\parsys=\parsystuple$ as follows.
%
We define the set of states by $\states=\setcomp{\state_s^n}{s\in\set{u,c,l}\band n\in\set{leaf,inner,root}}$ 
where the scripts $s$ and $n$ describe respectively the state and the nature of the node.
%
In the definition of the state ($s$), the letters $u$, $c$ and $l$ stand respectively for \emph{undefined}, 
\emph{child} and \emph{leader}.
%
In a similar manner to the previous section, we drop the script(s) whenever we mean that it can take any value 
(see caption of Figure~\ref{fig:arbiter:rules}).
% 

The rewrite rules $\rwrules$ are described below.
%
\begin{itemize}
\item The leaves initiate the communications:

\begin{center}
  \begin{tikzpicture}[show background rectangle]
    \tikzstyle{level 1}=[level distance=1cm,sibling distance=2cm]
    \tikzstyle{every node}=[draw,rounded corners,minimum width=1cm]
    \node {$\state_u$}
    child{node{$\state_u^{leaf}$/$\state_c$}}
    child[level distance=1cm,sibling distance=0cm]{node[draw=none]{} edge from parent[draw=none]};
  \end{tikzpicture}
%
\begin{tikzpicture}[show background rectangle]
    \tikzstyle{level 1}=[level distance=1cm,sibling distance=2cm]
    \tikzstyle{every node}=[draw,rounded corners,minimum width=1cm]
    \node {$\state_u$}
    child[level distance=1cm,sibling distance=0cm]{node[draw=none]{} edge from parent[draw=none]}
    child{node{$\state_u^{leaf}$/$\state_c$}};
\end{tikzpicture}
\end{center}

\item The inner nodes become children or wait for requests:

\begin{center}
  \begin{tikzpicture}[grow cyclic,show background rectangle]
    \tikzstyle{level 1}=[level distance=1cm,sibling angle=120]
    \tikzstyle{every node}=[draw,rounded corners,minimum width=1cm]
    \node {$\state_u$/$\state_c$}
    child {node {$\state_u$}}
    child[level distance=1.3cm] {node {$\state_c$}}
    child {node {$\state_c$}};
  \end{tikzpicture}
  % 
  \begin{tikzpicture}[grow cyclic,show background rectangle]
    \tikzstyle{level 1}=[level distance=1cm,sibling angle=120]
    \tikzstyle{every node}=[draw,rounded corners,minimum width=1cm]
    \node {$\state_u$/$\state_c$}
    child {node {$\state_c$}}
    child[level distance=1.3cm] {node {$\state_u$}}
    child {node {$\state_c$}};
  \end{tikzpicture}
  % 
  \begin{tikzpicture}[grow cyclic,show background rectangle]
    \tikzstyle{level 1}=[level distance=1cm,sibling angle=120]
    \tikzstyle{every node}=[draw,rounded corners,minimum width=1cm]
    \node {$\state_u$/$\state_c$}
    child {node {$\state_c$}}
    child[level distance=1.3cm] {node {$\state_c$}}
    child {node {$\state_u$}};
  \end{tikzpicture}
\end{center}

\item The leader is chosen:

\begin{center}
\begin{tikzpicture}[grow cyclic,show background rectangle]
\tikzstyle{every node}=[draw,rounded corners,minimum width=1cm]
\tikzstyle{level 1}=[level distance=1cm,sibling angle=120]
\node {$\state_u$/$\state_l$}
    child {node {$\state_c$}}
    child[level distance=1.6cm] {node {$\state_c$}}
    child {node {$\state_c$}};
\end{tikzpicture}
%
\begin{tikzpicture}[grow cyclic,show background rectangle]
\tikzstyle{every node}=[draw,rounded corners,minimum width=1cm]
\tikzstyle{level 1}=[level distance=1cm,sibling angle=120]
\node {$\state_c$}
    child{node {$\state_u^{leaf}$/$\state_l$}}
    child[level distance=0cm]{node[draw=none]{} edge from parent[draw=none]}
    child{node[draw=none]{} edge from parent[draw=none]};
\end{tikzpicture}
%
\begin{tikzpicture}[grow cyclic,show background rectangle]
\tikzstyle{every node}=[draw,rounded corners,minimum width=1cm]
\tikzstyle{level 1}=[level distance=1cm,sibling angle=120]
\begin{scope}[rotate=60]
  \node {$\state_c$}
    child{node {$\state_u^{leaf}$/$\state_l$}}
    child{node[draw=none]{} edge from parent[draw=none]}
    child[level distance=0cm]{node[draw=none]{} edge from parent[draw=none]};
\end{scope}
\end{tikzpicture}
\end{center}
\end{itemize}
%

The set of initial configurations $\init$ is represented by trees where all nodes
are in state undefined, and the set of bad constraints $\fconfs$ is represented by 
trees where at least 2 leaders are elected.
%
\begin{center}
\begin{tikzpicture}[level distance=1cm,show background rectangle]
\tikzstyle{every node}=[draw,rounded corners,minimum width=1cm]
\node {$\state_l$} child {node {$\state_l$}}
child[sibling distance=0cm]{node[draw=none]{} edge from parent[draw=none]};
\end{tikzpicture}
%
\begin{tikzpicture}[level distance=1cm,show background rectangle]
\tikzstyle{every node}=[draw,rounded corners,minimum width=1cm]
\node {$\state_l$}
child[sibling distance=0cm]{node[draw=none]{} edge from parent[draw=none]}
child {node {$\state_l$}};
\end{tikzpicture}
%
\begin{tikzpicture}[level distance=1cm,show background rectangle]
\tikzstyle{every node}=[draw,rounded corners,minimum width=1cm]
\node {$\state$} child {node {$\state_l$}} child {node {$\state_l$}};
\end{tikzpicture}
\end{center}
%
