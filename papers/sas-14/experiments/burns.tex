Burns' algorithm~\cite{Burns:protocol} implements a mutual exclusion
protocol and can be modeled as a parameterized system where the
processes are arranged in a linear topology. Each process can
communicate with and distinguish its neighbors on its right or its
left.

The local state of a process ranges over state $\set{1,\ldots,6}$ where
$6$ represents the critical section. Transitions are guarded with
conditions on the states of the neighbors, on the right, the left or
both and is enabled if the guard is not violated.

Initially, all processes are in state $1$. A bad configuration is
detected if 2 processes or more are in the critical section, ie if the
array contains at least 2 processes in state $6$.

The transitions are depicted in the following state diagram. $i$ is
the the current process, $j$ is another process and $\state{j}$ its
state.

%
\noindent%
\begin{minipage}{0.62\linewidth}
  % \lstinputlisting[caption={Burns' mutual exclusion protocol.}, style=custom, label=figure:BurnsCode]{burns-code.txt}
  \lstinputlisting[style=custom]{experiments/burns-code.txt}
\end{minipage}
%
\begin{minipage}{0.38\linewidth}
\hfill%
  \begin{tikzpicture}
    \node[state,state-i] (n1) at (0,0) {1};
    \node[state,state-n] (n2) at (2,0) {2};
    \node[state,state-n] (n3) at (4,0) {3};
    \node[state,state-n] (n4) at (4,-2) {4};
    \node[state,state-n] (n5) at (2,-2) {5};
    \node[state,state-b] (n6) at (0,-2) {6};
  
    \draw [->,myedge] (n1) -- (n2);
    % \draw [->,myedge] (n2) to[out=120,in=60] (n1) node[mylabel,above]{$\exists j<i: \state{j}\not\in\set{1,2,3}$};
    \draw [->,myedge] (n2) .. controls +(120:0.75) and +(45:0.75) .. (n1) node[mylabel,above]{$\exists j<i: \set{4,5,6}$};
    
    \draw [->,myedge] (n2) -- node[mylabel,above]{$\forall j<i: \set{1,2,3}$} (n3);
    
    \draw [->,myedge] (n3) -- (n4);
    
    \draw [->,myedge] (n4) -- node[mylabel,below]{$\forall j<i: \set{1,2,3}$} (n5);
    
    % \draw [->,myedge] (n5) to[out=90,in=-45] (n1) node[mylabel,pos=0.7,above right,sloped]{$\exists j<i: \set{4,5,6}$};
    \draw [->,myedge] (n4) to[out=135,in=-45] node[mylabel,pos=0.45,above,sloped]{$\exists j<i: \set{4,5,6}$} (n1);
    
    \draw [->,myedge] (n5) -- node[mylabel,below]{$\forall j>i: \set{1,2,3}$} (n6);
    
    \draw [->,myedge] (n6) -- (n1);
    
  \end{tikzpicture}
\end{minipage}
%\def\initstate{\tikz[baseline=(n.base)]\node[state,fill=green!20,scale=0.7](n){1};}
%\caption{A pseudocode of the $i$th process of
%  Bruns's protocol and the corresponding transition rules (in the form of a transition diagram). A state of a process is composed form
%  a program location and a value of the local variable
%  $\mathit{flag}[i]$. Since value of $\mathit{flag}[i]$ is invariant
%  at each location, states correspond to locations. Initially, all
%  processes are in state {\protect\initstate}.
%}
%\caption{Pseudocode and transition rules of Burns' protocol.}
