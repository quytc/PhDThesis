\tikzset{shared/.style={draw=green,fill=green!30!gray,thick}}
\tikzset{dottedrectangle/.style={rounded corners,draw,dotted,inner sep=0pt}}
\tikzset{localVars/.style={ellipse,minimum height=30mm,minimum width=12mm, shade, top color=yellow, bottom color=white}}
\tikzset{globalVars/.style={ellipse, shade, top color=blue!50!red, bottom color=white}}
\newcommand{\LD}[1]{\ensuremath{Read_{#1}}}
\newcommand{\ST}[1]{\ensuremath{Write_{#1}}}
%
\begin{wrapfigure}{r}[2pt]{0.6\textwidth}
\vspace{-10mm}
\begin{center}
    \begin{tikzpicture}[scale=0.7,
      show background rectangle,
      every edge/.style={draw,shorten >=1pt,>=stealth',semithick},
      every label/.append style={scale=0.7}
      ]

      \node(cheating) at(-2,0){};

      \node[place,state,label=right:$init$](in)  at (4,0) {?};
      \node[place,state](lockOut) at (0,0) {};
      \node[place,state](temp1) at (0,-2) {};
      \node[place,state](readOut) at (0,-4) {};
      \node[place,state](temp2) at (0,-6) {};
      \node[place,state](writeOut) at (0,-8) {};
      \node[place,state,label=right:$end$](unlockOut) at (4,-8) {};

      \node[transition,rotate=90,label={[scale=0.9]above left:$lock$}](tlock) at (2,0){} edge [pre] (in) edge [post] (lockOut);
      \node[transition](treadIn) at (0,-1){} edge [pre] (lockOut) edge [post] (temp1);
      \node[transition](treadOut) at (0,-3){} edge [pre] (temp1) edge [post] (readOut);
      \node[transition](twriteIn) at (0,-5){} edge [pre] (readOut) edge [post] (temp2);
      \node[transition](twriteOut) at (0,-7){} edge [pre] (temp2) edge [post] (writeOut);
      \node[transition,rotate=90,label={[scale=0.9]below right:$unlock$}](tunlock) at (2,-8){} edge [pre] (writeOut) edge [post] (unlockOut);

      \node[place,state,shared,label=right:$L$,tokens=1] (L)  at (6,-4) {};
      \node[place,state,shared,label=right:$\LD{counter}$] (Rc)  at (2,-3.5) {};
      \node[place,state,shared,label=right:$\ST{counter}$] (Wc)  at (2,-4.5) {};

      \draw[pre,myedge] (tlock.-165) to[out=0,in=90] (L);
      \draw[post,myedge] (tunlock.-15) to[out=0,in=-90] (L);

      \draw[post,myedge] (treadIn.-15) to[out=-90,in=120] (Rc.120);
      \draw[pre,myedge] (treadOut.15) to[out=90,in=150] (Rc.150);

      \draw[post,myedge] (twriteIn.-15) to[out=-90,in=-150] (Wc.-150);
      \draw[pre,myedge] (twriteOut.15) to[out=90,in=-120] (Wc.-120);

      \begin{pgfonlayer}{my background}
        \node[dottedrectangle,fit=(Rc) (Wc)] (r){};
        \node[dottedrectangle,fit=(treadIn) (treadOut)] (read){};
        \node[dottedrectangle,fit=(twriteIn) (twriteOut)] (write){};
      \end{pgfonlayer}
      
    \end{tikzpicture}    
\end{center}
\vspace{-10mm}
\end{wrapfigure}
%
We can model the critical section problem by read and write access to
a resource shared by multiple processes. There is no particular
topology so we will model it as an instance of a parametrized system
with multisets. More precisely, as a Petri Net.
%
As the petri net of a concurrent program, described at that level of
granularity, can quickly grow in size, we choose a short example: The
processes repeatedly grab the lock, increment a \emph{counter} and
release the lock. A bad configuration is detected when 2 or more
processes are having access to the shared variable simultaneously
while one is writing. 

The shared variable $counter$ is associated with two places,
$\LD{counter}$ and $\ST{counter}$. The tokens of a place in the petri
net represent the count of process in a given state, or the available
resources. %
A process places a token in $\LD{counter}$ (resp. $\ST{counter}$) if
it is currently accessing the variable $counter$ for reading
(resp. writing). We model read and write accesses to shared variables
with two transitions, denoted by the dotted rectangle in the following
figure.
%
There is a place $L$ associated with a lock. Intuitively, if $L$
contains a token, the lock is free, otherwise it is busy. This ensures
that only one process can hold the lock at a time. 
%
Note that $L$ is a global variable and that we omit the input places
used to balance out the number of tokens in the net.
%

Initially, the lock is free and the processes are in the initial state
\emph{init}. The petri contains then one token in $L$ and the others
in \emph{init}. 
%
A bad situation is detected when the petri net contains 2 or more
tokens in $\ST{counter}$ or when there is one (or more) token in
$\LD{counter}$ and one (or more) token in $\ST{counter}$.
