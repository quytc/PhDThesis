This algorithm implements a simple solution to the light system of an
office room. An arbitrary number of people may get in or out of the
room.
%
Initially, the light is off and everyone is outside the room.
%
The first person to enter the room turns the light on and the last
person to exit the room turns it off.
%

%
A bad configuration is detected when the light is on, but there is
noone in the room, or when the light is off while there are still
people in the room.

We model this algorithm with a Petri Net with Inhibitor arcs.  There
are 2 places associated with the \emph{on} and \emph{off} status of
the light. %
And there are 2 places associated with the fact that a person is
inside or outside the room. %
An extra place \emph{exiting} is used to check the person who wishes
to exit is the last one in the room. %
Initially, there is a token in the \emph{off} place, and all the other
tokens in the outside place. %
For readibility, we represent the transitions piecewise as follows.

%   let is_bad m = m.(bad) > 0
%     (* (m.(fst inside) + m.(fst exiting) > 0 && m.(fst off) > 0) || ( m.(fst inside) + m.(fst exiting) = 0 && m.(fst on) > 0) *) 

\begin{center}
\tikzset{shared/.style={draw=green,fill=green!30!gray,thick}}
\tikzset{dottedrectangle/.style={rounded corners,draw,dotted,inner sep=0pt}}
\begin{tikzpicture}[
  mylabel/.style={node distance=1mm,scale=0.8},
  show background rectangle,
  ]
  %% Going in
%     R.T ( [outside;off], [inside;on], "[outside;off] -> [inside;on]");
%     R.T ( [outside;on], [inside;on], "[outside;on] -> [inside;on]");

  \begin{scope}%[xshift=-4cm]

    \node[place,state](on) at (2,0) {};
    \node[place,state](off)  at (2,-2) {};
    \node[place,state](in)  at (0,0) {};
    \node[place,state](out)  at (0,-2) {};

    \node[mylabel,right=of on]{on};
    \node[mylabel,right=of off]{off};
    \node[mylabel,left=of in]{in};
    \node[mylabel,left=of out]{out};
    
    \node[transition](t1) at (1.7,-1){};
    \draw[<-,myedge] (t1.-165) to[out=-90,in=45] (out);
    \draw[<-,myedge] (t1.-15) to[out=-90,in=135] (off);
    \draw[->,myedge] (t1.165) to[out=90,in=-45] (in);
    \draw[->,myedge] (t1.15) to[out=90,in=-135] (on);

    \node[transition](t2) at (0.3,-1){};
    \draw[<-,myedge] (t2.-165) to[out=-90,in=135] (out);
    \draw[<-,myedge] (t2.-15) ..controls +(-65:10mm) and +(-160:13mm).. (on);
    \draw[->,myedge] (t2.165) to[out=90,in=-135] (in);
    \draw[->,myedge] (t2.15) to[out=90,in=180] (on);

    \node[caption] at (1,-2.2) {Going in};
  \end{scope}

  %% Going out
  \begin{scope}[shift={(4,0)}]

    \node[place,state](on) at (2,0) {};
    \node[place,state](off)  at (2,-2) {};
    \node[place,state](in)  at (0,0) {};
    \node[place,state](out)  at (0,-2) {};
    \node[place,state](exiting) at (1,-1) {};

    \node[mylabel,right=of on]{on};
    \node[mylabel,right=of off]{off};
    \node[mylabel,left=of in]{in};
    \node[mylabel,left=of out]{out};
    \node[mylabel,above=0mm of exiting]{exiting};

%     R.Inhibitor ( [inside], [exiting], [fst exiting], "[inside] empty[exiting] -> [exiting]");
    \node[transition](t1) at (-0.1,-1){};
    \draw[<-,myedge] (t1.165) to[out=90,in=-90] (in);
    \draw[*-,myedge] (t1.15) ..controls +(up:10mm) and +(left:6mm).. (exiting);
    \draw[->,myedge] (t1) ..controls +(-60:8mm) and +(down:8mm).. (exiting);

%     R.Inhibitor ( [exiting;on], [outside;off], [fst inside], "[exiting;on] empty[inside] -> [outside;off]");
    \node[transition](t2) at (2.1,-1){};
    \draw[<-,myedge] (t2.15) to[out=90,in=-90] (on);
    \draw[<-,myedge] (t2.165) ..controls +(up:10mm) and +(right:6mm).. (exiting);
    \draw[*-,myedge] (t2.90) to[out=90,in=0] (in);
    \draw[->,myedge] (t2.-165) to[out=-90,in=0] (out);
    \draw[->,myedge] (t2.-15) to[out=-90,in=90] (off);


    \node[caption] at (1,-2.2) {Going out};
  \end{scope}

  %% Still going out
  \begin{scope}[shift={(8,0)}]

    \node[place,state](on) at (2,0) {};
    \node[place,state](off)  at (2,-2) {};
    \node[place,state](in)  at (0,0) {};
    \node[place,state](out)  at (0,-2) {};
    \node[place,state](exiting) at (1,-1) {};

    \node[mylabel,right=of on]{on};
    \node[mylabel,right=of off]{off};
    \node[mylabel,left=of in]{in};
    \node[mylabel,left=of out]{out};
    \node[mylabel,above=0mm of exiting]{exiting};

%     R.T ( [exiting;inside], [outside;inside], "[exiting;inside] -> [outside;inside]");
    \node[transition](t1) at (-0.1,-1){};
    \draw[<-,myedge] (t1.165) to[out=90,in=-90] (in);
    \draw[<-,myedge] (t1.15) ..controls +(up:10mm) and +(left:6mm).. (exiting);
    \draw[->,myedge] (t1.-165) ..controls +(-135:10mm) and +(-135:8mm).. (in);
    \draw[->,myedge] (t1.-15) to[out=-90,in=45] (out);

    %% Bad cases
    \node[place,state,draw=black,fill=black!80!white](bad) at (3,-1) {};
    \node[mylabel,below=0mm of bad]{Bad};
%     R.T ( [inside;off], [(bad,2)], "[inside;off] -> [bad;bad]");
    \node[transition,rotate=90,minimum width=6mm,draw=black](t2) at (2.3,-1.5){};
    \draw[<-,myedge] (t2.35) ..controls +(left:5mm) and +(right:15mm).. (in.-10);
    \draw[<-,myedge] (t2.135) to[out=180,in=135] (off);
    \draw[->,myedge] (t2.-65) to[out=0,in=170] (bad);
    \draw[->,myedge] (t2.-125) to[out=0,in=-170] (bad);

%     R.Inhibitor ( [on], [(bad,1)], [fst inside; fst exiting], "[on] empty[inside;exiting] -> [bad]");
    \node[transition,rotate=90,minimum width=6mm,draw=black](t3) at (2.1,-0.7){};
    \draw[<-,myedge] (t3.15) ..controls +(left:5mm) and +(left:5mm).. (on);
    \draw[*-,myedge] (t3.90) ..controls +(left:5mm) and +(right:15mm).. (in.10);
    \draw[*-,myedge] (t3.165) to[out=180,in=0] (exiting);
    \draw[->,myedge] (t3.-90) to[out=0,in=135] (bad);


    \node[caption] at (1,-2.2) {Not the last out \& Bad};
  \end{scope}

  \draw[white,thick] (3,0) -- +(0,-2.5);
  \draw[white,thick] (7,0) -- +(0,-2.5);
\end{tikzpicture}
\end{center}
