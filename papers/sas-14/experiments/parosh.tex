This protocol ensures mutual exclusion between processes.
Each process has five local states %
\w[i]{0},%
\w[n]{1},%
\w[n]{2},%
\w[n]{3},%
\w[b]{4} %
and is initially in state\,\w[i]{0}. %
A process in the critical section is at state\,\w[b]{4}. %
The set of bad configurations contains exactly configurations with at
least two occurrences of state\,\w[b]{4}.

%
\noindent%
\begin{wrapfigure}{r}{0.5\textwidth}
  %\vspace{-24pt}
  \begin{center}
    \begin{tikzpicture}[%scale=0.8,
      % mystate/.style={circle,draw=none,fill=white,thick,inner sep=1pt,scale=0.8},
      edge/.style={->,>=stealth',thin},%,shorten >=1pt,semithick},
      mylabel/.style={midway,anchor=east,scale=0.9,thick},
      guard/.style={state,state-n,scale=0.8,thin}
      ]
      
      \node[state,state-i] (n0) at (0,0)     {0};
      \node[state,state-n] (n1) at (1,1.2)   {1};
      \node[state,state-n] (n2) at (2.4,1.2) {2};
      \node[state,state-n] (n3) at (3.4,0)   {3};
      \node[state,state-b] (n4) at (1.7,0)   {4};
      
      \draw [edge,->] (n0) to[out=90,in=-180] node[mylabel,above left,pos=0.25](l01){$\forall$} (n1);
      \draw [edge,->] (n1) -- (n2);
      \draw [edge,->] (n2) to[out=0,in=90] node[mylabel,above right,pos=0.7](l23){$\forall_L$} (n3);
      \draw [edge,->] (n3) -- node[mylabel,above,pos=0.6](l34){$\exists$} (n4);
      \draw [edge,->] (n4) -- (n0);
      \draw [edge,->] (n3)  ..controls +(-135:8mm) and +(-45:8mm).. (n0);

      %% 0 -> 1 forall 0,1,4
      \node[guard,fill=green!20,right=-0.5mm of l01]{0};
      \node[guard,right=1.5mm of l01]{1};
      \node[guard,fill=red!50,right=3.5mm of l01]{4};
      %% 2 -> 3 forall_left 0
      \node[guard,fill=green!20,right=0mm of l23]{0};
      %% 3 -> 4 exists 2
      \node[guard,right=-0.5mm of l34]{2};
    \end{tikzpicture}
  \end{center}
  \vspace{-6mm}
  \caption{State diagram (per process).}
  \label{figure:parosh}
  \vspace{-12pt}
\end{wrapfigure}
%
\noindent%
Processes move from state\,\w{0} to \w{1}, and
then \w{2}. %
Once the first process is in state\,\w{2}, it “closes the door”
on the processes which are still in\,\w{0}. They can no longer
leave state\,\w{0} until the door is opened again (when no
process is in state\,\w{2} or \w{3}). %
%
Moreover, a process is allowed to cross from state\,\w{3} to
\w{4} only if there is at least one process still in
state\,\w{2} (i.e., the door is still closed on the processes
in state\,\w{0}). %
%
This prevents a process to first reach state\,\w{4} along with
a process to its left starting to move from\,\w{0} all the way
to state\,\w{4} (thus violating mutual exclusion). %
From the set of processes which have left state\,\w{0} (and
which are now in state\,\w{1} or\,\w{2}), the leftmost
process has the highest priority and it is encoded in the global
condition: a process may move from \w{2} to\,\w{3}
only if all processes on its left are in state\,\w{0}.
%
