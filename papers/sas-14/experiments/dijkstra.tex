\label{app:dijkstra}
Dijsktra's algorithm %\cite{mutex:dijsktra}
 implements a mutual
exclusion protocol and can be modeled as a parameterized system where
the processes are arranged in a linear topology. Each process can
communicate with its neighbors and check their status.

\begin{wrapfigure}{r}[0pt]{0.55\linewidth}
  \hfill%
  \vspace{-24pt}
  \lstinputlisting[style=custom]{experiments/dijkstra-code.txt}
  \vspace{-20pt}
\end{wrapfigure}
%
%The algorithm is described in Figure~\ref{algo:dijkstra}.
The algorithm is described in code listing on the right.
%
It makes use of a pointer, i.e. a variable ranging over
process indices. We model this pointer by a local boolean variable $p$
for each process state.
%
$p$ is $\strue$ iff the pointer points to this current process. When
the pointer changes, this information must be passed onto all other
processes, which we model as a broadcast transition. Concretely, upon
pointer assignment, the current process sets its local variable $p$ to
$\strue$ and simultaneously sets $p$ to $\sfalse$ in all other
processes.

We denote the state of process as $St$ (resp. $Sf$) when the process
is in state $S$ and the pointer $p$ is $\strue$ (resp. $\sfalse$). The
state $S$ of a process ranges over $\set{1,\ldots,6}$ where $6$
represents the critical section.
%
Initially, one process is in state $1t$ and all other processes are in
state $1f$. A bad configuration is detected when 2 or more processes
are in the critical section, ie when their state is either $6t$ or
$6f$.

\begin{figure}[h]
  \begin{center}
    \vspace{-24pt}
\begin{tikzpicture}[node distance=15mm]

  \node[state,state-i] (n1t) {1f};
  \node[state,state-n] (n2t) [right=of n1t] {2f};
  \node[state,state-n] (n3t) [right=of n2t] {3f};
  \node[state,state-n] (n4t) [right=of n3t] {4f};
  \node[state,state-n] (n5t) [right=of n4t] {5f};
  \node[state,state-b] (n6t) [right=of n5t] {6f};

  \node[state,state-i] (n1f) [below=of n1t] {1t};
  \node[state,state-n] (n2f) [right=of n1f] {2t};
  \node[state,state-n] (n3f) [right=of n2f] {3t};
  \node[state,state-n] (n4f) [right=of n3f] {4t};
  \node[state,state-n] (n5f) [right=of n4f] {5t};
  \node[state,state-b] (n6f) [right=of n5f] {6t};

  \draw [->,myedge] (n1t) -- (n2t);
  \draw [->,myedge] (n2t) -- node[mylabel]{\begin{tabular}{c}$\forall j\neq i :$\\$\set{1t,2t,3t,4t,5t,6t,1f}$\end{tabular}} (n3t);
  \draw [->,myedge] (n4t) -- (n5t);
  \draw [->,myedge] (n5t) -- node[mylabel]{\begin{tabular}{c}$\forall j\neq i :$\\$\set{1t,1f,2t,2f,3t,3f,4t,4f}$\end{tabular}} (n6t);
  \draw [->,myedge] (n5t) .. controls +(-1,1) and +(1,1) .. (n1t) node[mylabel,above]{$\exists j\neq i :\set{5t,6t,5f,6f}$};
  \draw [->,myedge] (n6t) .. controls +(0,2) and +(0,2) .. (n1t);


  \draw [->,myedge] (n3t) -- node[mylabel,sloped,above]{\emph{Broadcast}} (n4f);

  \draw [->,myedge] (n1f) -- (n2f);
  \draw [->,myedge] (n2f) to[out=-30,in=-150] (n4f);

  \draw [->,myedge] (n4f) -- (n5f);
  \draw [->,myedge] (n5f) -- node[mylabel]{\begin{tabular}{c}$\forall  j\neq i : $\\$\set{1t,1f,2t,2f,3t,3f,4t,4f}$\end{tabular}} (n6f);
  \draw [->,myedge] (n5f) .. controls +(-1,-1) and +(1,-1) .. (n1f) node[mylabel]{$\exists j\neq i : \set{5t,6t,5f,6f}$};

  \draw [->,myedge] (n6f) .. controls +(0,-2) and +(0,-2) .. (n1f);

  \foreach \x in {1,2,3,4,5,6}{ \draw [<->,myedge,draw=gray] (n\x t) -- node[rotate=45,scale=0.5,midway]{\color{gray}Global change} (n\x f); }     

  %% Uncomment to see the limits of the bounding box
  % \fill[red] (current bounding box.north west) circle (2pt);
  % \fill[red] (current bounding box.south east) circle (2pt);
\end{tikzpicture}
\vspace{-12pt}
\caption{The transitions of Dijkstra's protocol}
\label{figure:dijkstra:transitions}
\end{center}
\end{figure}
