%% -------------------------------------------------
\subsection{Procedure}\label{section:procedure}
%% -------------------------------------------------

We describe now our verification procedure, inspired
from~\cite{AbHaHo:view:abstraction}. %
In this section, we fix a size $k$ for the views. For all the examples
from Section~\ref{section:experiments}, $k$ is in fact 2.

The basic idea is to compute a set $V$ of views (of size up to $k$)
that over-approximates the system configurations of any size, i.e.,
$\reach\subseteq\concretizeof k V$, such that $\concretizeof k V
\intersection \bad = \emptyset$.\footnote{%
%
  In the case where $\concretizeof k V \intersection \bad \neq
  \emptyset$, one must examine whether it is the result of a real
  error or if it is a spurious counterexample, i.e., an artifact of
  the abstraction. %
  In the latter case, as in~\cite{AbHaHo:view:abstraction}, it is
  possible to increase the precision of the abstraction (i.e.,
  increase $k$) and re-run the procedure.
%
} %
%

\noindent%
\begin{wrapfigure}{r}[0pt]{0.55\textwidth}
\vspace{-24pt}
\begin{minipage}{0.55\textwidth}
\begin{algorithm}[H]
\DontPrintSemicolon
\caption{Accelerated procedure}
\label{algo:accelerated}
$\worklist := \vinits$\;
$\visited := \emptyset$\;
\While{$\worklist \neq \emptyset$}{%
  remove some $\view$ from $\worklist$\;
  \lIf{$\view$ is bad\nllabel{line:isbad}}{\Return Unsafe}
  \eIf{$\exists v'\in\visited,\;v'\entails \view$\nllabel{line:discard}}{%
    discard $\view$ %
  }{%
    $X := \concretizeoflim k {k+1} {\set{\view} \union \visited}$\nllabel{line:extension}\;
    $\worklist := \worklist \;\union\; \abstr_k\circ\spostof X$\nllabel{line:post}\;
    $\visited := \set{\view} \union \setcomp{v'\in\visited}{\view\not\entails v'}$\nllabel{line:minset}\;
  }
}
\Return Safe
\end{algorithm}
\end{minipage}%
\vspace{-20pt}
\end{wrapfigure}
%
The procedure is implemented as a worklist algorithm and is described
in Algorithm~\ref{algo:accelerated}. %
It takes as input a set of initial views $\vinits$ and a function that
checks if a view corresponds to a bad configuration. %
It maintains two sets of views: %
(i) a set of views $\worklist$ that have not yet been analyzed. %
$\worklist$ is initialized to $\vinits$; and %
(ii) a set $\visited$, initially empty, that contains information
about the views that have already been analyzed.
%

Algorithm~\ref{algo:accelerated} uses the entailement relation and is
instantiated with the following parameters: %
\begin{enumerate}
\item[(i)]%
  $\vinits = \project k {\inits}$. %
  Computing the abstraction $\project k \inits$ of the initial
  configurations is usually an easy step. %
  For instance, in the case of Szymanski's protocol, all processes are
  initially in state\,\w{0}, hence $\project k \inits$ contains
  only views where the basis is the word $\word[1]{\w{0}} \ell$
  for $\ell\in\range 1 k$ and where the contexts are either \wc[]{}
  or \wc[0]{}. %
  Generally, $\inits$ is a~(very simple) regular set, and $\project k
  \inits$ is computed using a~straightforward automata construction.
\item[(ii)]%
  the function that returns if a given view represents a configuration from the set $\bad$.
  %
  Evaluating the test ``$\view$ is bad'' is carried out by checking if the view $\view$ is entailed by one of
  the view from the set $\project k {\bad}$.
  %
  Since $\bad$ is the upward closure of a~finite set $\minbad$, the
  test can be replaced with $\exists b\in\project k \minbad$ such that
  $b\entails\view$.
  %
  For instance, in the case of Szymanski's protocol, a view
  corresponds to a bad configuration if it is entailed by, e.g., 
  $\wc{9/,9/}$.
\end{enumerate}

%% ------------------------------------------------------------------
\myparagraph{Over-approximation.}
%% ------------------------------------------------------------------
The procedure reaches, by construction, a fixpoint $\visited$ which,
by lemma~\ref{lemma:apost}, over-approximates the fixpoint of the
abstract post starting from the set of views $\vinits$, if it does not
prematurely return with a negative answer (Unsafe). %
% Formally, we define $V_a := \mu X~.~\project k \inits \union \apostof
% k X$ and we have that $\visited \entails V_a$, i.e., $\visited$ is
% stronger than $V_a$ or that the set of configurations that $\visited$
% represents is larger than the one represented by $V_a$.
% %
% $\visited$ is an invariant for the instances of the system, that is,
% $\reach \subseteq \concretizeof k \visited$.%
%
%\TODO{Should I prove that, or it is clear? (like we did in VMCAI'13)}
Formally, we define $V_a := \mu X~.~\project k \inits \union \apostof
k X$ and we have that %
$\concretizeof k {V_a} \subseteq \concretizeof k \visited$.
%	
Since $\reach \subseteq \concretizeof k {V_a}$, it follows that %
$\reach \subseteq \concretizeof k \visited$ and $\visited$ is an
invariant for the instances of the system.%
\TODO{Use $\concretizeof k {gen(X)} = \concretizeof k X$ ?}%

Moreover, the algorithm also maintains, by construction, the following invariant:
$\concretizeof k \visited \intersection \bad = \emptyset$.
%
Therefore, we can use the (finite) set $\visited$ to approximate all
reachable configurations of system (of any size) and verify/falsify
safety properties for configurations of arbitrary size. 
%
The procedure is sound.
%
\TODO{Use: $\reach \subseteq \concretizeof k {\project k \reach}$, so
  we are clearly dealing with an over-approximation? %
}%
%

%% ------------------------------------------------------------------
\myparagraph{Acceleration.}
%% ------------------------------------------------------------------
We discuss a major acceleration for this procedure which leverages the
power of the entailment relation. %
It is based on the observation that $\reachof k$ contains
configurations of size k, which can be used as initial input for the
procedure. 
%
A careful reader will observe that $\project k {\reachof k}$ contains
views of size (up to) $k$ where all views of size $k$ have empty
contexts. These will be useful to cut down on the computations of the
symbolic post, %$\spost$
since its results will often be views that are already in $\project k
{\reachof k}$ but with larger contexts, and therefore immediately
discardable.
%
Indeed the successor of a view that is entailed by a view from
$\project{k} {\reachof {k}}$ would not bring any new information to
the analysis. Notice that we are not leveraging the monotonicity of
the transition system (because it is not!), but a rather simpler fact:
If a transition was to be enabled on \emph{both} views $\view_1$ and
$\view_2$ such that $\view_1\entails\view_2$, we can simply observe
that $\spostof{\view_1}\entails\spostof{\view_2}$. %
Since $\reachof {k}$ is stable by $\post$, $\project{k} {\reachof
  {k}}$ is also stable by $\spost$, we can safely discard the views
that are entailed by a view from $\project{k} {\reachof {k}}$. %
We could therefore use $\vinits = \project k {\reachof k}$.

Furthermore, it is possible to use %$\project {k+1} {\reachof {k+1}}$ %
$\mk{\reachof {k+1}}$ %
to cut down on the computation of the extensions
(line~\ref{line:extension}). For a given view $\view$,
$\concretizeoflim k {k+1} {\set{\view} \union \visited}$ will indeed
compute views of size $k+1$, with potentially non-empty contexts. If
this result is already entailed by a view in %
%$\project {k+1} {\reachof {k+1}}$, %
$\mk{\reachof {k+1}}$ %
it can safely be ignored, using the same idea as above.

Finally, since $\reachof k$ and $\reachof {k+1}$ are finite, their
computation can be done using any procedure for exact state-space
exploration and we use $\vinits = \project k {\reachof k} \union
\project k {\reachof {k+1}}$. %
We also use dynamic programming to compare the extensions with
%$\project {k+1} {\reachof {k+1}}$. %
$\mk{\reachof {k+1}}$. %
The net result of these accelerations is that most experiments happen
to be already at fixpoint, after the computation of $\vinits$, which
demonstrates the efficiency of the method
(cf. Section~\ref{section:experiments}) and that most behaviours are
captured small instances of the system.
%
% %% ------------------------------------------------------------------
% \myparagraph{Running example.}
% %% ------------------------------------------------------------------
\TODO{%
 Include the Running Example with Szymanski?%
}%
% When run on Szymanski's protocol, the accelerated version of
% Algorithm~\ref{algorithm:computable} starts by computing $\reachof 2
% = \{0,\ldots,6\}$.  Because $\reachof 1$ does not contain any bad
% configurations, the algorithm moves onto computing the fixpoint
% $V_1$ of line~\ref{line:fixpoint}. The iteration starts with $X =
% \alpha_1(\inits) = \{1\}$ and continues until $X = V_1 =
% \{1,\ldots,6\}$.  The test on line~\ref{line:test} subsequently
% fails since $\gammaof 1(V_1)$ contains $66$.  Since both tests fail,
% the first iteration does not allow us to conclude whether the
% protocol is safe or not, so the algorithm increases the precision of
% the abstraction by increasing $k$.
%
% In the second iteration with $k=2$, $\reachof 2$ is still safe.  The
% fixpoint computation starts with $X = \alpha_2(\inits) = \{1,11\}$.
% When $\apost 2$ is applied on $\{1,11\}$, we first construct the set 
% $\limgammaof 2 {2+1}(\{1,11\})$ which contains the extension $111$ of $11$, $11$ and $1$.
% Their successors are $2,12,21$, and $112,121,211$, which are abstracted into
% $\{1,2,11,12,21\}$.
% %
% The fixpoint computation continues with $X = \{1,2,11,12,21\}$ and
% constructs the concretization $\limgammaof 2 3 (X) = X\cup \{112,121,211\}$. Their successors are $2,3,12,21,22,31,13$, and $122$,
% $212$, $221$, $113$, $131$, $311$
% %
% which are abstracted into the views $1,2,3,11,12,21,22,31,13$.  
% %
% The next iteration will start with $X = \{1,2,3,11,12, 21,22, 13,31\}$.
% %
% The computation reaches, after 8 further iterations, the fixpoint
% $X=V_2$ which contains all words from $\{1,\ldots,6\}\cup\{1,\ldots,6\}^2$ except $65$
% and $66$.
% %
% This set satisfies the assumptions of Lemma~\ref{lemma:invariant}, and
% hence it is guaranteed to contain all views (of size at most 2) of all
% reachable configurations of the system.
% %
% Since the view $66$ is not present (recall $\alpha_2(\bad)=\{6,66\}$),
% no reachable configuration of the system is bad. The algorithm reached
% the cut-off point $k=2$ of Burns' protocol, and the system is proved
% safe.
%
