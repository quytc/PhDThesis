%-----------------------------------
\section{Conclusion and Future Work}
%\section{Conclusion}
\label{section:conclusion}
%-----------------------------------
%
We have presented a method for automatic verification of parameterized
systems which extends the view abstraction
from~\cite{AbHaHo:view:abstraction} but alleviates the lack of
precision it exhibits on systems with almost upward-closed invariants.
%
This is a unique method that combines the feature of discovering non
upward-closed invariant while allowing to model systems with
fine-grained transitions.

The method allows to perform parameterized verification by only
analyzing a small set of instances of the system (rather than the
whole family) and captures the reachability of bad configurations to
imply safety.
%
Our algorithm relies on a very simple abstraction function, where a
configuration of the system is approximated by breaking it down into
smaller pieces. This give rise to a finite representation of infinite
sets of configurations while retaining enough precision.
%
% Inspired by a strong empirical evidence, parameterized systems often
% enjoy a {\it small model property}.
% %
% More precisely, analyzing only a small number of processes (rather
% than the whole family) is sufficient to capture the reachability of
% bad configurations and therefore imply safety.
%
We have proven that the presented algorithm is complete for systems
with almost upward-closed invariants.
%
Based on the method, we have implemented a prototype which performs
efficiently on a wide range of benchmarks.
%

We are currently working on extending the method to the case of
multi-threaded programs running on machines with different memory
models.
%
These systems have notoriously complicated behaviors.
%
Showing that verification can be carried out
through the analysis of only a small number of threads
would allow for more efficient algorithms for these systems.
%
%

% Obviously, the bottleneck in the application of the method is when the
% computation of $\reachof k$ is necessary for high values of~$k$ or if
% the set of bad configurations is complex (in terms of size).
% %
% We plan therefore to integrate the method with advanced 
% tools that can perform efficient forward reachability analysis, like
% SPIN~\cite{SPINModelChecker}, and to use efficient
% symbolic encodings for compact representations for the set of views.
