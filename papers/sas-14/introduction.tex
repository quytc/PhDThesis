%-----------------------------------
\section{Introduction}
\label{section:introduction}
%-----------------------------------
%
We consider the verification of safety properties for
\emph{parameterized systems} with global
conditions. %
They typically consist of an arbitrary number of components
(processes) organized according to a certain topology. %
%
Processes in the system can perform a transition provided that they
satisfy the global condition associated with this transition. %
For example, in a linear topology, a process (at position~$i$) can
perform a transition only if all processes to its left (i.e. with
index~$j<i$) satisfy a property~$\varphi$. The latter is an
universally quantified transition. %
For an existentially quantified transition, it is required that
\emph{some} (rather than \emph{all}) processes satisfy~$\varphi$.
%
% For a ring topology, (resp.\ a~tree topology), global conditions might
% involve the immediate neighbor (resp.\ parent and/or children)
% processes.
% %
% Notice that the global conditions are not necessarily checked
% atomically: Other processes can perform transitions, even quantified,
% while the global condition check is carried out.
% %
% % Other topologies are examined in a companion
% % paper~\cite{AbHaHo:view:abstraction}.

%
The task is to perform \emph{parameterized verification}, i.e., to
verify correctness regardless of the number of processes. %
This amounts to the verification of an infinite family; namely one for
each possible size of the system. %
The term parameterized refers to the fact that the size of the system
is (implicitly) a parameter of the verification problem. %
%
Parameterized systems arise naturally in the modeling of mutual
exclusion algorithms, bus protocols, distributed algorithms,
telecommunication protocols, and cache coherence protocols. %
For instance, the specification of a mutual exclusion protocol may be
parameterized by the number of processes that participate in a given
session of the protocol. %
%
In such a case, it is interesting to verify correctness regardless of
the number of participants in a particular session. %
As usual, the verification of safety properties can be reduced to the
problem of checking the reachability of a set of bad configurations
(the set of configurations that violate the safety property).

%% --------------------------
The verification problem is known to be undecidable in general, but
decidable for the subclass of parameterized systems which are well
quasi-ordered
(WQO)~\cite{Parosh:Bengt:Karlis:Tsay:general,abdulla:well}.
%
There are numerous methods targeting primarily WQO systems and proven
to be complete for them. %
Universally quantified transitions are not monotonic, and systems with
such transitions are not WQO. %
Some of these methods are however still sound for such systems, and
were successfully used to verify many of them, mostly because these
systems have an inductive and upward-closed invariant which is strong
enough to imply safety. %
The WQO specialized methods are bound to fail, if they cannot derive
such upward-closed invariants.
%
Refer to Section~\ref{section:related_work} for more details on
related works.

%-----------------------------------
\myparagraph{Our contribution.}
%-----------------------------------
We observe that there exists in fact a significant subclass of systems
with ``almost'' upward-closed invariants, that is, invariants which
may be expressed as a list % unions
of \emph{finite} intersections of upward-closed sets.
%
Let us illustrate such a situation with a simple, yet characteristic,
example: the implementation of a barrier synchronization.
%
A process arriving at the barrier must block and cannot proceed until
all other processes reach this barrier.
%
The first process at the barrier acts as a \emph{pivot}. All other
processes must wait as long as there is a pivot. %
%
When all processes have arrived at the barrier, the pivot can proceed,
which in turn releases all the waiting processes.
%

\begin{wrapfigure}{r}{0.28\textwidth}
  \vspace{-24pt}
  %\hfill%
  \begin{center}
    \input demo
  \end{center}
  %\vspace{-6mm}
  %\caption{State diagram (per process).}
  \label{figure:demo}
  \vspace{-24pt}
\end{wrapfigure}
%
\noindent%
% A process can be in state $\sP$ (\textbf{p}ivot), $\sW$ (\textbf{w}aiting), $\sB$ or $\sA$
% (\textbf{b}efore and \textbf{a}fter the barrier).
A process can be in state $\sP$ (pivot), $\sW$ (waiting), $\sB$ or
$\sA$ (before and after the barrier).
%
All processes are initially in state $\sB$ and %
a bad configuration is detected when there is a process in state $\sA$,
while there is still a process in state $\sB$ %
% (cf Figure~\ref{figure:barrier}).
(cf the figure on the right).
%
It is clear that a pivot~$\sP$ can coexist with a process in state~$\sB$ and/or $\sW$, or
is alone, i.e.\ the set of configurations $\set{\sP\wedge\sB^*\wedge\sW^*}$. %
% and (ii) a process after the barrier can only coexist with other
% process after the barrier or still waiting, i.e.\ the set of
% configurations $\set{\sW^*\wedge\sA^+}$. %

We observe that a waiting process $\sW$ strongly counts on the fact that
if there is a delayed process $\sB$, then there is \emph{also} a process
$\sP$. %
Such a waiting process bases its decision to move through the barrier
\emph{solely} on the absence of the pivot (which implies the absence
of a process in the state $\sB$).
%
In other words, a configuration containing two processes in state
$\sW$ and $\sB$ \emph{must} also contain a pivot $\sP$, i.e.\ it must
be in $\uparrow\set{\sW^+\wedge\sB^+}\cap\uparrow\set{\sP}$.
%
An intersection of upward-closed sets is not generally upward-closed.
%
Deriving only $\uparrow\set{\sW^*\wedge\sB^*}$ is clearly not strong
enough, because, in such a set, a process in state $\sW$ could cross
the barrier, violating the safety property.
% (The latter is not even inductive).
%
We can recognize here a pattern, common for many systems with almost
upward-closed invariants: %
Two processes $p_1$ and $p_2$ count collectively on the presence of
another process $p_3$ in a certain state, in order to \emph{not}
perform a critical step. %
%This is captured by a \emph{context-sensitive} invariant.
Any invariant that does not derive the presence of the blocking
process $p_3$ is (potentially) not strong enough.
%
% The presence of an entity $E_1$ (here $\uparrow\set{W^+\wedge B^+}$) is implied by the presence of an entity $E_2$ ($\uparrow\set{P}$), 
% and the system performs a critical step which involves testing the absence of $E_1$ by the absence of $E_2$. 
% %
% It is clear that the strong enough inductive invariant must imply that
% $E_1 \Rightarrow E_2$, in other words, it cannot intersect with
% $\uparrow E_1 \cap \uparrow E_2$.

%% ---------------------------------
We propose a method that can \emph{fully automatically} infer such
\emph{context-sensitive} invariants. %
Of course, since we are dealing with an undecidable problem, the
method has its limitations. %
For instance, we are not able to infer (even to express) invariants
which involve equalities of numbers of processes in different states
(such invariant could be expressed as an \emph{infinite} intersection
of upward closed sets).

% This work introduces a context-sensitive view abstraction based
% on~\cite{AbHaHo:view:abstraction}, that alleviates the lack of
% precision that the latter can exhibit. %
Our method retains the simplicity and efficiency of the method
from~\cite{AbHaHo:view:abstraction} for verification of WQO systems,
on which it is based.
%
Thanks to its simplicity, we were able to extend it to handle
fine-grained parameterized systems, where the global (quantified)
conditions are checked non-atomically. %
%
The implementations of checking the global conditions are typically
implemented as for-loops ranging over process indices.
%
Verification of systems with non-atomically checked global conditions
is significantly harder. The models are closer to the real systems and
require to distinguish intermediate states of a for-loop performed by
a process. Moreover, any number of processes may be performing a
for-loop simultaneously. %at the same time.
%
% The models are closer to to the real systems and can reveal bugs which
% would otherwise stay hidden.
%

To the best of our knowledge, there is no other method which combines
the ability to infer non upward-closed invariants with the support of
fine-grained modeling.
%
In contrast to many landmark works on parameterized systems
\cite{CTV06,PRZ-tacas01,BHV04,AJNO:simple,APRXZ01,Namjoshi:VMCAI07,rmc:wo:transducers}
which either assume that the universally and existentially quantified
transitions are performed atomically or that use simplified models
that palliate to the non upward-closed invariants, %
we were able to verify, e.g., the full fine-grained version of
Szymanski's mutual exclusion protocol. %
The latter was an open problem in parameterized verification. %
% , which has been considered a
% challenge in parameterized verification. %
%
%Besides Szymanski's protocol, 
We have moreover tested our method on numerous other case studies,
some of which cannot be verified by any other methods we are aware of,
and we could observe the promising efficiency of the method.
%
% Finally, we show that when lifting our reasoning to a more abstract
% level, we obtain a verification scheme which is complete for systems
% with almost upward-closed invariants.
Finally, we show that our method gives rise to a verification scheme
which is complete for systems with almost upward-closed invariants.

%% -----------------------------------------------------
\myparagraph{Outline.}
%% -----------------------------------------------------
To simplify the presentation, we introduce the class of systems we
consider in a stepwise manner. First, we consider a basic model in
Section~\ref{section:paramsys} where we only allow atomically checked
global conditions. We describe how to derive the view abstraction that
retains enough information to carry out the verification procedure,
fully automatically, in Section~\ref{section:method}.
%
Then, we introduce in Section~\ref{section:non-atomic} how the
abstraction is adapted in order to handle the non-atomic setting. We
describe how the abstracted system subtly characterizes the original
system, with \emph{in-order} non-atomic checks, while retaining its
simplicity.
%
We report on our experimental results in
Section~\ref{section:experiments}, and describe related work in
Section~\ref{section:related_work}. Finally, we give some conclusions
and directions for future research in
Section~\ref{section:conclusion}.
