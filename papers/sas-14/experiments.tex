%-----------------------------------
\section{Implementation and Experiments}
\label{section:experiments}
%-----------------------------------
%
We have implemented a prototype in OCaml based on our method to verify
safety properties for a numerous protocols.
%
We report the results running on a 3.1~GHz computer with 4GB~memory.
%
Table~\ref{experiments:results-linear} displays, for various protocols
with linear topology (over 2 lines), the running times (in seconds),
the final number of views generated ($\sizeof V$) and the number of
bases.
%
The first line is the result of the atomic version of the protocol,
while the second line represents the non-atomic version.
%
The complete descriptions of the experiments can be found in
Appendix~\ref{appendix:experiments}.
%
In most cases, the method terminates almost immediately illustrating
the \emph{small model property}: all patterns occur for small
instances of the system.
%

\begin{wraptable}{r}[0pt]{0.63\textwidth}
%\begin{minipage}{0.68\textwidth}
\vspace{-32pt}
%\begin{table}[ht]%[!b]
\caption{Linear topologies. Atomic vs Non Atomic}
\label{experiments:results-linear}
%\centering
\hfill%
\begin{tabular}{|l||c||c|c|c|}\hhline{*{5}{=}}
 {\bf Protocol}                              & {\bf Time} & {\bf $\sizeof{bases}$} & {\bf $\sizeof V$} & {\bf status}\\\hhline{*{5}{=}}


 \multirow{2}{*}{Parosh's example}                    & 0.005      & 21     & 22     & $\safe$   \\
                                                      & 0.006      & -      & -      & $\unsafe$ \\\hline
 \multirow{2}{*}{Burns}                               & 0.004      & 34     & 34     & $\safe$   \\
                                                      & 0.011      & 34     & 64     & $\safe$   \\\hline
 \multirow{2}{*}{Dijkstra}                            & 0.027      & 93     & 93     & $\safe$   \\
                                                      & 0.097      & 93     & 222    & $\safe$   \\\hline
 \multirow{2}{*}{Szymanski}                           & 0.307      & 100    & 168    & $\safe$   \\
                                                      & 1.982      & 100    & 311    & $\safe$   \\\hline
 \multirow{2}{*}{Szymanski (compact)}                 & 0.006      & 48     & 48     & $\safe$   \\
                                                      & 0.557      & 48     & 194    & $\safe$   \\\hline
 Szymanski (random)                                   & 1.156      & -      & -      & $\unsafe$ \\\hline
 \multirow{2}{*}{Bakery}                              & 0.001      & 7      & 7      & $\safe$   \\
                                                      & 0.006      & 9      & 30     & $\safe$   \\\hline
 \multirow{2}{*}{Gribomont-Zenner}                    & 0.328      & 119    & 143    & $\safe$   \\
                                                      & 32.112     & 119    & 888    & $\safe$   \\\hline
 Simple Barrier                                       & 0.018      & 31     & 61     & $\safe$   \\
 (as array)                                           & 1.069      & 30     & 253    & $\safe$   \\\hhline{*{5}{=}}

\end{tabular}
%\end{table}
%\end{minipage}
\vspace{-20pt}
\end{wraptable}
%
For the first example of Table~\ref{experiments:results-linear} in the
case of non-atomicity, our tool reports the protocol to be
$\unsafe$. It is indeed a real error and not an artifact of the
over-approximation. The method is sound.
%
In fact, this is also the case when we intentionally tweak the
implementation of Szymanski's protocol and force the for-loops to
iterate randomly through the indices, in the non-atomic case. %
The tool reports a trace, that is, a sequence of configurations ---
here involving only 3 processes --- as a witness of an (erroneous)
scenario that leads to a violation of the mutual exclusion property.

%
\def\daznc{%
  \tikz[xshift=8ex,yshift=0.3ex]{%
    \node[left,scale=0.9,rotate=20,color=blue!20,overlay]{contexts};
    \node[left,scale=0.9,rotate=20,color=blue!20,overlay] at (5ex,0){needed};}
}
%
\begin{wraptable}{l}[0pt]{0.68\textwidth}
%\begin{minipage}{0.68\textwidth}
\vspace{-28pt}
%\begin{table}[ht]%[!b]
\caption{Petri Net with Inhibitor Arcs}
\label{experiments:results-pn}
\centering
\begin{tabular}{|l||c||c|c|c|}\hhline{*{5}{=}}
 {\bf Protocol}           &  {\bf Time}  & {\bf $\sizeof {bases}$}& {\bf $\sizeof V$} & {\bf status}\\\hhline{*{5}{=}}
 Critical Section with lock                           & 0.001      & 42     & 42     & $\safe$   \\
 Priority Allocator                                   & 0.001      & 33     & 33     & $\safe$   \\
 Barrier with Counters                                & 0.001      & 22     & 22     & $\safe$   \\\hline
 Simple Barrier       $\daznc$                        & 0.001      & 7      & 8      & $\safe$   \\
 Light Control                                        & 0.001      & 9      & 15     & $\safe$   \\
 List with Counter Automata                           & 0.002      & 20     & 38     & $\safe$   \\\hhline{*{5}{=}}

\end{tabular}
\vspace{-16pt}
%\end{table}
\end{wraptable}
%
\noindent%
We also ran the method to verify several examples with a multiset
topology: Petri nets with inhibitor arcs. %
Inhibitor places should retain a context in order to not fire the
transition and potentially make the over-approximation too coarse. The
bottom part of Table~\ref{experiments:results-pn} lists examples where
the contexts were necessary to verify the protocol, while the top part
lists examples that didn't require any context.


%\tikz[remember picture,overlay]{\node[right,rotate=30,color=blue!20] at (nc){contexts needed};}

%% ------------------------------------------------------------------
\myparagraph{Accelerations.}
%% ------------------------------------------------------------------
We discuss a major acceleration for the procedure
in~\ref{section:procedure} which leverages the power of the entailment
relation. %
It is based on the observation that $\reachof k$ contains
configurations of size $k$, which can be used as initial input
for the procedure.
%
A careful reader will observe that $\project k {\reachof k}$ contains
views of size (up to) $k$ where all views of size $k$ have empty
contexts. These will be useful to cut down on the computations of the
symbolic post, %$\spost$
since its results will often be views that are already in $\project k
{\reachof k}$ but with larger contexts, and therefore immediately
discardable by entailement.
%
Indeed the successor of a view that is entailed by a view from
$\project{k} {\reachof {k}}$ would not bring any new information to
the analysis. %
%
It is therefore interesting to seed the fixpoint computation with a
larger set than $\project k \inits$, namely $\project k {\reachof k}$.
%
% Notice that we are not leveraging the monotonicity of
% the transition system (because it is not!), but a rather simpler fact:
% If a transition was to be enabled on \emph{both} views $\view_1$ and
% $\view_2$ such that $\view_1\entails\view_2$, we can simply observe
% that $\spostof{\view_1}\entails\spostof{\view_2}$. %
% Since $\reachof {k}$ is stable by $\post$, $\project{k} {\reachof
%   {k}}$ is also stable by $\spost$, we can safely discard the views
% that are entailed by a view from $\project{k} {\reachof {k}}$. %
% We could therefore use $\vinits = \project k {\reachof k}$.
%-----------------------
Furthermore, it is possible to use $\project {k+1} {\reachof {k+1}}$ %
to cut down on the computation of the extensions
$\concretizeoflim{k}{k+1}{X}$. %
%
If a view $v$ from $\concretizeoflim k {k+1} {X}$ is already entailed
by a view $w$ in $\project {k+1} {\reachof {k+1}}$, %
it is again possible to safely ignore $v$, using the same argument
above ($w$ has an empty context) and since we already have computed
the symbolic post on the view $w$.
%-----------------------
Finally, since $\reachof k$ and $\reachof {k+1}$ are finite, their
computation can be done using any procedure for exact state-space
exploration and we seed the symbolic fixpoint computation with
$\project k {\reachof k \union \reachof {k+1}}$. %
%
The net result of these accelerations is that most experiments happen
to be already at fixpoint, which demonstrates the efficiency of the
method and that most behaviours are captured by small instances of the
system.

%% -----------------------------------------------------
\myparagraph{Heuristics.}
%% -----------------------------------------------------
If we can empirically observe that $\project 2 {\reachof 2 \union
  \reachof 3} = \project 2 {\reachof 2 \union \reachof 3 \union
  \reachof 4}$, we are probably already at fixpoint. It is therefore
interesting to inspect whether a new view could be inserted while
computing the symbolic post on $\concretizeoflim 2 3 {X}$, even though
we ignored contexts.
%
If not, we were indeed at fixpoint already and the invariant we
discovered is strong enough to imply safety.
% 
If so, we can stop the computations, fetch the views that caused the
insertion and remember their contexts for the next round of
computations.
%
This heuristic happen to be very successful in the case of Szymanski's
protocol (in its non-atomic full version).

On the other hand, this strategy can also be applied in general. We do
not remember any contexts, and if the procedure discovers a
false-positive, we trace the views that generated it and remember
their contexts for the next round of computations, in a CEGAR-like
fashion. This will in fact discover necessary contexts little at a
time. We are convinced that any smarter counter-example analysis,
other than the basic one we implemented, will lead to faster context
discovery, as this heuristic is interesting if few views need context.
%
It is however inefficient if it happens that mostly all views need a
context (as shown with the ring agreement example).
%
Table~\ref{experiments:accelerations} presents the results of using
the insertion and context discovery heuristics.
%
\begin{table}[ht]%[!b]
%\vspace{-6mm}
\caption{Leveraging the heuristics}
\label{experiments:accelerations}
\centering
\begin{tabular}{|rl||c||c|c|c|}\hhline{*{6}{=}}
 {\bf Protocol} &             &  {\bf Time}  & {\bf $\sizeof{bases}$}& {\bf $\sizeof V$} & {\bf iteration}\\\hhline{*{6}{=}}
 \multirow{3}{*}{Agreement}             & with insertion heuristic & 8.247      & 76     & 199    & 28        \\
                                        & with all contexts        & \tikz[baseline=(n.base)]{\node(e)[ellipse,color=blue!20,fill=blue!20]{3.950};\node(n){3.950};}      & 76     & 216    & 1         \\
                                        & with contexts discovery  & 166.893    & 77     & 121    & 4         \\\hline
 \multirow{2}{*}{Gribomont-Zenner}      & with insertion heuristic & \tikz[baseline=(n.base)]{\node(e)[ellipse,color=blue!20,fill=blue!20]{0.328};\node(n){0.328};}      & 119    & 143    & 7         \\
                                        & with all contexts        & 0.808      & 119    & 317    & 1         \\
                                        & with contexts discovery  & 50.049     & 119    & 217    & 3         \\\hline
 \multirow{2}{*}{Szymanski, non-atomic} & with insertion heuristic & \tikz[baseline=(n.base)]{\node(e)[ellipse,color=blue!20,fill=blue!20]{2.053};\node(n){2.053};}      & 100    & 311    & 26        \\
                                        & with all contexts        & 48.065     & 100    & 771    & 1         \\
                                        & with contexts discovery  & 732.643    & 100    & 896    & 7         \\\hhline{*{6}{=}}
\end{tabular}
%\vspace{-6mm}
\end{table}
