%-----------------------------------
\section{Non-atomically checked global conditions}
\label{section:non-atomic}
%-----------------------------------
%
We extend our model and method to handle parameterized systems where
global conditions are \emph{not} checked atomically.
%
We replace both existentially and universally guarded transition rules
by the following variant of a~for-loop rule:
$$%
\forrule \circ \witnesses s {s'} e%
$$
where $s\in\locs$ is a \emph{source} state, $s'\in\locs$ is a
\emph{target} state and $e\in Q$ is an \emph{escape}
state. $\circ\in\{<,>,\neq\}$ describes the \emph{range} of iterations
and $\witnesses\subseteq Q$ is a \emph{condition}.
%
For instance, line 3 of Szymanski's protocol would be replaced by 
\mbox{$\forrule \neq {\neg\set{1,2}} 3 7 4$}.

We extend the semantics of a~system with for-loop rules from
transition systems of Section~\ref{section:paramsys} in the following
way:
%
A~configuration is now a~pair $(\conf,\tick)$ where $\conf$ is a~word
over $Q$ and $\tick$ is a~partial boolean function over its (pairs of)
positions, i.e.,
$\tick:{\range{1}{\sizeof\conf}\times\range{1}{\sizeof\conf}}\rightarrow(\strue,\sfalse)$.
%
The boolean function $\tick$ is used to encode intermediate states of
for-loops. Intuitively, a~process at position $i$ performing
a~for-loop sets $\tickof{i}{j}$ to $\strue$ to mark that it
processed\TODO{\emph{in-order}?} the position $j$. It is otherwise
$\sfalse$.
%
Initially, $\tick$ is the false function: $\tickof i j = \sfalse$ for
all $i,j\in\range 1 {\sizeof \conf}$. 

Without loss of generality, we will assume that the for-loops iterate
through process indices in increasing order.
%
For convenience, we define the following operators: for a
(partial) %boolean
function $f$, we use $f\oplus(i,j)$, to denote the function $f'$ such
that $f'(x,y) = \strue$ if $x,y=i,j$ and $f'(x,y) = f(x,y)$
otherwise. We use $f\ominus i$ to denote the function $f'$ such that,
$f'(x,y) = \sfalse$ if $x=i$ and $f'(x,y) = f(x,y)$ otherwise (for all
values of $y$).

Formally, a~parameterized system $\parsys=(\locs,\rules)$ which
includes for-loop rules induces a~transition system $\transys =
(\confs,\trans)$ where $C \subseteq Q^+\times (\nat\times\nat)$. %
For the sake of simplicity, we assume that a~source of a~for-loop rule
in $\rules$ is not a~source of any other rule in $\rules$%
%
\footnote{This is a technicality, the state of a~process, without this
  restriction, would have to contain additional information recording
  which for-loop is the process currently performing. %
  Note that the restriction does not limit the power of our model. %
  Any potential branching may be moved to predecessors of the sources
  of the for-loop.}. %
%
Then every for-loop rule $\forrule \circ \witnesses s {s'} e$ induces
transitions $(\conf,\tick) \trans (\conf',\tick')$ with $\conf [i] =
s$ for some $i\in\range 1 {\sizeof \conf}$ which may be of the
following three forms: (illustrated using the aforementioned example
rule from Szymanski's protocol).

%\begin{description}
%\item[{Iteration:}]
\noindent\begin{minipage}[!t]{0.7\linewidth}%\vspace{0pt}
  %\vspace{-6mm}%
  {\bf Iteration:} The $i^{th}$ process checks that the state of
  the~\emph{first} unchecked process in the range is in the witness
  set $\witnesses$ and marks it. %
  That is, there is a lowest $j\in\range 1 {\sizeof \conf}$ with
  $j\circ i$, such that %
  $\tickof{i}{j}=\sfalse$, $\conf [j]\in\witnesses$, and the resulting
  configuration has $\conf'=\conf$ and $\tick'=\tick\oplus(i,j)$.
\end{minipage}
\hfill%
\begin{minipage}[!t]{0.29\linewidth}%\vspace{0pt}
  %\vspace{6mm}
  \inputtikz{\begin{tikzpicture}[%
  scale=0.5%
  ,state/.append style={scale=0.9,semithick}%
  ,myedge/.append style={very thin,green!20!orange}%
  ]

  \begin{scope}
    \foreach \n in {1,2,3,4,6}{\node[state,state-n](n\n) at (\n,0){ {\hspace{2ex}} };}
    \node[state,state-n](n5) at (5,0){$3$};

    %\draw[myedge,->,blue!70!red] (n5) .. controls +(135:6mm) and +(30:6mm) .. (n4);
    %\draw[myedge,->,blue!70!red] (n5) .. controls +(135:8mm) and +(30:10mm) .. (n3);
    \draw[myedge,->,blue!70!red] (n5) .. controls +(135:10mm) and +(30:14mm) .. (n2);
    \draw[myedge,->,blue!70!red] (n5) .. controls +(135:12mm) and +(30:18mm) .. (n1);

    \draw[myedge,->] (n1) .. controls +(-45:6mm) and +(-150:6mm) .. (n2);
    \draw[myedge,->] (n1) .. controls +(-45:8mm) and +(-150:10mm) .. (n3);
    \draw[myedge,->] (n1) .. controls +(-45:10mm) and +(-150:14mm) .. (n4);

    \begin{pgfonlayer}{my background}
      \node[scale=0.5,fit=(n1)(n6),background rectangle,double=white,fill=black!10!white](bgFrom){};
    \end{pgfonlayer}
  \end{scope}

  \node[rotate=-90] at (5,-1){$\trans$};

  \begin{scope}[shift={(0,-2)}]
    \foreach \n in {1,2,3,4,6}{\node[state,state-n](n\n) at (\n,0){ {\hspace{2ex}} };}
    \node[state,state-n](n5) at (5,0){$3$};

    %\draw[myedge,->,blue!70!red] (n5) .. controls +(135:6mm) and +(30:6mm) .. (n4);
    \draw[myedge,->,blue!70!red] (n5) .. controls +(135:8mm) and +(30:10mm) .. (n3);
    \draw[myedge,->,blue!70!red] (n5) .. controls +(135:10mm) and +(30:14mm) .. (n2);
    \draw[myedge,->,blue!70!red] (n5) .. controls +(135:12mm) and +(30:18mm) .. (n1);

    \draw[myedge,->] (n1) .. controls +(-45:6mm) and +(-150:6mm) .. (n2);
    \draw[myedge,->] (n1) .. controls +(-45:8mm) and +(-150:10mm) .. (n3);
    \draw[myedge,->] (n1) .. controls +(-45:10mm) and +(-150:14mm) .. (n4);

    \begin{pgfonlayer}{my background}
      \node[scale=0.5,fit=(n1)(n6),background rectangle,double=white,fill=black!10!white](bgTo){};
    \end{pgfonlayer}
  \end{scope}

  \begin{pgfonlayer}{my background}
    \draw[background rectangle,double=white] ([shift={(-0.5,0.5)}]bgFrom.north west) rectangle ([shift={(0.5,-0.6)}]bgTo.south east);
  \end{pgfonlayer}

  \node[looplabel] at ([shift={(0,-0.6)}]bgTo.south){Iteration};
\end{tikzpicture}
}
\end{minipage}

%\vspace{-4mm}% Cheating
%\item[{Escape:}]
\noindent\begin{minipage}[!t]{0.7\linewidth}%\vspace{0pt}
  {\bf Escape:} If the state of the~\emph{first} unchecked process in
  the range violates the loop condition, then the $i^{th}$ process may
  escape to the state $e$. %
  %
  That is, there is a lowest $j\in\range 1 {\sizeof \conf}$ with
  $j\circ i$ such that $\tickof{i}{j}=\sfalse$, and $\conf [j]\not\in
  \witnesses$. %
  The resulting configuration has $\conf' [i] = e$ and for all
  $\ell\neq i$, $\conf' [\ell] = \conf [\ell]$. The execution of the
  for-loop ends and the marks of process $i$ are reset, i.e.,
  $\tick'=\tick\ominus i$.
\end{minipage}
\hfill%
\begin{minipage}[!t]{0.29\linewidth}%\vspace{0pt}
  %\vspace{-6mm}
  \inputtikz{\begin{tikzpicture}[%
  scale=0.5%
  ,state/.append style={scale=0.9,semithick}%
  ,myedge/.append style={very thin,green!20!orange}%
  ]

  \begin{scope}
    \foreach \n in {1,2,4,6}{\node[state,state-n](n\n) at (\n,0){ {\hspace{2ex}} };}
    \node[state,state-n](n5) at (5,0){$3$};
    \node[state,state-n,draw=red!50](n3) at (3,0){$1$};


    %\draw[myedge,->,blue!70!red] (n5) .. controls +(135:6mm) and +(30:6mm) .. (n4);
    \draw[myedge,->,blue!70!red,densely dashed] (n5) .. controls +(135:8mm) and +(30:10mm) .. (n3);
    \draw[myedge,->,blue!70!red] (n5) .. controls +(135:10mm) and +(30:14mm) .. (n2);
    \draw[myedge,->,blue!70!red] (n5) .. controls +(135:12mm) and +(30:18mm) .. (n1);

    \draw[myedge,->] (n1) .. controls +(-45:6mm) and +(-150:6mm) .. (n2);
    \draw[myedge,->] (n1) .. controls +(-45:8mm) and +(-150:10mm) .. (n3);
    \draw[myedge,->] (n1) .. controls +(-45:10mm) and +(-150:14mm) .. (n4);

    \begin{pgfonlayer}{my background}
      \node[scale=0.5,fit=(n1)(n6),background rectangle,double=white,fill=black!10!white](bgFrom){};
    \end{pgfonlayer}
  \end{scope}

  \node[rotate=-90] at (5,-1){$\trans$};

  \begin{scope}[shift={(0,-2)}]
    \foreach \n in {1,2,4,6}{\node[state,state-n](n\n) at (\n,0){ {\hspace{2ex}} };}
    \node[state,state-b](n5) at (5,0){$4$};
    \node[state,state-n,draw=red!50](n3) at (3,0){$1$};

    \draw[myedge,->] (n1) .. controls +(-45:6mm) and +(-150:6mm) .. (n2);
    \draw[myedge,->] (n1) .. controls +(-45:8mm) and +(-150:10mm) .. (n3);
    \draw[myedge,->] (n1) .. controls +(-45:10mm) and +(-150:14mm) .. (n4);
    
    \begin{pgfonlayer}{my background}
      \node[scale=0.5,fit=(n1)(n6),background rectangle,double=white,fill=black!10!white](bgTo){};
    \end{pgfonlayer}
  \end{scope}

  \begin{pgfonlayer}{my background}
    \draw[background rectangle,double=white] ([shift={(-0.5,0.5)}]bgFrom.north west) rectangle ([shift={(0.5,-0.6)}]bgTo.south east);
  \end{pgfonlayer}

  \node[looplabel] at ([shift={(0,-0.6)}]bgTo.south){Escape};
\end{tikzpicture}
}
\end{minipage}

%\item[{Terminal:}]
\noindent\begin{minipage}[!t]{0.7\linewidth}%\vspace{0pt}
  {\bf Terminal:} When the states of all processes from the range have
  been successfully checked, the for-loop ends and the $i^{th}$
  process moves to the terminal state $s'$. %
  That is, if there is no $j\in\range 1 {\sizeof \conf}$ with $j\circ
  i$ and $\tickof{i}{j}=\sfalse$, then $\conf' [i] = s'$ and for all
  $\ell\neq i$, $\conf' [\ell] = \conf [\ell]$. %
  Furthermore, the marks of the $i^{th}$ process are reset, i.e.,
  $\tick' = \tick\ominus i$.
\end{minipage}
\hfill%
\begin{minipage}[!t]{0.29\linewidth}%\vspace{0pt}
  %\vspace{6mm}
  \inputtikz{\begin{tikzpicture}[%
  scale=0.5%
  ,state/.append style={scale=0.9,semithick}%
  ,myedge/.append style={very thin,green!20!orange}%
  ]

  \begin{scope}
    \foreach \n in {1,2,3,4,6}{\node[state,state-n](n\n) at (\n,0){ {\hspace{2ex}} };}
    \node[state,state-n](n5) at (5,0){$3$};

    \draw[myedge,->,blue!70!red] (n5) .. controls +(45:6mm) and +(150:6mm) .. (n6);
    \draw[myedge,->,blue!70!red] (n5) .. controls +(135:6mm) and +(30:6mm) .. (n4);
    \draw[myedge,->,blue!70!red] (n5) .. controls +(135:8mm) and +(30:10mm) .. (n3);
    \draw[myedge,->,blue!70!red] (n5) .. controls +(135:10mm) and +(30:14mm) .. (n2);
    \draw[myedge,->,blue!70!red] (n5) .. controls +(135:12mm) and +(30:18mm) .. (n1);

    \draw[myedge,->] (n1) .. controls +(-45:6mm) and +(-150:6mm) .. (n2);
    \draw[myedge,->] (n1) .. controls +(-45:8mm) and +(-150:10mm) .. (n3);
    \draw[myedge,->] (n1) .. controls +(-45:10mm) and +(-150:14mm) .. (n4);

    \begin{pgfonlayer}{my background}
      \node[scale=0.5,fit=(n1)(n6),background rectangle,double=white,fill=black!10!white](bgFrom){};
    \end{pgfonlayer}
  \end{scope}

  \node[rotate=-90] at (5,-1){$\trans$};

  \begin{scope}[shift={(0,-2)}]
    \foreach \n in {1,2,3,4,6}{\node[state,state-n](n\n) at (\n,0){ {\hspace{2ex}} };}
    \node[state,state-n,draw=green!50!blue,fill=green!20](n5) at (5,0){$7$};

    \draw[myedge,->] (n1) .. controls +(-45:6mm) and +(-150:6mm) .. (n2);
    \draw[myedge,->] (n1) .. controls +(-45:8mm) and +(-150:10mm) .. (n3);
    \draw[myedge,->] (n1) .. controls +(-45:10mm) and +(-150:14mm) .. (n4);
    
    \begin{pgfonlayer}{my background}
      \node[scale=0.5,fit=(n1)(n6),background rectangle,double=white,fill=black!10!white](bgTo){};
    \end{pgfonlayer}
  \end{scope}

  \begin{pgfonlayer}{my background}
    \draw[background rectangle,double=white] ([shift={(-0.5,0.5)}]bgFrom.north west) rectangle ([shift={(0.5,-0.6)}]bgTo.south east);
  \end{pgfonlayer}

  \node[looplabel] at ([shift={(0,-0.6)}]bgTo.south){Terminal};
\end{tikzpicture}
}
\end{minipage}
%\end{description}

Other rules behave as before on the $\conf$ part of configurations and
they do not influence the function $\tick$.  That is, a~local,
broadcast, or rendez-vous rule%
\TODO{introduce them?}%
induces transitions $(\conf,\tick) \trans (\conf',\tick)$ where
$\conf\trans \conf'$ is a~transition induced by the rule as described
in Section~\ref{section:paramsys}.


%% -----------------------------------------------------
%\myparagraph{Abstraction.}
\myparagraph{Extended view.}
%% -----------------------------------------------------
We extend the notion of view with a partial function over the indices
of the basis elements. That is, a view of size $k$ is now a pair
$(\view,\tick)$ where $\view = R_0b_1R_1\ldots b_kR_k$ is defined in
Section~\ref{section:abstraction} and $\tick$ is a~three-valued
function
$\tick:\range{1}{k}\times\range{1}{k+1}\rightarrow\set{\strue,\sdunno,\sfalse}$.
($k$ processes can keep track of $k+1$ contexts!).
%
% Intuitively, a view $w =(\view,\tick)$ is seen as a~graph, where the
% vertices are the positions of $\view$ and the edges are defined by (i)
% the ordering of the positions and (ii) the relation $\tick$. The
% vertices are labeled by the state of processes at those
% positions. $\project k w$ then returns the set of subgraphs of $w$
% where every subgraph contains a~subset of at most $k$ vertices of
% $\view$ (positions of $\view$) and the maximal subset of edges of $w$
% adjacent with the chosen vertices.
%
%% -----------------------------------------------------
\myparagraph{Operational semantics on views.}
%% -----------------------------------------------------
Rules others than the for-loops behave as before on the $\view$ part
of views and they do not influence the $\tick$ part. %
That is, a~local, broadcast, or rendez-vous\TODO{idem} rule induces
transitions $(\view,\tick) \trans (\view',\tick)$ where $\view\trans
\view'$ is a~transition induced by the rule as described in
Section~\ref{section:paramsys}.

We describe the symbolic post operator for the (in-order) for-loops. 
%
Consider a for-loop rule $\forrule \circ \witnesses s {s'} e $ and a
view $(\view,\tick)$, of size $k$, where $\view = R_0b_1R_1\ldots
b_kR_k$.
%
If there exists $i\in\range 1 k$ such that $b_i = s$, we define
$\spostof {\view,\tick}$ to be the view $(\view',\tick')$ where
$\view' = R_0b_1R_1\ldots R_{i-1}b'_iR_i\ldots b_kR_k$. A for-loop is
size-preserving and doesn't not update the contexts, nor the elements
of the basis other than $b_i$. We now only need to define $b'_i$ and
$\tick'$.
%
The $i^{th}$ process selects the~\emph{first} unchecked position,
depending on the range. %
We define the set %
$Pos(i,\tick) = \setcomp{j\in D}{\tick(i,j)=\sfalse}$ where
$D=\range{1}{i},\range{i}{k+1},\range{1}{k+1}$ if $\circ$ is
$<,>,\neq$ respectively. Note that we extend the range in order to
capture the last context.
%
If this set is empty, that is, the whole view has been successfully
scanned, process $i$ then moves to the target state $s'$.
%
If the set is not empty, its minimum exists, say $\ell$, and process
$i$ checks if process $\ell$'s immediate left context $R_{\ell-1}$
does not violate the condition and it does also so with $b_\ell$ (only
if $b_\ell$ exists, that is, $\ell\neq i$ and $\ell\neq k+1$). If so,
process $i$ only marks position $\ell$. If this is not the case, it
moves to the escape state $e$.
%
This corresponds formally to the 3 following scenari:
%
(i) {\bf Iteration:} $Pos(i,\tick)\neq\emptyset$ and
$\ell=min(Pos(i,\tick))$. %
$R_{\ell-1}\subseteq\witnesses$%
\TODO{Change that to $R_{\ell-1}=\emptyset$}%
and %
if $\ell$ is neither $i$ nor $k+1$, then $b_\ell\in\witnesses$.
% 
If those conditions hold, we then have $b'_i=b_i$ and
$\tick'=\tick\oplus(i,\ell)$. %
Notice that all the elements of the context $R_{\ell-1}$ are checked
at once. We will show in the next section how/why this is an
over-approximation of the system.
%
(ii) {\bf Escape:} $Pos(i,\tick)\neq\emptyset$ and
$\ell=min(Pos(i,\tick))$. %
$R_{\ell-1}\intersection\witnesses\neq\emptyset$ and %
if $\ell$ is neither $i$ nor $k+1$, then $b_\ell\not\in\witnesses$.
% 
If those conditions hold, we then have $b'_i=e$ and
$\tick'=\tick\ominus i$.
% 
(iii) {\bf Terminal:} if $Pos(i,\tick)=\emptyset$ then $b'_i=s'$ and
$\tick'=\tick\ominus i$.
%
\begin{center}
  \inputtikz{\begin{tikzpicture}[scale=0.8]

  \begin{scope}
    \matrix (m) [na-matrix] {
      |[na-context]|
      & |[na-state]| & |[na-context]|
      & |[na-state]| & |[na-context]|
      & |[na-state]| & |[na-context]|
      & |[na-state]|3 & |[na-context]| \\
    };

    \makehighgroup{m-1-1}{m-1-2}{2}
    \makehighgroup{m-1-3}{m-1-4}{4}
    \makehighgroup[densely dotted]{m-1-5}{m-1-6}{6}
    \draw[na-edge,->] (m-1-8) .. controls +(115:10mm) and +(30:7mm) .. (n2);
    \draw[na-edge,->] (m-1-8) .. controls +(115:8mm) and +(30:5mm) .. (n4);
    \draw[na-edge,->,densely dotted] (m-1-8) .. controls +(115:6mm) and +(30:3mm) .. (n6);

    \matrix[na-matrix] (m2) at (0,-1.5) {
      |[na-context]|
      & |[na-state]| & |[na-context]|
      & |[na-state]| & |[na-context]|
      & |[na-state]| & |[na-context]|
      & |[na-state]|3 & |[na-context]| \\
    };

    \makehighgroup{m2-1-1}{m2-1-2}{2}
    \makehighgroup{m2-1-3}{m2-1-4}{4}
    \makehighgroup{m2-1-5}{m2-1-6}{6}
    \draw[na-edge,->] (m2-1-8) .. controls +(115:10mm) and +(30:7mm) .. (n2);
    \draw[na-edge,->] (m2-1-8) .. controls +(115:8mm) and +(30:5mm) .. (n4);
    \draw[na-edge,->] (m2-1-8) .. controls +(115:6mm) and +(30:3mm) .. (n6);

    \begin{pgfonlayer}{my background}
      \draw[background rectangle,double=white] ([shift={(-0.3,0.5)}]m.north west) rectangle ([shift={(0.3,-0.5)}]m2.south east);
    \end{pgfonlayer}

    \node[rotate=-90] at (barycentric cs:m-1-8=0.5,m2-1-8=0.5){$\trans$};
    \node[na-looplabel] at ([shift={(0,-0.5)}]m2.south) {Iteration};
  \end{scope}

  %% ================================================

  \begin{scope}[xshift=5cm]
    \matrix (m) [na-matrix] {
      |[na-context]|
      & |[na-state]| & |[na-context]|
      & |[na-state]| & |[na-context,draw=red!50]|5
      & |[na-state]| & |[na-context]|
      & |[na-state]|3 & |[na-context]| \\
    };

    \makehighgroup{m-1-1}{m-1-2}{2}
    \makehighgroup{m-1-3}{m-1-4}{4}
    \makehighgroup[draw=red,densely dotted]{m-1-5}{m-1-6}{6}
    \draw[na-edge,->] (m-1-8) .. controls +(115:10mm) and +(30:7mm) .. (n2);
    \draw[na-edge,->] (m-1-8) .. controls +(115:8mm) and +(30:5mm) .. (n4);
    \draw[na-edge,->,draw=red,densely dotted] (m-1-8) .. controls +(115:6mm) and +(30:3mm) .. (n6);

    \matrix[na-matrix] (m2) at (0,-1.5) {
      |[na-context]|
      & |[na-state]| & |[na-context]|
      & |[na-state]| & |[na-context,draw=red!50]|5
      & |[na-state]| & |[na-context]|
      & |[na-state,draw=red!50,fill=red!20]|4 & |[na-context]| \\
    };

    \begin{pgfonlayer}{my background}
      \draw[background rectangle,double=white] ([shift={(-0.3,0.5)}]m.north west) rectangle ([shift={(0.3,-0.5)}]m2.south east);
    \end{pgfonlayer}

    \node[rotate=-90] at (barycentric cs:m-1-8=0.5,m2-1-8=0.5){$\trans$};
    \node[na-looplabel] at ([shift={(0,-0.5)}]m2.south) {Escape};
  \end{scope}

  %% ================================================
  \begin{scope}[xshift=10cm]
    \matrix (m) [na-matrix] {
      |[na-context]|
      & |[na-state]| & |[na-context]|
      & |[na-state]| & |[na-context]|
      & |[na-state]| & |[na-context]|
      & |[na-state]|3 & |[na-context]| \\
    };

    \makehighgroup{m-1-1}{m-1-2}{2}
    \makehighgroup{m-1-3}{m-1-4}{4}
    \makehighgroup{m-1-5}{m-1-6}{6}

    \draw[blue!70!red] (m-1-7.north west) ++(0,0.5mm) -- +(0,1mm) -- ([yshift=1mm]m-1-7.north east) coordinate[pos=0.4](n7) -- +(0,-0.5mm);
    \draw[blue!70!red] (m-1-9.north east) ++(0,0.5mm) -- +(0,1mm) -- ([yshift=1mm]m-1-9.north west) coordinate[pos=0.4](n9) -- +(0,-0.5mm);

    \draw[na-edge,->] (m-1-8) .. controls +(115:10mm) and +(30:7mm) .. (n2);
    \draw[na-edge,->] (m-1-8) .. controls +(115:8mm) and +(30:5mm) .. (n4);
    \draw[na-edge,->] (m-1-8) .. controls +(115:6mm) and +(30:3mm) .. (n6);
    \draw[na-edge,->] (m-1-8) .. controls +(115:5mm) and +(30:2mm) .. (n7);
    \draw[na-edge,->] (m-1-8) .. controls +(65:5mm) and +(150:2mm) .. (n9);

    \matrix[na-matrix] (m2) at (0,-1.5) {
      |[na-context]|
      & |[na-state]| & |[na-context]|
      & |[na-state]| & |[na-context]|
      & |[na-state]| & |[na-context]|
      & |[na-state,draw=green!50!blue,fill=green!20]|7 & |[na-context]| \\
    };

    \begin{pgfonlayer}{my background}
      \draw[background rectangle,double=white] ([shift={(-0.3,0.5)}]m.north west) rectangle ([shift={(0.3,-0.5)}]m2.south east);
    \end{pgfonlayer}

    \node[rotate=-90] at (barycentric cs:m-1-8=0.5,m2-1-8=0.5){$\trans$};
    \node[na-looplabel] at ([shift={(0,-0.5)}]m2.south) {Terminal};
  \end{scope}

\end{tikzpicture}
}
\end{center}
%

%% -----------------------------------------------------
\myparagraph{Abstraction, Entailment and Concretization.}
%% -----------------------------------------------------
We define an extension of the abstraction function from
Section~\ref{section:abstraction}. %

\noindent%
\begin{wrapfigure}{r}{0.45\textwidth}
  \vspace{-24pt}
  \hfill
  \inputtikz{\tikzfading[name=f, top color=transparent!100, bottom color=transparent!0, middle color=transparent!80]
\begin{tikzpicture}
  \matrix(m1)[matrix of math nodes,nodes in empty cells,nodes={na-cell}]{%
    |[ful]|&|[fu]|&|[keep,fu]|&|[fu]|&|[fu]|&|[fu]|&|[keep,fu]|&|[fu]|&|[fur]|\\
    |[fl]|& &|[keep]|& & & &|[keep]|& &|[fr]|\\
    |[keep,fl]|\tick&|[keep]|\tick&|[keep]|\tick&|[check]|\tick&|[check]|\tick&|[check]|\tick&|[select]|\tick&|[keep]|\tick&|[keep,fr]|\tick\\
    |[fl]|& &|[keep]|& & & &|[keep]|& &|[fr]|\\
    |[fl]|& &|[keep]|& & & &|[keep]|& &|[fr]|\\
    |[keep,fl]|\tick&|[keep]|\tick&|[keep]|\tick&|[check]|\tick&|[check]|\tick&|[check]|\times&|[select]|\times&|[keep]|\times&|[keep,fr]|\times\\
    |[fl]|& &|[keep]|& & & &|[keep]|& &|[fr]|\\
    |[fdl]|&|[fd]|&|[keep,fd]|&|[fd]|&|[fd]|&|[fd]|&|[keep,fd]|&|[fd]|&|[fdr]|\\
  };

  \foreach \n in {1,2,...,8}{
    \draw[snake] (m1-2-\n.north east) -- +(up:3ex);
    \draw[snake] (m1-7-\n.south east) -- +(down:3ex);
  }
  \foreach \n in {1,2,...,7}{
    \draw[snake] (m1-\n-2.south west) -- +(left:3ex);
    \draw[snake] (m1-\n-8.south east) -- +(right:3ex);
  }

  \node[above=0pt of m1-1-3.north,anchor=center,circular glow={fill=white}]{ h(j-1) };
  \node[above=0pt of m1-1-7.north,anchor=center,circular glow={fill=white}]{ h(j) };
  \node[left=0pt of m1-3-1]{ h(i) };
  \node[left=0pt of m1-6-1]{ h(i+1) };
  
  \node[fit=(m1-3-4)(m1-3-5)(m1-3-6),draw=red,very thick](c3){};
  \draw[<->,na-edge,very thick,draw=red](c3.north) to[out=45,in=135] (m1-3-7.north);

  \node[fit=(m1-6-4)(m1-6-5)(m1-6-6),draw=red,very thick](c6){};
  \draw[<->,na-edge,very thick,draw=red](c6.north) to[out=45,in=135] (m1-6-7.north);

  % xshift = 5 cm
  \matrix(m2)[yshift=-40mm,matrix of math nodes,nodes in empty cells,nodes={na-cell}]{%
    |[ful]|&|[keep,fu]|&|[keep,fu]|&|[fur]|\\
    |[fl,keep]|\tick&|[keep]|\tick&|[select]|\tick&|[fr,keep]|\tick\\
    |[fl,keep]|\tick&|[keep]|\tick&|[select]|\sdunno&|[fr,keep]|\times\\
    |[fdl]|&|[keep,fd]|&|[keep,fd]|&|[fdr]|\\
  };

  \foreach \n in {1,2,3}{
    \draw[snake] (m2-2-\n.north east) -- +(up:2ex);
    \draw[snake] (m2-3-\n.south east) -- +(down:2ex);
  }
  \foreach \n in {1,2,3}{
    \draw[snake] (m2-\n-2.south west) -- +(left:2ex);
    \draw[snake] (m2-\n-3.south east) -- +(right:2ex);
  }
  \node[above=0pt of m2-1-2.north,anchor=center,circular glow={fill=white}]{ j-1 };
  \node[above=0pt of m2-1-3.north,anchor=center,circular glow={fill=white}]{ j };
  \node[left=0pt of m2-2-1]{ i };
  \node[left=0pt of m2-3-1]{ i+1 };

  %\draw[->,shorten >=7mm,shorten <=3mm] (m1) -- (m2) node[pos=0.4,above]{project};
  \draw[->,shorten >=3mm,shorten <=3mm] (m1) -- (m2) node[midway,right=2pt]{$\Pi_{h}$};

%   \matrix(m3)[matrix of math nodes,nodes={na-cell},right=10pt of m1.south east, anchor=south west]{%
%     |[check]|\tick&|[check]|\tick&|[check]|\times&|[select]|\times\\
%   };
%   \node[fit=(m3-1-1)(m3-1-2)(m3-1-3),draw=red,very thick](c){};
%   \draw[<->,na-edge,very thick,draw=red](c.north) to[out=45,in=135] (m3-1-4.north);
  
%   \draw[->,shorten <=3mm] (m3-1-4) -- ++(right:15mm) node[pos=0.6,above]{project} node[pos=1.1,right]{$\notdefined$};

\end{tikzpicture}
}
  \vspace{-6pt}
  \caption{$\tick$ markings projection}
  \label{figure:projection}
  \vspace{-12pt}
\end{wrapfigure}
%
Let us consider $k\in\nat$, a view $w = (\view,\tick)$ with $\sizeof
\view = n$ and a strictly increasing injection $h\in H_{n}^{k}$.
% $h\in H_{n+1}^{k+1}$ (to account for the last context too!). %
% %

We define a projection function that redefines the (boolean) function
for the positions of processes that are not in $h(\range{1}{k})$. %
Informally, when we project away a process, we inspect the ``status''
of the contexts at its immediate left and right position. If they are
both marked or both not marked, they have the same status and keep it
after merging. If they have a different status, we merge the contexts
and mark the resulting context with $\sdunno$.
%
% That is, if $\tickof{h(i)}{h(j)}=\strue$, we check if it is also the
% case that $\tickof{h(i)}{\ell}=\strue$ for all
% $\ell\in\rangeopen{h(j-1)}{h(j)}$. %
% Conversely, if $\tickof{h(i)}{h(j)}=\sfalse$, we check if it is also
% the case that $\tickof{h(i)}{\ell}=\sfalse$ for all
% $\ell\in\rangeopen{h(j-1)}{h(j)}$. %
% Otherwise, the projection is not defined.
%
% Formally, we define $\tick'$ for
% $i,j\in\range{1}{k}\times\range{1}{k+1}$ such that %
% $\tick'(i,j) = \tickof{h(i)}{h(j)}$ if
% $\forall\ell\in\spanof{h}{i},\tickof{h(i)}{\ell}=\tickof{h(i)}{h(j)}$
% and $\tick'(i,j) = \sdunno$ otherwise.
%
% Formally, we define $\tick'$ for
% $i\in\range{1}{k}$ such that %
% \TODO{Make a picture?}%
% $$
% \tick'(i,j) = \left\{
%   \begin{array}{l@{\hspace{2ex}}l}
%     \tickof{h(i)}{h(j)} & \text{if } \forall\ell\in\spanof{h}{i},\tickof{h(i)}{\ell}=\tickof{h(i)}{h(j)} \\
%     \sdunno & \text{otherwise} \\
%   \end{array}\right.
% $$

We use $\spanof h i$ to denote the set $\range{1}{h(1)}$ if $i=1$, the
set $\rangesopen{h(i-1)}{h(i)}$ if $1<i\leq k$, and the set
$\rangesopen{h(k)}{n+1}$ if $i=k+1$.
%
For $i\in\range 1 n$ and a subset $D\subseteq\range 1 {n+1}$, we use
$\status{i}{D}$ to denote the collective status of all markings in $D$,
that is, $\status{i}{D} = \tick(i,max(D))$ if $\forall\ell\in{D},
\tickof{i}{\ell}=\tickof{i}{max(D)}$ and $\status{i}{D} = \sdunno$ otherwise.
%
We can now define $\proj{h}{\tick}$ to be the function $\tick'$ such
that, for $i,j\in\range{1}{k}\times\range{1}{k+1}$, %
$\tick'(i,j) = \status{h(i)}{\spanof h j}$.
%
Intuitively, if the marking function is depicted as a matrix (cf
Fig.~\ref{figure:projection}), the projection function removes rows
and columns by collapsing columns if the cell values are consistent
with each other, in the collapsed cells or by introducing an imperfect
information with $\sdunno$.
%
Consequently, we do allow a view to encode partial marking information
about a context.
%
Finally, we define the extended abstraction function as %
$$
\project k {w} = \setcomp{%
  (\proj h {\view}, \proj h \tick)%
}{%
  h\in H_{n}^{k} %
}%
$$
%

%
% Formally, we define $\tick'$ such that for $i\in\range{1}{k}$%
% $$
% \begin{array}{rcl}
%   \tick'(i,1) & = & \left\{
%     \begin{array}{l@{\hspace{2ex}}l}
%       \tickof{h(i)}{h(1)} & \text{if }\forall\ell\leq h(1), \tickof{h(i)}{\ell}=\tickof{h(i)}{h(1)} \\
%       \sdunno & \text{otherwise} \\
%     \end{array}\right. \\[3ex]
%   % 
%   \forall j\in\range 2 k, \tick'(i,j) & = & \left\{
%     \begin{array}{l@{\hspace{2ex}}l}
%       \tickof{h(i)}{h(j)} & \text{if } \forall\ell\in\rangesopen{h(j-1)}{h(j)},\tickof{h(i)}{\ell}=\tickof{h(i)}{h(j)} \\
%       \sdunno & \text{otherwise} \\
%     \end{array}\right. \\[3ex]
%   %
%   \tick'(i,k+1) & = & \left\{
%     \begin{array}{l@{\hspace{2ex}}l}
%       \tickof{h(i)}{n+1} & \text{if }\forall\ell>h(k), \tickof{h(i)}{\ell}=\tickof{h(i)}{n+1} \\
%       \sdunno & \text{otherwise} \\
%     \end{array}\right. \\
% \end{array}%
%$$
%
% Finally, $\tproj{h}{\tick} = \tick'$ if and only if $\tick'$ is
% well-defined on all values of its domain.
%
% Consequently, we do not allow a view to encode partial marking
% information about a context.
%
% Consequently, we do allow a view to encode partial marking information
% about a context. %
% Finally, we denote $\tick'$ as $\proj{h}{\tick}$ and define the
% extended abstraction function as %
% $$
% \project k {w} = \setcomp{%
%   (\proj h {\view}, \proj h \tick)%
% }{%
%   h\in H_{n}^{k} %
% }%
% $$
% %
% Intuitively, if we the marking function is depicted as a matrix (cf Fig.~\ref{figure:projection}), the projection function
% removes rows and columns by collapsing columns if the cell values are
% consistent, in the collapsed cells or by introducing an imperfect information with $\sdunno$.


The entailement is defined naturally as follows: %
$(v,\tick)\entails(v',\tick')$ iff $v\entails v'$ and $\tick =
\tick'$.
%
The notion of concretization is defined in
the same manner as in Section~\ref{section:method} based on
based on $\entails$ and $\abstr$.

%\newpage
%\NOTE{Luk\'{a}\v{s}'s Part}
%%% -----------------------------------------------------
%\myparagraph{Configurations, Views, Abstraction, Concretization, and Post, Sketch.}
%%% -----------------------------------------------------
%A configuration is a sequence $c = x_1\cdots x_n$ where each $x_i,1\leq i \leq n$, is a pair $b_i^{\tick^i}$, called ticked state, where $b_i$ is a state and $\tick^i$ is a subset of $\range 1 n$. 
%A ticked state $x_i$ encodes that the state of the $i$th process of $c$ is $b_i$ and that the $i$th process has \emph{not} been ticked by states at positions in $\tick^i$.
%A sequence $\view = R_1x_1R_2\ldots x_nR_{n+1}$ where $R_1,\ldots,R_{n+1}$ are sets of states ticked by subsets of $\range 1 n$ is a view of size $n$.
%For a set of ticked states $R$, we define $R(i)$ as the set of states who are present in $R$ as not ticked by $i$, formally, $R(i) = \{q\mid \exists \tick:q^\tick\in R\land i\in\tick\}$.
%We define the operation $\Cup$ of merging two sets $R,R'$ of ticked states by uniting ticks of their elements, 
%formally, we let $R\Cup R' = \{q^{\tick\cup \tick'}\mid q^{\tick},q^{\tick'} \in R\cup R'\}$.
%
%Let us have a rule $\delta = \forrule \circ \witnesses {s} {t} {e}$, 
%and $i\in\range 1 {n}$. 
%Then the post image $\delta(c,i)$ of $c$ by a move of the $i$th process is defined if $b_i = s$ and then it is the configuration obtained from $c$ by one of the following three transformations:
%If there is no $\ell\in \range 1 n$ s.t. $\ell\circ i$ and $i\in\tick^\ell$ then the $i$th process changes its state to $t$ and $i$ is removed from $\tick^j$ for all $j\in\range 1 n$.
%In this case, we say that $\delta(c,i)$ was obtained from $c$ by terminal transition of the $i$th process. 
%Otherwise, let an $\ell$ of the above properties exist and let it be the smallest such index. Then there are two possibilities: 
%if $b_{\ell}\in S$ then $i$ is added to the set $\tick^{\ell}$ and we speak about iteration,
%else if $b_{\ell}\not\in S$ then the state of the $i$th process changes to $e$, $i$ is again removed from $\tick^j$ for all $j\in\range 1 n$, and we speak about escape transition.
%Let $\delta(c,i)$ be defined (that is, $b_i = s$) and let it equal $x'_1\ldots x'_n$.
%The post image of the view $\view$ is then defined as the view $\delta(\view,i) =  R_1' x'_1 R_2' \cdots x'_n R_{n+1}'$ or possibly undefined depending on what type of transition $\delta(c,i)$ represents.
%If $\delta(c,i)$ is terminal, then $\delta(v,i)$ is defined only if $R_{\ell+1}(i)=\emptyset$ (or only if $R_{\ell+1}=\emptyset$?),
%and if $\delta(c,i)$ is escape or iteration, then $\delta(v,i)$ is defined only if $R_{\ell}(i)=\emptyset$. 
%Last, for all $j\in\range 1 n$, $R_j' = R_j$ if $\delta(c,i)$ is iteration, otherwise (in the case of escape and terminal) $R_j'$ arises from $R_j$ by removing $i$ from ticks of all its elements.
%
%We define the entailment relation $\sentailed$ between two sets of ticked states such that $R \sentailed R'$ iff for each $q^{\tick}\in R$ there is $q^{\tick'}\in R'$ with $\tick\supseteq\tick'$,
%and we define entailment between two views of the same size such as $R_1 x_1 \ldots x_n R_{n+1} \sentailed R_1' x_1 \ldots x_n R'_{n+1}$ iff $R_i\sentailed R_i'$ for all $i\in\range 1 n$. For a set of views $X$, $\max X$ is the set of the minimal elements of $X$ w.r.t. $\sentails$, that is, elements $x$ of $X$ for which there is no other element $y$ of $X$ with $x\sentails y$. Note that since $\sentails$ is antisymmetric,
%every element $y$ of $X$ has a representant $x\in\max X$ such that $y\sentails x$.
%
%Let us consider a strictly increasing injection $h\in H_{n}^{k}$.
%We use $\spanof h i$ to denote the set $\range{1}{h(1)}$ if $i=1$, the
%set $\rangesopen{h(i-1)}{h(i)}$ if $1<i\leq k$, and the set
%$\rangesopen{h(k)}{n+1}$ if $i=k+1$.
%%
%Then 
%$\proj h \view = R_1' x_{h(1)} R_2' x_{h(2)} \cdots x_{h(k)} R_{k+1}'$ where 
%for each $1\leq i \leq k+1$, $R_i' = (\Cup_{j\in\spanof h i} R_j) \Cup (\Cup_{j\in\spanof h i\setminus \{i\}} \{x_j\})$. 
%\footnote{
%A smarter version of projection remembers only relevant ticks. Particularly, if the set of witnesses of a forall rule with the source $q$ is $S$, then only ticks of states from $Q\setminus S$ may contain positions on which state $q$ appears (other states just ignore ticks of processes that are in state $q$). Ticks of states which are not starting points of forall rules are also ignored.}
%Abstraction is then defined as the function $\alpha_k(c) = \max{\{\proj h c\mid h\in H_n^k\}}$. For a set $X$ of views or configurations, $\alpha_k(X) = \max \bigcup_{x\in X} \alpha_k{X}$.
%Finally, we define concretization with a parameter $k$ of a set of views $X$ as $\gamma_k(X) = \{\view \mid X \sentailed \alpha_k(\view)\}$.
%We define $\gamma_k^\ell(X)$. for $k < \ell$, as $\gamma_k(X) \cap \{v\mid |v|\leq \ell\}$,
%and we define abstract post transformer $\Delta^{\#}_k = \alpha^k \Delta \gamma_k^\ell$.
%
%
%
%\note{for two views with $\view \sentailed \view'$ and any set $X$ of views, $\gamma(X\cup \view) \subseteq \gamma(X\cup\view')$ (prove)?}
%
%\newpage
%
%We first prove some auxiliary lemma:
%\begin{lemma}
%\begin{enumerate}
%\item
%Let $v,v'$ be views of some size $k$, let $i\leq k$ and $\delta\in \Delta$.
%If $\delta(v',i)$ is defined then $\delta(v,i)$ is defined and $v\sentails v'$ then $\delta(v,i)\sentails \delta(v',i)$.
%\item
%Let $c$ be a configuration of size $n$.
%Let $h_1\in H_{n}^{k}, h_2\in H_{k}^{\ell}$ for some $\ell \leq k \leq n$ such that $h_2\circ h_1 = h$. 
%Then $\proj h c = \proj {h_2} {\proj {h_1} {c}}$.
%\item
%Let $\delta(c,i)$ be defined as well as $\delta(\proj h c,h^{-1}(i))$.
%Then $\proj h {\delta (c,i)} = \delta (\proj h b, h^{-1}(i))$.
%\end{enumerate}
%\end{lemma}
%
%\begin{lemma}
%$\abstractof k\circ\Delta\circ\concretizeof k X \sentailed \abstractof k\circ\Delta\circ\concretizeoflim k {k+2} X \cup X$.
%\end{lemma}
%
%\begin{proof}
%By definition of $\alpha_k$, every view in $\abstractof k \circ \Delta \circ \concretizeof k X$ equals $\proj h {c'}$ for some configuration $c'\in \Delta \circ \gamma_k(X)$ and some $h\in H_{|c|}^{k}$.
%We will prove the lemma by showing that there is some $v'\in \abstractof k\circ\Delta\circ\concretizeoflim k {k+1} X$ such that $v'\sentails \proj h {c'}$.
%Since $c'\in\Delta\circ\gamma_k(X)$, there is a configuration $c\in\gamma_k(X)$,
%a transition $\delta\in\Delta$ and $i\in\range 1 {|c|}$ such that $c' = \delta(c,i)$.
%Since $c\in\gamma_k(X)$, there is some $v\in X$ such that $v\sentails \proj c h$. 
%We will extend $h$ into a $g\in H_{|c|^{k+x}}$ by extending its range by the minimal set $X$ of positions of $c$ so that $\delta(\proj g, g^{-1}(i))$ is defined. We will show that the size of $X$ is at most 2 by an analysis of possible transitions that the $i$th process of $c$ could make:
%\begin{itemize}
%\item {\it Iteration:}
%For iteration to be enabled, it is only neccessary for the range of $g$ to contain $i$, hence the set of $X$ contains at most $i$.
%\item {\it Terminal transition:}
%Similarly as in the case of iteration, terminal transition is enabled if the range of $g$ contains $i$, hence $X$ contains at most $i$.
%\item {\it Escape transition:}
%For escape to be enabled, the range of $g$ must contain $i$ and the position $\ell$ of the first process of $c$ to be ticked by the $i$th process. Hence, the set $X$ can contain at most $i$ and $\ell$.
%\end{itemize}
%\end{proof}
%
%
%
%
%
%\begin{proof}
%By definition of $\alpha_k$, every view in $\abstractof k \circ \Delta \circ \concretizeof k X$ equals $\proj h {c'}$ for some configuration $c'\in \Delta \circ \gamma_k(X)$ and some $h\in H_{|c|}^{k}$.
%We will prove the lemma by showing that there is some $v'\in \abstractof k\circ\Delta\circ\concretizeoflim k {k+1} X$ such that $v'\sentails \proj h {c'}$.
%Since $c'\in\Delta\circ\gamma_k(X)$, there is a configuration $c\in\gamma_k(X)$,
%a transition $\delta\in\Delta$ and $i\in\range 1 {|c|}$ such that $c' = \delta(c,i)$.
%Since $c\in\gamma_k(X)$, there is some $v\in X$ such that $v\sentails \proj c h$.
%Let $\delta = \forrule \circ \witnesses s t e$.
%Then the $i$th  position of $c$ must be $s$, otherwise $\delta(c,i)$ could not be defined.
%If $\proj h c = \proj  h {c'}$ then we can let $v' = v'$ and conclude the proof, 
%so we further assume that $\proj h c = \proj  h {c'}$.
%We split the proof according to whether $c'$ is obtained from $c$ by an iteration, terminal, or escape.
%
%\begin{itemize}
%\item {\it Iteration:}
%Let $\ell$ be the first position in $c$ with $i\circ \ell$ which has not been yet checked by the $i$th process (such $\ell$ exists since $i$ since the move is an iteration).
%Then $c'$ is the same as $c$ up to that $i\in\tick^\ell$ in $c'$ but not in $c$.
%Since $\proj h c \neq \proj h {c'}$, the range of $h$ must contain both $i$ and $\ell$, from the following reasons:
%Position $i$ is contained since the only difference between $c$ and $c'$ and hence between $\proj h c$ and $\proj h {c'}$ involves ticks of the $i$th process which are invisible in projection which do not contain the $i$th process.
%If the position $\ell$ was not present, the state of the newly ticked $\ell$th process would appear only in some context of $\proj h {c'}$. Since the move is an iteration, the state of the $\ell$th process $x$ is in $S$, and hence contexts of $\proj h {c'}$ does not store any information about whether the $i$th process has or has not ticked a process in state $x$.  
%
%Notice that $\delta(\view,h(i))$ entails $\proj h c$ because 
%$\view$ entails $\proj h c$ and $\delta(\view,h(i))$ differs from $\view$ in exactly the same way as $\proj h {c'}$ differs from $\proj h c$, that is, by that $\tick^{h(\ell)}$ contains $h(i)$.  
%%
%Since $v\in X$, $\Delta^\#(X)$ contains a view which entails $\delta(\view,h(i))$. 
%We can hece let $v'$ be the view of $\Delta(X)$ which enaitls $\delta(\view,h(i))$ (such view is present by the definition of $\Delta^\#$). 
%
%\item {\it Terminal transition:}
%In the case of terminal transition, $c'$ differs from $c$ in that the $i$th process changes its state to $t$ and all its ticks are erased. 
%Let $h^{+1}\in H_n^k$ be the mapping with the range equal the range of $h$ united with $i$ ($h^{+1}$ can be possible equal to $h$). 
%Since $\gamma_k(X)$ contains $c$, then $\gamma_k^{k+1}(X)$ contains some view $w$ which entails $\proj {h^{+1}} c$. Let $w' = \delta(w,g^{-1}(i))$ where $g\in H_{k+1}^k$ is the mapping such that $g\circ h^{+1} = h$. We have that $w'\sentails \proj g c'$.
%
%
%
%,  contains a view $w$ which entails $\proj {h^{+1}} c$ differs from $\proj {h^{+1}} {c'}$ in that $h(i)$ changes its state to $t$ and ticks of $t$ are reset in case $i$ is in the range of $h$, otherwise both projections are the same.
%Since we assume that they are not the same, $i$ must be in the range of $h$. 
%As before, we know that by definition, $\Delta^\#$ contains a view which entials $\delta(v,h(i))$, and we let this view be $v'$. Notice that the difference between $v$ and $\delta(v,h(i))$ is the change of the state of $h(i)$th process to $t$ and reset of all ticks of $h(i)$. The terminal transition is indeed perfored since $v$ contains the moving process $h(i)$ $v$ is a view of $c$ which could make the terminal transition, there can be no unticked process by $i$ to in the base of $v$ or in some its context which would prevent the transition.
%\footnote{we could prove it for the different post where we require the right most process to be empty, using extension}
%
%\item {\it Escape:}
%Last, let $\delta(c,i)$ be excape. As before, we can argue that range of $h$ contains $i$. Let $\ell$ be the index of the process which is to be ticked by the $i$th process in $c$. Its state is not in $S$ (because the transition is escape).
%Let $h^\ell$ be the mapping from $H_n^k$ which is the same as $h$ except that its range includes also $\ell$. Let $h_1,\ldots,h_{k+1}$ be the views which arise from $h^+$ by excluding one position form its range. For every view $\proj {h_i} c$, there is (by definition of $\gamma_k$) a view $\view_i\in X$ which entails $\proj {h_i} c$. We have that $\proj {h^+} c \in \gamma_k^{k+1}\{v-1,\ldots,v_{k+1}\}$ (some lemma?). This or stronger (w.r.t. $\sentails$) view $w$ is constructed by $\gamma_k^{k+1}$.
%We can thus take $\proj {h'} {\delta(w,i')}$ as $v$ where domain of $h'$ is $\range 1 {k+1}\setminus \{{h^+}(\ell)\}$ and $i'$ is ${h^+}(i)$.
%Notice that $\delta(w,i')$ captures the change of the moving process since it is in its domain as well as the process outside $S$ which makes the post possible.
%
%\item
%Local moves are simulated by forall moves.
%\end{itemize}
%
%\end{proof}
%
%\NOTE{End of Luk\'{a}\v{s}'s Part}


%% -----------------------------------------------------
\myparagraph{Verification procedure.}
%% -----------------------------------------------------
Same as before, so it doesn't seem useful.

%
Lemma~\ref{lemma:apost} holds here in the same wording (as shown in
the appendix).
%
Thus the verification method for systems with for-loops is analogous
to the method of Section~\ref{section:method}.
