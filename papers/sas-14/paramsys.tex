%-----------------------------------
%\section{Parametrized Systems on Arrays}
\section{Coverability in Parameterized Systems on Arrays}
\label{section:paramsys}
%-----------------------------------
%
We introduce a standard notion of parameterized systems operating on a
linear topology, where processes may perform local transitions or
universally/existentially guarded transitions --- standard model used
e.g. in \cite{PRZ-tacas01,CTV06,rmc:wo:transducers,Namjoshi:VMCAI07}.

A parameterized system is a pair $\parsys=(\locs,\rules)$ where
$\locs$ is a finite set of \emph{local states} of a process and
$\rules$ is a set of \emph{transition rules} over $\locs$.  A
transition rule is either {\it local} or {\it global}.
%
A local rule is of the form $\loc\trans\loc'$, where the process
changes its local state from $\loc$ to $\loc'$ independently of the
local states of the other processes.
%
A global rule is of the form: 
$\quantrule{\loc}{\loc'}{\quantify}{\circ}{S}$, where
$\quantify\in\set{\exists,\forall}$, $\circ\in\set{<,>,\neq}$ and
$S\subseteq\locs$.
%
Here, the $i^{th}$ process checks also the local states of the other
processes when it makes the move\footnote{For the sake of
presentation, we only consider here a version where a process checks
atomically the other processes. The non-atomic case will be
introduced in Section~\ref{section:non-atomic}.}.
%
For instance, the condition $\forall\,j<i:\,\witnesses$ means that
“every process $j$, with a lower index than $i$, should be in a local
state that belongs to the set $\witnesses$”; the condition
$\forall\,j{\neq}i:\,\witnesses$ means that “the local state of all
processes, except the $i^{th}$ one, should be one from the set
$\witnesses$”; etc...
%

%A parameterized system $\parsys =(\locs,\rules)$ induces a
%\emph{transition system} (TS) $\transys = (\confs,\trans)$ where $\confs =
%\locs^*$ is the set of its \emph{configurations} and ${\trans}$,  \emph{transition relation}, is the union $\bigcup_{i\in\nat} {\trans_i}$ of \emph{indexed transition relations} ${\trans_i}\subseteq\confs\times\confs$. 
We use $\conf[i]$ to denote the state of the $i^{th}$ process within the
configuration~$\conf$.
%
For a configuration~$c$, $i\leq |c|$, and $\delta\in\Delta$, we define
the immediate successor $\delta(c,i)$ of $c$ under a $\delta$-move of
the $i^{th}$ process such that $\delta(c,i) = c'$ iff $\conf[i] =
\loc$, $\conf'[i] = \loc'$, $\conf[j]=\conf'[j]$ for all $j:j\neq i$
and either (i) $\delta$ is a local rule $\loc\trans\loc'$, or (ii)
$\delta$ is a global rule of the form
$\quantrule\loc{\loc'}{\quantify}\circ S$, where one of the following
conditions is satisfied:
%\begin{compactitem}
\begin{itemize}%[topsep=0pt]%package enumitem
\item 
$\quantify=\forall$ and for all $j:1\leq j\leq |\conf|$ such that $j\circ i$, it holds that $\conf[j]\in \witnesses$.
\item 
$\quantify=\exists$ and there exists $j:1\leq j\leq |\conf|$ such that $j\circ i$ and $\conf[j]\in \witnesses$.
%\end{compactitem}
\end{itemize}
%

%\myparagraph{Coverability problem}
%The verification problem of our concern is the problem of coverability of a given bad configuration. Formally, the verification problem is given by a system $\paramsys$ and a finite set of bad configurations. The answer to the problem is yes (safe) if no bad configuration is coverable, that is, there is no reachable configuration which contains a bad configuration as a subword.

\begin{figure}[!t]
  \centering
  \inputtikz{\lstinputlisting[style=custom]{experiments/szymanski-implementation.txt}}
  \vspace{-3mm}
  \caption{Szymanski's protocol implementation (for process $i$)}
  \label{figure:SzymanskiImplementation}
\end{figure}


%% -------------------------------------------------------
%% REACHABILITY PROBLEM
%% -------------------------------------------------------
An instance of the \emph{reachability problem} is defined by a
parameterized system $\parsys = (\locs,\rules)$, a regular set
$\inits\subseteq\locs^+$ of \emph{initial configurations}, and a set
$\bad\subseteq \locs^+$ of \emph{bad configurations}. 
%
Let $\subword$ be the usual \emph{subword relation}, i.e.,
$u\subword\loc_1\ldots\loc_n$ iff $u=\loc_{i_1}\ldots\loc_{i_k},1\leq
i_1,{\ldots},i_k\leq n$ and $i_j<i_{j+1}$ for all $j:1\leq j < k$.
%
We assume that $\bad$ is the upward closure $\{\conf\mid\exists b
\in\minbad:b\subword \conf\}$ of a given \emph{finite} set
$\minbad\subseteq\locs^+$ of \emph{minimal bad configurations}. 
%
This is a common way of specifying bad configurations which often
appears in practice, see e.g. the running example of Szymanski's
mutual exclusion protocol below.
%
We say that $\conf\in \confs$ is \emph{reachable} iff there are
$\conf_0,\ldots, \conf_l\in\confs$ such that $\conf_0\in \inits$,
$\conf_l = \conf$, and for all $0\leq
i < l$, there are $\delta\in\Delta$ and $j\leq |\conf_i|$ such that $\conf_{i+1} = \delta(\conf_{i},j)$. 
%
%We use $\reach$ to denote the set of all reachable configurations. 
%
We say that the system $\parsys$ is \emph{safe} w.r.t.\
$\inits$~and~$\bad$ if no bad configuration is reachable, i.e.\
$\reach\cap\bad = \emptyset$, where $\reach$ is the set of all
reachable configurations.
\todo{a note about that we can have more general bad specified as an intersection of upward closed sets?}


We define the \emph{post-image} of a set $X\subseteq\confs$
 as the set $\postof{X} = \{\delta(\conf,i)\mid \conf\in X,i\leq|\conf|,\delta\in\Delta\}$.
%
For $n\in\nat$ and a set of configurations $X \subseteq \confs$, we
use $X_n$ to denote its subset $\setcomp{c\in X}{|c|{\leq}n}$ of
configurations of size up to $n$.
%
% For $\circ\in\{<,\leq,\geq,>\}$, we write $S_{\circ n}$ to denote the
% union $\cup_{i\in\nat:i\circ n}S_i$.
%
%and 
%configurations of length $n$ and $\reachof n$ to denote the set of
%reachable configurations of length $n$.  For
%$\circ\in\{<,\leq,\geq,>\}$, we write $\reachof {\circ n}$ to denote
%the union $\cup_{i\in\nat:i\circ n}\reachof i$.
%

%% -----------------------------------------------------
\myparagraph{Example: Szymanski's mutual exclusion protocol.}
%% -----------------------------------------------------
We illustrate the notion of a parameterized systems with the example
of Szymanski's mutual exclusion protocol~\cite{Szymanski:protocol}.
%
The protocol ensures exclusive access to a shared resource in a system
consisting of an unbounded number of processes organized in an array.
%
\begin{wrapfigure}{r}{0.6\linewidth}
  \vspace{-24pt}
  \begin{center}
    \inputtikz{\input{Szymanski.tex}}
  \end{center}
  \vspace{-6mm}
  \caption{Szymanski's protocol transition system}
  \label{figure:SzymanskiTS}
  \vspace{-12pt}
\end{wrapfigure}
%
% The pseudocode of the process at the $i^{th}$ position of the array is
% given in Fig.~\ref{figure:SzymanskiCode} and the transition rules of
% the parameterized system in Fig.~\ref{figure:SzymanskiTS}.
The transition rules of the parameterized system are given in
Fig.~\ref{figure:SzymanskiTS}.
%
A state of the $i^{th}$ process consists of a program location and a
value of the local variable $\mathit{flag}[i]$. Since the value of
$\mathit{flag}[i]$ is invariant at each location, states correspond to
locations.

A configuration of the induced transition system is a word over the
alphabet $\{\w{0},\ldots,\w{11}\}$ of local process
states.
%
The task is to check that the protocol guarantees exclusive access to
the shared resource regardless of the number of processes. A
configuration is considered to be bad if it contains two occurrences
of state\,\w[b]{9} or \w[b]{10}, i.e., the set of minimal bad
configurations $\minbad$ is
$\set{\w[b]{9,9},\w[b]{9,10},\w[b]{10,9},\w[b]{10,10}}$. %
Initially, all processes are in state\,\w[i]{0}, i.e.\
$\inits~=~{\w[i]{0}}^+$.

% \begin{center}
% %\begin{minipage}{0.58\linewidth}
%   %\makebox[\textwidth][c]{\framebox[1.8\width][c]{For Process $i$}}%
%   %\makebox[\textwidth][c]{For Process $i$}%
%   % \lstinputlisting[caption={Szymanski's mutual exclusion protocol.}, style=custom, label=figure:SzymanskiCode]{szy-code.txt}
%   \inputtikz{\lstinputlisting[style=custom]{experiments/szymanski-implementation.txt}}
%   \vspace{-3mm}
%   %\noindent\hrulefill%
%   \captionof{figure}{Szymanski's protocol implementation.}
%   \label{figure:SzymanskiImplementation}
% %\end{minipage}%
% \end{center}
