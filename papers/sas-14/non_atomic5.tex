
%-----------------------------------
\section{Non-atomically checked global conditions}
\label{section:non-atomic}
%-----------------------------------
We extend our model and method to handle parameterized systems where
global conditions are \emph{not} checked atomically.
%
We replace both existentially and universally guarded transition rules
by the following variant of a~for-loop rule:
$$%
\forrule \circ \witnesses s {s'} e%
$$
where $s\in\locs$ is a \emph{source} state, $s'\in\locs$ is a
\emph{target} state and $e\in Q$ is an \emph{escape}
state. $\circ\in\{<,>,\neq\}$ describes the \emph{range} of iterations
and $\witnesses\subseteq Q$ is a \emph{condition}.
%
For instance, line 3 of Szymanski's protocol would be replaced by
\mbox{$\forrule \neq {\neg\set{1,2}} 3 7 4$}. 

For a configuration $c$, a process at position $i$ inspects the state
of another process at position $j$, \emph{in-order}.
%
If the latter is not a condition for the $i^{th}$ process to escape,
we say that $i$ \emph{ticks} $j$, unless there is no more process to
inspect and the $i^{th}$ completes its transition.
%
Without loss of generality, we will assume that the for-loops iterate
through process indices in increasing order.

%% -------------------------------------------------
\myparagraph{Concrete semantics.}
%% -------------------------------------------------
%
We extend the semantics of a~system with for-loop rules from
transition systems of Section~\ref{section:paramsys} in the following
way: %
A configuration is now a pair $c = (q_1\cdots q_n,\tick)$ where
$q_1\cdots q_n\in Q^+$ as before and $\tick:\range 1 n \rightarrow
\range 0 n$ is a total map which assigns to every position $i$ of $c$
the last position which has been ticked by the $i^{th}$ process. %
Initially, before ticking any process, it is assigned $0$.


Let us have a rule $\delta = \forrule \circ \witnesses {s} {t} {e}$ from $\rules$, a configuratoin $c$ with $|c| = n$,
and $i\in\range 1 {n}$. 
We will distinguish three types of the post-image of $c$ by a move of the $i$th process, $\delta_i(c,i)$ for iteration, $\delta_e(c,i)$ for escape, and $\delta_t(c,i)$ for terminal. All of them can be defined only if $b_i = s$. 
Let us define $\next(i)$ as the position which should the $i$th process of $c$ inspect next, that is, as the smallest position in $\range 1 n$ larger than $\tick(i)$ which satisfies $\next(i)\circ i$. Notice that if the $i$th process has already ticked the right-most position $j$ with $j\circ i$, then (and only then) $\next(i)$ is undefined.
\begin{itemize}
\item
$\delta_t(c,i)$ is defined if $\next(i)$ is undefined, and it is obtained from $c$ by changing the state of the $i$th process to $t$ and setting $\tick(i)$ to $0$. 
\item
$\delta_i(c,i)$ is defined if $\next(i)$ is defined and $q_{\next(i)}\in S$. It is obtained form $c$ by
setting $\tick(i)$ to $\next(i)$.
\item
$\delta_e(c,i)$ is defined if $\next(i)$ is defined and $q_{\next(i)}\not\in S$. It is obtained  from $c$ by changing the state of the $i$th process to $e$ and  setting $\tick(i)$ to $0$.
\end{itemize}

%
\myparagraph{Instantiation of the view abstract domain.}
We now instantiate the abstract domain described in Section~\ref{section:method} by specifying new notions projection and entailment on the views. They will be analogies of the original ones, customized to the new notion of configuration and successor function.
A view is now a 
pair $\view = ((q_1 \cdots q_n,\tick),(R_0,\cdots,R_n,\unticked))$, often written as $(R_0q_1R_2\ldots q_nR_{n},\tick,\unticked)$, where
$(q_1\cdots q_n,\tick)$ is a configuration called the \emph{base} of $v$, 
and $(R_0,\cdots,R_n,\unticked)$ is a \emph{context}, where
$R_0,\ldots,R_{n}\subseteq Q$ and
$\unticked:\range 1 n\rightarrow 2^\locs$ is a total map which assigns 
a subset of $R_{\tick(i)}$ 
%a subset of $\importantof i \cap R_{\tick(i)}$ 
to every $i\in\range 1 n$. 
Intuitivelly, $\loc \in R_{\tick(i)}$ symbolises existence of a process in a state $\loc$ which 
%is important for the $i$th process, 
appears between the ${\tick(i)}$th and the ${\tick(i+1)}$th position and has not yet been ticked by the $i$th process. 

%
Projection is defined analogically as in Section~\ref{section:method},
plus that the right-most ticked position and the set of not-yet ticked process states must be properly updated.
For $h\in H_k, k\leq n$,
$\proj h c = (\proj h {q_1\cdots q_n},\tick',\unticked')$ where
for $i\in \range 1 k$, 
\begin{inparaenum}[(i)]
\item
$\tick'(i)$ equals the largest $j$ for which $h(j)\leq \tick(i)$, and
\item
given the smallest $\ell\in\range 1 k$ such that $\tick(h(i))\leq h(\ell)$,
%$\unticked(i) = \importantof i \cap \{b_j\mid \tick'(h(i)) < j < h(\ell)\}$. 
$\unticked(i) = \{b_j\mid \tick'(h(i)) < j < h(\ell)\}$. 
\end{inparaenum}
%

\todo{dubious paragraph:}
Observe that the for-all transition combines certain form of both universal and existential guards. The $i$th process can escape or perform an iteration only if there \emph{exists} an unicked process at the $j$th position, and, at the same time, there \emph{do not exist} other processes at positions between the postion last ticked by the $i$th process and $j$. A terminal transition can be triggered only if there \emph{do not exist} any other unticked processes. The regurements on non-existence of a process may be seen as variants of universal guards.
We use precisely the same definition of symbolic post as in Section~\ref{section:method}, where the role of contexts is only to block the universal guards of transitions by asserting presence some processes. 
%
%For a rule $\delta\in\Delta$, a view $v$ and $i\leq v$, $\delta^S(v,i)$ is defined differently for different types of transitions.

The entialment relation between views 
$v = (R_0q_1R_2\ldots q_nR_{n},\tick,\unticked)$,
$v' = (R_0'q_1'R_2'\ldots q_n'R_{n}',\tick',\unticked')$,
is defined so that $u\entails v$ iff  
\begin{enumerate}
%\item
%$R_0q_1R_2\ldots q_nR_{n}\entails R_0'q_1'R_2'\ldots q_n'R_{n}'$, 
\item
both have the same base,
\item
$R_0\subseteq R_0',\ldots,R_n\subseteq R_n'$, and
\item
$\unticked(1)\subseteq \unticked'(1),\ldots,\unticked(n)\subseteq \unticked'(n)$.
%for all $i\in\range 1 n$, ${\unticked(i)}\subseteq{\unticked'(i)}$.
\end{enumerate}
This intuitivelly reflects that
more unticked states within a context can only block more effectivelly the progress of ticking, and larger contexts are more likely to contain non-ticked states in the future, both pose more restrictions on configurations.

%Let us consider a strictly increasing injection $h\in H_{n}^{k}$.
%We use $\spanof h i$ to denote the set $\range{1}{h(1)}$ if $i=0$, the
%set $\rangesopen{h(i)}{h(i+1)}$ if $0<i< k$, and the set
%$\rangesopen{h(k)}{n}$ if $i=k$. 

%
Abstraction, concretization, and $\spost$ are then defined based on the definition of projection and entailment and $\post$ in the same way as before.

%\note{for two views with $\view \sentailed \view'$ and any set $X$ of views, $\gamma(X\cup \view) \subseteq \gamma(X\cup\view')$ (prove)?}

%\newpage
%\newcommand{\imap}{h}
%\newcommand{\imapf}{f}
%\newcommand{\imapg}{g}
%\newcommand{\imapplus}{h_+}
%\newcommand{\imapminus}{h_-}
%\newcommand{\imapprime}{h'}
%\newcommand{\imapsof}[2]{H_{#1}^{#2}}
%\renewcommand{\proj}[2]{\Pi_{#1}(#2)}
%\renewcommand{\apost}{\Delta^{\#}_k}
%\renewcommand{\apost}{\abstractof k\circ\Delta\circ{\max}\circ\gamma_k^{k+2}}
%\renewcommand{\proj}[2]{#2 {\downharpoonright}_{{#1}}}
---------------------------------------------\\
In order to prove that $\apost$ overapproximates the abstract post (Lemma~2), we first prove an auxiliary Lemma~\ref{lemma:aux}.
\begin{lemma}\label{lemma:aux}
Let $\delta\in\Delta$, 
$\conf\in\confs,|c|=n$, let $k \leq m \leq n$,
and $\imap\in \imapsof n k$.
\begin{enumerate}
\item\label{item:composition}
For $\imapf\in \imapsof {n} {m}, \imapg\in \imapsof {m} {k}$, $\Pi_{\imapg\circ\imapf} = \Pi_{\imapg} \circ \Pi_\imapf$.
\item\label{item:samepost}
If $\delta(c,k)$ and $\delta(\proj \imap c ,\imap^{-1}(k))$ are defined then
$\proj \imap {\delta (c,k)} = \delta (\proj \imap {c}, \imap^{-1}(k))$.
\item\label{item:monotonicity}
For two view of size $n$ s.t. $v\sentails v'$, we have that 
\begin{inparaenum}
\item\label{item:post}
$\delta(v,k)\sentails \delta(v',k)$ and 
\item\label{item:projection}
$\proj h v \sentails \proj h {v'}$.
\end{inparaenum}
\end{enumerate}
\end{lemma}
\begin{proof}TODO
%\begin{enumerate}
%\item
%%We will show claim for $\imapf$ and $\imapg$ such that the range of $\imapg$ equals the range of $\imapf$ plus one element, say $\ell$. The claim of the lemma (with general $\imapg$ and $\imapf$) then follows by simple induction to the size of the difference of ranges of $\imapf$ and $\imapg$.
%
%Let us have a view $v = R_1x_1\cdots x_n R_{n+1}$, then, by definition,  
%$\proj \imapf v = R_1' x_{\imapf(1)} R_2' x_{\imapf(2)} \cdots x_{\imapf(m)} R_{m+1}'$ where 
%for each $1\leq i \leq m+1$,\\ $R_i' = (\Cup_{j\in\spanof \imapf i} R_j) \Cup (\Cup_{j\in\spanof \imapf i\setminus \{i\}} \{x_j\})$, 
%and\\
%$\proj \imapg {v'} = 
%R_1'' x_{\imapg(\imapf(1))} R_2'' x_{\imapg(\imapf(2))} \cdots x_{\imapg(\imapf(m))} R_{k+1}''$ 
%where 
%for each $1\leq i \leq k+1$,\\ $R_i'' = (\Cup_{j\in\spanof \imapg i} R'_j) \Cup (\Cup_{j\in\spanof \imapg i\setminus \{i\}} \{x_j\})$ which after substitution equals\\
%$
%\bigl(\Cup_{j\in\spanof \imapg i} 
%[(\Cup_{\ell\in\spanof \imapf j} R_\ell) 
%\Cup 
%(\Cup_{\ell\in\spanof \imapf j\setminus \{j\}} \{x_\ell\})]\bigr) 
%\Cup 
%\bigl(\Cup_{j\in\spanof {\imapg\circ\imapf} i\setminus \{i\}} \{x_j\}\bigr)
%$.
%The last expression may be show to be uqual to\\
%$(\Cup_{j\in\spanof {\imapg\circ\imapf} i} R_j) \Cup (\Cup_{j\in\spanof {\imapg\circ\imapf} i\setminus \{i\}} \{x_j\})$,\\
%%
%which proves that
%$\proj \imapg {v'} = \proj {\imapg\circ\imapf} v$.
%
%\item
%$\delta(v,i)$ is defined onnly if the iterating process as well as the first process in its checking range, if it exists, are in the range of $\imap$.
%$\delta(v,i)$ then differs from $v$ only in one of those two elements of the base.
%
%
%%Let $v' = R_1x_1\cdots x_k R_{k+1} = \proj \imap  v$ for some view $v$ of size $n$ where $\imap\in \imapsof n k$. Assume that $k<n$ and that $i\leq n$ is not in the range of $\imap$.
%%%
%%The view $\proj \imapf {v'} = R_1 x_1 \cdots x_{i-1} R_{i} \Cup \{i\} \Cup R_{i+1} x_{i+1} R_{i+1} \cdots x_k R_{k+1}$. This is the same as the view $\proj {\imapf\circ\imapg} v$.
%%%
%%$\proj h \view = R_1' x_{h(1)} R_2' x_{h(2)} \cdots x_{h(k)} R_{k+1}'$ where 
%%for each $1\leq i \leq k+1$, $R_i' = (\Cup_{j\in\spanof h i} R_j) \Cup (\Cup_{j\in\spanof h i\setminus \{i\}} \{x_j\})$. 
%
%
%\end{enumerate}

\end{proof}

\begin{lemma}
$\abstractof k\circ\Delta\circ\concretizeof k X \sentailed \apost(X) \cup X$.
\end{lemma}

\begin{proof}
By definition of $\alpha_k$, every view in $\abstractof k \circ \Delta \circ \concretizeof k X$ equals $\proj \imap {c'}$ for some configuration $c'\in \Delta \circ \gamma_k(X)$ with $|c'|=n$ and some $\imap\in \imapsof n {k}$.
We will prove the lemma by showing that there is some $v'\in \apost (X)$ 
such that $v'\sentails \proj \imap {c'}$. The case when $\proj \imap {c'}\sentailed X$ is trivia, we therefore further assume that $\proj \imap {c'}\not\sentailed X$.
Since $c'\in\Delta\circ\gamma_k(X)$, there is a configuration $c\in\gamma_k(X)$,
a transition $\delta\in\Delta$ and $i\in\range 1 {n}$ such that $c' = \delta(c,i)$.
Since $c\in\gamma_k(X)$, there is some $v\in X$ such that $v\sentails \proj \imap c$. 
We will extend $\imap$ into a $\imapplus\in \imapsof {n} {k+|X|}$ by extending its range by the minimal set $X$ of positions of $c$ so that $\delta(\proj \imapplus c, \imapplus^{-1}(i))$ is defined. 

Notice that $|X|$ is at most $2$. Recall the definition of post on views. For local and terminal transition, to enable post, it is only neccessary to include in $X$ the position $i$ of the process performing the transition. For iteration and escape transition, post is enabled if the $i$th positiont together with the position $\ell$ of the first procecess to be checked by the $i$th one is present (the $\ell$th process is either ticked or it causes the $i$th process to escape). Hence $X$ indeed has to contain at most 2 positions.

We show that $\proj \imapplus c \in \gamma_{k}^{k+|X|}$.
By defitition of $\gamma_k^{k+|X|}$, this is true if every element of $\alpha_k(\proj \imapplus c)$ is entailed by an element of $X$.
An element of $\alpha_k(\proj \imapplus c)$ equals $\proj \imapprime {\proj \imapplus c}$ for some $\imapprime \in \imapsof {k+|X|} k$. 
%The mapping  $\imapprime\circ\imapplus$ is an element of $\imapsof n k$. 
By Lemma~\ref{lemma:aux}.\ref{item:composition}, $\proj \imapprime {\proj \imapplus c} = \proj {\imapprime\circ\imapplus} c$. 
Since 
$\imapprime\circ\imapplus\in\imapsof n k$ and
$c\in \gamma_k(X)$,
$\proj {\imapprime\circ\imapplus} c$ must be entailed by an element of $X$ by definition of $\gamma_k$, which proves our point.

Since $\proj \imapplus c \in \gamma_{k}^{k+|X|}$, there is $w\in\max(\gamma_{k}^{k+|X|})$ such that $w\sentails\proj \imapplus c$.
We will show that $\apost(X)$ contains the wanted $v'$ which entails $\proj \imapminus {\delta(w,\imapplus^{-1}(i))}$. %Since $w\in \gamma_{k}^{k+|X|}$, $v\in \Delta$
By Lemma~\ref{lemma:aux}.\ref{item:samepost}, we have that $\delta(\proj \imapplus c,\imapplus^{-1}(i)) = \proj \imapplus {\delta(c,i)}$. 
Since $\imap= \imapminus\circ\imapplus$ and by Lemma~\ref{lemma:aux}.\ref{item:composition}, 
we have that $\proj \imapminus {\proj \imapplus {\delta(c,i)}} = \proj {\imapminus\circ\imapplus} {\delta(c,i)} = \proj \imap {\delta(c,i)} =  \proj \imap {c'}$. 
By Lemma~\ref{lemma:aux}.\ref{item:post} and due to $w\sentails \proj \imapplus c$, we have $\delta(w,\imapplus^{-1}(i))\sentails\delta(\proj \imapplus c,\imapplus^{-1}(i))$, and from here, by Lemma~\ref{lemma:aux}.\ref{item:projection},  
we have
$\proj {\imapminus} {\delta(w,\imapplus^{-1}(i))}\sentails\proj \imapminus {\delta(\proj \imapplus c,\imapplus^{-1}(i))} = \proj \imap {c'}$.
Since $\imapminus\in\imapsof {k+|X|} k$ and $w\in \max(\gamma_k^{k+|X|}(X))$,
$\delta(\proj \imapplus c,\imapplus^{-1}(i))\in\Delta\circ\max\circ\gamma_k^{k+|X|}(X)$,$\delta(\proj \imapplus c,\imapplus^{-1}(i))\in\Delta\circ\max\circ\gamma_k^{k+|X|}(X)$,  and hence $\apost(X)$ must contain some view $v'$ which entails $\imapminus {\delta(\proj \imapplus c,\imapplus^{-1}(i))}$ by the definition of $\alpha_k$.
\qed
\end{proof}





%\begin{proof}
%By definition of $\alpha_k$, every view in $\abstractof k \circ \Delta \circ \concretizeof k X$ equals $\proj h {c'}$ for some configuration $c'\in \Delta \circ \gamma_k(X)$ and some $h\in H_{|c|}^{k}$.
%We will prove the lemma by showing that there is some $v'\in \abstractof k\circ\Delta\circ\concretizeoflim k {k+1} X$ such that $v'\sentails \proj h {c'}$.
%Since $c'\in\Delta\circ\gamma_k(X)$, there is a configuration $c\in\gamma_k(X)$,
%a transition $\delta\in\Delta$ and $i\in\range 1 {|c|}$ such that $c' = \delta(c,i)$.
%Since $c\in\gamma_k(X)$, there is some $v\in X$ such that $v\sentails \proj c h$.
%Let $\delta = \forrule \circ \witnesses s t e$.
%Then the $i$th  position of $c$ must be $s$, otherwise $\delta(c,i)$ could not be defined.
%If $\proj h c = \proj  h {c'}$ then we can let $v' = v'$ and conclude the proof, 
%so we further assume that $\proj h c = \proj  h {c'}$.
%We split the proof according to whether $c'$ is obtained from $c$ by an iteration, terminal, or escape.
%
%\begin{itemize}
%\item {\it Iteration:}
%Let $\ell$ be the first position in $c$ with $i\circ \ell$ which has not been yet checked by the $i$th process (such $\ell$ exists since $i$ since the move is an iteration).
%Then $c'$ is the same as $c$ up to that $i\in\tick^\ell$ in $c'$ but not in $c$.
%Since $\proj h c \neq \proj h {c'}$, the range of $h$ must contain both $i$ and $\ell$, from the following reasons:
%Position $i$ is contained since the only difference between $c$ and $c'$ and hence between $\proj h c$ and $\proj h {c'}$ involves ticks of the $i$th process which are invisible in projection which do not contain the $i$th process.
%If the position $\ell$ was not present, the state of the newly ticked $\ell$th process would appear only in some context of $\proj h {c'}$. Since the move is an iteration, the state of the $\ell$th process $x$ is in $S$, and hence contexts of $\proj h {c'}$ does not store any information about whether the $i$th process has or has not ticked a process in state $x$.  
%
%Notice that $\delta(\view,h(i))$ entails $\proj h c$ because 
%$\view$ entails $\proj h c$ and $\delta(\view,h(i))$ differs from $\view$ in exactly the same way as $\proj h {c'}$ differs from $\proj h c$, that is, by that $\tick^{h(\ell)}$ contains $h(i)$.  
%%
%Since $v\in X$, $\Delta^\#(X)$ contains a view which entails $\delta(\view,h(i))$. 
%We can hece let $v'$ be the view of $\Delta(X)$ which enaitls $\delta(\view,h(i))$ (such view is present by the definition of $\Delta^\#$). 
%
%\item {\it Terminal transition:}
%In the case of terminal transition, $c'$ differs from $c$ in that the $i$th process changes its state to $t$ and all its ticks are erased. 
%Let $h^{+1}\in H_n^k$ be the mapping with the range equal the range of $h$ united with $i$ ($h^{+1}$ can be possible equal to $h$). 
%Since $\gamma_k(X)$ contains $c$, then $\gamma_k^{k+1}(X)$ contains some view $w$ which entails $\proj {h^{+1}} c$. Let $w' = \delta(w,g^{-1}(i))$ where $g\in H_{k+1}^k$ is the mapping such that $g\circ h^{+1} = h$. We have that $w'\sentails \proj g c'$.
%
%
%
%,  contains a view $w$ which entails $\proj {h^{+1}} c$ differs from $\proj {h^{+1}} {c'}$ in that $h(i)$ changes its state to $t$ and ticks of $t$ are reset in case $i$ is in the range of $h$, otherwise both projections are the same.
%Since we assume that they are not the same, $i$ must be in the range of $h$. 
%As before, we know that by definition, $\Delta^\#$ contains a view which entials $\delta(v,h(i))$, and we let this view be $v'$. Notice that the difference between $v$ and $\delta(v,h(i))$ is the change of the state of $h(i)$th process to $t$ and reset of all ticks of $h(i)$. The terminal transition is indeed perfored since $v$ contains the moving process $h(i)$ $v$ is a view of $c$ which could make the terminal transition, there can be no unticked process by $i$ to in the base of $v$ or in some its context which would prevent the transition.
%\footnote{we could prove it for the different post where we require the right most process to be empty, using extension}
%
%\item {\it Escape:}
%Last, let $\delta(c,i)$ be excape. As before, we can argue that range of $h$ contains $i$. Let $\ell$ be the index of the process which is to be ticked by the $i$th process in $c$. Its state is not in $S$ (because the transition is escape).
%Let $h^\ell$ be the mapping from $H_n^k$ which is the same as $h$ except that its range includes also $\ell$. Let $h_1,\ldots,h_{k+1}$ be the views which arise from $h^+$ by excluding one position form its range. For every view $\proj {h_i} c$, there is (by definition of $\gamma_k$) a view $\view_i\in X$ which entails $\proj {h_i} c$. We have that $\proj {h^+} c \in \gamma_k^{k+1}\{v-1,\ldots,v_{k+1}\}$ (some lemma?). This or stronger (w.r.t. $\sentails$) view $w$ is constructed by $\gamma_k^{k+1}$.
%We can thus take $\proj {h'} {\delta(w,i')}$ as $v$ where domain of $h'$ is $\range 1 {k+1}\setminus \{{h^+}(\ell)\}$ and $i'$ is ${h^+}(i)$.
%Notice that $\delta(w,i')$ captures the change of the moving process since it is in its domain as well as the process outside $S$ which makes the post possible.
%
%\item
%Local moves are simulated by forall moves.
%\end{itemize}
%
%\end{proof}

%\NOTE{End of Luk\'{a}\v{s}'s Part}
