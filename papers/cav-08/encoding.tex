\section{Proofs of Lemmas - Section \ref{termination:section}}
\label{encoding:section}
We show termination of the reachability algorithm
in several steps.
%
First, we recall basic concepts from the theory of well quasi-orderings.
%
Then, we identify particular properties of our graphs which are implied by
the fact that each node has at most one successor.
%
More precisely, we will define a certain measure on graphs,
which we call the {\it degree} of the graph, and show that
the graphs which are generated during the reachability algorithm 
have degrees which are bounded by some natural number.
%
This will lead to a new entailment ordering on graphs, defined in terms
of particular encodings of the graph structures.
%
We show that the new ordering is a WQO, and that it is stronger than the ordering $\lt$
used in the reachability algorithm; leading to the termination proof.
%
Let $\pintgrs$ denote the set of positive integers.

\paragraph{\bf Well Quasi-Orderings}
For a set $A$ and a preorder on $A$, we say that $\preceq$ is 
a {\it well quasi-ordering (WQO)} on $A$ if the following property is
satisfied:
for any infinite sequence $a_0,a_1,a_2,\ldots$ of elements in $A$, there are 
$i,j$  such that $i<j$ and $a_i\preceq a_j$.
%
A simple example of a WQO is the standard ordering on natural numbers.
%
We extend the ordering $\preceq$ to an ordering $\preceq^*$ on
the set $A^*$ of finite words over $A$ as follows:
$a_1 a_2\cdots a_m\preceq^* b_1 b_2\cdots b_n$ if there is an injection $h$
from the set
from $\set{1,2,\ldots,m}$ to the set $\set{1,2,\ldots,n}$ such that
$i<j$ implies $h(i)<h(j)$ and $a_i\preceq b_{h(i)}$ for all $i,j:1\leq i,j\leq m$.
%
In our proofs, we will use the following property of WQOs (see e.g.\cite{ACJT00}).
%
\begin{lemma}
\label{WQO:properties:lemma}
If $\preceq$ is a WQO on $A$ then $\preceq^*$ is a WQO on $A^*$.
\end{lemma}
In other words, WQOs extend to finite words under the mentioned ordering.
%
As special cases, WQOs also extend to multisets, vectors of length $k$ for a given
$k\in\pintgrs$, etc.
%
For instance, the lemma implies that vectors of multisets of natural numbers are WQO.
%


\paragraph{\bf Blocks}
In this paragraph, we identify some particular patterns in the class of
graphs we consider in this paper.
%
Consider a graph $\graph=\tuple{\vertices,\succ,\labeling}$.
%
A vertex $\vertex$ is said to be a {\it leaf} if $\succ^{-1}(\vertex)=\emptyset$;
and a {\it root} if $\succ(\vertex)=\undef$.
%
A graph is said to be a {\it tree} if it contains exactly one root, and it
is called a {\it star} if it contains no root.

%
A graph $\graph'=\tuple{\vertices',\succ',\labeling'}$ is said to be a {\it block}
in $\graph$ if $\vertices'$ is a maximal subset of $\vertices$
containing a vertex $\vertex$  such that, for all $\vertex'\in\vertices'$, it is the
case that $\tuple{\vertex',\vertex}\in\succ^*$.
%
In other words, there is path along the edges of $\graph$ from $\vertex'$ to $\vertex$.
%
Furthermore,
$\succ'$ and $\labeling'$ are the restrictions of $\succ$ and $\labeling$ to
$\vertices'$, respectively.
%
In other words, $\succ'(\vertex)=\vertex'$ iff $\vertex'\in\vertices'$
and $\succ(\vertex)=\vertex'$; and
$\labeling(\xvar)=\vertex$ iff
$\vertex\in\vertices'$ and
$\labeling(\xvar)=\vertex$.
%
Each graph can obviously be uniquely partitioned into a finite set
of blocks.
%
Furthermore, in our graphs, since each vertex has at most one successor, it follows that 
each block in $\graph$ is either a tree or a star.
%

%
A vertex is said to be {\it unguarded} if it is a leaf and there is no
variable $\xvar\in\xvars$ with $\labeling(\xvar)=\vertex$.
%
For a graph $\graph$,
we define the {\it degree} of $\graph$ as
$\degof{\vertex}:=d_2-d_1$ where $d_2$ is the number of unguarded vertices and
$d_1$ is the number of roots in $\graph$.
%

A graph is said to be {\it compact} if it does not contain simple vertices.
%
Intuitively, a graph is {\it compact} if it cannot be reduced
due to contraction.


%
The proof of the following 
lemma exploits the fact that each block is either a tree or a star.
%
\begin{lemma}
\label{finite:deg:lemma}
For any $k\in\pintgrs$, there are only finitely many compact graphs $\graph$ 
such that $\degof{\graph}\leq k$ and such that $\graph$ does not contain
any blocks with degree $0$.
\end{lemma}
%
\begin{proof}
Suppose that there is a $k\in\pintgrs$ such that $\degof{\graph}\leq k$ for 
each $\graph\in\graphs$.
%
First we show the claim for graphs which consist of a single block.
%
Consider such a graph $\graph$.
%
Since there is at most one root in $\graph$,
the number of leafs is bounded by $\degof{\graph}+\sizeof{\xvars}$.
%
Furthermore, there is at most $\sizeof{\xvars}$ internal nodes (non-leafs) which have only 
one child (the nodes labeled by some variable).
%
The claim follows immediately.

Now, suppose that $\graph$ contains several blocks.
%
Since each block has a positive degree, it follows that the number of blocks 
is bounded by $k$, and
each of them has a degree which is less than $k$.
%
This means that the size of the set is bounded by at most $k$ multiplied by the number
of blocks of degree at most $k$ (which is bounded as explained above).
\end{proof}

In particular, Lemma~\ref{finite:deg:lemma} implies that if we take any 
(infinite) set of compact graphs
and delete from each graph the blocks which are of degree $0$, then we will be left
with only finitely many graphs.
%
This will be reflected in our termination proof, where we will distinguish
between blocks of degree $0$ and blocks with positive degrees.
%
We define $\zeroreduce{\graph}$ to be the graph
we get from $\graph$ by keeping only the blocks
with degree $0$.
%
Analogously, we define $\posreduce{\graph}$ to be the graph
we get from $\graph$ by keeping only the blocks
with positive degrees.


\paragraph{\bf Encodings and Signatures}
%

%
An {\it encoding} is a tuple $\encoding=\tuple{\vertices,\succ,\labeling,\cnt}$ where
$\graph=\tuple{\vertices,\succ,\labeling}$ is a compact graph, and
$\cnt:\vertices\times\vertices\rightarrow\pintgrs$ is a partial mapping such that
$\cnt(\vertex_1,\vertex_2)\neq\undef$ iff $\vertex_2=\succ(\vertex_1)$.
%
In other words, $\cnt$ associates a positive integer to each edge in $\graph$.
%
We call $\graph$ the {\it signature} of $\encoding$ and denote it by $\sigof{\encoding}$.

Fix a graph $\graph=\tuple{\vertices,\succ,\labeling}$.
%
For non-simple vertices $\vertex,\vertex'\in\vertices$, we say that
$\vertex'$ is the {\it target} of $\vertex$ if there are
vertices $\vertex_0,\vertex_1,\ldots,\vertex_{n}$ with $n\geq 1$ such that
$\vertex_0=\vertex$, $\vertex_{n}=\vertex'$, $\vertex_i$ is simple for all $i:1\leq i< n$, and
$\vertex_i=\succ(\vertex_{i-1})$ for all $i:1\leq i\leq n$.
%
In other words, there is a path of simple nodes leading from $\vertex$ to $\vertex'$.
%
In such a case we define the {\it distance} between $\vertex$ and $\vertex'$ by
$\dist(\vertex,\vertex'):=n$.
%
Notice that the target $\vertex'$ of vertex $\vertex$ is unique if it exists.
%
We define $\encodingof{\graph}$ to be the encoding
$\tuple{\vertices',\succ',\labeling',\cnt}$, where
$\vertices'$ is the set of non-simple nodes in $\vertices$,
$\succ(\vertex)=\vertex'$ if $\vertex'$ is the target of $\vertex$ in $\graph$,
$\labeling'=\labeling$, and
$\cnt(\vertex,\vertex')=\dist(\vertex,\vertex')$.
%
We define $\sigof{\graph}:=\sigof{\encodingof{\graph}}$.
%
We will extend graph concepts to encodings by 
interpreting them on their signature.
%
For instance, when we say that encoding $\encoding$ has degree $k$ then
we mean that $\graph$ (i.e., $\sigof{\encoding}$) has degree $k$;
and write $\zeroreduce{\encoding}$ to denote $\encodingof{\zeroreduce{\graph}}$, etc.


%
We will define a WQO $\encodingorder$ on encodings.
%
The definition of $\encodingorder$ will be derived from two relations, namely
$\incencodingorder$ and $\isencodingorder$.
%
The relation  $\incencodingorder$ is
based on graph inclusion and will be applied on blocks with degree $0$;
while the relation  $\isencodingorder$ is
based on graph isomorphism and will be applied to blocks with positive degrees.

Consider compact graphs $\graph=\tuple{\vertices,\succ,\labeling}$ and 
$\graph'=\tuple{\vertices',\succ',\labeling'}$.
%
For an injection $h:\vertices\rightarrow\vertices'$, 
we say that $\graph$ is {\it included} in $\graph'$
with respect to $h$, denoted $\graph\included_h\graph'$
if, for all vertices $\vertex,\vertex'\in\vertices$ and variables $\xvar\in\xvars$,
it is the case that
(i) if $\vertex'=\succ(\vertex)$ then $h(\vertex')=\succ'(h(\vertex))$; and
(ii) if $\labeling(\xvar)=\vertex$ then $\labeling'(\xvar)=h(\vertex)$.
%
%
For encodings $\encoding,\encoding'$, we write $\encoding\incencodingorder_h\encoding'$ if 
$\sigof{\encoding}\included_h\sigof{\encoding'}$ and for all vertices $\vertex,\vertex'$,
if $\cnt(\vertex,\vertex')\neq\undef$ then  $\cnt(\vertex,\vertex')\leq\cnt(h(\vertex),h(\vertex'))$.
%
%We write $\encoding\incencodingorder\encoding'$ to denote that
%$\encoding\incencodingorder_h\encoding'$ for some $h$.
%


For a bijection
$h:\vertices\rightarrow\vertices'$, we say that $\graph_1$ and $\graph_2$ are {\it isomorphic}
with respect to $h$, denoted $\graph\sim_h\graph'$
if, for all vertices $\vertex,\vertex'\in\vertices$ and variables $\xvar\in\xvars$,
it is the case that
(i) $\vertex'=\succ(\vertex)$ iff $h(\vertex')=\succ'(h(\vertex))$; and
(ii) $\labeling(\xvar)=\vertex$ iff $\labeling'(\xvar)=h(\vertex)$.
%
We say that $\graph$ and $\graph'$ are isomorphic, denoted $\graph\isomorphic\graph'$
if $\graph\isomorphic_h\graph'$ for some $h$; and
%
For encodings $\encoding,\encoding'$, we write $\encoding\isencodingorder_h\encoding'$ if 
$\sigof{\encoding}\isomorphic_h\sigof{\encoding'}$ and for all vertices $\vertex,\vertex'$,
if $\cnt(\vertex,\vertex')\neq\undef$ then  $\cnt(\vertex,\vertex')\leq\cnt(h(\vertex),h(\vertex'))$.
%
%We write $\encoding\isencodingorder\encoding'$ to denote that
%$\encoding\isencodingorder_h\encoding'$ for some $h$.



We use $\encoding\encodingorder_h\encoding'$ to denote that both
$\zeroreduce{\encoding}\incencodingorder_{h_1}\zeroreduce{\encoding'}$,
and
$\posreduce{\encoding}\isencodingorder_{h_2}\posreduce{\encoding'}$
where $h_1$ and $h_2$ are the restrictions of $h$ to the sets of vertices in
$\zeroreduce{\encoding}$ and $\posreduce{\encoding}$ respectively.
%
We write $\encoding\encodingorder\encoding'$ to denote that
$\encoding\encodingorder_h\encoding'$ for some $h$.
%
For graphs $\graph,\graph'$, we write $\graph\encodingorder\graph'$
to denote that $\encodingof{\graph}\encodingorder\encodingof{\graph'}$.


\begin{lemma}
\label{encoding:WQO:lemma}
The ordering $\encodingorder$ is a WQO on any set $\graphs$ of graphs with degrees bounded
by some $k\in\pintgrs$.
\end{lemma}
%
\begin{proof}
First, we show that $\incencodingorder$ is a WQO on encodings of degree $0$.
%
A block in a compact graph $\graph$ (underlying such an encoding)
with degree $0$ is of one of four forms (in the first three cases, the block
does not contain any vertex labeled by a variable):
(i) a single isolated vertex; or (ii) a single vertex with a self-loop (an edge
starting from and ending at the vertex); or
(iii) two vertices with an edge between them; or
(iv) the block contains at least one vertex labeled by a variable.
%
Since we have only finitely many variables, the set of possible blocks which can occur
in $\graph$ is finite
(although there is no bound on the number of blocks which may occur in $\graph$).
%
This means that each encoding with degree $0$ can be considered as a vector of multisets
of natural numbers: 
each entry of the vector corresponds to the natural numbers which appear
on the edges of blocks of a certain type.
%
For instance, if we consider blocks which are self-loops and without variables,
then the corresponding entry will record the natural numbers which appear on the loop.
%
Furthermore, we need multisets since there is no bound on the number
of blocks of that particular type which may occur in $\graph$, and therefore there is no bound
either on the number of times the same natural number may occur in the different blocks
of that type.
%
The well quasi-ordering of $\incencodingorder$ on encodings of degree $0$, then follows
from Lemma~\ref{WQO:properties:lemma}.


Next, we show that $\isencodingorder$ is a WQO on encodings of positive degrees
which are bounded by some $k$.
%
Consider such a set of encodings, and the corresponding set of underlying compact graphs.
%
From Lemma~\ref{finite:deg:lemma}, we know that there are only finitely many such compact
graphs.
%
Therefore, the set of encodings can be viewed as vectors of natural numbers:
each entry of a vector is indexed by one edge in one of the (finitely many) compact graphs.
%
The value of an entry is given by the natural number on the corresponding edge.
%
The WQO of $\isencodingorder$ then follows from Lemma~\ref{WQO:properties:lemma}.

Finally, we show that $\encodingorder$ is a WQO on
the set of encodings with degrees bounded by $k$.
%
Each underlying compact graph can be considered as consisting of two parts; namely the blocks
with degree $0$, and those with positive degrees (but with degrees which
are bounded by $k$).
%
The encoding of such a graph can be described a pair, where the first element
is a vector of multisets of natural numbers, and the second part is a multiset of natural numbers
(as described above).
%
Since both parts are WQOs, it follows by  Lemma~\ref{WQO:properties:lemma}
that the set of pairs of such elements is also a WQO.
\end{proof}

The next lemma shows the relation between the orderings $\encodingorder$  and
$\lt$.
%
For graphs $\graph,\graph'$,
suppose that $\graph\encodingorder\graph'$.
%
Then, we can derive $\graph$ from $\graph'$ by
applying a finite sequence of variable, vertex, and edge deletions
on blocks with degree $0$, and
applying a finite sequence of contraction
operations on blocks with positive degrees.
%
%More precisely, for each pair of vertices $\vertex,\vertex'$ with
%$m=\cnt(\vertex,\vertex')\leq\cnt'(\vertex,\vertex')=n$, we contract
% $n-m$ edges between $\vertex$ and $\vertex'$.
%
This gives the following lemma.
%
\begin{lemma}
\label{encoding:implication:lemma}
For graphs $\graph_1,\graph_2$, if 
$\graph_1\encodingorder\graph_1$ then
$\graph_1\lt\graph_2$.
\end{lemma}

The following lemma implies that the degree of the configurations generated during the reachability
algorithm is bounded by the degree of the configurations in the set $\finalconfs$.
%
The proof of the lemma can be carried out by the fact that no unguarded vertices
are introduced in the computation of $\pmovesto{}$.
%
\begin{lemma}
\label{backward:bounded:degree:lemma}
For graphs $\graph_1,\graph_2$, if $\graph_1\pmovesto{}\graph_2$
then $\degof{\graph_2}\leq\degof{\graph_1}$.
\end{lemma}
\begin{proof}
Suppose that $\graph_1\pmovesto{\action}\graph_2$.
%
We inspect the cases in the definition of $\pmovesto{}$ one by one, and 
verify that no unguarded vertices are introduced in the derivation 
of $\graph_2$ from $\graph_1$.
%
We observe that the degree can be increased by removing a variable from a leaf 
(which is not pointed by any other variable),
or by adding an edge leading to a node which is not unguarded.
%
\begin{enumerate}
\item
$\action$ is of the form $\xvar\equals\yvar$.
%
In all cases, no variables or edges are deleted.
\item
$\action$ is of the form $\xvar\assigned\yvar$.
%
We inspect the different subcases:
\begin{enumerate}
\item $\xvar$ is removed; however $\yvar$ still points to the vertex.
\item $\xvar$ is removed; however $\yvar$ is made to point to the vertex.
\item No variables or edges are removed.
\item No variables or edges are removed.
\end{enumerate}
\item $\action$ is of the form  $\xvar\nequals\yvar$.
In all the case, no variables or edges are deleted.
%
\item $\action$ is of the form $\xvar\assigned\nxt\yvar$.
%
We inspect the different subcases:
\begin{enumerate}
\item $\xvar$ is removed; however the vertex is still the successor of the vertex pointed to by $\yvar$.
\item $\xvar$ is removed; however the vertex is now pointed to by the successor of vertex pointed to by $\yvar$.
\item $\xvar$ is removed; however the vertex is now pointed to by the successor of vertex pointed to by $\yvar$.
\item No variables or edges are removed.
\item No variables or edges are removed.
\item No variables or edges are removed.
\end{enumerate}
\item $\action$ is of the form $\nxt\xvar\assigned\yvar$
%
We inspect the different subcases:
\begin{enumerate}
\item The edge from the vertex pointed by \xvar to the vertex pointed by \yvar is removed; 
  however both vertices are still pointed to by \xvar and \yvar.
\item The edge from the vertex pointed by \xvar to its successor is removed; 
however, the successor is now pointed to by \yvar.
\item No variables or edges are removed.
\item No variables or edges are removed.
\item No variables or edges are removed.
\item No variables or edges are removed.
\end{enumerate}
\end{enumerate}
\end{proof}


Now, we are ready to prove Theorem~\ref{termination:theorem}.
%
Suppose that that algorithm does not terminate.
%
Then, during the course of the algorithm, we add to the set
$\toexplore$ an infinite sequence $\conf_0,\conf_1,\conf_2,\ldots$ of configurations,
where $\conf_i$ is of the form $\tuple{\cstate,\graph_i}$, and
and where $\conf_i\not\lt\conf_j$ for all $i,j$ with $i<j$.
%
From Lemma~\ref{backward:bounded:degree:lemma}, we know that that 
the set $\set{\graph_0,\graph_1,\graph_2,\ldots}$ has bounded degree, and therefore
by Lemma~\ref{encoding:WQO:lemma}, it follows that there
are $i,j$ such that $i<j$ $\encoding(\graph_i)\isencodingorder\encoding(\graph_j)$.
%
From Lemma~\ref{encoding:implication:lemma} it follows that
$\graph_i\lt\graph_j$, and hence $\conf_i\lt\conf_j$ which is a contradiction.
%
This gives Theorem~\ref{termination:theorem}.
%
