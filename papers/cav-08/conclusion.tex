\section{Conclusions}
\label{conclusion:section}
 
We have presented a new approach for automatic verification of
programs with dynamic heaps.  The proposed approach is based on a
simple algorithmic principle, and is fully automatic.  The main idea
is to perform an abstract (over-approximate) reachability analysis
using upward-closed sets w.r.t. a suitable preorder on heap
graphs. This preorder is shown to be a well-quasi ordering, which
guarantees the termination of the
analysis. %Intuitively, the considered preorder corresponds to a notion of graph embedding modulo contraction of paths of simple vertices (i.e., vertices having both out- and in-degree equal to 1).

The results of this paper concern the case of heap structures with
1-next selector. Our approach can however be generalized to heap
structures with multiple next selectors. Several extensions of our
framework can be done by refining the considered preorder (and the
abstraction it induces). For instance, it could be possible (1) to
take into account data values attached to objects in the heap, (2) to
consider constraints on (and relating) the lengths of (contracted)
paths, and (3) to consider in integer program variables whose values
are related to the lengths of paths in the heap. Such extensions with
arithmetical reasoning can be done in our framework by considering
preorders involving for instance
%(approximations of the constraints by)
 gap-order constraints.
