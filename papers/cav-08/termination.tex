\newcommand{\block}{b}
\section{Termination}
\label{termination:section}

In this section, we give an overview of the termination proof for
the reachability algorithm.
%
The full details can be found in~\cite{it:2008-015}.
%
Let $\pintgrs$ denote the set of positive integers.
%
For a set $A$ and a preorder on $A$, we say that $\preceq$ is 
a {\it well quasi-ordering (WQO)} on $A$ if the following property is
satisfied:
for any infinite sequence $a_0,a_1,a_2,\ldots$ of elements in $A$, there are 
$i,j$  such that $i<j$ and $a_i\preceq a_j$.
%
A simple example of a WQO is the standard ordering on natural numbers.
%
We extend the ordering $\preceq$ to an ordering $\preceq^*$ on
the set $A^*$ of finite words over $A$ as follows:
$w_1 \preceq^* w_2$ if there is an order-preserving injection
from $w_1$ to $w_2$ such that each element in $w_1$ is mapped to an element
in $w_2$ which is larger wrt $\preceq$.
%
It is well-known that $\preceq^*$ is a WQO (see e.g.\cite{ACJT00}).
%
Since multisets and vectors are special cases of words it follows,
for instance, that vectors of multisets of vectors of natural numbers are WQO
(this particular property will be used later in the proof).

%

Consider a graph $\graph=\tuple{\vertices,\succ,\labeling}$.
%
A graph $\block$ is said to be a {\it block}
of $\graph$ if $\block$ is a maximal part of $\graph$ which is connected.
%
A vertex is said to be {\it unguarded} if it is a leaf and there is no
variable $\xvar\in\xvars$ with $\labeling(\xvar)=\vertex$.
%
For a graph $\graph$,
we define the {\it degree} of $\graph$, denoted $\degof{\graph}$, to be
the number of unguarded vertices in $\graph$.
%
A graph is said to be {\it compact} if it does not contain simple vertices.
%
Intuitively, a graph is {\it compact} if it cannot be reduced
due to contraction.
%
An {\it encoding} is a tuple $\encoding=\tuple{\vertices,\succ,\labeling,\cnt}$ where
$\graph=\tuple{\vertices,\succ,\labeling}$ is a compact graph, and
$\cnt:\vertices\times\vertices\rightarrow\pintgrs$ is a partial mapping such that
$\cnt(\vertex_1,\vertex_2)\neq\undef$ iff $\vertex_2=\succ(\vertex_1)$.
%
In other words, $\cnt$ associates a positive integer to each edge in $\graph$.
%
%We call $\graph$ the {\it signature} of $\encoding$ and denote it by $\sigof{\encoding}$.
%

Fix a graph $\graph=\tuple{\vertices,\succ,\labeling}$.
%
We define $\encodingof{\graph}$ to be the encoding we get from $\graph$ 
by applying contraction as much as possible (until the resulting
graph cannot be reduced any more using contraction).
%
Furthermore, for vertices $\vertex$ and $\vertex'$, the value
of $\cnt(\vertex,\vertex')$ gives the number of edges between $\vertex$ and $\vertex'$
in $\graph$.
%
We will define an ordering $\encodingorder$ on graphs 
(the formal definition of the relation is in~\cite{it:2008-015}). %the appendix).
%
Consider graphs $\graph,\graph'$ with encodings 
$\encodingof{\graph}=\tuple{\vertices,\succ,\labeling,\cnt}$
and $\encodingof{\graph'}=\tuple{\vertices',\succ',\labeling',\cnt'}$.
%
Roughly speaking,  $\graph\encodingorder\graph'$ means that for each
block $\block$ in $\encodingof{\graph}$ there is an corresponding
isomorphic block $\block'$ in $\encodingof{\graph'}$ such that
such that $\cnt(\vertex_1,\vertex_2)\leq\cnt'(\vertex'_1,\vertex'_2)$ %$\cnt(\vertex_1,\vertex_2)\leq\cnt'(h(\vertex'_1),h(\vertex'_2))$
for all vertices $\vertex_1,\vertex_2$ in $\block$ and their images $\vertex'_1,\vertex'_2$
in $\block'$.
%
The ordering  $\encodingorder$ is a WQO on the set of compact graphs whose degrees are bounded by some
$k\in\pintgrs$.
%
The reason is that in any such a graph, there number of leaves is bounded by
$\sizeof{\vars}+k$, and hence there are only finitely many types of blocks which may
occur in any such graph.
%
This means that an encoding of such a graph can be represented by vectors of multisets of vectors of natural numbers
as follows.
%
Suppose that there are $\ell$ types of blocks.
%
Then, an element of the representation will be of the form $\tuple{m_1,\ldots,m_\ell}$,
where each $m_i$ is a multiset of vectors of natural numbers corresponding to one type of block:
an entry of the vector corresponds to an edge in the block; the natural numbers correspond to
the ones which appear on the edges.  
%
We need multisets since there is no bound on the number
of blocks (of a certain type) which may appear in the graph.
%
This means that $\encodingorder$ is a WQO.
%
Also, $\graph\encodingorder\graph'$ implies $\graph\lt\graph'$,
since  if $\graph\encodingorder\graph'$ then we can derive $\graph$ from $\graph'$
through the application of a finite sequence of variable deletion, vertex deletion, 
edge deletion, and contraction operations.
%
Finally, we observe that in the definition $\pmovesto{}$ no unguarded vertices are introduced.
%
This means that all the configurations which are generated in the reachability algorithm are bounded
by some $k$, and are therefore WQO.
%
This gives the following theorem.
%
\begin{theorem}
\label{termination:theorem}
The reachability algorithm is guaranteed to terminate.
\end{theorem}






%
