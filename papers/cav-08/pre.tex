%
\section{Computing Predecessors}
\label{section:pre}
%\newpage
%
The main idea behind our algorithm to solve the coverability problem,
is to perform backward reachability analysis.
%
The basic step of the algorithm uses a relation $\pmovesto{}$ 
defined on the set of configurations.
%
Intuitively, $\conf\pmovesto{}\conf'$ means that, from $\conf'$,
we can perform a transition and reach a configuration in the upward closure of
$\conf$.
%
First, we give the formal definition of $\pmovesto{}$,
and then describe some of its properties, and in particular how it relates
to the transition relation $\movesto{}$.


For a graph $\graph=\tuple{\vertices,\suc,\labeling}$, a graph
$\graph'$, and an action $\action$, we write
$\graph\pmovesto{\action}\graph'$ to denote that one of the following
conditions is satisfied:
%
%
\begin{enumerate}
\item
\label{pre:case:equals}%
  %% Test of x == y
  %% ----------------------------------
  $\action$ is of the form $\xvar\equals\yvar$ and one of the
  following conditions is satisfied:
  % 
  \begin{enumerate}
  \item
    \label{pre:case:equalsA}%
    $\labeling(\xvar)\nequals\undef$, %
    $\labeling(\yvar)\nequals\undef$, %
    $\labeling(\xvar)=\labeling(\yvar)$ and $\graph'=\graph$.
  \item
    \label{pre:case:equalsB}%
    $\labeling(\xvar)\neq\undef$, %
    $\labeling(\yvar)=\undef$ and
    $\graph'=\graph\addEqVar{\xvar}{\yvar}$.
  \item
    \label{pre:case:equalsC}%
    $\labeling(\xvar)=\undef$, %
    $\labeling(\yvar)\neq\undef$ and
    $\graph'=\graph\addEqVar{\yvar}{\xvar}$.
  \item
    \label{pre:case:equalsD}%
    $\labeling(\xvar)=\undef$, %
    $\labeling(\yvar)=\undef$ and
    $\graph'=\graph_1\addEqVar{\xvar}\yvar$ for some
    $\graph_1\in\left(\graph\addVar{\xvar}\right)$.
  \end{enumerate}
In order to be able to perform the action, the variables
$\xvar$ and $\yvar$ should point to the same vertex.
%
If one (or both) of them are missing, then we add them to the graph
(with the restriction that they point to the same vertex).
\item
  %% Test of x =/= y
  %% ----------------------------------
  $\action$ is of the form $\xvar\nequals\yvar$ and one of the
  following conditions is satisfied:
  \begin{enumerate}
  \item \label{pre:case:notEqualsA}%
    $\labeling(\xvar)\nequals\undef$, %
    $\labeling(\yvar)\nequals\undef$, %
    $\labeling(\xvar)\neq\labeling(\yvar)$ and $\graph'=\graph$.
    % 
  \item \label{pre:case:notEqualsB}%
    $\labeling(\xvar)\neq\undef$, %
    $\labeling(\yvar)=\undef$ and %
    $\graph'\in\graph\addNotEqVar{\xvar}{\yvar}$.
    % 
  \item \label{pre:case:notEqualsC}%
    $\labeling(\xvar)=\undef$, %
    $\labeling(\yvar)\neq\undef$ and %
    $\graph'\in\graph\addNotEqVar{\yvar}{\xvar}$.
    % 
  \item \label{pre:case:notEqualsD}%
    $\labeling(\xvar)=\undef$, %
    $\labeling(\yvar)=\undef$ and %
    $\graph'\in\graph_1\addNotEqVar{\xvar}{\yvar}$ for some
    $\graph_1\in\left(\graph\addVar{\xvar}\right)$.
    % 
  \end{enumerate}
We proceed as in case~\ref{pre:case:equals}, but now under the restriction that
$\xvar$ and $\yvar$ point to different vertices (rather than to the same vertex).
\item
\label{pre:case:aasigned}
  %% Assignment x:=y
  %% ----------------------------------
  $\action$ is of the form $\xvar\assigned\yvar$ and
  one of the following conditions is satisfied:
  \begin{enumerate}
  \item \label{pre:case:assignedA}%
    $\labeling(\xvar)\nequals\undef$, %
    $\labeling(\yvar)\nequals\undef$, %
    $\labeling(\xvar)=\labeling(\yvar)$ and
    $\graph'=\graph\delVar{\xvar}$.
    % 
  \item \label{pre:case:assignedB}%
    $\labeling(\xvar)\neq\undef$, %
    $\labeling(\yvar)=\undef$ and
%    $\graph'=\left(\graph\addEqVar{\xvar}{\yvar}\right)\delVar\xvar$.
    $\graph'=\graph_1\delVar\xvar$ where $\graph_1=\graph\addEqVar{\xvar}{\yvar}$.
    % 
  \item \label{pre:case:assignedC}%
    $\labeling(\xvar)=\undef$, %
    $\labeling(\yvar)\nequals\undef$ and %
    $\graph'=\graph$.
    % 
  \item \label{pre:case:assignedD}%
    $\labeling(\xvar)=\undef$, %
    $\labeling(\yvar)\equals\undef$ and %
    $\graph'\in(\graph\addVar{\yvar})$.
    % 
  \end{enumerate}
The difference compared to case~\ref{pre:case:equals} is that
the variable
$\xvar$ may have had any value before performing the assignment.
%
Therefore, we remove $\xvar$ from the graph.
\item \label{pre:case:assignedDotNext}%
  %% Assignment x:=y.next
  %% ----------------------------------
  $\action$ is of the form $\xvar\assigned\nxt\yvar$ and
  one of the following conditions is satisfied:
  \begin{enumerate}
  \item \label{pre:case:assignedDotNextA}%
    $\labeling(\xvar)\neq\undef$, %
    $\labeling(\yvar)\neq\undef$, %
    $\suc(\labeling(\yvar))\neq\undef$, %
    $\suc(\labeling(\yvar))=\labeling(\xvar)$ and
    $\graph'=\graph\delVar{\xvar}$.
    %
  \item \label{pre:case:assignedDotNextB}%
    $\labeling(\xvar)\neq\undef$, %
    $\labeling(\yvar)\neq\undef$, %
    $\labeling(\yvar)\neq\labeling(\nil)$, %
    $\suc(\labeling(\yvar))=\undef$ and
    $\graph'=\graph_1\delVar{\xvar}$, where
    $\graph_1=\graph\addEdgeBetween{\yvar}{\xvar}$.
  \item \label{pre:case:assignedDotNextC}%
    $\labeling(\xvar)\neq\undef$, %
    $\labeling(\yvar)=\undef$ %
    and there are graphs $\graph_1,\graph_2$ such that
    $\graph'=\graph_2\delVar{\xvar}$, %
    $\graph_2=\graph_1\addEdgeBetween{\yvar}{\xvar}$ and %
    $\graph_1\in\graph\addVarAsPredOf{\xvar}{\yvar}$.
  \item \label{pre:case:assignedDotNextD}%
    $\labeling(\xvar)=\undef$, %
    $\labeling(\yvar)\neq\undef$, %
    $\suc(\labeling(\yvar))\neq\undef$ and %
    $\graph'=\graph$.
  \item \label{pre:case:assignedDotNextE}%
    $\labeling(\xvar)=\undef$, %
    $\labeling(\yvar)\neq\undef$, %
    $\labeling(\yvar)\neq\labeling(\nil)$, %
    $\suc(\labeling(\yvar))=\undef$ and %
    $\graph'\in\graph\addEdge{\yvar}$.
  \item \label{pre:case:assignedDotNextF}%
    $\labeling(\xvar)=\undef$, %
    $\labeling(\yvar)=\undef$ %
 and   there are graphs $\graph_1,\graph_2,\graph_3$ such that
    $\graph_1\in\graph\addVar{\xvar}$, %
    $\graph_2\in\graph_1\addVarAsPredOf{\xvar}{\yvar}$, %
    $\graph_3\equals\graph_2\addEdgeBetween{\yvar}{\xvar}$ and %
    $\graph'=\graph_3\delVar{\xvar}$.
%     and %
%     $\graph'\in \graph_1\addVarAsPredOf{\xvar}{\yvar}$ for some
%     $\graph_1\in \graph\addVar{\xvar}$.
%    Observe that \xvar\ just serves here as a temporary variable.
  \end{enumerate}
Similarly to case~\ref{pre:case:aasigned} we remove $\xvar$ from the graph.
%
The successor of $\yvar$ should be defined and point to the same vertex as $\xvar$.
%
In case the successor is missing, we add an edge explicitly from the vertex
labeled by $\yvar$ to the vertex labeled by $\xvar$.
%
Furthermore, if $\xvar$ is missing then the successor of $\yvar$ may point
anywhere inside the graph.
\item
  %% Assignment x.next:=y
  %% ----------------------------------
  $\action$ is of the form $\nxt\xvar\assigned\yvar$ and
  one of the following conditions is satisfied:
  \begin{enumerate}
  \item \label{pre:case:dotNextAssignedA}%
    $\labeling(\xvar)\neq\undef$, %
    $\suc(\labeling(\xvar))\neq\undef$, %
    $\labeling(\yvar)\neq\undef$, %
    $\suc(\labeling(\xvar))=\labeling(\yvar)$ and
    $\graph'=\graph\delEdge{\xvar}$.
  \item \label{pre:case:dotNextAssignedB}%
    $\labeling(\xvar)\neq\undef$, %
    $\suc(\labeling(\xvar))\neq\undef$, %
    $\labeling(\yvar)=\undef$ and %
    $\graph'=\graph_1\delEdge{\xvar}$, %
    where $\graph_1=\graph\addVarAsSuccOf{\xvar}{\yvar}$.
  \item \label{pre:case:dotNextAssignedC}%
    $\labeling(\xvar)\neq\undef$, %
    $\suc(\labeling(\xvar))=\undef$, %
    $\labeling(\yvar)\neq\undef$, %
    $\labeling(\xvar)\neq\labeling(\nil)$ %
    and %
    $\graph'=\graph$.
  \item \label{pre:case:dotNextAssignedD}%
    $\labeling(\xvar)\neq\undef$, %
    $\suc(\labeling(\xvar))=\undef$, %
    $\labeling(\yvar)=\undef$, %
    $\labeling(\xvar)\neq\labeling(\nil)$ and %
    $\graph'\in\graph\addVar{\yvar}$.
  \item \label{pre:case:dotNextAssignedE}%
    $\labeling(\xvar)=\undef$, %
    $\labeling(\yvar)\neq\undef$ and %
    $\graph'=\graph_1\delEdge{\xvar}$, where %
    $\graph_1\in\graph\addVarAsPredOf{\yvar}{\xvar}$.
  \item \label{pre:case:dotNextAssignedF}%
    $\labeling(\xvar)=\undef$, %
    $\labeling(\yvar)=\undef$ and %
    there are graphs $\graph_1,\graph_2$ such that
    $\graph_1\in\graph\addVar{\yvar}$, %
    $\graph_2\in\graph_1\addVarAsPredOf{\yvar}{\xvar}$ and %
    $\graph'=\graph_2\delEdge{\xvar}$.
  \end{enumerate}
After performing the action, the successor of the vertex labeled by
$\xvar$ should be the same vertex as the one labeled by $\yvar$.
%
Before performing the action, the successor could have been anywhere inside the graph,
and the corresponding edge is therefore removed.
\end{enumerate}
%
{\bf Remark}
In the above definition, we assume that $\xvar$ and $\yvar$ are different variables.
%
It is straightforward to handle the case where they are the same variable.

%
For a transition $\transition=\tuple{\cstate_1,\action,\cstate_2}$ and
configurations $\conf=\tuple{\cstate,\graph}$ and
$\conf'=\tuple{\cstate',\graph'}$, we write
$\conf\pmovesto{\transition}\conf'$ to denote that
$\cstate=\cstate_1$, $\cstate'=\cstate_2$ and
$\graph\pmovesto{\action}\graph'$.
%
We use $\conf\pmovesto{}\conf'$ to denote that
$\conf\pmovesto{\transition}\conf'$ for some $\transition\in\transitions$.
%
For a set $\confs$ of configurations and a configuration $\conf$, we
define $\rank{\confs}{\conf}$ to be the smallest $n$ such that there
is a sequence
$\conf_0\pmovesto{}\conf_1\pmovesto{}\cdots\pmovesto{}\conf_n$ where
$\conf_0=\conf$ and there is a $\conf'\in\confs$ such that
$\conf_n\lt\conf'$.

The following lemma states that small configurations simulate larger ones
with respect to the backward relation.

\begin{lemma}
\label{pre:js:monotonic:lemma}
For configurations $\conf_1$, $\conf_2$ and $\conf_3$, if
$\conf_1\pmovesto{}\conf_2$ and $\conf_3\lt\conf_1$ then there
is a configuration $\conf_4$ such that $\conf_3\pmovesto{}\conf_4$ and
$\conf_4\lt\conf_2$.
\end{lemma}

The following lemma relates the backward and forward transition
relations.

\begin{lemma}
\label{backwards:iff:forwards:lemma}
Consider configurations $\conf_1$ and $\conf_2$.
%
If $\conf_1\pmovesto{}\conf_2$ then $\conf_2\movesto{}\uc{\conf_1}$.
%
If $\conf_1\movesto{}\uc{\conf_2}$
then $\conf_2\pmovesto{}\dc{\conf_1}$.
\end{lemma}
%
\ignore{
\begin{tikzpicture}[show background rectangle,>=stealth]
  \begin{scope}[xscale=2]
    \draw[fill=white] (0.6,1.2) parabola[bend pos=0.5] bend +(0,-1.5) +(0.8,0);
  \end{scope}
  \begin{scope}[blue,xscale=2]
    \draw (0.2,0) node(c2){$\conf_2$};
    \draw(0.55,0) node[rotate=180,scale=1.5](b){$\pmovesto{}$};
    \draw (1,0) node (c1){$\conf_1$};
  \end{scope}
  \begin{scope}[red,xscale=2]
    \draw (1,0.5) node[rotate=90]{$\lt$};
    \draw (1,1) node(c3){$\conf_3$};
    \draw (c3) node[left](e){$\exists$};
    \draw[->] (c2) -- (e.west);
  \end{scope}
\end{tikzpicture}
}
%
