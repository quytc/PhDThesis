\section{Proofs of Lemmas - Section \ref{section:pre}}

%\subsection*{Lemma~\ref{pre:js:monotonic:lemma}}
\input ladder

\subsection*{Lemma~\ref{backwards:iff:forwards:lemma}}

Suppose $\graph_1\pmovesto{\action}\graph_2$, 
we show there is
a $\graph_3$ such that $\graph_1\lt\graph_3$ and
$\graph_2\absmovesto{\action}\graph_3$ 
(in fact $\graph_2\movesto{\action}\graph_3$).
%
Let $\graph_i=\tuple{\vertices_i,\succ_i,\labeling_i}$
for $i\geq 1$.
%and
%$\graph_2=\tuple{\vertices_2,\succ_2,\labeling_2}$.
%
We consider five cases depending on the type of $\action$:
\begin{enumerate}
\item
$\action$ is of the form $\xvar\equals\yvar$.
%
We define $\graph_3=\graph_2$ and
show $\graph_1\lt\graph_3$ by considering 
the following four subcases.
%
For each one of these subcases, $\graph_2\movesto{\action}\graph_3$ 
follows directly from the definitions:
\begin{enumerate}
\item
$\labeling_1(\xvar)\nequals\undef$, 
$\labeling_1(\yvar)\nequals\undef$,
$\labeling_1(\xvar)=\labeling_1(\yvar)$ and 
$\graph_2=\graph_1$.
%
$\graph_1\lt\graph_3$ follows from $\graph_3=\graph_2=\graph_1$.
\item
\label{equal:defined:undefined}
$\labeling_1(\xvar)\neq\undef$, $\labeling_1(\yvar)=\undef$, and
$\graph_2=\graph_1\addEqVar{\xvar}{\yvar}$.
%
%
$\graph_1\lt\graph_3$ follows by deletion of the variable $\yvar$.
\item $\labeling_1(\xvar)=\undef$ and $\labeling_1(\yvar)\neq\undef$ 
is similar to case~(\ref{equal:defined:undefined}).
\item
$\labeling_1(\xvar)=\undef$, $\labeling_1(\yvar)=\undef$
and there is a $\graph_4$ in 
$\graph_1\addVar{\xvar}$ such that
$\graph_2=\graph_4\addEqVar{\xvar}{\yvar}$.
%
Observe that $\graph_4\lt\graph_3$ holds by deletion of the variable $\yvar$.
%
We show $\graph_1\lt\graph_4$ by considering the three subcases 
resulting from $\graph_1\addVar{\xvar}$:
%
\begin{enumerate}
\item
\label{equal:undefined:undefined:addisolated}
There is a $\vertex\not\in\vertices_1$ such that
$\graph_4=\gupdate{(\graph_1\addVertex{\vertex})}{\labeling}\updateby{\xvar}{\vertex}$.
%
$\graph_1\lt\graph_4$ follows by deletion of the variable $\xvar$ 
and the vertex $\vertex$.
\item
\label{equal:undefined:undefined:addvariable}
There is a $\vertex\in\vertices_1$ such that
$\graph_4=\gupdate{\graph_1}{\labeling}\updateby{\xvar}{\vertex}$.
%
$\graph_1\lt\graph_4$ follows by deletion of the variable $\xvar$.
\item
\label{equal:undefined:undefined:addsimple}
There are $\vertex_1\in\vertices_1$ 
and $\vertex_2\not\in\vertices_1$ with $\succ_1(\vertex_1)\neq\undef$, 
and graphs $\graph_5,\graph_6$ and $\graph_7$ such that
$\graph_7=\graph_1\addVertex{\vertex_2}$,
$\graph_6=\gupdate{\graph_7}{\succ}\updateby{\vertex_2}{\succ_7(\vertex_1)}$,
$\graph_5=\gupdate{\graph_6}{\succ}\updateby{\vertex_1}{\vertex_2}$, and
$\graph_4=\gupdate{\graph_5}{\labeling}\updateby{\xvar}{\vertex_2}$.
%
$\graph_1\lt\graph_4$ follows by deletion of the variable $\xvar$, and elimination
of the vertex $\vertex_2$ by contraction.
\end{enumerate}
%
\end{enumerate}
%
%
%
\item
$\action$ is of the form $\xvar\nequals\yvar$.
%
%
We define $\graph_3=\graph_2$.
%
We show $\graph_1\lt\graph_3$ by considering 
the following four subcases.
%
For each one of these subcases, $\graph_2\movesto{\action}\graph_3$ 
follows directly from the definitions:
\begin{enumerate}
\item
$\labeling_1(\xvar)\nequals\labeling_1(\yvar)$ and 
$\graph_2=\graph_1$.
%
$\graph_1\lt\graph_3$ follows from $\graph_3=\graph_2=\graph_1$.
\item
\label{nequal:defined:undefined}
$\labeling_1(\xvar)\nequals\undef$, $\labeling_1(\yvar)=\undef$
and $\graph_2\in\graph_1\addNotEqVar{\xvar}{\yvar}$.
%
Define $\graph_3=\graph_2$. 
%
$\graph_3$ is in $\graph_1\addVar{\yvar}$ 
and $\labeling_3(\xvar)\nequals\labeling_3(\yvar)$.
%
%Clearly $\graph_4\lt\graph_2$. 
%
We show $\graph_1\lt\graph_3$
by the arguments in~(\ref{equal:undefined:undefined:addisolated}), 
(\ref{equal:undefined:undefined:addvariable}) and
(\ref{equal:undefined:undefined:addsimple}).
%
\item
$\labeling_1(\xvar)=\undef$ and $\labeling_1(\yvar)\nequals\undef$
is similar to~(\ref{nequal:defined:undefined}) where $\xvar$ and $\yvar$
are permuted.
\item
$\labeling_1(\xvar)=\undef$, $\labeling_1(\yvar)=\undef$ and
there is a $\graph_4$ in $\graph_1\addVar{\xvar}$
such that $\graph_2\in\graph_4\addNotEqVar{\xvar}{\yvar}$.
%
We get $\graph_1\lt\graph_4$ by the arguments 
in~(\ref{equal:undefined:undefined:addisolated}), 
(\ref{equal:undefined:undefined:addvariable}) and
(\ref{equal:undefined:undefined:addsimple}).
%
$\graph_4\lt\graph_2$ by the arguments of~(\ref{nequal:defined:undefined})
where $\graph_1$ is replaced by $\graph_4$.
\end{enumerate}
%
%
%
\item
$\action$ is of the form $\xvar\assigned\yvar$.
%
We consider 
the four subcases of the definition of $\graph_1\pmovesto{\action}\graph_2$.
% 
%$\graph_2\absmovesto{\action}\graph_3$ follows directly.
%
%For each one of the subcases, we show 
%$\graph_2\absmovesto{\action}\uc{\graph_1}$:
\begin{enumerate}
\item
\label{assignment:defined:defined}
$\labeling_1(\xvar)\nequals\undef$,
$\labeling_1(\yvar)\nequals\undef$
and $\labeling_1(\xvar)=\labeling_1(\yvar)$ and 
$\graph_2=\graph_1\delVar{\xvar}$.
%
Observe $\labeling_2(\yvar)\nequals\undef$, $\suc_1=\suc_2$ and
$\labeling_1=\labeling_2\updateby{\xvar}{\labeling_2(\yvar)}$.
%
Hence $\graph_2\movesto{\action}\graph_1$.
%
We define $\graph_3=\graph_1$.
\item
\label{assignment:defined:undefined}
$\labeling_1(\xvar)\nequals\undef$, $\labeling_1(\yvar)=\undef$
and there exists a $\graph_3=\graph_1\addEqVar{\xvar}{\yvar}$
such that $\graph_2=\graph_3\delVar{\xvar}$.
%
Observe 
$\labeling_3(\xvar)\nequals\undef$,
$\labeling_3(\yvar)\nequals\undef$ and 
$\labeling_3(\xvar)=\labeling_3(\yvar)$.
%
Using arguments similar to~(\ref{assignment:defined:defined}),
we deduce $\graph_2\movesto{\action}\graph_3$.
%
Furthermore, we have $\graph_1\lt\graph_3$ by deletion of the 
variable $\xvar$. 
%
\item
$\labeling_1(\xvar)=\undef$, $\labeling_1(\yvar)\nequals\undef$
and $\graph_2=\graph_1$.
%
We have $\labeling_2(\yvar)\nequals\undef$, and define 
$\graph_3=\gupdate{\graph_2}{\labeling}\updateby{\xvar}{\labeling_2(\yvar)}$.
%
We have by construction $\graph_2\movesto{\action}\graph_3$, and
$\graph_1\lt\graph_3$ by deletion the variable $\xvar$.
%
\item 
$\labeling_1(\xvar)=\undef$, $\labeling_1(\yvar)=\undef$ and
$\graph_2\in\graph_1\addVar{\yvar}$.
%
We have $\labeling_2(\yvar)\nequals\undef$, and define 
$\graph_3=\gupdate{\graph_2}{\labeling}\updateby{\xvar}{\labeling_2(\yvar)}$.
%
We have by construction $\graph_2\movesto{\action}\graph_3$, and
$\graph_1\lt\graph_3$ by deletion of both variables $\xvar$ and $\yvar$
if $\labeling_2(\yvar)=\undef$.
%
\end{enumerate}
%
%
%
\item
$\action$ is of the form $\xvar\assigned\nxt\yvar$.
%
For each one of the six subcases corresponding to the definition 
of $\graph_1\pmovesto{\action}\graph_2$, we define $\graph_3$ 
such that $\graph_2\movesto{\action}\graph_3$
and $\graph_1\lt\graph_3$.
% 
%Observe that $\graph_2\absmovesto{\action}\graph_3$ follows directly.
%
\begin{enumerate}
\item
\label{assignmentnext:defined:defined:defined}
$\labeling_1(\xvar)\nequals\undef$ , $\labeling_1(\yvar)\nequals\undef$, 
$succ_1(\labeling_1(\yvar))\neq\undef$, 
$\succ(\labeling_1(\yvar))=\labeling_1(\xvar)$ and 
$\graph_2=\graph_1\delVar{\xvar}$.
%
We define $\graph_3=\graph_1$.
%
%Clearly $\graph_2\movesto{\action}\graph_3$
%and $\graph_1\lt\graph_3$.
\item
\label{assignmentnext:defined:defined:undefined}
$\labeling_1(\xvar)\neq\undef$,
$\labeling_1(\yvar)\neq\undef$, 
$\labeling_1(\yvar)\neq\labeling_1(\nil)$, 
$\succ_1(\labeling_1(\yvar))=\undef$,
and there is a $\graph_3=\graph_1\addEdgeBetween{\yvar}{\xvar}$
such that
$\graph_2=\graph_3\delVar{\xvar}$.
%
%Observe that $\graph_2\movesto{\action}\graph_3$.
%
%We show $\graph_1\lt\graph_4$ as follows.
%
%Observe also that $\graph_3=\gupdate{\graph_1}{\succ}
%\updateby{\labeling_1(\yvar)}{\labeling_1(\xvar)}$.
%
We deduce $\graph_1\lt\graph_3$ holds by edge deletion.
% as defined in
%$\graph_1=\gupdate{\graph_3}{\succ}\updateby{\labeling_1(\yvar)}{\undef}$.
%
\item
\label{assignmentnext:defined:undefined}
$\labeling_1(\xvar)\neq\undef$,
$\labeling_1(\yvar)=\undef$,
and there are graphs $\graph_3,\graph_4$ such that
$\graph_4\in\graph_1\addVarAsPredOf{\xvar}{\yvar}$,
$\graph_3=\graph_4\addEdgeBetween{\yvar}{\xvar}$, and
$\graph_2=\graph_3\delVar{\xvar}$.
%
Observe that 
$\labeling_3(\xvar)\nequals\undef$,
$\labeling_3(\yvar)\nequals\undef$,
$\succ_3(\labeling_3(\yvar))$,
$\succ_3(\labeling_3(\yvar))=\labeling_3(\xvar)$ and 
$\graph_2=\graph_3\delVar{\xvar}$,
meaning $\graph_2\movesto{\action}\graph_3$.
%
We show $\graph_1\lt\graph_4$
and $\graph_4\lt\graph_3$ as follows.
%
We start with $\graph_1\lt\graph_4$, and consider the 
three subcases corresponding to the definition 
of $\graph_4\in\graph_1\addVarAsPredOf{\xvar}{\yvar}$.
\begin{enumerate}
\item 
\label{assignmentnext:defined:undefined:vertex}
There is a $\vertex\not\in\vertices_1$ such that
$\graph_4=\gupdate{(\graph_1\addVertex{\vertex})}{\labeling}\updateby{\yvar}{\vertex}$.
%
$\graph_1\lt\graph_4$ by deleting the variable $\yvar$ and the vertex $\vertex$.
%
\item
\label{assignmentnext:defined:undefined:variable}
There is a $\vertex\in\vertices_1$ such that 
$\labeling_1(\vertex)\nequals\labeling_1(\nil)$
with $\succ_1(\vertex)=\undef$ or $\succ_1(\vertex)=\labeling_1(\xvar)$ 
and $\graph_4=\gupdate{\graph_1}{\labeling}\updateby{\yvar}{\vertex}$.
%
$\graph_1\lt\graph_4$ by deleting the variable $\yvar$.
%
\item
\label{assignmentnext:defined:undefined:contraction}
There are $\vertex_1\in\vertices_1$, 
$\vertex_2\not\in\vertices_1$ and graphs $\graph_5, \graph_6$
such that 
$\succ_1(\vertex_1)=\labeling_1(\xvar)$,
$\graph_6=\gupdate{(\graph_1\addVertex{\vertex_2})}{\succ}
\updateby{\vertex_2}{\succ_1(\vertex_1)}$,
$\graph_5=\gupdate{\graph_6}{\succ}\updateby{\vertex_1}{\vertex_2}$, and 
$\graph_4=\gupdate{\graph_5}{\labeling}\updateby{\yvar}{\vertex_2}$.
%
$\graph_1\lt\graph_4$ by deleting the variable $\yvar$ and 
eliminating $\vertex_2$ by contraction.
\end{enumerate}
%
We show $\graph_4\lt\graph_3$ by considering the two cases 
$\succ_4(\labeling_4(\yvar))=\labeling_4(\xvar)$
and 
$\succ_4(\labeling_4(\yvar))=\undef$
when defining $\graph_3$ in 
$\graph_3=\graph_4\addEdgeBetween{\yvar}{\xvar}$.
%
Observe that these are the only possible values for 
$\succ_4(\labeling_4(\yvar))$
according to the definition of $\graph_4$.
%
In the first case, we get $\graph_4=\graph_3$. % (and hence $\graph_4\lt\graph_3$).
%
The second case gives
$\graph_3=\gupdate{\graph_4}{\succ}
\updateby{\labeling_4(\yvar)}{\labeling_4(\xvar)}$,
and hence $\graph_4\lt\graph_3$ by edge deletion.
% as defined in 
%$\graph_4=\gupdate{\graph_3}{\succ}
%\updateby{\labeling_3(\yvar)}{\undef}$.
%
\item
$\labeling_1(\xvar)=\undef$,
$\labeling_2(\yvar)\neq\undef$,
$\succ_1(\labeling_1(\yvar))\neq\undef$, and
$\graph_2=\graph_1$.
%
We define 
$\graph_3=\gupdate{\graph_2}{\labeling}
\updateby{\xvar}{\succ_2(\labeling_2(\yvar))}$.
% 
%Observe $\graph_2\movesto{\action}\graph_3$.
% 
We have $\graph_1\lt\graph_3$ by deleting variable $\xvar$.
%
\item
$\labeling_1(\xvar)=\undef$,
$\labeling_1(\yvar)\neq\undef$,
$\suc_1(\labeling_1(\yvar))=\undef$, 
$\labeling_1(\yvar)\nequals\labeling_1(\nil)$ and
$\graph_2$ in the set $\graph_1\addEdge{\yvar}$.
%
Notice that $\suc_2(\labeling_2(\yvar))\nequals\undef$.
%
We define $\graph_3=\graph_2\addVarAsSuccOf{\yvar}{\xvar}$. 
%
%Observe that $\graph_2\movesto{\action}\graph_3$, 
%and 
$\graph_2\lt\graph_3$ holds by deletion of variable 
$\xvar$.
% 
We show $\graph_1\lt\graph_2$ by considering the
three subcases corresponding to the definition of 
$\graph_2$ in the set $\graph_1\addEdge{\yvar}$:
\begin{enumerate}
\item 
\label{assignmentnext:undefined:defined:undefined:isolate}
There is a $\vertex\not\in\vertices_1$ and a graph $\graph_2$ 
such that $\graph_2=\gupdate{\graph_4}{\succ}
\updateby{\labeling_4(\yvar)}{\vertex}$ and 
$\graph_4=\graph_1\addVertex{\vertex}$.
%
We have $\graph_1\lt\graph_4$ by deleting $\vertex$, while
$\graph_4\lt\graph_2$ by edge deletion.
% as defined in 
%$\graph_4=\gupdate{\graph_2}{\suc}\updateby{\labeling_1(\yvar)}{\undef}$.
%
\item
\label{assignmentnext:undefined:defined:undefined:notisolate}
There is a 
$\vertex\in\vertices$ such that $\graph_2=\gupdate{\graph_1}{\succ}
\updateby{\labeling_1(\yvar)}{\vertex}$.
%
%We have 
$\graph_1\lt\graph_2$ holds by edge deletion.
%  as defined in 
%$\graph_1=\gupdate{\graph_2}{\suc}\updateby{\labeling_1(\yvar)}{\undef}$.
%
\item
\label{assignmentnext:undefined:defined:undefined:simple}
There are $\vertex_1\in\vertices_1$, 
$\vertex_2\not\in\vertices_1$ and graphs $\graph_4, \graph_5$
such that 
$\graph_5=\gupdate{(\graph_1\addVertex{\vertex_2})}{\succ}
\updateby{\vertex_2}{\succ_1(\vertex_1)}$,
$\graph_4=\gupdate{\graph_5}{\succ}\updateby{\vertex_1}{\vertex_2}$, 
and $\graph_3=\gupdate{\graph_4}{\suc}\updateby{\labeling(\yvar)}{\vertex_2}$.
%
$\graph_1\lt\graph_5$ holds by deleting an edge
% as defined in  
%$\graph_4=\gupdate{\graph_1}{\suc}\updateby{\labeling(\yvar)}{\undef}$
and by eliminating $\vertex_2$ with contraction.
\end{enumerate}
%
\item
$\labeling_1(\xvar)=\undef$,
$\labeling_1(\yvar)=\undef$
and there are $\graph, \graph', \graph''$ such that 
$\graph\in\graph_1\addVar{\xvar}$,
$\graph'\in\graph'\addVarAsPredOf{\xvar}{\yvar}$,
$\graph''=\graph'\addEdgeBetween{\yvar}{\xvar}$ and
$\graph_2=\graph''\delVar{\xvar}$.
We define $\graph_3=\graph''$.
$\graph_1\lt\graph_3$ by similar arguments
to ~\ref{equal:undefined:undefined:addisolated}-\ref{equal:undefined:undefined:addvariable}-\ref{equal:undefined:undefined:addsimple} 
(in $\graph'\in\graph_1\addVar{\xvar}$) and 
to ~\ref{assignmentnext:defined:undefined} 
(in $\graph'\in\graph'\addVarAsPredOf{\xvar}{\yvar}$ and
$\graph''=\graph'\addEdgeBetween{\yvar}{\xvar}$).
Also, $\graph_2\movesto{\action}\graph_3$ because
$\labeling_3(\xvar)\nequals\undef$ , $\labeling_3(\yvar)\nequals\undef$, 
$succ_3(\labeling_3(\yvar))\neq\undef$, 
$\succ(\labeling_3(\yvar))=\labeling_3(\xvar)$ and 
$\graph_2=\graph_3\delVar{\xvar}$.
%
\end{enumerate}
%
%
%
\item
$\action$ is of the form $\nxt\xvar\assigned\yvar$.
%
For each one of the six subcases corresponding 
to the definition 
of $\graph_1\pmovesto{\action}\graph_2$, 
we find a graph $\graph_3$ 
such that $\graph_2\movesto{\action}\graph_3$
and $\graph_1\lt\graph_3$:
% 
%Observe that $\graph_2\absmovesto{\action}\graph_3$ 
%follows directly.
\begin{enumerate}
\item
\label{nextassignment:defined:defined:defined}
$\labeling_1(\xvar)\neq\undef$,
$\labeling_1(\yvar)\neq\undef$,
$\suc_1(\labeling_1(\xvar))\neq\undef$,
$\suc_1(\labeling_1(\xvar))=\labeling_1(\yvar)$ and
$\graph_2=\graph_1\delEdge{\xvar}$.
%
Define $\graph_3=\graph_1$. % (hence $\graph_1\lt\graph_3$).
%
Observe $\graph_3=\gupdate{\graph_2}{\succ}
\updateby{\labeling_2(\xvar)}
{\labeling_2(\yvar)}$,
meaning $\graph_2\movesto{\action}\graph_3$.
%
\item
$\labeling_1(\xvar)\neq\undef$,
$\suc_1(\labeling_1(\xvar))\neq\undef$,
$\labeling_1(\yvar)=\undef$ and there is a 
$\graph_3=\graph_1\addVarAsSuccOf{\xvar}{\yvar}$
such that
$\graph_2=\graph_3\delEdge{\xvar}$.
%
%Observe $\labeling_4(\xvar)\neq\undef$,
%$\suc_4(\labeling_4(\xvar))
%=\labeling_4(\yvar)$.
%
%Using similar arguments 
%to those of~(\ref{nextassignment:defined:defined:defined}),
%we deduce $\graph_2\movesto{\action}\graph_4$.
%
Observe that $\graph_1\lt\graph_3$ by deletion
of variable $\yvar$.
%
%We define $\graph_3=\graph_4$.
\item
$\labeling_1(\xvar)\neq\undef$,
$\suc_1(\labeling_1(\xvar))=\undef$,
$\labeling_1(\yvar)\neq\undef$, 
$\labeling_1(\xvar)\neq\labeling_1(\nil)$ 
and
$\graph_2=\graph_1$.
%
We define $\graph_3=\gupdate{\graph_2}{\succ}
\updateby{\labeling_2(\xvar)}
{\labeling_2(\yvar)}$.
%
%We get $\graph_2\movesto{\action}\graph_3$ by definition
%of $\graph_3$.
%
Also, observe 
%$\graph_1=\gupdate{\graph_3}{\succ}
%\updateby{\labeling_3(\xvar)}{\undef}$, meaning
$\graph_1\lt\graph_3$ by edge deletion.
\item
$\labeling_1(\xvar)\neq\undef$,
$\suc_1(\labeling_1(\xvar))=\undef$,
$\labeling_1(\yvar)=\undef$, 
$\labeling_1(\xvar)\neq\labeling_1(\nil)$, and
$\graph_2$ in the set $\graph_1\addVar{\yvar}$.
%
Observe $\labeling_2(\yvar)\nequals\undef$.
%
%
We define $\graph_3=\gupdate{\graph_2}{\succ}
\updateby{\labeling_2(\xvar)}
{\labeling_2(\yvar)}$.
%
%By definition, we get $\graph_2\movesto{\action}\graph_3$.
%
%Also, observe $\graph_2=\gupdate{\graph_3}{\succ}
%\updateby{\labeling_3(\xvar)}{\undef}$,
%meaning 
Observe $\graph_2\lt\graph_3$ holds by edge deletion,
%
%Furthermore, $\graph_1=\gupdate{\graph_2}{\labeling}
%\updateby{\yvar}{\undef}$, meaning 
and $\graph_1\lt\graph_2$
by deletion of variable $\yvar$.
\item
\label{nextassignment:undefined:undefined}
$\labeling_1(\xvar)=\undef$,
$\labeling_1(\yvar)\neq\undef$, and
there is a $\graph_3$ such that
$\graph_2=\graph_3\delEdge{\xvar}$ where
$\graph_3\in(\graph_1\addVarAsPredOf{\yvar}{\xvar})$.
%
We have $\graph_3=\gupdate{\graph_2}{\succ}
\updateby{\labeling_2(\xvar)}
{\labeling_2(\yvar)}$, meaning 
$\graph_2\movesto{\action}\graph_3$.
%
Furthermore, $\graph_1\lt\graph_3$ by arguments
similar to those 
in~(\ref{assignmentnext:defined:undefined:vertex}),
(\ref{assignmentnext:defined:undefined:variable})
and (\ref{assignmentnext:defined:undefined:contraction}).
\item
$\labeling_1(\xvar)=\undef$,
$\labeling_1(\yvar)=\undef$, and
there  graphs $\graph_3,\graph_4$ such that
$\graph_4\in\graph_1\addVar{\yvar}$,
$\graph_3\in\graph_4\addVarAsPredOf{\yvar}{\xvar}$, and
$\graph_2=\graph_3\delEdge{\xvar}$.
%
We use similar arguments to 
those in~(\ref{nextassignment:undefined:undefined}) 
to deduce $\graph_2\movesto{\action}\graph_3$
and $\graph_1\lt\graph_3$.
\end{enumerate}
\end{enumerate}


%\subsection*{Lemma~\ref{forwards:implies:bakwards:lemma}}
%
Assume $\graph_1\absmovesto{\action}\graph_2$.
%
This implies the existence of a $\graph_3$ such that
$\graph_3\movesto{\action}\graph_2$ 
and $\graph_3\lt\graph_1$.
%
Showing that $\graph_2\pmovesto{\action}\graph_3$
is sufficient for proving $\graph_2\pmovesto{}\dc{\graph_1}$.
%
%Lemma~(\ref{pre:js:monotonic:lemma}) gives that
%for any $\graph_2$ verifying 
%$\graph_2\lt\graph_3$, there exists a $\graph_4$
%such that 
%$\graph_4\lt\graph_1$ and $\graph_2\pmovesto{\action}\graph_4$.
%
We let in the following 
%that $\graph_1\movesto{\action}\graph_3$
%implies $\graph_3\pmovesto{\action}\graph_1$.
%
%We let 
$\graph_i=\tuple{\vertices_i,\succ_i,\labeling_i}$ for $i\geq0$; 
and consider five cases depending on the type of $\action$:
\begin{enumerate}
\item
  $\action$ is of the form $\xvar\equals\yvar$. 
  %
  By definition:
  $\labeling_3(\xvar)\nequals\undef$,
  $\labeling_3(\yvar)\nequals\undef$, 
  $\labeling_3(\xvar)=\labeling_3(\yvar)$ 
  and $\graph_3=\graph_2$.
  %
  We have $\graph_2\pmovesto{\action}\graph_3$ 
  by case~(\ref{pre:case:equalsA}) in Section~\ref{section:pre}.
  %
\item
  $\action$ is of the form $\xvar\nequals\yvar$.
%
  By definition:
  $\labeling_3(\xvar)\nequals\undef$, 
  $\labeling_3(\yvar)\nequals\undef$, 
  $\labeling_3(\xvar)\neq\labeling_3(\yvar)$ 
  and $\graph_3=\graph_2$.
  % 
  We have $\graph_2\pmovesto{\action}\graph_3$ 
  by case~(\ref{pre:case:notEqualsA}) in Section~\ref{section:pre}.
  %
\item
  $\action$ is of the form $\xvar\assigned\yvar$.
  %
  By definition: 
  $\labeling_3(\xvar)\nequals\undef$,
  $\labeling_3(\yvar)\nequals\undef$, 
  $\labeling_3(\xvar)=\labeling_3(\yvar)$ and
  $\graph_3=\graph_2\delVar{\xvar}$.
  %
  We have $\graph_2\pmovesto{\action}\graph_3$ 
  by case~(\ref{pre:case:assignedA}) in Section~\ref{section:pre}.
  %
\item
$\action$ is of the form  $\xvar\assigned\nxt\yvar$.
  By definition:
  %
  $\labeling_3(\xvar)\neq\undef$, %
  $\labeling_3(\yvar)\neq\undef$, %
  $\suc_3(\labeling_3(\yvar))\neq\undef$, %
  $\suc_3(\labeling_3(\yvar))=\labeling_3(\xvar)$ and
  $\graph_3=\graph_2\delVar{\xvar}$.
  %
  We have $\graph_2\pmovesto{\action}\graph_3$ 
  by case~(\ref{pre:case:assignedDotNextA}) in Section~\ref{section:pre}.
  %
\item
$\action$ is of the form $\nxt\xvar\assigned\yvar$.
  By definition:
  $\labeling_3(\xvar)\neq\undef$, %
  $\suc_3(\labeling_3(\xvar))\neq\undef$, %
  $\labeling_3(\yvar)\neq\undef$, %
  $\suc_3(\labeling_3(\xvar))=\labeling_3(\yvar)$ and
  $\graph_3=\graph_3\delEdge{\xvar}$.
  %
  We have $\graph_2\pmovesto{\action}\graph_3$ 
  by case~(\ref{pre:case:dotNextAssignedA}) in Section~\ref{section:pre}.
\end{enumerate}

%



