%
\section{Preliminaries}
\label{section:prels}
%
We consider programs that operate on data structures
with one next-pointer such as traditional singly-linked lists and circular
lists (possibly sharing their parts).
%
We represent the store as a graph, where the vertices represent the list cells,
and the successor of a vertex represents the cell pointed to by the current one.
%
The graphs are of a special form in the sense that each vertex has at most one successor.
%
A program also uses a finite set of pointers which we call {\it variables}.
%
A cell is labeled by the (possibly empty) set of variables pointing to it.
%
%%
\localnote{Fred: Include a picture here}
%%

For simplicity of presentation, we will treat the constant $\nil$ as a
variable, with the special property that whenever a vertex is labeled by $\nil$,
the successor of the cell is undefined.

%We use $\nat$ to denote the set of natural numbers.
%
%For a set $A$, we use $A^*$ to denote the set of finite words over $A$.
%
For a partial function $f$, we write $f(a)=\undef$ to denote that $f(a)$ is undefined.
%
For a (partial) function $f$ , we use
$f\updateby{a}{b}$ to denote the function $f'$ such that
$f'(a)=b$ and $f'(x)=f(x)$ if $x\neq a$.
%
%We use
%$f\restrict{a}$ to denote the restriction of $f$ to elements different from $a$, i.e.,
%$f\restrict{a}$ is a function $f'$ such that
% $f'(a)=\undef$ and $f'(b)=f(b)$ if $b\neq a$.
%
%We write $f(a)=f(b)$ to denote that both $f(a)$ and $f(b)$ are defined and they are equal.
%
%In other words, $f(a)=f(b)$ does not holds in case either $f(a)=\undef$ or $f(b)=\undef$ 
%(or if both $f(a)=\undef$ and $f(b)=\undef$).
%
%Similar relations are interpreted analogously.
%
%For instance, $f(a)\neq f(b)$ holds iff $f(a)$ and $f(b)$ are defined and their values are different.

Formally, we assume a finite set $\vars$ of variables including the element
$\nil$.
%
A {\it program} $\prg$ is a pair $\prgtuple$
where $\cstates$ is a finite set of {\it control states} and
$\transitions$ is a finite set of {\it transitions}. 
%
A transition is a triple $\tuple{\cstate_1,\action,\cstate_2}$ where
$\cstate_1,\cstate_2\in\cstates$ are control states and $\action$ is an {\it action}.
%
An {\it action} is of one of the following forms
%$\pointer \xvar$
$\xvar\equals\yvar$, $\xvar\nequals\yvar$,
$\xvar\assigned\yvar$ where $\xvar\neq\nil$,
$\nxt\xvar\equals\yvar$ where $\xvar\neq\nil$,
or
$\xvar\assigned\nxt\yvar$ where $\xvar,\yvar\neq\nil$.
%
The transition corresponds to the program changing control state from
$\cstate_1$ to $\cstate_2$ while performing the operation described in $\action$ on the data structure.
%
We choose to work with the above minimal set of operations.
%
Other operations, e.g., $\xvar\equals\nxt\yvar$, $\xvar\nequals\nxt\yvar$, etc, can be expressed
using the given set.


A  {\it graph}
$\graph$ is a triple $\tuple{\vertices,\suc,\labeling}$
where $\vertices$ is a finite set of {\it vertices},
$\suc$ is a partial function from $\vertices$ to $\vertices$, and
$\labeling$ is a partial function from $\xvars$ to $\vertices$.
%
Furthermore, it is always the case that $\suc(\labeling(\nil))=\undef$.
%
Intuitively, the vertices correspond to the list cells.
%
The function $\succ$ defines the successors of the cells.
%
If $\succ(\vertex)=\undef$, %then
the cell represented by $\vertex$ has currently no successor.
%
The function $\labeling$ defines the cell to which a given variable points.
%
If $\labeling(\xvar)=\undef$, %then
the value of variable (pointer) $\xvar$ is undefined.
%
%We use $\varsof{\graph}$ to denote the set of variables which occur in 
%$\graph$, i.e., $\varsof{\graph}=\setcomp{\xvar}{\labeling(\xvar)\neq\undef}$.
% 

A {\it configuration} $\conf$ is a pair $\tuple{\cstate,\graph}$ where
$\cstate\in\cstates$ is a control state and $\graph$ is a graph.
%
We define a transition relation on configurations as follows.
%
Let $\transition=\tuple{\cstate_1,\action,\cstate_2}$ be a transition and let
$\conf=\tuple{\cstate,\graph}$ and $\conf'=\tuple{\cstate',\graph'}$ be configurations.
%
We write $\conf\movesto{\transition}\conf'$ to denote that 
$\cstate=\cstate_1$, $\cstate'=\cstate_2$, and $\graph\movesto{\action}\graph'$,
where
$\graph\movesto{\action}\graph'$ holds
if one of the following conditions is satisfied:
\begin{itemize}
\item
$\action$ is of the form $\xvar\equals\yvar$,
$\labeling(\xvar)\neq\undef$,
$\labeling(\yvar)\neq\undef$,
$\labeling(\xvar)=\labeling(\yvar)$, and
$\graph'=\graph$.
\item
$\action$ is of the form $\xvar\nequals\yvar$,
$\labeling(\xvar)\neq\undef$,
$\labeling(\yvar)\neq\undef$,
$\labeling(\xvar)\neq\labeling(\yvar)$, and
$\graph'=\graph$.
\item
$\action$ is of the form $\xvar\assigned\yvar$, 
$\labeling(\yvar)\neq\undef$,
$\suc'=\suc$,
and $\labeling'=\labeling\updateby{\xvar}{\labeling(\yvar)}$.
\item %\localnote{ahmed: removed condition $\labeling(\xvar)\neq\undef$}
$\action$ is of the form $\xvar\assigned\nxt\yvar$,
$\labeling(\yvar)\neq\undef$,
$\succ(\labeling(\yvar))\neq\undef$,
$\suc'=\suc$, and
$\labeling'=\labeling\updateby{\xvar}{\suc(\labeling(\yvar))}$.
\item % \localnote{ahmed: restored condition $\labeling(\xvar)\neq\labeling(\nil)$}
$\action$ is of the form $\nxt\xvar\assigned\yvar$,
$\labeling(\xvar)\neq\undef$,
$\labeling(\yvar)\neq\undef$,
$\labeling(\xvar)\neq\labeling(\nil)$,
$\labeling'=\labeling$, and
$\suc'=\suc\updateby{\labeling(\xvar)}{\labeling(\yvar)}$.
\end{itemize}
%
We define $\movesto{}$ as $\bigcup_{\transition\in\transitions}\movesto{\transition}$ and 
use $\movesto{*}$ to denote the reflexive transitive closure of $\movesto{}$.
%
For sets $\confs_1$ and $\confs_2$ of configurations, 
we use $\confs_1\movesto{}\confs_2$ to denote that
$\conf_1\movesto{}\conf_2$ for some $\conf_1\in\confs_1$ and $\conf_2\in\confs_2$.
%
By  $\conf_1\movesto{}\confs_2$ we mean 
$\set{\conf_1}\movesto{}\confs_2$.
%
We define $\conf_1\movesto{*}\confs_2$, $\confs_1\movesto{*}\confs_2$, etc in a similar manner to above.
