%
\section{Operations on Graphs}
\label{section:graphops}
In this section, we define a number of operations on graphs which we
use in the subsequent sections.
%
In the rest of the section, we assume a graph
$\graph=\tuple{\vertices,\succ,\labeling}$.
%

%% Notation (g.succ)[v1<-v2]
%% ----------------------------------------
For $\vertex_1,\vertex_2\in\vertices$, we use
$\gupdate{\graph}{\succ}\updateby{\vertex_1}{\vertex_2}$ to denote the
graph $\graph'=\tuple{\vertices',\succ',\labeling'}$ where
$\vertices'=\vertices$, $\labeling'=\labeling$, and
$\succ'=\succ\updateby{\vertex_1}{\vertex_2}$.
%
Intuitively, we only modify $\graph$ so that $\vertex_2$ becomes the
successor of $\vertex_1$.
%
%% Notation (g.lambda)[x<-v2]
%% ----------------------------------------
We define $\gupdate{\graph}{\labeling}\updateby{\xvar}{\vertex}$
analogously. That is, we make \xvar\ point to \vertex.
%

%% Definition of a simple vertex
%% ----------------------------------------
%\paragraph{\bf Simple vertex.}
%
For a vertex $\vertex\in\vertices$, we say that $\vertex$ is {\it simple} if
$\sizeof{\succ^{-1}(\vertex)}=1$, $\succ(\vertex)\neq\undef$, and
there is no $\xvar\in\xvars$ with $\labeling(\xvar)=\vertex$.
%
In other words, $\vertex$ has exactly one predecessor, one successor 
and no label.
%
%
\ignore{%
  % For a graph $\graph$, we abuse notation and introduce the
  % following notations.
  % %
  % $\vertices(\graph)$, $\succ(\graph)$, and $\labeling(\graph)$ to
  % denote the set of vertices, the successor function, and the
  % labeling function respectively in $\graph$.
  % %
  % For instance, if
  % $\graph_1=\tuple{\vertices_1,\succ_1,\labeling_1}$, then
  % $\vertices(\graph_1)=\vertices_1$, $\succ(\graph_1)=\succ_1$, and
  % $\labeling(\graph_1)=\labeling_1$.
  % %
  % If $\labeling(\graph)(\xvar)\neq\undef$ then we sometimes write
  % $\succ(\graph)(\xvar)$ instead of
  % $\succ(\graph)(\labeling(\graph)(\xvar))$.
  % %
} % end of ignore
%
%% Definition of an isolated vertex
%% ----------------------------------------
%\paragraph{\bf Isolated vertex.}
%
We say that $\vertex$ is %
{\it isolated} in $\graph$ if $\succ(\vertex)=\undef$,
$\succ^{-1}(\vertex)=\emptyset$, and there is no $\xvar\in\xvars$ with
$\labeling(\xvar)=\vertex$.
%
In other words, $\vertex$ has no successors or predecessors and it is
not labeled by any variable.
%

\graphopsection{\bf Operations on vertices.}
%% Definition of adding a vertex
%% ----------------------------------------
%\graphopsection{\bf Vertex addition.}
%
\begin{graphop}[$\graph\addVertex{\vertex}$]
  For $\vertex\not\in\vertices$, we use $\graph\addVertex{\vertex}$ to
  denote the graph $\graph'=\tuple{\vertices',\succ',\labeling'}$ such
  that $\vertices'=\vertices\cup\set{\vertex}$, $\succ'=\succ$, and
  $\labeling'=\labeling$, i.e.\ we add a new vertex to $\graph$.
  Observe that the added vertex is then isolated.
\end{graphop}
%


%% Definition of deleting a vertex
%% ----------------------------------------
%\graphopsection{\bf Vertex deletion.}
%
\begin{graphop}[$\graph\delVertex{\vertex}$]
  For an {\it isolated} vertex $\vertex\in\vertices$, we use
  $\graph\delVertex{\vertex}$ to denote the graph
  $\graph'=\tuple{\vertices',\succ',\labeling'}$ such that
  $\vertices'=\vertices-\set{\vertex}$, $\succ'=\succ$, and
  $\labeling'=\labeling$.
\end{graphop}
%

\graphopsection{\bf Operations on variables.}
%% Definition of deleting a variable
%% ----------------------------------------
%\graphopsection{\bf Variable addition.}
%
\begin{graphop}[$\graph\addVar{\xvar}$]
  % If  $\labeling(\xvar)=\undef$,
  We define $\graph\addVar{\xvar}$ to be the set of graphs we get from
  $\graph$ by letting $\xvar$ point %to
  anywhere inside $\graph$.
  % 
  Formally, we define $\graph\addVar{\xvar}$ to be the smallest set
  containing each graph
  $\graph'$ %$\tuple{\vertices',\succ',\labeling'}$%
  such that one of the following conditions is satisfied:
  \ignore{%
  \begin{enumerate}
  \item \label{addVar1}%
    There is a $\vertex\not\in\vertices$ such that
    $\graph'=\gupdate{(\graph\addVertex{\vertex})}{\labeling}\updateby{\xvar}{\vertex}$,
    $\ie$ we add a vertex to $\graph$ and make $\xvar$ point to it.
    % 
  \item \label{addVar2}%
    There is a $\vertex\in\vertices$ such that
    $\graph'=\gupdate{\graph}{\labeling}\updateby{\xvar}{\vertex}$,
    % 
    $\ie$ we make \xvar\ point to some vertex %node
    in $\graph$.
    % 
  \item \label{addVar3}%
    \ignore{%
      \localnote{ahmed: modified from: There are
        $\vertex_1\in\vertices$, $\vertex_2\not\in\vertices$, and
        graphs $\graph_1,\graph_2,\graph_3,\graph_4$, such that
        $\succ(\vertex_1)\neq\undef$,
        $\graph_1=\graph\addVertex\vertex_2$,
        $\graph_2=\gupdate{\graph_1}{\succ}\updateby{\vertex_1}{\vertex_2}$,
        $\graph_3=\gupdate{\graph_2}{\succ}\updateby{\vertex_2}{\succ(\vertex_1)}$,
        and
        $\graph'=\gupdate{\graph_4}{\labeling}\updateby{\xvar}{\vertex_2}$.}
    } % end of ignore
    % 
    There are $\vertex_1\in\vertices$, $\vertex_2\not\in\vertices$,
    and graphs $\graph_i=(\vertices_i,\succ_i,\labeling_i)$ for
    $i=1,2,3$, such that $\succ(\vertex_1)\neq\undef$,
    $\graph_1=\graph\addVertex\vertex_2$,
    $\graph_2=\gupdate{\graph_1}{\succ}\updateby{\vertex_2}{\succ_1(\vertex_1)}$,
    $\graph_3=\gupdate{\graph_2}{\succ}\updateby{\vertex_1}{\vertex_2}$,
    and
    $\graph'=\gupdate{\graph_3}{\labeling}\updateby{\xvar}{\vertex_2}$,
    % 
    $\ie$ we add a new vertex %node
    $\vertex_2$ in between $\vertex_1$ and its successor and make
    \xvar\ point to $\vertex_2$.
    % 
  \end{enumerate}
  }%end of ignore
(i) \label{addVar1}%
    there is a $\vertex\not\in\vertices$ such that
    $\graph'=\gupdate{(\graph\addVertex{\vertex})}{\labeling}\updateby{\xvar}{\vertex}$,
    $\ie$ we add a vertex to $\graph$ and make $\xvar$ point to it.
    % 
(ii) \label{addVar2}%
    there is a $\vertex\in\vertices$ such that
    $\graph'=\gupdate{\graph}{\labeling}\updateby{\xvar}{\vertex}$,
    % 
    $\ie$ we make \xvar\ point to some vertex %node
    in $\graph$.
    % 
(iii) \label{addVar3}%
    \ignore{%
      \localnote{ahmed: modified from: There are
        $\vertex_1\in\vertices$, $\vertex_2\not\in\vertices$, and
        graphs $\graph_1,\graph_2,\graph_3,\graph_4$, such that
        $\succ(\vertex_1)\neq\undef$,
        $\graph_1=\graph\addVertex\vertex_2$,
        $\graph_2=\gupdate{\graph_1}{\succ}\updateby{\vertex_1}{\vertex_2}$,
        $\graph_3=\gupdate{\graph_2}{\succ}\updateby{\vertex_2}{\succ(\vertex_1)}$,
        and
        $\graph'=\gupdate{\graph_4}{\labeling}\updateby{\xvar}{\vertex_2}$.}
    } % end of ignore
    % 
    there are $\vertex_1\in\vertices$, $\vertex_2\not\in\vertices$,
    and graphs $\graph_i=(\vertices_i,\succ_i,\labeling_i)$ for
    $i=1,2,3$, such that $\succ(\vertex_1)\neq\undef$,
    $\graph_1=\graph\addVertex\vertex_2$,
    $\graph_2=\gupdate{\graph_1}{\succ}\updateby{\vertex_2}{\succ_1(\vertex_1)}$,
    $\graph_3=\gupdate{\graph_2}{\succ}\updateby{\vertex_1}{\vertex_2}$,
    and
    $\graph'=\gupdate{\graph_3}{\labeling}\updateby{\xvar}{\vertex_2}$,
    % 
    $\ie$ we add a new vertex %node
    $\vertex_2$ in between $\vertex_1$ and its successor and make
    \xvar\ point to $\vertex_2$.
    % 
\end{graphop}
%

%% Definition of add y at place =x
%% ----------------------------------------
\begin{graphop}[$\graph\addEqVar{\xvar}{\yvar}$]
  For variables \xvar\ and \yvar\ with $\labeling(\xvar)\neq\undef$, 
  we define $\graph\addEqVar{\xvar}{\yvar}$ to be the graph
  $\graph'=\gupdate{\graph}{\labeling}\updateby{\yvar}{\labeling(\xvar)}$,
  % 
  $\ie$ we make \yvar\ point to the same vertex %node
  as $\xvar$.
\end{graphop}
%
%% Definition of add y but not at place x
%% ----------------------------------------
%If $\labeling(\xvar)\neq\undef$
\begin{graphop}[$\graph\addNotEqVar{\xvar}{\yvar}$]
  Furthermore, we define $\graph\addNotEqVar{\xvar}{\yvar}$ to be the
  smallest set containing each
  graph %$\graph'=\tuple{\vertices',\succ',\labeling'}$
  $\graph'$ %=\tuple{\vertices,\succ,\labeling'}$%
  such that $\graph'\in(\graph\addVar{\yvar})$ and
  $\labeling'(\yvar)\neq\labeling'(\xvar)$, $\ie$ we make $\yvar$
  point anywhere inside $\graph$ except to the vertex %node
  pointed to by $\xvar$.
\end{graphop}
%

%% Definition of add y such that succ(x)=y
%% ----------------------------------------
\begin{graphop}[$\graph\addVarAsSuccOf{\xvar}{\yvar}$]
  For variables \xvar\ and \yvar\ with $\labeling(\xvar)\neq\undef$
  and $\succ(\labeling(\xvar))\nequals\undef$,
  % If $\labeling(\xvar)\nequals\undef$ and
  % $\labeling(\yvar)\equals\undef$, %
  % we define $\graph\addVarAsSuccOf{\xvar}{\yvar}$ to be the graph
  % $\gupdate{\graph}{\succ}\updateby{\labeling(\xvar)}{\labeling(\yvar)}$.
  we use $\graph\addVarAsSuccOf{\xvar}{\yvar}$ to denote the graph
  $\gupdate{\graph}{\labeling}\updateby{\yvar}{\succ(\labeling(\xvar))}$,
  % $\gupdate{\graph}{\succ}\updateby{\labeling(\xvar)}{\labeling(\yvar)}$.
  % 
  $\ie$ we make $\yvar$ point to the successor of the vertex %node
  pointed to by $\xvar$.
  % 
  % Observe that it eventually adds an arrow if there was none.
  % \localnote{ahmed: To do : add
  %   $\succ(\labeling(\xvar))\nequals\undef$, change
  %   $\gupdate{\graph}{\succ}\updateby{\labeling(\xvar)}{\labeling(\yvar)}$
  %   into
  %   $\gupdate{\graph}{\labeling}\updateby{\yvar}{\succ(\labeling(\xvar))}$,
  %   and remove"eventually adds an arrow".
  %   
  %   If the arrow that is to be added is the one between
  %   $\labeling(\xvar)$ and its not yet defined successor, then we
  %   need to concretize the successor at each possible position.}
\end{graphop}
%
%% Definition of add y such that succ(y)=x
%% ----------------------------------------
\begin{graphop}[$\graph\addVarAsPredOf{\xvar}{\yvar}$]
  For variables $\xvar$ and $\yvar$ with $\labeling(\xvar)\neq\undef$,
  we define $\graph\addVarAsPredOf{\xvar}{\yvar}$ to be the set of
  graphs we get from $\graph$ by letting $\yvar$ point to any
  vertex %node
  where the successor %of the node
  is either undefined or pointed to by $\xvar$.
  % 
  Formally, we define $\graph\addVarAsPredOf{\xvar}{\yvar}$ to be the
  smallest set containing each graph $\graph'$ such that one of the
  following conditions is satisfied:
  \ignore{%
  \begin{enumerate}
  \item \label{addVarAsPred1}%
    There is a $\vertex\not\in\vertices$ such that
    $\graph'=\gupdate{(\graph\addVertex\vertex)}{\labeling}\updateby{\yvar}{\vertex}$.
    % 
  \item \label{addVarAsPred2}%
    There is a $\vertex\in\vertices$ such that
    $\vertex\neq\labeling(\nil)$, either $\succ(\vertex)=\undef$ or
    $\succ(\vertex)=\labeling(\xvar)$, and
    $\graph'=\gupdate{\graph}{\labeling}\updateby{\yvar}{\vertex}$.
    % 
    That is, we place \yvar\ on the vertices without a successor
    or the ones whose successor is pointed to by \xvar.
    % 
  \item \label{addVarAsPred3}%
    There are $\vertex_1\in\vertices$, $\vertex_2\not\in\vertices$,
    and graphs $\graph_i=\left(\vertices_i,\suc_i,\labeling_i\right)$
    for $i=1,2,3$, % $\graph_1$, $\graph_2$, $\graph_3$ %
    such that $\succ(\vertex_1)=\labeling(\xvar)$,
    $\graph_1=\graph\addVertex\vertex_2$,
    $\graph_2=\gupdate{\graph_1}{\succ}\updateby{\vertex_2}{\labeling(\xvar)}$,
    $\graph_3=\gupdate{\graph_2}{\succ}\updateby{\vertex_1}{\vertex_2}$,
    and
    $\graph'=\gupdate{\graph_3}{\labeling_3}\updateby{\yvar}{\vertex_2}$.
    % 
    Intuitively, we add a new vertex %node
    $\vertex_2$ in between the vertex %node
    pointed by $\xvar$ and its predecessors and make \yvar\ point to
    $\vertex_2$.
    % 
  \end{enumerate}
  }% end of ignore
(i) \label{addVarAsPred1}%
    there is a $\vertex\not\in\vertices$ such that
    $\graph'=\gupdate{(\graph\addVertex\vertex)}{\labeling}\updateby{\yvar}{\vertex}$.
    % 
(ii) \label{addVarAsPred2}%
    there is a $\vertex\in\vertices$ such that
    $\vertex\neq\labeling(\nil)$, either $\succ(\vertex)=\undef$ or
    $\succ(\vertex)=\labeling(\xvar)$, and
    $\graph'=\gupdate{\graph}{\labeling}\updateby{\yvar}{\vertex}$.
    % 
    That is, we place \yvar\ on the vertices without a successor
    or the ones whose successor is pointed to by \xvar.
    % 
(iii) \label{addVarAsPred3}%
    there are $\vertex_1\in\vertices$, $\vertex_2\not\in\vertices$,
    and graphs $\graph_i=\left(\vertices_i,\suc_i,\labeling_i\right)$
    for $i=1,2,3$, % $\graph_1$, $\graph_2$, $\graph_3$ %
    such that $\succ(\vertex_1)=\labeling(\xvar)$,
    $\graph_1=\graph\addVertex\vertex_2$,
    $\graph_2=\gupdate{\graph_1}{\succ}\updateby{\vertex_2}{\labeling(\xvar)}$,
    $\graph_3=\gupdate{\graph_2}{\succ}\updateby{\vertex_1}{\vertex_2}$,
    and
    $\graph'=\gupdate{\graph_3}{\labeling_3}\updateby{\yvar}{\vertex_2}$.
    % 
    Intuitively, we add a new vertex %node
    $\vertex_2$ in between the vertex %node
    pointed by $\xvar$ and its predecessors and make \yvar\ point to
    $\vertex_2$.
    % 
\end{graphop}
%

%% Definition of deleting a variable
%% ----------------------------------------
%\graphopsection{\bf Variable deletion.}
%
\begin{graphop}[$\graph\delVar{\xvar}$]
  \ignore{%
    For a variable $\xvar$, we define $\graph\delVar{\xvar}$ to be the
    graph $\graph'$ where
    $\graph'=\gupdate{\graph}{\labeling}\updateby{\xvar}{\undef}$.%
  }
  % 
  For a variable $\xvar$, we use $\graph\delVar{\xvar}$ to denote
  $\gupdate{\graph}{\labeling}\updateby{\xvar}{\undef}$.
\end{graphop}
%

\graphopsection{\bf Operations on edges.}
%\graphopsection{\bf Edge addition.}
%% Definition of edge addition from x to y
%% ------------------------------------------
\begin{graphop}[$\graph\addEdgeBetween{\xvar}{\yvar}$]
  If $\labeling(\xvar)\neq\undef$, $\labeling(\yvar)\neq\undef$ and
  $\labeling(\xvar)\neq\labeling(\nil)$, we use
  $\graph\addEdgeBetween{\xvar}{\yvar}$ to denote
  %$\graph'= %
  $\gupdate{\graph}{\succ}\updateby{\labeling(\xvar)}{\labeling(\yvar)}$,
  % 
  $\ie$ we delete the edge between the vertex
  $\labeling(\xvar)$ and its successor (if any) and add %then 
  an edge
  from $\labeling(\xvar)$ to $\labeling(\yvar)$.
\end{graphop}
%

%% Definition of edge addition from x to anything
%% ----------------------------------------------
\begin{graphop}[$\graph\addEdge{\xvar}$]
  If $\labeling(\xvar)\nequals\undef$ and
  $\labeling(\xvar)\nequals\labeling(\nil)$, we define
  %% That the intuition
  $\graph\addEdge{\xvar}$ to be the set of graphs we
  get from $\graph$ by letting $\nxt\xvar$ point %to
  anywhere inside $\graph$.
  % 
  Formally, we define $\graph\addEdge{\xvar}$ to be the smallest set
  containing each graph $\graph'$ such that one of the following
  conditions is satisfied:
  \ignore{%
  \begin{enumerate}
  \item \label{addEdge1}%
    There is a $\vertex\not\in\vertices$ such that
    $\graph'=\gupdate{(\graph\addVertex{\vertex})}{\succ}\updateby{\labeling(\xvar)}{\vertex}$.
    % 
  \item \label{addEdge2}%
    There is a $\vertex\in\vertices$ such that
    $\graph'=\gupdate{\graph}{\succ}\updateby{\labeling(\xvar)}{\vertex}$.
    % 
  \item \label{addEdge3}%
    There are $\vertex_1\in\vertices$, $\vertex_2\not\in\vertices$,
    and graphs $\graph_i=\left(\vertices_i,\suc_i,\labeling_i\right)$
    for $i=1,2,3$, such that $\succ(\vertex_1)\neq\undef$,
    $\graph_1=\graph\addVertex\vertex_2$,
    $\graph_2=\gupdate{\graph_1}{\succ}\updateby{\vertex_2}{\succ_1(\vertex_1)}$,
    $\graph_3=\gupdate{\graph_2}{\succ}\updateby{\vertex_1}{\vertex_2}$,
    and
    $\graph'=\gupdate{\graph_3}{\succ}\updateby{\labeling_3(\xvar)}{\vertex_2}$.
    % 
  \end{enumerate}
  %
  } %end of ignore
(i) \label{addEdge1}%
    there is a $\vertex\not\in\vertices$ such that
    $\graph'=\gupdate{(\graph\addVertex{\vertex})}{\succ}\updateby{\labeling(\xvar)}{\vertex}$.
    % 
(ii) \label{addEdge2}%
    there is a $\vertex\in\vertices$ such that
    $\graph'=\gupdate{\graph}{\succ}\updateby{\labeling(\xvar)}{\vertex}$.
    % 
(iii) \label{addEdge3}%
    there are $\vertex_1\in\vertices$, $\vertex_2\not\in\vertices$,
    and graphs $\graph_i=\left(\vertices_i,\suc_i,\labeling_i\right)$
    for $i=1,2,3$, such that $\succ(\vertex_1)\neq\undef$,
    $\graph_1=\graph\addVertex\vertex_2$,
    $\graph_2=\gupdate{\graph_1}{\succ}\updateby{\vertex_2}{\succ_1(\vertex_1)}$,
    $\graph_3=\gupdate{\graph_2}{\succ}\updateby{\vertex_1}{\vertex_2}$,
    and
    $\graph'=\gupdate{\graph_3}{\succ}\updateby{\labeling_3(\xvar)}{\vertex_2}$.
    % 
\end{graphop}
%

%% Definition of edge deletion
%% ------------------------------------------
%\graphopsection{\bf Edge deletion.}
%
\begin{graphop}[$\graph\delEdge{\xvar}$]
  If $\labeling(\xvar)\neq\undef$, we denote $\graph\delEdge{\xvar}$
  as
  $\gupdate{\graph}{\succ}\updateby{\labeling(\xvar)}{\undef}$,
  % 
  $\ie$ we remove the edge from the vertex %node
  pointed to by $\xvar$ and its successor (if any).
  % 
  \forget{%
    If $\labeling(\xvar)\neq\undef$, $\labeling(\yvar)\neq\undef$ and
    $\labeling(\xvar)\nequals\labeling(\nil)$ then we define
    $\graph\addedge{\xvar}\yvar$ to be the graph
    $\graph'=\gupdate{\graph}{\succ}\updateby{\labeling(\xvar)}{\labeling(\yvar)}$.
    % 
    % $\graph'=\gupdate{\graph}{\labeling}
    % \updateby{\yvar}{\succ(\labeling(\xvar))}$\localnote{ahmed:
    %   changed from:
    %   $\graph'=\gupdate{\graph}{\succ}\updateby{\labeling(\xvar)}{\labeling(\yvar)}$}.
    % 
    In other words, we make the vertex pointed to by $\yvar$ the
    successor of the vertex pointed to by $\xvar$.
    % 
    % In other words, we make $\yvar$ point to the successor of the
    % vertex pointed to by $\xvar$\localnote{ahmed: changed from: In
    %   other words, we make the vertex pointed to by $\yvar$ the
    %   successor of the vertex pointed to by $\xvar$.}.
  }%
\end{graphop}
%
