%
\section{Ordering}
\label{section:orderings}
In this section, we introduce an ordering on configurations.
%
Based on the ordering, we will define the coverability problem
which we use to check safety properties, and
define the abstract transition relation.
%
The latter is an over-approximation of the concrete transition relation.

%\paragraph{\bf Ordering.} 
\smallskip\noindent{\bf Ordering.} 
Let $\graph=\tuple{\vertices,\suc,\labeling}$ and
$\graph'=\tuple{\vertices',\suc',\labeling'}$.
%
We write $\graph\jlt\graph'$ to
denote that one of the following properties is satisfied:
(i)
  {\it Variable Deletion}: %
  $\graph=\graph'\delVar{\xvar}$ for some variable $\xvar$,
  % 
(ii)
  {\it Vertex Deletion}: %
  $\graph=\graph'\delVertex{\vertex}$ for some isolated vertex
  $\vertex\in\vertices'$,
  % 
(iii)
  {\it Edge Deletion}: %
  $\graph=\gupdate{\graph'}{\suc}\updateby{\vertex}{\undef}$ for some
  $\vertex\in\vertices'$, or
  % 
(iv)
  {\it Contraction}: %
  there are vertices $\vertex_1,\vertex_2,\vertex_3\in\vertices'$ and
  graphs $\graph_1,\graph_2$ such that $\vertex_2$ is simple,
  $\suc'(\vertex_1)=\vertex_2$, $\suc'(\vertex_2)=\vertex_3$,
  $\graph_1=\gupdate{\graph'}{\suc}\updateby{\vertex_2}{\undef}$,
  $\graph_2=\gupdate{\graph_1}{\suc}\updateby{\vertex_1}{\vertex_3}$
  and $\graph=\graph_2\delVertex{\vertex_2}$.
  % 
  \ignore{%
  \begin{minipage}{0.3\textwidth}
    \begin{tikzpicture}[show background rectangle]
      \begin{scope}[scale=0.5]
        \node[xshift=-1cm,shape=circle,fill=white,draw=none,scale=0.7] (v1){$\vertex_1$};
        \node[shape=circle,fill=white,draw=none,scale=0.7] (v2){$\vertex_2$};
        \node[xshift=1cm,shape=circle,fill=white,draw=none,scale=0.7] (v3){$\vertex_3$};
        \node[draw=none,above=3mm] at (v2) {Before contraction};
      \end{scope}
      \begin{scope}[scale=0.5,yshift=-2.3cm]
        \node[xshift=-1cm,shape=circle,fill=white,draw=none,scale=0.7] (v4){$\vertex_1$};
        \node[draw=none] (v5){};
        \node[xshift=1cm,shape=circle,fill=white,draw=none,scale=0.7] (v6){$\vertex_3$};
        \node[draw=none,above=3mm] at (v5) {After contraction};
      \end{scope}
      \draw[->] (v1) -- (v2);
      \draw[->] (v2) -- (v3);
      \draw[->] (v4) -- (v6);
    \end{tikzpicture}
  }%

%
%
We write $\graph\lt\graph'$ to denote that there are
$\graph_0\jlt\graph_1\jlt\graph_2\jlt\cdots\jlt\graph_n$ with $n\geq
0$, $\graph_0=\graph$, and $\graph_n=\graph'$.
%
That is, we can obtain $\graph$ from $\graph'$ by performing a finite
sequence of variable deletion, vertex deletion, edge deletion and contraction operations.
%
For configurations $\conf=\tuple{\cstate,\graph}$ and
$\conf'=\tuple{\cstate',\graph'}$, we write $\conf\lt\conf'$ to
denote that $\cstate'=\cstate$ and $\graph\lt\graph'$.
%

For a configuration $\conf$, we use $\uc{\conf}$ to denote the %
{\it upward closure} of $\conf$, $\ie$
$\uc{\conf}=\setcomp{\conf'}{\conf\lt\conf'}$.
%
We use $\dc{\conf}$ to denote the
{\it downward closure} of $\conf$, $\ie$
$\dc{\conf}=\setcomp{\conf'}{\conf'\lt\conf}$.
%
For a set $\confs$ of configurations, we define $\uc{\confs}$ as
$\bigcup_{\conf\in\confs}\uc{\conf}$.
%
We define $\dc{\confs}$ analogously.
%For sets $\confs_1$ and $\confs_2$ of configurations, we write
%$\confs_1\lt\confs_2$ to denote that $\conf_1\lt\conf_2$ for some
%$\conf_1\in\confs_1$ and $\conf_2\in\confs_2$.

%\paragraph{\bf Safety Properties.}
\smallskip\noindent{\bf Safety Properties.}
In order to analyze safety properties, we
study the {\it coverability problem} defined below.
%

\probbox{Coverability}
{%
  \item Sets $\initconfs$ and $\finalconfs$ of configurations.
}{%
Is it the case $\initconfs\movesto{*}\uc{\finalconfs}$?
}
\newline
Intuitively, $\uc{\finalconfs}$ represents a set of ``bad'' states which we do not 
want to reach during the execution of the program.
\localnote{Fred: should we add ``starting from any configuration in $\initconfs$''}
%
This set is represented by a set $\finalconfs$ of minimal elements.
%
In Section~\ref{section:experiments}, we will describe how to encode
properties such as garbage generation, dereferencing and shape
violation as reachability of upward closed sets of configurations
represented by finite sets of minimal elements.
%
Therefore, checking safety with respect to these properties
amounts to solving the coverability problem.

%\paragraph{\bf Abstract Transition Relation.}
\smallskip\noindent{\bf Abstract Transition Relation.}
We write $\conf_1\absmovesto{\transition}\conf_2$ to denote that there is a $\conf_3$ such that
$\conf_3\lt\conf_1$ and $\conf_3\movesto{\transition}\conf_2$.
%
In other words, a step of the abstract transition relation consists of first moving to 
a smaller configuration (wrt $\lt$) and then performing a step of the concrete transition relation.
%
Notice that the abstraction corresponds to an over-approximation and therefore
any safety property which holds in the abstract system will also hold in the 
concrete one.\localnote{Fred:we can add a figure here}
