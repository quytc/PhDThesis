
\newcommand{\forget}[1]{}
\newcommand{\ignore}[1]{}

%% Having a note in the margin. Notes are numbered.
%% ---------------------------------------------
\newcommand{\NoteComment}[2]{%
  \stepcounter{NoteCounter#1}%
  {\scriptsize\bf$^{(\arabic{NoteCounter#1})}$}%
  \marginpar[\fbox{
    \parbox{1.4cm}{\raggedleft
      \scriptsize$^{({\bf{\arabic{NoteCounter#1}{#1}}})}$%
      \scriptsize #2}}]%
  {\fbox{\parbox{1.4cm}{\raggedright 
      \scriptsize$^{({\bf{\arabic{NoteCounter#1}{#1}}})}$%
      \scriptsize #2}}}
}
\newcounter{NoteCounter}
\newcommand{\Note}[1]{\NoteComment{}{#1}}
% \renewcommand{\Note}[1]{}
\newcommand{\parosh}[1]{\Note{Parosh: #1}}
\newcommand{\ahmed}[1]{\Note{Ahmed: #1}}
\newcommand{\daz}[1]{\Note{Fred: #1}}
\newcommand{\jonathan}[1]{\Note{Jonathan: #1}}

%% Local note as footnotes.
%% ---------------------------------------------
\newcommand{\localnote}[1]{\footnote{#1}}
\renewcommand{\localnote}[1]{}
%% Uncomment to remove the comment footnote

%% TikZ definitions
%% ---------------------------------------------
\colorlet{figbg}{gray!30!white}
\tikzstyle{background rectangle}=[draw=none,fill=figbg,rounded corners=8pt]

%% Color definitions
%% ---------------------------------------------
\newcommand{\red}[1]{\textcolor{red}{#1}}
\newcommand{\green}[1]{\textcolor{green}{#1}}
\newcommand{\blue}[1]{\textcolor{blue}{#1}}
\newcommand{\black}[1]{\textcolor{black}{#1}}

%% Other definitions
%% ---------------------------------------------
\newcommand{\ie}{i.e.\ }

\newcommand{\tuple}[1]{\left(#1\right)}
\newcommand{\set}[1]{\left\{#1\right\}}
\newcommand{\setcomp}[2]{\left\{#1|\,#2\right\}}
%\newcommand{\nat}{\mathbb{N}}
\newcommand{\vars}{\ensuremath{X}}
\newcommand{\xvars}{\ensuremath{X}}
\newcommand{\yvars}{\ensuremath{Y}}
\newcommand{\varsof}[1]{{\it Var}\!\left(#1\right)}
\newcommand{\var}{\ensuremath{x}}
\newcommand{\xvar}{\ensuremath{x}}
\newcommand{\yvar}{\ensuremath{y}}
\newcommand{\zvar}{\ensuremath{z}}
\newcommand{\tvar}{\ensuremath{t}}
\newcommand{\pointer}{{\it pointer}\;}
\newcommand{\assigned}{:=}
\newcommand{\equals}{=}
\newcommand{\nequals}{\neq}
\newcommand{\nil}{{\it null}}
\newcommand{\nxt}[1]{#1.{\it next}}
\newcommand{\emptystring}{\varepsilon}


\newcommand{\labeling}{\lambda}
\newcommand{\vertex}{\ensuremath{v}}
\newcommand{\vertices}{\ensuremath{V}}
\newcommand{\suc}{{\it succ}}
\renewcommand{\succ}{\suc}
\newcommand{\graph}{\ensuremath{g}}
\newcommand{\graphs}{\ensuremath{G}}
\newcommand{\vertexof}[1]{{\it vertex}\left(#1\right)}
\newcommand{\transition}{t}
\newcommand{\transitions}{T}
\newcommand{\cstates}{Q}
\newcommand{\cstate}{q}
\newcommand{\action}{a}
\newcommand{\prg}{P}
\newcommand{\conf}{c}
\newcommand{\confs}{C}
\newcommand{\initconfs}{\confs_{\it Init}}
\newcommand{\finalconfs}{\confs_F}
\newcommand{\prgtuple}{\tuple{\cstates,\transitions}}
\newcommand{\movesto}[1]{\stackrel{#1}{\longrightarrow}}
\newcommand{\pmovesto}[1]{\stackrel{#1}{\leadsto}}
\newcommand{\absmovesto}[1]{\stackrel{#1}{\longrightarrow}_A}
\newcommand{\updateby}[2]{\ensuremath{\left[#1\leftarrow#2\right]}}
\newcommand{\gupdate}[2]{\left(#1.#2\right)}
\newcommand{\undef}{\bot}
%\newcommand{\restrict}[1]{\backslash #1}
\newcommand{\jlt}{\lhd}
\newcommand{\jleq}{\unlhd}
\newcommand{\lt}{\preceq}
\newcommand{\sizeof}[1]{|#1|}
\newcommand{\uc}[1]{#1\!\!\uparrow}
\newcommand{\dc}[1]{#1\!\!\downarrow}
% \newcommand{\uc}[1]{
%   \begin{tikzpicture}[baseline=(n.base)]
%     \node[draw=none,inner xsep=2pt,inner ysep=1pt](n){\ensuremath{#1}};
%     \draw[line width=1pt]%
%     ([yshift=2pt]n.north west)--(n.north west)--(n.north east)--([yshift=2pt]n.north east);
% %     (n.west)-- %(n.north west)--
% %     ([yshift=1ex]n.north)-- %(n.north east)--
% %     (n.east);
%   \end{tikzpicture}
% }
% \newcommand{\dc}[1]{
%   \begin{tikzpicture}[baseline=(n.base)]
%     \node[draw=none,inner xsep=2pt,inner ysep=1pt](n){\ensuremath{#1}};
%     \draw[line width=1pt]%
%     ([yshift=-2pt]n.north west)--(n.north west)--(n.north east)--([yshift=-2pt]n.north east);
% %     (n.west)-- %(n.north west)--
% %     ([yshift=1ex]n.north)-- %(n.north east)--
% %     (n.east);
%   \end{tikzpicture}
% }
\newcommand{\pre}{{\it Pre}}
\newcommand{\rank}[2]{{\it Rank}(#1)(#2)}
\newcommand{\toexplore}{{\tt ToExplore}}
\newcommand{\explored}{{\tt Explored}}

%% =======================================

\newcommand{\addRound}{\ensuremath{\oplus}}
\newcommand{\delRound}{\ensuremath{\ominus}}
%\newcommand{\mybox}[1]{\ensuremath{\,{\tikz[baseline=(n.base)]\node(n)[draw,inner sep=1pt]{#1};}\,}}
%\newcommand{\addBox}{\ensuremath{\mybox{{\small +}}}}
%\newcommand{\delBox}{\ensuremath{\mybox{{\Large -}}}}
% \setlength{\fboxsep}{0pt}
\newcommand{\addBox}{\ensuremath{\boxplus}}
\newcommand{\delBox}{\ensuremath{\boxminus}}

\newcommand{\addVar}[1]{\ensuremath{\addRound{#1}}}
\newcommand{\delVar}[1]{\ensuremath{\delRound{#1}}}

\newcommand{\addVertex}[1]{\ensuremath{\addRound{#1}}}
\newcommand{\delVertex}[1]{\ensuremath{\delRound{#1}}}

\newcommand{\addEqVar}[2]{\ensuremath{\addRound_{={#1}}{#2}}}
\newcommand{\addNotEqVar}[1]{\ensuremath{\addRound_{\neq{#1}}}}

%\newcommand{\resaddvar}[1]{\odot_{{#1}}}
\newcommand{\addVarAsSuccOf}[2]{\ensuremath{\addRound_{{#1}\rightarrow}{#2}}}
\newcommand{\addVarAsPredOf}[2]{\ensuremath{\addRound_{{#1}\leftarrow}{#2}}}

\newcommand{\delEdge}[1]{\ensuremath{\delBox({#1}\rightarrow)}}
\newcommand{\addEdge}[1]{\ensuremath{\addBox({#1}\rightarrow)}}
\newcommand{\addEdgeBetween}[2]{\ensuremath{\addBox({#1}\rightarrow{#2})}}

%% Definition of the Graph Operation environment
%% ---------------------------------------------
\usepackage{ifthen}
%% ===== For the graph operation section =====
%\newcommand{\graphopsection}[1]{\paragraph{#1}}
\newcommand{\graphopsection}[1]{\noindent{\bf #1}}
%\newcommand{\graphopsection}[1]{\subsection{#1}}
%\newcommand{\graphopsection}[1]{}


%% ===== For the graph operation environment =====
% \newcommand{\GrOp}[1]{\marginpar[\hfill#1]{\hfill #1}}
% \newlength{\myparindent}
% \setlength{\myparindent}{0pt}% or {\parindent}
% \newboolean{noop}
% \newenvironment{graphop}[1][]{%
% \ifthenelse{\equal{#1}{}}{\setboolean{noop}{true}}{\setboolean{noop}{false}}
% %\setboolean{noop}{true}
% \ifthenelse{\boolean{noop}}{}{
% \hfill\\
% %\fbox{%
% \begin{minipage}[t]{0.15\textwidth}%
% {#1}%
% \end{minipage}%
% %}
% \begin{minipage}[t]{0.85\textwidth}%
% \setlength{\parindent}{\myparindent}%
% }% end of if
% }{%
% \ifthenelse{\boolean{noop}}{}{
% \end{minipage}\\%
% }% End of graph op
% }%
% \newenvironment{graphop}[1][]{%
% \ifthenelse{\equal{#1}{}}{\setboolean{noop}{true}}{\setboolean{noop}{false}}
% %\setboolean{noop}{true}
% \ifthenelse{\boolean{noop}}{}{\GrOp{#1}}%
% }{}% End of graph op
% %
%
%% Environment canceled
\newenvironment{graphop}[1][]{}{}%


\newcommand{\pintgrs}{{\mathbb N}^{>0}}
\newcommand{\encoding}{e}
\newcommand{\encodings}{E}
\newcommand{\cnt}{\#}
\newcommand{\sigof}[1]{{\it sig}\left(#1\right)}
\newcommand{\encodingof}[1]{{\it enc}\left(#1\right)}
\newcommand{\dist}{{\it dist}}
\newcommand{\Between}{{\it between}}
\newcommand{\degof}[1]{{\it deg}\left(#1\right)}
\newcommand{\isencodingorder}{\sqsubseteq^2}
\newcommand{\incencodingorder}{\sqsubseteq^1}
\newcommand{\encodingorder}{\sqsubseteq}
\newcommand{\similar}{\sim}
\newcommand{\isomorphic}{\sim}
\newcommand{\included}{\subseteq}
%\newcommand{\graphsubtract}{\setminus}
\newcommand{\zeroreduce}[1]{{#1}_{=0}}
\newcommand{\posreduce}[1]{{#1}_{>0}}


\newcommand{\probbox}[3]{%
  \vspace{3mm}
  \noindent
  \framebox{
    \begin{minipage}{0.95\hsize}
      \noindent\underline{#1}

      \noindent{\bf Instance}
      \vspace{-2mm}
      \begin{itemize}
        #2
      \end{itemize}
      \vspace{-2mm}
      \noindent{\bf Question}
      #3
    \end{minipage}
  }
  \vspace{2mm}
}


\renewcommand{\algname}[2]{\alg@margin}
