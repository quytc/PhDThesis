\section{Introduction}
\label{section:introduction} 

%\paragraph{Background.}
%
We address verification of safety properties
for {\it parameterized systems} that
consist of arbitrary numbers
of components (processes)  organized
according to a regular pattern.
%
%
The task is to perform \emph{parameterized verification}, i.e., to
verify correctness regardless of the number of processes.
%
%
This amounts to the verification of an infinite family;
namely one for each possible size of the system.
%
The term {\it parameterized} refers to the fact that the size
of the system is (implicitly) a parameter of the verification problem.
%
Parameterized systems arise naturally in the modeling of mutual
exclusion algorithms, bus protocols, distributed algorithms,
telecommunication protocols, and cache coherence protocols.
%
For instance, the specification of a mutual exclusion protocol may be 
parameterized by the number of processes that
participate in  a given session of the protocol.
%
In such a case, it is interesting to verify 
correctness regardless of the number of participants in a particular session.
%
As usual, the verification of safety properties can be
reduced
to the problem of checking the reachability of a set of {\it bad
configurations} (the set of configurations that violate the safety property).

\paragraph{Existing approaches.}
%
An important approach to parameterized verification has been
{\it regular model checking}
\cite{KMMPS2001,AJNO:simple,BLW03} in which
regular languages are used
as symbolic representations of infinite sets of 
system configurations, and automata-based techniques are employed to implement the verification
procedure.
%
The main problem with such techniques is that  they
are heavy since they usually rely on several layers of computationally
expensive automata-theoretic constructions, in many cases
leading to a state space explosion that
severely limits their applicability.
%
Another class of methods analyze {\it approximated} system
behavior through the use of abstraction techniques.
%
Such methods include  {\it counter abstraction}
\cite{GerSis:many,PXZ02},
 {\em invisible invariant} generation~\cite{APRXZ01,PRZ-tacas01},
{\em environment abstraction}~\cite{CTV06}, and
 {\it monotonic  abstraction}~\cite{rmc:wo:transducers}
(see Section~\ref{section:related_work}).


In a similar manner to \cite{cutoff}, this work is inspired by a
strong empirical evidence that parameterized systems often enjoy a
{\it small model property}.
%
More precisely, analyzing only a small number of processes (rather
than the whole family) is sufficient to capture the reachability of
bad configurations.
%
%There are two reasons for that.
%
On the one hand, bad configurations can often be characterized by
minimal conditions that are possible to specify through a fixed number
of {\it witness} processes.
%
For instance, in a mutual exclusion protocol, a bad configuration
contains {\it two} processes in their critical sections; and in a
cache coherence protocol, a bad configuration contains {\it two} cache
lines in their {\it exclusive} states.
%
In both cases, having the two witnesses is sufficient
to make the configuration bad 
(regardless of the actual size of the configuration).
%
On the other hand, it is usually the case that such bad patterns (if
existing) appear already in small instances of the system, as observed
in our experimental section.%~\ref{section:experiments}.

\paragraph{Our approach.}
%
We introduce a method that exploits the small model property, and
performs parameterized verification by only inspecting a small set of
{\it fixed} instances of the system.
%
% The main advantage of the method is its surprising simplicity which
% makes it efficient, extensible, and easy to understand.  It only
% performs straightforward manipulations with a finite number of
% configurations of the system that do not involve any complicated
% symbolic representation or intricate algorithms.
%
Furthermore, the instances that need to be considered are 
often small in size
(typically three or four processes) which allows for a very
efficient verification procedure.
%
The framework can be applied uniformly
to generate {\it fully automatic} verification algorithms
for wide classes of parameterized systems including ones that
operate on linear, ring, or tree-like topologies, or systems
that contain unbounded collections of anonymous processes
 (the latter class is henceforth referred to as having~a~{\it
multiset}~topology).
%
% The main characteristic feature of this method is its surprising
% simplicity which makes it efficient, extensible, and also easy to
% understand and implement.

At the heart of the method is an operation that allows to detect {\it
  cut-off} points beyond which the verification procedure need not
continue.
%
Intuitively, reaching a cut-off point means that we need not inspect % ...search, explore...
larger instances of the system: the information collected so far
during the exploration of the state space allows us to conclude safely
that no bad configurations will occur in the larger instances.
%
The cut-off analysis is executed {\it dynamically} in the sense that
it is performed on-the-fly during the verification procedure itself.
%
It is based on an abstraction function, called {\it view abstraction},
parameterized by a constant $k$, and it approximates a
configuration %$\conf$
by the set of all its projections containing at most
$k$ processes. We call the sub-configurations {\it views}.
%
For instance, when a configuration is a word of process states
(represented as an array of processes), its abstraction is the set of
all its subwords of length at most $k$.
%
Furthermore, for a given set of views $X$, its concretization, denoted
as $\gammaof k (X)$, is the set of configurations (of any size) for which {\it all}
their views belong to $X$.

The verification method performs two search procedures in parallel.
%
The first performs a standard (explicit-state) forward reachability
analysis trying to find a bad configuration among system
configurations of size $k$ (for some natural number $k$).
%
If a bad configuration is encountered then the system is not safe.
%
The second procedure performs a {\it symbolic} forward reachability
analysis in the abstract domain of sets of views of size at most $k$.
%
%
When the computation terminates, it will have collected an 
over-approximation of all views of size up to $k$ of all reachable configurations (of all sizes).
%
If there is no bad configuration in the concretization of this set, 
then a cut-off point has been found and the system can be claimed safe.
%
If neither of the parallel procedures reaches a conclusion during iteration
$k$, the value of $k$ is increased by one (thus increasing
the precision of the abstraction).
%
Notice that the abstract search requires computing the \emph{abstract post-image} of a set $X$ of 
views of size at most $k$, which is the set $X'$ of views (of size at most $k$) of successors of $\gammaof k (X)$.
%
Obviously, this cannot be performed straightforwardly since
the set of configurations $\gammaof k (X)$ is infinite.
%
A crucial contribution of the paper is to show that,
for all the classes of parameterized systems that we consider,
it is sufficient to only compute successors of configurations from 
$\gammaof k (X)$ that are of the size at most $k+\ell$, where $\ell$ is a small constant, typically 1.
%
Intuitively, the reason is that the precondition for firing a
transition is the presence of a {\it bounded } number of processes in
certain states.
%
The views need only to encompass these processes in order to determine
the successor view.
%
This property is satisfied by a wide class of concurrent systems including the
ones we consider in this paper.
%
For instance, in rendez-vous communication between a pair of
processes, the transition is conditioned by the states of {\it two}
processes; in broadcast communication, {\it one} process initiates the
transition (while the other processes may be in any state); in
existential global transitions (see below), we need {\it two}
processes, namely the witness and the process performing the
transition; in Petri nets, the number of required processes is bounded
by the in-degree of the transitions (which is fixed for a given Petri
net), etc.
%
We will show formally that this property is satisfied by all the types
of transitions we consider.

\paragraph{Applications.}
%
We have instantiated the method to obtain automatic verification
procedures for four classes of parameterized systems, namely
systems where the processes are organized as arrays, 
rings, trees, or multisets.
%
Each instantiation is straightforward and is achieved by defining
the manner in which we define the views of a configuration.
%
More precisely, these views are (naturally)  defined as
subwords, cyclic subwords, subtrees, resp.\ subsets for the
above four classes.
%
Once the views are fixed we obtain a fully 
automatic procedure for all parameterized systems in the class.
%
In the systems we consider, we allow a rich set of features, 
in which processes may perform local transitions, rendez-vous,
broadcasts, and universally or existentially guarded transitions.
%
In a universally guarded transition, the process checks whether the
states of {\it all} other processes inside the system satisfy a given
constraint before it performs the transition.
%
In an existentially quantified transition, the processes
checks that there is {\it at least one} other process
 satisfying the condition.
%
Furthermore, we allow dynamic behaviors such as
the creation and deletion of processes during 
the execution of the system.


In the basic variant of our method, we assume that existential and
universal global conditions of transitions are checked atomically.
%
The same assumption is made in many landmark works on parameterized
systems
(e.g. \cite{CTV06,PRZ-tacas01,BHV04,AJNO:simple,APRXZ01,Namjoshi:VMCAI07,rmc:wo:transducers}).
%
However, actual implementations of global checks are usually not
atomic.
%
They are typically implemented as for-loops ranging over indices of
processes.
%
Iterations of such a loop may be interleaved with transitions of other
processes, therefore modeling the loop as an atomic transition means
under-approximating the behavior of the system.
%
Verification of systems with non-atomic global checks is significantly
harder.
%
It requires to distinguish intermediate states of a for-loop performed
by a process. Their number is proportional to the number of processes
in the system.
%
Moreover, any number of processes may be performing a for-loop at the
same time.
%
% This makes even number of states of individual process unbounded since
% their sizes are now parameterized by the number of processes in the
% system.
%
As we will show, our method can be easily adapted to this setting,
while retaining its simplicity and efficiency.


\paragraph{Implementation.}
%
%% I changed the order of the 2 following paragraphs (cuz it mentionned "the tool"). /Fred
%
We have implemented a prototype based on the method and run
it on a wide class of benchmarks, including 
mutual exclusion protocols on arrays
 (e.g., Burns', Szymanski's, and Dijkstra's protocols),
cache coherent protocols (e.g., MOSI and German's protocol),
different protocols on tree-like architectures
(e.g. percolate, arbiter, and leader election),
ring protocols (token passing), % and leader election), 
and different Petri nets.
%

The class of systems we consider have undecidable
reachability properties, and hence our method
is necessarily incomplete
(the verification procedure is not guaranteed to terminate in case
the safety property is satisfied).
%
However, as shown by our experimentation, the tool
terminates efficiently on all the tested benchmarks.

\paragraph{Completeness.}
Although the method is not complete in general, we show that is
complete for a large class of systems, namely those
that induce
{\it well quasi-ordered} transition systems \cite{Parosh:Bengt:Karlis:Tsay:general,abdulla:well} and satisfy certain additional technical requirements.
%
This implies that our method is complete for e.g., Petri nets.
%lossy channel systems \cite{AbJo:lossy}, and systems that
%are generated by monotonic abstraction \cite{rmc:wo:transducers}.
%
Notice that, as evident from our experiments, 
the method can in practice handle even systems that are outside the class.


\paragraph{Outline.}
To simplify the presentation, we instantiate our framework in a
step-wise manner.
%
In Section~\ref{section:parameterized_systems}, we introduce our model
for parameterized systems operating on linear topologies and describe
our verification method in Section~\ref{section:verification_method}.
%
In Section~\ref{section:extensions}, we describe how the framework can
be extended to incorporate other kinds of transitions such as
broadcast, rendez-vous, dynamic process deletion/creation, and
non-atomic checks of global conditions; and to cover other classes of
topologies such as ring, multiset, and tree-like structures.
%
The completeness of our method for well quasi-ordered systems is shown
in Section~\ref{section:completeness}.
%
We report on our experimental results in 
Section~\ref{section:experiments}, and 
describe related work in Section~\ref{section:related_work}.
%
Finally, we give some conclusions and directions for future
research in Section~\ref{section:conclusion}.
%
