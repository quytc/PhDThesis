%
%--------------------------------------------------------
\section{Extensions}
\label{section:extensions}
%--------------------------------------------------------
In this section, we describe how to extend the class of parameterized
systems that we presented in
Section~\ref{section:parameterized_systems}.
%
The extensions are obtained 
1) by extending the types of transition rules that we allow, 
2) by replacing transitions with atomically checked global conditions by more realistic for-loops, and
3) by considering topologies other than the linear ones. 
% 
As we shall see below, the extensions can be handled by our method
with straightforward extensions of the method of
Section~\ref{section:verification_method}.


%--------------------------------------------------------
\subsection{More Communication Mechanisms}
%--------------------------------------------------------

%--------------------------------------------------------
\paragraph{Broadcast.}
\label{section:broadcast}
%--------------------------------------------------------
In a~broadcast transition, an arbitrary number of processes change
states simultaneously. A broadcast rule is a~pair 
$(\loc\trans\loc',\{\locr_1\trans\locr_1',\ldots,\locr_m\trans\locr_m'\})$.
%
It is deterministic in the sense that 
$\locr_i\neq\locr_j$ for $i\neq j$.
%
The broadcast is initiated by a~process, 
called the \emph{initiator}, 
which triggers the transition 
rule $\loc\trans\loc'$. 
%
Together with the initiator, an arbitrary number of processes, 
called
the \emph{receptors}, change state simultaneously.
%
More precisely, if the local
state of a~process is $\locr_i$, then the process changes
its local state to $\locr_i'$. 
%
Processes whose local states are different from
$\loc,\locr_1,\ldots\locr_m$ remain passive during the broadcast.
%
Formally, the broadcast rule induces transitions $\conf\trans\conf'$
of $\system$ where for some $i:1\leq i\leq |\conf|$, $\conf[i]=\loc$,
$\conf'[i]=\loc'$, and for each $j:1\leq j\neq i \leq |\conf|$, if
$\conf[j] = \locr_k$ (for some $k$) then $\conf'[j] = \locr_k'$,
otherwise $\conf[j]=\conf'[j]$.
%

In a similar manner to globally guarded transitions, broadcast
transitions have small preconditions.  Namely, to fire a~transition,
it is enough that an initiator is present in the transition.
%
More precisely, for parameterized systems with local, global, and
broadcast transitions, Lemma~\ref{lemma:apost} still holds (in the
proof of Lemma~\ref{lemma:apost}, the initiator is treated analogously
to a~witness of an existential transition).
%
Therefore, the verification method from
Section~\ref{section:verification_method} can be used without any
change.


%--------------------------------------------------------
\paragraph{Rendez-vous.}
%--------------------------------------------------------

In rendez-vous, multiple processes change their states simultaneously.
A \emph{simple rendez-vous} transition rule is a~tuple of local rules
$\delta = (\locr_1\trans\locr_1',\ldots,\locr_m\trans\locr_m'),m>1$.
%
Multiple occurrences of local rules with the same source state $\locr$
in the tuple does not mean non-determinism, but that the rendez-vous
requires multiple occurrences of $\locr$ in the configuration to be
triggered.
%
Formally, the rule induces transitions $\conf\trans\conf'$ of
$\system$ such that there are ${i_1},\ldots,{i_m}$ with $i_j\neq i_k$
for all $j\neq k$, such that $\conf[i_1]\cdots\conf[i_m] =
\locr_1\cdots\locr_m$, $\conf'[i_1]\cdots\conf'[i_m] =
\locr_1'\cdots\locr_m' $, and $\conf'[\ell]=\conf[\ell]$ if
$\ell\not\in\{{i_1},\ldots,{i_m}\}$.

%In rendez-vous, multiple processes change their states simultaneously.
%A rendez-vous transition rule is a~tuple of local rules
%$\delta = (\locr_1\trans\locr_1',\ldots,\locr_m\trans\locr_m'),m>1$.
%%
%Multiple occurrences of local rules with the same source state $\locr$
%in the tuple does not mean non-determinism, but that the rendez-vous
%requires multiple occurrences of $\locr$ in the configuration to be
%triggered.
%
%Additionaly, in order to model creation and deletion of processes
%and also Petri net transitions that change numbers of tokens in the net, 
%we allow $\delta$ contain two types of special transitions. 
%These are rules of the form $\bullet\trans r$, $\bullet\not\in\locs$, used to model creation of a process in a state $r$,
%and $r\trans\bullet$, used to model deletion of a process in a state $r$.
%$\bullet$ is a special symbol s.t. $\bullet\not\in\locs$.

%
%Formally, the rule induces transitions $\conf\trans\conf'$
%where $\conf'$ is generated from $\conf$ as follows.
%First, we create an intermediate configuration $\conf^\bullet\in(\locs\cup\{\bullet\})^*$
%such that there are ${i_1},\ldots,{i_m}$ with $i_j\neq i_k$
%for all $j\neq k$, such that $\conf[i_1]\cdots\conf[i_m] =
%\locr_1\cdots\locr_m$, $\conf'[i_1]\cdots\conf'[i_m] =
%\locr_1'\cdots\locr_m' $, and $\conf'[\ell]=\conf[\ell]$ if
%$\ell\not\in\{{i_1},\ldots,{i_m}\}$.

%\newcommand\placeholder\bot
\newcommand\placeholder\bullet

Additionally, we define a \emph{generalized rendez-vous (or just
  rendez-vous) transition rules} in order to model \emph{creation and
  deletion} of processes and also Petri net transitions that change
the number of tokens in the net. 
%
A generalized rendez-vous rule $\delta$ is as a simple rendez-vous
rule, but it can in addition to the local rules contain two types of
special rules: of the form $\placeholder\trans
r,\placeholder\not\in\locs$ (acting as a placeholder), which are used
to model creation of processes, and of the form $r\trans
\placeholder$, which are used to model deletion of processes.
%
When a generalized rendez-vous rule is fired, the starting
configuration is first enriched with $\placeholder$ symbols in order
to facilitate creation of processes by the rule $\placeholder\trans
r$, then the rule is applied as if it was a simple rendez-vous rule,
treating $\placeholder$ as a normal state (states of the processes
that are to be deleted are rewritten to $\placeholder$ by the rules
$r\trans\placeholder$).  Finally, all occurrences of $\placeholder$
are removed.
%
Formally, a generalized rendez-vous rule induces a transition
$\conf\trans\conf'$ if there is
$\conf_\placeholder\in\{\placeholder\}^*\conf[1]\{\placeholder\}^*\cdots\{\placeholder\}^*\conf[|\conf|]\{\placeholder\}^*$
such that $\conf_\placeholder\trans\conf_\placeholder'$ is a
transition of the system with states $Q\cup\{\placeholder\}$ induced
by $\delta$ (treated as a simple rendez-vous rule), and $\conf'$
arises from $\conf_\placeholder'$ by erasing all occurrences of
$\placeholder$.

Rendez-vous transitions have small preconditions, but unlike
existentially quantified transitions, firing a transition may require
presence of more than two (but still a fixed number) processes in
certain states (the number is the arity of the transition).
%
It essentially corresponds to requiring the presence of more than one
witness.
%
This is why Lemma~\ref{lemma:apost} holds here only in the weaker variant:

\begin{lemma}
\label{lemma:rendez-vous}
Let $\rules$ contain rules of any previously described type (i.e.,
local, global, broadcast, rendez-vous), and let $m+1$ is the largest
arity of a~rendez-vous rule in $\rules$.  Then, for any $k$ and
$\X\subseteq\viewsof k$, $\projof {\postof {\gammaof k (\X)}} k\,
\cup\, V\; =\; \projof {\postof {\limgammaof k {k+m} (\X)}} k\,
\cup\, V$.
\end{lemma}



%--------------------------------------------------------
\paragraph{Global variables.}
%--------------------------------------------------------
Communication via shared variables is modeled using a~special
process, called \emph{controller}. Its local state records the state
of all shared variables in the system.  A~configuration of a~system
with global variables is then a~word $s_1\ldots s_nc$ where
$s_1,\ldots,s_n$ are the states of individual processes and $c$ is the state
of the controller. An individual process can read and update a shared
variable.
%
A~read is modeled by a~rendez-vous rule of the form $(s\trans
s',c\trans c)$ where $c$ is a state of the controller and $s,s'$ are
states of the process. An update is modeled using a~rendez-vous rule
$(s\trans s',c\trans c')$.

To verify systems with shared variables of finite domains, we use
a~variant of the abstraction function which always keeps the state of
the controller in the view.
%
Formally, for a~configuration $wc$ where $w\subseteq\locs^+$ and $c$
is the state of the controller, $\alpha_k$ returns the set of words $vc$
where $v$ is a~subword of $w$ of length at most $k$. The concretization and
abstract-post image are then defined analogously as before, based on
$\alpha_k$, Lemma~\ref{lemma:apost} and Lemma~\ref{lemma:invariant}
still hold. The method of Section~\ref{section:method} can be thus
used in the same way as before.

%Broadcast communication may be used to model global variables. 
%Communication via a~shared variable can be modelled in our framework using already described mechanisms.
%A state of a~global variable with a~\emph{finite domain} is recorded in
%a~local state of every process in the system. An assignment to such
%variable is modelled by a~broadcast transition which updates the value of
%local copy of each process.

Another type of global variable is a~\emph{process pointer}, i.e., a
variable ranging over process indices. This is used, e.g., in
Dijkstra's mutual exclusion
protocol. %(see Appendix~\ref{app:dijkstra}).
A process pointer is modeled by a~local Boolean flag $p$ for each
process state.  The value of $p$ is $\strue$ iff the pointer points to
the process (it is true for precisely one process in every
configuration). An update of the pointer is modeled by a~rendez-vous
transition rule which sets to $\sfalse$ the flag of the process
currently pointed to by the pointer and %simultaneously
sets to $\strue$ the flag of the process which is to become the target
of the pointer.

\subsection{Transitions that do not Preserve Size}
\label{section:nonsizepreserving}
We now discuss the case when the transition relation does not preserve size of configurations, which happens in the case of generalised rendez-vous.
%We have to cope with the issue that computation of $\reachof k$ becomes problematic. 
$\reachof k$ then cannot be computed straightforwardly since
computations reaching configurations of the size up to $k$ may traverse configurations of larger sizes.
%Notice that computing $\reachof k$ on line~\ref{line:exact} of Algorithm~\ref{algorithm:basic} is straightforward only because 
%(i) the number of configurations is of the size $k$ is finite and 
%(ii) the transition relation does not change the size of configurations.
Therefore, similarly as in \cite{Raskin:Viagra}, we only consider runs of the system visiting configurations of the size up to $k$.
That is, on line~\ref{line:exact} of Algorithm~\ref{algorithm:basic}, instead of computing
$\reachof k = \mu X.\,\inits_k\cup\post(X)$,
we compute its underapproximation 
$\mu X.\,(\inits\cup\post(X))\cap\confsof k$.
%, the set of configurations that can be reached traversing configurations of size at most $k$.
The computation terminates provided that $\confsof k$ is finite.
The algorithm is still guaranteed to return Unsafe if a configuration in $\bad$ is reachable, 
since then there is $k\in\nat$ such that the bad configuration is reachable by a finite path traversing configurations of the size at most $k$.



%When the pointer changes, this
%information must be passed onto all other processes, which we model as
%a~broadcast transition. Concretely, upon a~pointer assignment, the
%current process sets its local variable $p$ to $\strue$ and
%simultaneously sets $p$ to $\sfalse$ in all other processes.

%%--------------------------------------------------------
%\paragraph{Dynamic creation and deletion of processes.}
%%--------------------------------------------------------
%We cannot model dynamic creation and deletion of processes
%in a straightforward manner since the proof of Lemma~\ref{lemma:invariant}
%requires the transition relation to preserve the length of
%configurations.
%%
%However, we can overcome this issue by using the length-preserving
%rules described above, together with a~simple mechanism of
%placeholders. A placeholder is a~process in a~dummy state
%$\bullet$. It holds a~place for a~process that can be possibly
%created.  That is, the placeholder can take a~local transition
%$\bullet\trans\loc$ to an initial state $\loc$ and become a~regular
%process.  Conversely, a~process that is to be deleted in some local
%state $\loc$ takes the transition $\loc\trans\bullet$.  An initial
%configuration from $\inits$ may contain any number of
%placeholders.
%% In every run, the number of place holders is bounded by the size of
%% configuration.
%Every finite run of a~system creates only a~finite number $n$ of
%processes, therefore it can be modeled by running the system from an
%initial configuration with $n$ placeholders.

%--------------------------------------------------------
\subsection{Non-atomic Global Conditions} \label{section:non-atomic}
%--------------------------------------------------------
We extend our method to handle systems where global conditions are not checked atomically.
%
We replace both existentially and universally guarded transition rules by a~simple variant of a~for-loop rule:
$$%
\forrule q r \circ S s%
$$
where $q,r,s\in Q$ is resp.\ a~source state, a~target state, and an
escape state, $\circ\in\{<,>,\neq\}$, and $S\subseteq Q$ is a
\emph{condition}.
%
For instance, line 2 of Burns' protocol would be replaced by
% $
% \KwSty{if forall}\, j < i: \mathit{flag}[j] = 0 \ \KwSty{then\ goto}\ 3\ \KwSty{else\ goto}\ 1.
% $
$
\forrule 2 3 < {\{1,2,3\}} 1
$.


The semantics of a~system with for-loop rules is defined as an
extension of the transition system from
Section~\ref{section:parameterized_systems}.  Configurations are
extended with a~binary relation over their positions, that is,
a~configuration is now a~pair $(\conf,\tick)$ where $\conf$ is a~word
over $Q$ and $\tick$ is a~binary relation over its positions
$\{1,\ldots,|\conf|\}$.
%
The relation $\tick$ is used to encode intermediate states of
for-loops. Intuitively, a~process at position $i$ performing
a~for-loop puts $(i,j)$ into $\tick$ to mark that it has processed the
position $j$.

Formally, a~parameterized system $\parsys=(\locs,\rules)$ which
includes for-loop rules induces a~transition system $\transys =
(\confs,\trans)$ where $C \subseteq Q^+\times (\nat\times\nat)$.  For
technical convenience, we assume that a~source of a~for-loop rule in
$\rules$ is not a~source of any other rule in $\rules$.%
%
\footnote{Without this restriction, the state of a~process would have
  to contain additional information recording which for-loop is the
  process currently performing.  Note that the restriction does not
  limit the modeling power of the formalism.  Any potential branching
  may be moved to predecessors of the sources of the for-loop.}%
%
Then every for-loop rule $ \forrule q r \circ S s $ induces
transitions $t = (w,\tick) \trans (w',\tick')$ with $w[i] = q$ for
some $i:1\leq i \leq |w|$ which may be of the following three forms:
(illustrated using the aforementioned example rule from Burn's
protocol).

\tikzset{looplabel/.style={scale=0.7,draw=gray!10,fill=white,rounded corners=2pt,anchor=center}}
%\begin{description}
%\item[{Iteration:}]
\noindent\begin{minipage}[!t]{0.7\linewidth}%\vspace{0pt}
  %\vspace{-6mm}%
  {\bf Iteration:} The $i$th process checks that the state of a~next
  unchecked process in the range is in $S$ and marks it.  That is,
  there is $j:1\leq j \leq |w|$ with $j\circ i$, $(i,j)\not\in\tick$,
  $w[j]\in S$, and the resulting configuration has $w'=w$ and
  $\tick'=\tick\cup\{(i,j)\}$.
\end{minipage}
\hfill%
\begin{minipage}[!t]{0.29\linewidth}%\vspace{0pt}
  %\vspace{6mm}
  \input{Ticks.iteration.tex}
\end{minipage}

%\vspace{-4mm}% Cheating
%\item[{Escape:}]
\noindent\begin{minipage}[!t]{0.7\linewidth}%\vspace{0pt}
  {\bf Escape:} If the state of some process in the range which is
  still to be checked violates the loop condition, then the $i$th
  process may escape to the state $s$.  That is, there is $j:1\leq j
  \leq |w|$ with $j\circ i$, $(i,j)\not\in\tick$, and $w[j]\not\in S$.
  The resulting configuration has $w'[k]=w[k]$ for all $k\neq i$ and
  $w[i] = s$. The execution of the for-loop ends and the marks of
  process $i$ are reset, i.e., $\tick' = \tick\setminus\{(i,k)\mid
  k\in \nat\}$.
\end{minipage}
\hfill%
\begin{minipage}[!t]{0.29\linewidth}%\vspace{0pt}
  %\vspace{-6mm}
  \input{Ticks.escape.tex}
\end{minipage}

%\item[{Terminal:}]
\noindent\begin{minipage}[!t]{0.7\linewidth}%\vspace{0pt}
  {\bf Terminal:} When the states of all processes from the range have
  been successfully checked, the for-loop ends and the $i$th process
  moves to the terminal state $r$.  That is, if there is no $j:1\leq j
  \leq |w|$ with $j\circ i$ and $(i,j)\not\in\tick$, then $w'[k]=w[k]$
  for all $k\neq i$, $w'[i] = r$, and $\tick' =
  \tick\setminus\{(i,k)\mid k\in \nat\}$.
%marks of the $i$th process are reset. 
\end{minipage}
\hfill%
\begin{minipage}[!t]{0.29\linewidth}%\vspace{0pt}
  %\vspace{6mm}
  \input{Ticks.terminal.tex}
\end{minipage}
%\end{description}

Other rules behave as before on the $w$ part of configurations and they do not influence the $\tick$ part.
That is, a~local, broadcast, or rendez-vous rule induces transitions $(w,\tick) \trans (w',\tick)$ where 
$w\trans w'$ is a~transition induced by the rule as described in Section~\ref{section:parameterized_systems}.  


\paragraph{Verification.}
To verify systems with for-loop rules using our method, we define an abstraction $\project_k$.
Intuitively, we view a~configuration $c = (w,\tick)$ as a~graph with vertices being the positions of $w$ and edges being defined by (i) the ordering of the positions and (ii) the relation $\tick$. The vertices are labeled by the states of processes at the positions. $\alpha_k(c)$ then returns the set of subgraphs of $c$ where every subgraph contains a~subset of at most $k$ vertices of $c$ (positions of $w$) and the maximal subset of edges of $c$ adjacent with the chosen vertices.

Formally, given a~configuration $\conf = (w,\tick)$, 
$\projof \conf k$ is the set of views $\view = (w',\tick')\in\confs$ 
of size at most $k$ (i.e., $|w'| = l \leq k$) 
such that there exists an injection $\rho:\{1,\ldots l\}\rightarrow\{1,\ldots,|w|\},l\leq k$ where
for all $i,j:1\leq i,j \leq l$:
\begin{itemize}
\item
$i<j$ iff $\rho(i)<\rho(j)$,
\item
$w'[i]=w[\rho(i)]$ (i.e., $w'\subword w$), and
\item
$(i,j)\in\tick'$ iff $(\rho(i),\rho(j))\in\tick$.
\end{itemize}
%
The notions of concretization and abstract post-image are defined in
the same manner as in Section~\ref{section:verification_method} based on
based on $\alpha$.
%
Lemma~\ref{lemma:apost} holds here in the same wording (as shown in
the appendix).
%
Thus the verification method for systems with for-loops is analogous
to the method of Section~\ref{section:verification_method}.



%--------------------------------------------------------
\subsection{Tree Topology}
\label{section:trees}
%--------------------------------------------------------
We extend our method to systems where configurations are trees.  For
simplicity, we restrict ourselves to complete binary trees.
%
\paragraph{Trees.}
Let $\nodes$ be a~prefix closed set of words over the alphabet
$\{0,1\}$ called \emph{nodes} and let $\locs$ be a~finite set.
%
A (binary) \emph{tree} over $\locs$ is a~mapping
$\tree:\nodes\rightarrow\locs$.
%
The node $\epsilon$ is called the \emph{root}, nodes that are not
prefixes of other nodes are called \emph{leaves}.
%
For a~node $\node = \node'i,i\in\{0,1\}$, $\node'$ is the
\emph{parent} of $\node$, the node $\node 0$ is the \emph{left child}
of $\node$ and $\node 1$ is its \emph{right child}.
%
Every node $\node' =\node w,w\in\{0,1\}^+$ is a~\emph{descendant} of $\node$. 
%and $\node$ is its \emph{ancestor}.
The \emph{depth} of the tree is the length of the longest leaf.  A
tree is \emph{complete} if all its leaves have the same length and
every non-leaf node has both children.
%
A tree $\tree':\nodes'\rightarrow\locs$ is a~\emph{subtree} of
$\tree$, denoted $\tree'\subtree\tree$, iff there exists a~injective
map $\embed:\nodes'\rightarrow \nodes$ which respects the descendant
relation and labeling. That is, $\tree'(\node) =
\tree(\embed(\node))$ and $\node$ is a~descendant of $\node'$ iff
$\embed(\node)$ is a~descendant of $\embed(\node')$.
%

\paragraph{Parameterized systems with tree topology.}
The definitions for parameterized systems with a~tree topology are
 analogous to the definitions for systems with a~linear topology 
(Section~\ref{section:parameterized_systems}).
%
%Here, we only describe the differences.  
%points where the parameterized systems with tree topology differ from systems with linear topology.   
A parameterized system $\parsys = (\locs,\rules)$ induces a~transition
system $\transys = (\confs,\trans)$ where $\confs$ is the set of
complete trees over $\locs$.  The set $\rules$ of transition rules is
a~set of {\it local} and \emph{tree} transition rules.  The
transitions of $\trans$ are obtained from rules of $\rules$ as
follows.  A {\it local} rule is of the form $\loc\trans\loc'$ and it
locally changes the label of a~node from $\loc$ to $\loc'$.  A {\it
  tree} rule is a~triple
$\loc(\loc_0,\loc_1)\trans\loc'(\loc_0',\loc_1')$. The rule can be
applied to a~node $\node$ and its left and right children $\node_0$,
$\node_1$ with labels $\loc$, $\loc_0$, and $\loc_1$, respectively,
and it changes their labels to $\loc'$, $\loc_0'$, and $\loc_1'$,
respectively.

The reachability problem is defined in a~similar manner to the case of
linear systems. The set $\minbad$ of minimal bad configurations is
a~finite set of trees over $\locs$, $\inits$ is a~regular
tree-language, and $\bad$ is the upward closure of $\minbad$ w.r.t.\ the
subtree relation $\subtree$.  In the notation $\confsof n$ and
$\reachof n$, $n$ refers to the depth of trees rather than to the
length of words.

\paragraph{Verification.}
The verification method of Section~\ref{section:verification_method}
is easily extended to the tree topology. The text of
Section~\ref{section:verification_method} can be taken almost verbatim
with the difference that instead of words, we manipulate complete
trees, subword relation is replaced by subtree relation, and $k$ now
refers to the depth of trees rather than the length of words.
%
That is, a~view of size $k$ is a~tree of depth $k$ and the
abstraction $\projof \tree k$ returns all complete subtrees of 
depth at most $k$ of the tree $\tree$. 
Concretization and abstract post-image are defined analogously as in Section~\ref{section:verification_method}, based on $\alpha_k$. 
The set $\inits$ may be given in the form of a~tree automaton. 
The computation of $\projof \inits k$ may be then done over the structure of the tree automaton.
We can compute the abstract
post-image since Lemma~\ref{lemma:apost} holds here in the same wording as in Section~\ref{section:method}. 
The test $\gammaof k(\Vk)\cap\bad=\emptyset$ is carried out in the same way as in Section~\ref{section:method} since $\bad$ is an upward closure of a~set $\minbad$ w.r.t.
$\subtree$. The points 1-4 of Section~\ref{section:method} are thus satisfied and Algorithm~\ref{algorithm:basic} can be used as a~verification procedure for systems with tree topology.


%--------------------------------------------------------
\subsection{Ring Topology}
\label{section:rings}
%--------------------------------------------------------
The method can be extended also to systems with a~ring topology.
In a~parameterized system with ring topology, processes are organized in a~circular array and they synchronize by near-neighbor communication.
We model system with a~ring topology as systems with linear topology of Section~$\ref{section:parameterized_systems}$, where a~configuration $c\in Q^+$ is interpreted as a~circular word. 
%
The set $\rules$ may contain local and \emph{near-neighbor} transition rules. 
A near-neighbor rule is a~pair $(\loc_1\trans\loc_1',\loc_2\trans\loc_2')$.
% It induces the transition $\conf\trans\conf'$ of $\trans$ if either 
% $\conf = \conf_L\,\loc_1\loc_2\,\conf_R$ and $\conf' = \conf_L\,\loc_1'\loc_2'\,\conf_R$ or 
% $\conf = \loc_2\,\bar\conf\,\loc_1$ and  $\conf' = \loc_2'\,\bar\conf\,\loc_1'$. 
% The latter case is needed because configurations encode circular words. 
% The two ends must be allowed to communicate in the same way as other pairs of %neighbouring
% adjacent processes.  
It induces the transition $\conf\trans\conf'$ of $\trans$ if either
$\conf = \conf_L\,\loc_1\loc_2\,\conf_R$ and $\conf' =
\conf_L\,\loc_1'\loc_2'\,\conf_R$ (i.e.\ the 2 processes are adjacent
in the configuration $c$) or $\conf = \loc_2\,\bar\conf\,\loc_1$ and
$\conf' = \loc_2'\,\bar\conf\,\loc_1'$ (i.e.\ the 2 processes are
positioned at the end of the configuration $c$). The latter case
covers the communication between the extremities since configurations
encode circular words.

\paragraph{Verification.}
A word $u$ is a~\emph{circular subword} of a~word $v$, denoted $u\csubword v$, iff there are $v_1,v_2$ such that $v = v_1v_2$ and $u \subword v_2\,v_1$.
The only difference compared to the method for the systems with a~linear topology is  
that the standard subword relation is in all definitions replaced by
the circular subword relation $\csubword$. 
%Moreover, an abstraction $\projof \conf k$ of a~configuration $\conf$ returns all circular subwords of $\conf$ of length~$k$.
%
An equivalent of Lemma~\ref{lemma:apost} holds here in unchanged
wording, points 1-4 are satisfied, and Algorithm~\ref{algorithm:basic}
is thus a~verification procedure for systems with ring topology.

%--------------------------------------------------------
\subsection{Multiset Topology}
\label{section:multisets}
%--------------------------------------------------------
Systems which we refer to as systems with multiset topology are a
special case of the systems with 
a~linear topology of Section~\ref{section:parameterized_systems}. 
Typical representatives of these systems are Petri nets, which correspond precisely to systems of Section~\ref{section:extensions} with only (generalized) rendez-vous transitions.
Systems with multiset topology may contain all types of transitions
including local, global, broadcast, and rendez-vous, with the
exception of global transitions with the scope of indices
$j>i$ and $j<i$ (i.e., only $j\neq i$ is permitted). 
Since the processes have no way of distinguishing their respective positions within a~configuration,
the notion of ordering of positions within a~configuration is not meaningful and configurations can be represented as multisets. 
