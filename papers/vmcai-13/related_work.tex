%
%--------------------------------------------------------
\section{Related Work}
\label{section:related_work}
%--------------------------------------------------------
An extensive amount of work has been devoted to regular model checking,
e.g.~\cite{KMMPS2001,DLS01}; and in particular augmenting regular
model checking with techniques such as widening~\cite{BLW03,Tou01},
abstraction~\cite{BHV04}, and acceleration~\cite{AJNO:simple}. All
these works rely on computing the transitive closure of transducers or
on iterating them on regular languages. Our method is significantly simpler and more efficient. 

%Roughly comparable running times presents only \cite{BHV04}. Although \cite{BHV04} is also very general, it is not clear how it could be extended to handle non-atomically checked global conditions \cite{private}.
%

A technique of particular interest for parameterized systems is that
of {\it counter abstraction}. The idea is to keep track of the number
of processes which satisfy a certain
property~\cite{GerSis:many,Esparza:Finkel:Mayr:LICS,Delzanno:cache,Delzanno:futurebus,PXZ02}.
%
In general, counter abstraction is designed for systems with
unstructured or clique architectures.
%
As mentioned, our method can cope with these kinds of systems but also
with more general classes of topologies.
%
%Several works reduce parameterized verification to the verification of
%finite-state models. Among these, the {\em invisible invariants}
%method~\cite{APRXZ01,PRZ-tacas01} exploits cut-off properties to check
%invariants for mutual exclusion protocols. % like German's protocol.
%%
%The success of the method depends on the heuristic used in the
%generation of the candidate invariant. This method sometimes (e.g.\
%for German's protocol) requires insertion of auxiliary program
%variables for completing the proof.
Several works reduce parameterized verification to the verification of
finite-state models. Among these, the {\em invisible invariants}
method~\cite{APRXZ01,PRZ-tacas01} and the work of~\cite{Namjoshi:VMCAI07} exploit cut-off properties to check
invariants for mutual exclusion protocols. % like German's protocol.
%
%Lukas
%The success of the method depends on the heuristic used in the
%generation of the candidate invariant. It can generate inductive invariants of the form 
The success of the method depends on the heuristic used in the
generation of the candidate invariant. This method sometimes (e.g.\
for German's protocol) requires insertion of auxiliary program
variables for completing the proof. 
The nature of invariants generated by our method is similar to that of the aforementioned works, since our invariant sets of views of size at most $k$ can be seen as universally quantified assertions over reachable $k$-tuples of processes.  
%$\forall i:\theta(i)$ and $\forall i,j:\theta(i,j)$ where $i$ and $j$ are indices of processes and $\theta(i,j)$ is a formula that relates states of proceses $i$ and $j$. In order to generate invariants of different form, especially with mixed quantifiers, the method requires requires insertion of auxiliary program
%variables for completing the proof (e.g.\
%for German's protocol).
%Using the acelerated algorithm of Section~\ref{section:verification_method:acceleration}, our method actually generates and proves the strongest inductive invariant of the form 
%Similarly as the method of \cite{Namjoshi:VMCAI07}, our method actually generates the strongest inductive invariant of the form 
%$\forall i_1,\ldots,i_k:\Theta(i_1,\ldots,i_k)$ where $\theta{}$. In its current form, our method cannot generate and prove invariants with mixed quantifiers.
%

In~\cite{BLS02}, finite-state abstractions for verification of systems
specified in WS1S are computed on-the-fly by using the weakest
precondition operator. The method requires the user to provide a set
of predicates on which to compute the abstract model.
%

The idea of refining the view abstraction by increasing $k$ is similar in spirit to the work of~\cite{MalkisPR07} which discusses increasing precision of thread modular verification (Cartesian abstraction) by remembering some relationships between states of processes.
Their refinement mechanism is more local, targeting the source of undesirable imprecision; however, it is not directly applicable to parameterized verification. 

{\em Environment abstraction}~\cite{CTV06} combines predicate
abstraction with the counter abstraction. The technique is applied to
Szymanski's algorithm. The model of~\cite{CTV06} contains a more
restricted form of global conditions than ours, and also does not
include features such as broadcast communication, rendez-vous
communication, and dynamic creation and deletion of processes.

Recently, we have introduced the method of {\it monotonic
  abstraction}~\cite{rmc:wo:transducers} that combines regular model
checking with abstraction in order to produce systems that have
monotonic behaviors w.r.t.\ a well quasi-ordering on the state space.
%
%
In contrast to the method of this paper, the abstract system still
needs to be analyzed using full symbolic reachability analysis on an
infinite-state system.
%
The only work we are aware of which attempts to automatically verify systems
with non-atomic global transitions is \cite{parosh:non-atomic} which applies monotonic abstraction.
%
The abstraction in this case 
amounts to a verification procedure that operates on unbounded
graphs, and thus is a non-trivial extension of the existing framework.
%
As we saw, our method is easily extended to the case of non-atomic
transitions.

The method of~\cite{Raskin:Viagra,Raskin:experiments:German} and its
reformulated, generic version of~\cite{Ganty:Complete} are in
principle similar to ours.  They come with a complete method for
well-quasi ordered systems which is an alternative to backward
reachability analysis based on a forward exploration.
%
Unlike our method, they target well-quasi ordered systems only and
have not been instantiated for topologies other than multisets and
lossy channel systems.

Constant-size cut-offs have been defined for ring networks
in~\cite{Emerson:Namjoshi:POPL95} where communication is only allowed
through token passing.
%
More general communication mechanisms such as guards over local and shared
variables are described in~\cite{EmKa:manyfew}.
%
However, the cut-offs are linear in the number of states of the
components, which makes the verification task intractable on most of our
examples.

The closest work to ours is the one in~\cite{cutoff} that also relies
on dynamic detection of cut-off points.
%
The class of systems considered in~\cite{cutoff} corresponds
essentially to Petri nets.
%
In particular, it cannot deal with systems with linear or tree-like
topologies.
%
The method relies on the ability to perform backward reachability
analysis on the underlying transition system.
%
This means that the algorithm of~\cite{cutoff} cannot be applied on systems with undecidable reachability problems (such as the ones
we consider in this paper).
%
The method of~\cite{cutoff} is yet complete.
%On the other hand, the method of~\cite{cutoff} is complete.
%
% This is necessarily not true in our case since we are dealing with an
% undecidable problem.

