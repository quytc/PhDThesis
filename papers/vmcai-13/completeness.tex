\section{Completeness for Well Quasi-Ordered Systems}\label{section:completeness}
%
In this section, will show that the scheme desribed by Algorithm~\ref{algorithm:basic}
%
is complete for a wide class of well-quasi ordered systems.
%
%
%we show that Algorithm~\ref{algorithm:basic} is
%guaranteed to terminate if the transition relation is monotonic w.r.t.\ a well quasi-ordering.  
%The
%completeness result may be actually stated for a much wider class of
%transition systems than the one induced by the parameterized systems of
%Section~\ref{section:parameterized_systems}. 
%
To state the result in general terms, we will first give some
definitions from the theory of well quasi-ordered systems
(c.f. \cite{abdulla:well}).

A \emph{well quasi-ordering} (WQO) is a preorder $\preceq$ over a set
$S$ such that for every infinite sequence $s_1,s_2,\ldots$ of elements
of $S$, there exists $i$ and $j$ such that $i<j$ and $s_i\preceq s_j$. 
%
The \emph{upward-closure} $\ucl T$ of a set $T\subseteq S$ w.r.t. $\preceq$ is the
set $\{s\in S\mid \exists t\in T:t\preceq s\}$ and its 
\emph{downward-closure} is the set $\dcl T = \{s\in S\mid \exists t\in T:s\preceq t\}$.
%
A set is \emph{upward-closed} if it equals its upward-closure and it is \emph{downward-closed} if it equals its downward-closure.
%
If $T$ is upward closed, 
%
its complement $S\setminus T$ is downward closed and, conversely, 
%
if $T$ is downward closed, its complement is upward closed. 
%
For every upward closed set $T$, there exists a minimal (w.r.t $\subseteq$) set $\gen$ such that $\ucl\gen = T$, 
%
called \emph{generator} of $T$, 
%
which is finite.
%
If moreover $\preceq$ is a partial order, 
%
then $\gen$ is unique. 



A relation $\rel\subseteq S\times S$ is \emph{monotonic} w.r.t. $\preceq$ if whenever
$(s_1,s_2)\in\rel$ and $s_1\preceq s_1'$, then there is $s_2'$ with
$(s_1',s_2')\in\rel$ and $s_2\preceq s_2'$.  
%
Given a relation $f\subseteq S\times S$ monotonic w.r.t. $\preceq$ 
%
and a set $T\subseteq S$, 
%
it holds that if $f(T)\subseteq T$, 
%
then $f(\dcl T)\subseteq \dcl T$, 
%
where $f(T)$ is the image of $T$ 
%
defined as $\{t'\mid\exists t\in T:(t,t')\in f\}$.




The reasoning in Section~\ref{section:method} is based on the natural
notion of a size of a configuration.  Its generalization is the notion
of a \emph{discrete measure} over a set $S$, a function
$|.|:S\trans\nat$ which fulfills the property that for every
$k\in\nat$, $\{s\in S\mid|s| = k\}$ is finite.
%
A discrete measure is necessary to obtain the completeness result as
it allows enumerating elements of $S$ of the same size.
%
In particular, this property guarantees termination of the fixpoint
computation on Line~\ref{line:fixpoint} of
Algorithm~\ref{algorithm:basic}.
%
We note that the existence of a discrete measure is implied by a
stronger restriction of \cite{Bingham:empirically} to the so called
discrete transition systems.
%
%\comment{Throw away the reference to \cite{Bingham:empirically}?}

We say that a transition system $\transys = (\confs,{\trans})$ is
\emph{well-quasi ordered} by a WQO ${\preceq}\subseteq\confs\times\confs$ if
$\trans$ is monotonic w.r.t. ${\preceq}$.
%
Given a well-quasi ordered transition system and
a measure $|.|:\confs\rightarrow\nat$,
we define an abstraction function $\alpha_k,k\in\nat$ such that
$\abstr k \conf = \{\conf'\in\confs\mid\conf'\preceq\conf\}$. The 
corresponding concretization $\gammaof k$ and abstract post-image
$\mathit{Apost}_k$ are then defined based on $\alpha_k$ and $|.|$ as in
Section~\ref{section:abstraction}. 
%

Lemma~\ref{lemma:invariant} holds here in the same wording as in Section~\ref{section:method}.
%
%
%We first state a generalized variant of
%Lemma~\ref{lemma:invariant}. 
%%
%\begin{lemma}
%\label{lemma:invariant:general}
%Let $\transys$ be a TS well-quasi ordered by $\preceq$ with a discrete measure $|.|$.
%Then for any $i\in\nat$ and $X\subseteq\viewsof i$, 
%$\projof{\inits}{i}\subseteq{X} \; \land \; \apostof{X}{i}\subseteq X\implies\projof{\reach}{i}\subseteq X$.  
%\end{lemma}
%\comment{The assumptions of discreteness of $|.|$ and monotonicity w.r.t. a WQO are in fact not used in the proof. Or are they?}\\
%
%Let $S\subseteq \confs$ and let 
%$\preceq$ be a WQO over $\confs$.
%
%Its upward closure is the set $\ucl(V) = \{w\in Q^*\mid \exists v\in V:v\subword w\}$ 
%
%Its downward closure is the set $\dcl(S) = \{\conf\in \confs\mid\exists \conf'\in S:\conf\preceq \conf'\}$.
%
%A set is downward closed iff it equals its downward closure.
%
%
The main component of the completeness result is  the following theorem. 
%
\newcommand{\theoremtext}{
Let $\transys = (\confs,{\trans})$ be a well-quasi ordered transition system with a measure $|.|$.
Let $\inits$ be any subset of $\confs$ and let $\bad$ be upward-closed w.r.t. $\preceq$.
Then, if $\transys$ is safe w.r.t. $\inits$ and $\bad$, then there is $k\in \nat$ such that for  $\Vk = \mu X. \projof \inits k \cup \apostof X k$, $\bad \cap \gammaof k (\Vk) =\emptyset$.
}
\begin{theorem} 
\label{theorem:completeness}
\theoremtext
\end{theorem}

\begin{proof}
Recall first that $\gammaof k, \post, \apost k, \abstraction k$ are monotonic functions w.r.t. $\subseteq$ for all $k\in\nat$. 
%
Let $\gen$ be the minimal generator of the upward closed set $\confs\setminus\dcl\reach$.
%
We will prove that $k$ can be chosen as $k=\max\{|\conf|\mid\conf\in\gen\}$.
%
Such $k$ exists because $\gen$ is finite.



We first show an auxiliary claim that $\gammaof k(\thing) \subseteq \dcl\reach$.
%
Let $s\in \gammaof k(\thing)$. 
%
For the sake of contradiction, suppose that $s\not\in \dcl\reach$.
%
%By the definition of $\gamma$, we know that if $|s|\leq k$ then $s\in \dcl\reach$, 
%
%which contradicts our assumption.
%
%Let us therefore supppose that $|s| > k$.
%
We have that $s\in \confs\setminus\dcl\reach  =\ucl\gen$
%
and there is a generator $t\in\gen$ with $t\preceq s$.
%
By the definition of $k$, $|t|\leq k$. 
%
Since $t\in\gen$, $t\not\in\dcl\reach$ and hence $t\not\in\alpha_k(\dcl\reach)$. %
But due to this and since %$|t|\leq k$ and 
$t\preceq s$,
%Notice that $\thing = [\dcl\reach]_{\leq k}$ 
%
%(by the definition of $\alpha_k$) 
%
%and thus $t\not \in [\dcl\reach]_{\leq k}$.
%
%Since $t\preceq s$, 
%
we have that $s\not\in\gammaof k(\thing)$ (by the definition of $\gammaof k$)
%
which contradicts the initial assumption and the claim is proven.

Next, we argue that ${\thing}$ is stable under abstract post, that is,
%
$\apostof {{\thing}} k \subseteq {\thing}$.
%
Since $\reach$ is stable under $\post$ and $\post$ is monotonic w.r.t. $\preceq$,
%
we know that $\dcl\reach$ is stable under $\post$ 
%
(that is, $\post(\dcl\reach)\subseteq \dcl\reach$).
%
%By the definition of $\abstraction k$,  
%
%$\abstr k {\dcl\reach} = \thing$.
%
Then,  by the definition of $\apost k$,
%
and by monotonicity of $\abstraction k$ w.r.t. $\subseteq$, 
%
we have
$\apostof {\thing} k = 
%
\abstr k {\post(\gammaof k (\thing))} \subseteq 
%
\abstr k {\post(\dcl\reach)}\subseteq 
%
%\abstr k {\dcl\reach} = 
%
\thing$. 
%
%We have thus shown that $\abstr k {{{\dcl_k\reach}}}$ is closed under $\apost k$.

Since $\dcl\reach$ contains $\inits$, $\abstr k \inits  \subseteq \thing$.
%
$\thing$ is thus a fixpoint of $\lambda X.\alpha_k(\inits)\cup\apost k (X)$.
%
Because $\Vk$ is the least fixpoint of $\lambda X.\alpha_k(\inits)\cup\apost k (X)$, 
%
$\Vk\subseteq \thing$.
%
From, $\reach\cap\bad = \emptyset$ and since $\bad$ is upward closed, 
we know that $\dcl\reach\cap\bad = \emptyset$.
%
Because $\gammaof k (\Vk)\subseteq 
%
\gammaof k (\thing) \subseteq  
%
%\thing  \subseteq 
%
\dcl\reach$
%
and $\dcl\reach\cap \bad = \emptyset$,
%
$\gammaof k (\Vk)\cap\bad=\emptyset$.
\qed
\end{proof}

%\paragraph{Sketch of the proof.}
%The complement $\confs\setminus\dcl\reach$ of the downward-closed set
%$\dcl\reach$ is upward-closed.
%Since $\preceq$ is a WQO, there exists no infinite sequence of
%unrelated elements.
%%
%The set
%$\confs\setminus\dcl\reach$~{\tikz[scale=0.25]{\fill[zone=yellow!60]
%    (-2ex,5.5ex) parabola[bend pos=0.5] bend (0,0) +(4ex,0) --
%    cycle;}} is therefore generated by a finite number of
%elements~{\tikz[scale=1.1]{\fill[black,draw=gray!20] +(-2pt,2pt) rectangle
%    +(2pt,-2pt);}}.
%%
%Let $\gen$ be this finite set.
%%
%The proof of Theorem~\ref{theorem:completeness} is based on showing
%that $k$ can be taken as the size of largest element in $\gen$.
%Using the assumption of monotonicity of the
%transition relation, it can be shown that the fixpoint $\Vk$ is a
%subset of $\alpha_k(\dcl\reach)= (\dcl\reach)_{\leq k}$
%%$\dcl\reach\cap\confs_{\leq k}$ 
%(denoted by~%
%{\tikz[scale=1.1]{\fill[black,draw=gray!20] circle (2pt);}}).
%%
%The definition of $\gammaof k$ gives that $\gammaof k(\alpha_k(\dcl\reach))
%= \dcl\reach$ (i.e.\ the zone~%
%%
%{\tikz[scale=0.2,baseline=(n.base)]{
%    \coordinate(n) at (10ex,2ex);
%    \fill[zone=blue!20,draw=white] rectangle (20ex,10ex);
%%    \pattern[pattern color=white,pattern=bricks] rectangle (20ex,10ex);
%  }}%
%%
%) and thus $\gammaof k(\Vk) \subseteq \dcl\reach$.
%%
%Since $\bad$~%
%{\tikz[scale=0.25]{
%    \def\mypara{(-2ex,5ex) parabola[bend pos=0.5] bend (0,0) +(4ex,0) -- cycle}
%    \fill[zone=red!80] \mypara;
%    \pattern[pattern color=white,pattern=north west lines] \mypara;
%    %\fill[red,draw=white] (0,-1ex) circle (4pt);
%  }} %
%%
%is upward-closed and $\reach\cap\bad=\emptyset$, we know that $\dcl\reach\cap\bad = \emptyset$ and thus
%$\gammaof k(\Vk)\cap\bad=\emptyset$.
%%
%%
%The situation is illustrated in Figure~\ref{figure:completeness} and
%the detailed proof may be found in Appendix~\ref{app:completeness}.
%%(The full proof requires both monotonicity of $\trans$ w.r.t. $\preceq$ and discreteness of $|.|$).
%\begin{figure}
%\label{figure:completeness}
%\input{completeness.pic.tex}
%\caption{The different sets involved in the completeness proof.}
%\end{figure}
%\qed
%\medskip
%
Theorem~\ref{theorem:completeness} guarantees that for a safe well
quasi-ordered system, there exists $k$ for which the test on
line~\ref{line:test} of Algorithm~\ref{algorithm:basic} succeeds.
%
Conversely, Lemma~\ref{lemma:invariant}, which, as mentioned above,
still holds for the general class of well-quasi ordered systems, then
assures than if the test on line~\ref{lemma:invariant} succeeds, the
system is indeed safe.

\paragraph{Complete algorithm.}
The schema described by Algorithm~\ref{algorithm:basic} (or its
variant from Section~\ref{section:nonsizepreserving} if the transition
relation is not size-preserving) gives a complete verification procedure for a well quasi-ordered system provided that all the four steps of its for-loop can be effectively evaluated.
This is guaranteed by the following requirements:
%\vspace{-2mm}
\begin{enumerate}[i.]
\item \label{ii}
$\alpha_k(I)$ can be computed,
\item \label{iv}
the measure $|.|$ is discrete, %
\item \label{iii}
for a configuration $c$, $\post(c)$ and $\alpha_k(c)$ can be computed, 
\item \label{v}
for a finite set of views $V$, $\gamma_k^{k+1}(V)$ can be computed, and
\item \label{vi}
a variant of Lemma~\ref{lemma:apost} holds.
%\item \label{i}
%the system is well quasi-ordered w.r.t.\ a WQO $\preceq$.
\end{enumerate}
%
%
%(iii) and the conditions 1-4 from Section~\ref{section:method} are
%satisfied. 
%
Point (\ref{ii}) is point 1 of Section~\ref{section:method}.
Points (\ref{iv})-(\ref{vi}) guarantee that we can compute abstract post-image (point 2 of Section~\ref{section:method}). 
We can test $\gamma_k(V)\cap\bad = \emptyset$ (point 3 of Section~\ref{section:method}) since due to (\ref{iv}),
%and also (\ref{iv}), 
$V$ is always finite.
Exact reachability analysis of configurations of a bounded size (point 4 of Section~\ref{section:method}) can be carried out since we can iterate $\post$ due to (\ref{iii}) and the iteration terminates after a finite number of steps  due to (\ref{iv}). 
%
Point (\ref{iv}) also assures termination of the computation of the fixpoint on line~\ref{line:fixpoint} ($V$ is always finite).
%of Algorithm~\ref{algorithm:basic}, 
%since it implies that $V$ is always finite. 

%Theorem~\ref{theorem:completeness} together with (\ref{i}) guarantees that if the system is safe, there is indeed a cutoff point for which safety will be concluded on line~\ref{line:test}. This conclusion is always sound since, as mentioned above, Lemma~\ref{lemma:invariant} holds in the general setting too.

Overall, Algorithm~\ref{algorithm:basic} is a complete verification procedure for 
parameterized systems of
Section~\ref{section:parameterized_systems} with local and existential
transitions rules, broadcast and rendez-vous. %
The induced transition relation is indeed monotonic w.r.t.\ the
preorder $\subword$ which is a WQO and the length of a configuration
is a discrete measure.
%
An important subclass of such systems are Petri nets, which, as
mentioned in Section~\ref{section:extensions}, correspond to systems
with multiset topology and generalized rendez-vous transition rules.
%
Systems of Section~\ref{section:parameterized_systems} with
universally guarded transition rules do not satisfy the assumptions: 
the induced transition relation is not monotonic.
%
%%\paragraph{Completeness for Petri nets.}
%%Let us point out that Theorem~\ref{theorem:completeness} allows to
%%turn Algorithm~\ref{algorithm:basic} into a complete procedure for the
%%general class of Petri nets (where our verification problem
%%corresponds to coverability). As mentioned in
%%Section~\ref{section:extensions}, Petri nets may be seen as
%%parameterized systems with a multiset topology. Transitions of Petri
%%nets are however not size-preserving since they may change the
%%number of tokens in the net. 
%%%
%%Recall that size-preserving is required by the definition of
%%compatible transition systems.
%%% , and it is needed in the proof of
%%% Lemma~\ref{lemma:invariant:general} and
%%% Theorem~\ref{theorem:completeness}.
%%%
%%Such transitions may be modelled in our
%%framework using the technique of placeholders described in
%%Section~\ref{section:extensions}.  The original Petri net is
%%transformed into a size-preserving one by adding a special place
%%$p$. Every transition of the original net which increases the number
%%of tokens by $n$ will take $n$ tokens from $p$ and, conversely, every
%%transition of the original net that decreases the number of tokens by
%%$n$ will add $n$ tokens into $p$.  The transitions of the transformed
%%Petri net then correspond precisely to the rendez-vous rules of
%%Section~\ref{section:extensions}. 
%%%
%%In the transformed net, $p$ can initially have any number of
%%processes.  It holds that a marking is coverable from an initial
%%marking $M$ of the original net if, and only if, it is coverable from
%%an initial marking $M'$ of the transformed net which is obtained from
%%$M$ by adding an arbitrary number of tokens into $p$.
%%%
%%The WQO used for Petri nets is the standard sub-multiset relation on
%%markings and the compatible measure is simply the size of a multiset.
