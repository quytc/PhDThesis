\section{Conclusion and Future Work}
\label{section:conclusion}
We have  presented a uniform framework
for automatic verification of 
different classes of parameterized systems
with topologies such as words, trees, rings, or
multisets, with an extension to handle non-atomic global conditions.
%
The framework allows to perform parameterized verification by
only considering a small set of instances of the system.
%
We have proved that the presented algorithm is complete for a wide
class of well quasi-ordered systems.
%
Based on the method, we have implemented a prototype which performs
efficiently on a wide range of benchmarks.
%

We are currently working on extending the framework to the case of
multi-threaded programs operating on dynamic heap structures.
%
These systems have notoriously complicated behaviors.
%
Showing that verification can be carried out
through the analysis of only a small number of threads
would allow for more efficient algorithms for these systems.
%
Furthermore,
our algorithm relies on a very simple abstraction function, 
where a configuration of the system is approximated by its
sub-structures (subwords, subtrees, etc.).
%
We believe
that our approach can be lifted to more general classes
of abstractions.
%
This would allow for abstraction schemes that are more
precise than existing ones, e.g.,
 thread-modular abstraction \cite{ThreadModular}
and 
Cartesian abstraction \cite{TMisCA}.
%

Obviously, the bottleneck in the application
of the method is when the cut-off condition is only satisfied at high
values of
$k$ (see e.g., the {\it Kanban} example in
Section~\ref{section:experiments}).
%
We plan therefore to integrate the method with advanced 
tools that can perform efficient forward reachability analysis, like
SPIN~\cite{SPINModelChecker}, and to use efficient
symbolic encodings for compact representations for the set of views.
