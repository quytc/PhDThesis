The MOSI protocol is an extension of the basic MSI cache coherency
protocol. It is a snoop-based protocol. It adds the state Owned ($O$),
which indicates that the current processor owns this block, and will
service requests from other processors for the block. This also
reduces the amount of write-back data upon cache eviction.

The protocol can be modeled as a parameterized system where the
processes are arranged in a multiset. Each process can communicate
with its neighbors by broadcasting a message on the connecting bus.

\begin{wrapfigure}{r}[2pt]{0.25\textwidth}
\vspace{-10mm}
\newcommand{\wrong}{\tikz[baseline=-3pt]\node[scale=0.8]{\color{red}$\skull$};}
\newcommand{\valid}{{\color{green}\bf$\checkmark$}}
\newcommand{\cache}[1]{\bf\small\emph{#1}}
\begin{tabular}{c|cccc}
                 & {\cache{M}} & {\cache{O}} & {\cache{S}} & {\cache{I}} \\\hline
  {\cache{M}} & {\wrong}       & {\wrong}       & {\wrong}       & {\valid}       \\
  {\cache{O}} & {\wrong}       & {\wrong}       & {\valid}       & {\valid}       \\
  {\cache{S}} & {\wrong}       & {\valid}       & {\valid}       & {\valid}       \\
  {\cache{I}} & {\valid}       & {\valid}       & {\valid}       & {\valid}       \\
\end{tabular}
\vspace{-10mm}
\end{wrapfigure}
The state of a cache line can be \emph{Modified}, \emph{Owned},
\emph{Shared}, and \emph{Invalid} (resp. \emph{M,O,S,I}). The broadcast
communication is depicted using the 2 following automata. Initially,
all cache lines are in state \emph{Invalid} (\emph{I}). The permitted
states of any given pair of cache lines is given in the table beside.
%

The first automaton represents the action taken when the given process
issues the broadcast message and/or when it manipulates the cache
line.
%
The cache can be written (\emph{CPUwrite}), caused by a store miss, it
can be read (\emph{CPUread}) caused by a load miss and finally, it can
be replaced (\emph{CPUrepl}).
%
The message sent on the bus vary depending on the state of the current
cache line. The active process can send a \emph{read-to-share}
(\emph{BUSrts}) request, a \emph{read-to-write} (\emph{BUSrtw})
request, a \emph{write-back} (\emph{BUSwb}) request (in case of cache
eviction) or an \emph{invalidate} (\emph{BUSinv}) request. We label
each edge of the automaton with the action %
\begin{tikzpicture}[baseline=(n.center)]
  \path node[message](n){\emph{action}\nodepart{second}\emph{message}} [clip] (n.west) rectangle (n.north east);
\end{tikzpicture} %
issued by the current process
and with the message %
\begin{tikzpicture}[baseline=(n.south)]
  \path node[message](n){\emph{action}\nodepart{second}\emph{message}} [clip,use as bounding box] (n.south west) rectangle (n.east);
\end{tikzpicture} %
 it sent on the bus. We use %
\begin{tikzpicture}[baseline=(n.south)]
  \path node[message](n){\emph{action}\nodepart{second}$-$} [clip,use as bounding box] (n.south west) rectangle (n.east);
\end{tikzpicture} %
when no message is sent on the bus.

\begin{center}
\begin{tikzpicture}[node distance=3cm,
  state/.append style={scale=1.5},
  ]

  \node[state] (m)	        {\emph{M}};
  \node[state] (o) [right=4cm of m] {\emph{O}};
  \node[state] (s) [below=of m] {\emph{S}};
  \node[state] (i) [below=of o] {\emph{I}};

  \draw [->,myedge] (m) .. controls +(-0.5,1) and +(-1,0.5) ..
                        node[message,anchor=north east,pos=0.7]{\emph{CPUread}\nodepart{second}$-$}
                        node[message,anchor=south east,pos=0.3]{\emph{CPUwrite}\nodepart{second}$-$} (m);
  \draw [->,myedge] (m) to[out=-30,in=120] node[message,anchor=south,sloped]{\emph{CPUrepl}\nodepart{second}\emph{BUSwb}} (i);
  \draw [->,myedge] (o) -- node[message,above]{\emph{CPUwrite}\nodepart{second}\emph{BUSinv}} (m);
  \draw [->,myedge] (o) -- node[message,right]{\emph{CPUrepl}\nodepart{second}\emph{BUSwb}} (i);
  \draw [->,myedge] (o) .. controls +(0.5,1) and +(1,0.5) .. node[message,anchor=south west]{\emph{CPUread}\nodepart{second}$-$} (o);
  \draw [->,myedge] (s) to[out=-30,in=-150] node[message,anchor=north]{\emph{CPUrepl}\nodepart{second}$-$} (i);
  \draw [->,myedge] (s) .. controls +(-1,-0.5) and +(-0.5,-1) .. node[message,anchor=north east]{\emph{CPUread}\nodepart{second}$-$} (s);
  \draw [->,myedge] (s) -- node[message,left]{\emph{CPUwrite}\nodepart{second}\emph{BUSinv}} (m);
  \draw [->,myedge] (i) -- node[message,anchor=south]{\emph{CPUread}\nodepart{second}\emph{BUSrts}} (s);
  \draw [->,myedge] (i) to[out=150,in=-60] node[message,anchor=north,sloped]{\emph{CPUwrite}\nodepart{second}\emph{BUSrtw}} (m);

\end{tikzpicture}
\end{center}

The second automaton represents how other processes react upon
reception of a message \tikz[baseline=(n.south)]{\path node[message](n){\emph{action}\nodepart{second}\emph{message}} [clip] (n.south west) rectangle (n.east);}.

\begin{center}
\begin{tikzpicture}[node distance=3cm,
  state/.append style={scale=1.5},
  message/.append style={rectangle split parts=1,fill=pink!10!white,rectangle split part fill={yellow!10!white}},
  ]

  \node[state] (m)	        {\emph{M}};
  \node[state] (o) [right=4cm of m] {\emph{O}};
  \node[state] (s) [below=of m] {\emph{S}};
  \node[state] (i) [below=of o] {\emph{I}};
  
  \draw [->,myedge] (m) -- node[message,anchor=south]{\emph{BUSrts} (+Data)} (o);
  \draw [->,myedge] (m) -- node[message,sloped,above]{\emph{BUSrtw} (+Data)} (i);
  \draw [->,myedge] (o) -- node[message,anchor=west] {\ensuremath{\begin{tabular}{l} \emph{BUSrtw} (+Data) \\ \emph{BUSinv}\end{tabular}}} (i);
  \draw [->,myedge] (s) -- node[message,anchor=south east,pos=0.6]{\ensuremath{\begin{tabular}{l}\emph{BUSrtw}\\\emph{BUSinv}\end{tabular}}} (i);
  \draw [->,myedge] (s) .. controls +(-1,-0.5) and +(-0.5,-1) .. node[message,anchor=east]{\ensuremath{\begin{tabular}{l}\emph{BUSrts}\\\emph{BUSwb}\end{tabular}}} (s);
  \draw [->,myedge] (i) .. controls +(1,-0.5) and +(0.5,-1) .. node[message,anchor=south west]{\ensuremath{\begin{tabular}{l}\emph{BUSrts}\\\emph{BUSrtw}\\\emph{BUSinv}\\\emph{BUSwb}\end{tabular}}} (i);
  \draw [->,myedge] (o) .. controls +(0.5,1) and +(1,0.5) .. node[message,anchor=south west]{\emph{BUSrts} (+Data)} (o);
\end{tikzpicture}
\end{center}
