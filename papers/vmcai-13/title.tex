
%\title{Parametrized Verification\\through $k$-View Abstraction}
\title{All for the Price of Few}
\subtitle{\footnotesize (Parameterized Verification through View Abstraction)}



% \author{Parosh Aziz Abdulla\inst{1}
% \texttt{parosh@it.uu.se}
% \and \\
% Fr\'ed\'eric Haziza\inst{1}
% \texttt{daz@it.uu.se}
% \and \\
% Luk\'a\v s Hol\'ik\inst{1}\inst{2}
% \texttt{holik@fit.vutbr.cz}
% }
%\institute{}

\author{Parosh Aziz Abdulla\inst{1}  \and Fr\'ed\'eric Haziza\inst{1} \and Luk\'a\v s Hol\'ik\inst{1}\inst{2} }
\institute{Uppsala University, Sweden \and Brno University of Technology, Czech Republic.}

\maketitle



\begin{abstract}
  We present a simple and efficient framework for automatic
  verification of systems with a parameteric number of communicating
  processes.
  % 
  The processes may be organized in various topologies
  such as words, multisets, rings, or trees.
  % 
  Our method needs to inspect only a small number of processes in
  order to show correctness of the whole system. It relies on an
  abstraction function that views the system from the perspective of a
  fixed number of processes. The abstraction is used during the
  verification procedure in order to dynamically detect cut-off points
  beyond which the search of the state space need not continue.
  % 
  We show that the method is complete for a large class of well
  quasi-ordered systems including Petri nets.
  %
  Our experimentation on a variety of benchmarks demonstrate that the
  method is highly efficient and that it works well even for classes
  of systems with undecidable verification problems.
\end{abstract}




%\begin{abstract}
%  We present a simple and efficient framework for automatic
%  verification of parameterized systems. The method needs to run only
%  a small number of processes in order to show correctness of the
%  whole system.  The method relies on an abstraction function that
%  views the system from the perspective of a parametric number $k$ of
%  processes. The abstraction is used during the verification procedure
%  in order to dynamically detect cut-off points beyond which the
%  search of the state space need not continue.  We show how the
%  framework can be instantiated for a wide range of parameterized
%  systems with different topologies, namely words, multisets, rings,
%  and trees.  Our experimental results demonstrate the efficiency of
%  the approach on a variety of benchmarks belonging to the
%  above-mentioned classes of systems.
%\end{abstract}
