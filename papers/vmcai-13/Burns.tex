\begin{center}
\vspace{-6mm}
\begin{tikzpicture}[node distance=8mm]
		\node(code) at (-4.0,-0.8) {
			\begin{minipage}{60mm}
			\mbox{
			\begin{procedure}[H]
				%\While{\emph{true}}{
					\caption{Burns($i$)}
                                        %\tikz[baseline=(init.base)]\node[rounded corners=1ex,inner ysep=2pt,draw=blue!50,very thick,fill=green!20](init){Init};
					$\mathit{flag}[i]:= 0$\;
					\existsstmt {<} {\mathit{flag}[i] = 1} 1\;
					$\mathit{flag}[i]:= 1$\;
					\existsstmt {<} {\mathit{flag}[i] = 1} 1\;
					\KwSty{await} $ \forall j > i:\mathit{flag}[j] \neq 1$\;
					$\mathit{flag}[i]:= 0$;\,\KwSty{goto} 1\hfill\tikz[baseline=(cs.base)]\node[rounded corners=8pt,inner xsep=10pt,draw=blue!50,very thick,fill=red!50](cs){CS};
				%}
			\end{procedure}
			}
			\end{minipage}
		};

  \node[state] (n1) at (0,0) [fill=green!20]{1};
  \node[state] (n2) at (2,0) {2};
  \node[state] (n3) at (4,0) {3};
  \node[state] (n4) at (4,-2) {4};
  \node[state] (n5) at (2,-2) {5};
  \node[state] (n6) at (0,-2) [fill=red!50] {6};

  \draw [->,myedge] (n1) -- (n2);
  %\draw [->,myedge] (n2) to[out=120,in=60] (n1) node[mylabel,above]{$\exists j<i: \state{j}\not\in\set{1,2,3}$};
  \draw [->,myedge] (n2) .. controls +(120:0.75) and +(45:0.75) .. (n1) node[mylabel,above]{$\exists j<i: \set{4,5,6}$};

  \draw [->,myedge] (n2) -- (n3) node[mylabel,above]{$\forall j<i: \set{1,2,3}$};

  \draw [->,myedge] (n3) -- (n4);

  \draw [->,myedge] (n4) -- (n5) node[mylabel,below]{$\forall j<i: \set{1,2,3}$};

%  \draw [->,myedge] (n5) to[out=90,in=-45] (n1) node[mylabel,pos=0.7,above right,sloped]{$\exists j<i: \set{4,5,6}$};
  \draw [->,myedge] (n4) to[out=135,in=-45] (n1) node[mylabel,pos=0.45,above,sloped]{$\exists j<i: \set{4,5,6}$};

  \draw [->,myedge] (n5) -- (n6) node[mylabel,below]{$\forall j>i: \set{1,2,3}$};

  \draw [->,myedge] (n6) -- (n1);

\end{tikzpicture}
\vspace{-6mm}
\end{center}
%\def\initstate{\tikz[baseline=(n.base)]\node[state,fill=green!20,scale=0.7](n){1};}
%\caption{A pseudocode of the $i$th process of
%  Bruns's protocol and the corresponding transition rules (in the form of a transition diagram). A state of a process is composed form
%  a program location and a value of the local variable
%  $\mathit{flag}[i]$. Since value of $\mathit{flag}[i]$ is invariant
%  at each location, states correspond to locations. Initially, all
%  processes are in state {\protect\initstate}.
%}
\caption{
Pseudocode and transition rules of Burns' protocol.
}
