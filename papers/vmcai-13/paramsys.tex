%
%--------------------------------------------------------
\section{Parameterized Systems}
\label{section:parameterized_systems}
%--------------------------------------------------------

We introduce a standard notion of a parameterized system operating on
a linear topology, where processes may perform local transitions or
universally/existentially guarded transitions (this is the standard
model used e.g. in
\cite{PRZ-tacas01,CTV06,rmc:wo:transducers,Namjoshi:VMCAI07}).

A parameterized system is a pair $\parsys=(\locs,\rules)$ where
$\locs$ is a finite set of \emph{local states} of a process and
$\rules$ is a set of \emph{transition rules} over $\locs$.  A
transition rule is either {\it local} or {\it global}.
%
A local rule is of the form $\loc\trans\loc'$, where the process
changes its local state from $\loc$ to $\loc'$ independently of the
local states of the other processes.
%
A global rule is of the form $\quantrule\loc{\loc'}{\quantify}\circ
S$, where $\quantify\in\{\exists,\forall\}$, $\circ\in\{<,>,\neq\}$
and $S\subseteq\locs$.
%
Here, the $i$th process checks also the local states of the other
processes when it makes the move.
%
For instance, the condition $\forall j< i: \pstates$ means that ``for
every $j$ such that $j< i$, the $j$th process should be in a local
state that belongs to the set $\pstates$''; the condition $\forall
j\neq i: \pstates$ means that ``all processes except the $i$th one
should be in local states that belong to the set $\pstates$''; etc.

A parameterized system $\parsys =(\locs,\rules)$ induces a
\emph{transition system} (TS) $\transys = (\confs,\trans)$ where $\confs =
\locs^*$ is the set of its \emph{configurations} and ${\trans}
\subseteq\confs\times\confs$ is the \emph{transition relation}. We use
$\conf[i]$ to denote the state of the $i$th process within the
configuration $\conf$.
%
The transition relation $\trans$ contains a transition
$\conf\trans\conf'$ with $\conf[i] = \loc$, $\conf'[i] = \loc'$,
$\conf[j]=\conf'[j]$ for all $j:j\neq i$ iff either (i) $\rules$
contains a local rule $\loc\trans\loc'$, or (ii) $\rules$ contains a
global rule $\quantrule\loc{\loc'}{\quantify}\circ S$, and one of the
following conditions is satisfied:
%\begin{compactitem}
\begin{itemize}%[topsep=0pt]%package enumitem
\item 
$\quantify=\forall$ and for all $j:1\leq j\leq |\conf|$ such that $j\circ i$, it holds that $\conf[j]\in \pstates$.
\item 
$\quantify=\exists$ and there exists $j:1\leq j\leq |\conf|$ such that $j\circ i$ and $\conf[j]\in \pstates$.
%\end{compactitem}
\end{itemize}

\vspace{1ex}%
An instance of the \emph{reachability problem} is defined by a
parameterized system $\parsys = (\locs,\rules)$, a regular set
$\inits\subseteq\locs^+$ of \emph{initial configurations}, and a set
$\bad\subseteq \locs^+$ of \emph{bad configurations}.  Let $\subword$
be the usual \emph{subword relation}, i.e.,
$u\subword\loc_1\ldots\loc_n$ iff $u=\loc_{i_1}\ldots\loc_{i_k},1\leq
i_1{\ldots}i_k\leq n$ and $i_j<i_{j+1}$ for all $j:1\leq j < k$.  We assume
that $\bad$ is the upward closure $\{\conf\mid\exists b
\in\minbad:b\subword \conf\}$ of a given \emph{finite} set
$\minbad\subseteq\locs^+$ of \emph{minimal bad configurations}.  This
is a common way of specifying bad configurations which often appears
in practice, see e.g. the running example of Burn's mutual exclusion
protocol below.  We say that $\conf\in \confs$ is \emph{reachable} iff
there are $\conf_0,\ldots, \conf_l\in\confs$ such that $\conf_0\in
\inits$, $\conf_l = \conf$, and $\conf_i\rightarrow \conf_{i+1}$ for
all $0\leq i < l$.  We use $\reach$ to denote the set of all reachable
configurations.  We say that the system $\parsys$ is \emph{safe} w.r.t.\
$\inits$~and~$\bad$ if no bad configuration is reachable, i.e.\
$\reach\cap\bad = \emptyset$.

We define the \emph{post-image} of a set $\C\subseteq\confs$ to be the
set $\postof \C := \{\conf'\mid \conf\trans\conf'\wedge\conf\in\C\}$.
%
For $n\in\nat$ and a set of configurations $S \subseteq \confs$, we
use $S_n$ to denote its subset $\{c\in S\mid |c|{\leq}n\}$ of
configurations of size up to $n$.
%
% For $\circ\in\{<,\leq,\geq,>\}$, we write $S_{\circ n}$ to denote the
% union $\cup_{i\in\nat:i\circ n}S_i$.
%
%and 
%configurations of length $n$ and $\reachof n$ to denote the set of
%reachable configurations of length $n$.  For
%$\circ\in\{<,\leq,\geq,>\}$, we write $\reachof {\circ n}$ to denote
%the union $\cup_{i\in\nat:i\circ n}\reachof i$.
%

\paragraph{Running example.}
\newcommand{\bubblestate}[2]{\tikz[baseline=(n.base)]\node[state,fill=#2,scale=0.7](n){#1};}
\begin{figure}[!b]
  \begin{center}
\vspace{-6mm}
\begin{tikzpicture}[node distance=8mm]
		\node(code) at (-4.0,-0.8) {
			\begin{minipage}{60mm}
			\mbox{
			\begin{procedure}[H]
				%\While{\emph{true}}{
					\caption{Burns($i$)}
                                        %\tikz[baseline=(init.base)]\node[rounded corners=1ex,inner ysep=2pt,draw=blue!50,very thick,fill=green!20](init){Init};
					$\mathit{flag}[i]:= 0$\;
					\existsstmt {<} {\mathit{flag}[i] = 1} 1\;
					$\mathit{flag}[i]:= 1$\;
					\existsstmt {<} {\mathit{flag}[i] = 1} 1\;
					\KwSty{await} $ \forall j > i:\mathit{flag}[j] \neq 1$\;
					$\mathit{flag}[i]:= 0$;\,\KwSty{goto} 1\hfill\tikz[baseline=(cs.base)]\node[rounded corners=8pt,inner xsep=10pt,draw=blue!50,very thick,fill=red!50](cs){CS};
				%}
			\end{procedure}
			}
			\end{minipage}
		};

  \node[state] (n1) at (0,0) [fill=green!20]{1};
  \node[state] (n2) at (2,0) {2};
  \node[state] (n3) at (4,0) {3};
  \node[state] (n4) at (4,-2) {4};
  \node[state] (n5) at (2,-2) {5};
  \node[state] (n6) at (0,-2) [fill=red!50] {6};

  \draw [->,myedge] (n1) -- (n2);
  %\draw [->,myedge] (n2) to[out=120,in=60] (n1) node[mylabel,above]{$\exists j<i: \state{j}\not\in\set{1,2,3}$};
  \draw [->,myedge] (n2) .. controls +(120:0.75) and +(45:0.75) .. (n1) node[mylabel,above]{$\exists j<i: \set{4,5,6}$};

  \draw [->,myedge] (n2) -- (n3) node[mylabel,above]{$\forall j<i: \set{1,2,3}$};

  \draw [->,myedge] (n3) -- (n4);

  \draw [->,myedge] (n4) -- (n5) node[mylabel,below]{$\forall j<i: \set{1,2,3}$};

%  \draw [->,myedge] (n5) to[out=90,in=-45] (n1) node[mylabel,pos=0.7,above right,sloped]{$\exists j<i: \set{4,5,6}$};
  \draw [->,myedge] (n4) to[out=135,in=-45] (n1) node[mylabel,pos=0.45,above,sloped]{$\exists j<i: \set{4,5,6}$};

  \draw [->,myedge] (n5) -- (n6) node[mylabel,below]{$\forall j>i: \set{1,2,3}$};

  \draw [->,myedge] (n6) -- (n1);

\end{tikzpicture}
\vspace{-6mm}
\end{center}
%\def\initstate{\tikz[baseline=(n.base)]\node[state,fill=green!20,scale=0.7](n){1};}
%\caption{A pseudocode of the $i$th process of
%  Bruns's protocol and the corresponding transition rules (in the form of a transition diagram). A state of a process is composed form
%  a program location and a value of the local variable
%  $\mathit{flag}[i]$. Since value of $\mathit{flag}[i]$ is invariant
%  at each location, states correspond to locations. Initially, all
%  processes are in state {\protect\initstate}.
%}
\caption{
Pseudocode and transition rules of Burns' protocol.
}

  \label{figure:Burns}
\end{figure}
We illustrate the notion of a parameterized systems with the example
of Burns' mutual exclusion protocol~\cite{burnsinside}.
%
The protocol ensures exclusive access to a shared resource in a system
consisting of an unbounded number of processes organized in an array.
%
The pseudocode of the process at the $i$th position of the array and
the transition rules of the parameterized system are given in
Figure~\ref{figure:Burns}.
%
A state of the $i$th process consists of a program location and a
value of the local variable $\mathit{flag}[i]$. Since the value of
$\mathit{flag}[i]$ is invariant at each location, states correspond to
locations.

A configuration of the induced transition system is a word over the
alphabet $\{1,\ldots,6\}$ of local process states.
%
The task is to check that the protocol guarantees exclusive access to
the shared resource (line~6) regardless of the number of processes. A
configuration is considered to be bad if it contains two occurrences
of state \bubblestate{6}{red!50}, i.e., the set of minimal bad
configurations $\minbad$ is
$\{\bubblestate{6}{red!50}\bubblestate{6}{red!50}\}$. Initially, all
processes are in state {\bubblestate{1}{green!20}}, i.e.\ $\inits =
{\bubblestate{1}{green!20}}^+$.
