\Section{Model}
\label{section:model}

%% ---------------------------------------
\SubSection{Petri Nets}
%% ---------------------------------------
%\myparagraph{Definition.}
A Petri net $\pn$ is a tuple $\pntuple$ 
where $\pnplaces$ is a finite set of \emph{places},
$\pntransitions$ is a finite set of \emph{transitions}
and $\flowrel\subseteq(\pnplaces\times\pntransitions)\union(\pntransitions\times\pnplaces)$
is the \emph{flow relation}, such that $\pnplaces\intersect\pntransitions=\emptyset$.

%
If $\tuple{\pnplace,\pntransition}\in\flowrel$, $\pnplace$ is said to
be an \emph{input} place of $\pntransition$ and if
$\tuple{\pntransition,\pnplace}\in\flowrel$, $\pnplace$ is said to be
an \emph{output} place of $\pntransition$.
%
We use $\inputof{\pntransition}=\setcomp{\place\in\places}{\tuple{\place,\transition}\in\flowrel}$ and
$\outputof{\pntransition}=\setcomp{\place\in\places}{\tuple{\transition,\place}\in\flowrel}$ to denote the sets of input places and
output places of $\pntransition$ respectively.

%
A \emph{configuration} $\conf$ of a Petri net, often called a
\emph{marking} in the literature, is a multiset over $\pnplaces$ and
represents a valuation of %defines
the number of \emph{tokens} in each place.
%

%\myparagraph{Operational semantics. }
%
The transition system induced by a Petri net consists of the set
configurations together with the transition relation defined on them.
%
The operational semantics of a Petri net is defined through the notion
of \emph{firing} transitions. This gives a transition relation on the
set of configurations.
%
More precisely, a transition fires by removing tokens from its input
places and creating new tokens which are distributed to its output
places. A transition is \emph{enabled} if each of its input places has
at least one token. A transition may fire if it is
enabled. \footnote{Petri nets allow multiple arcs from a place to a
  transition (and vice-versa), where as many tokens are removed (or
  created) when an enabled transition fires. But we will not use that
  construct.}%
%
Formally, when a transition $\pntransition$ is enabled, we write
$\conf\movesto{\pntransition}\conf'$ if $\conf'$ is the result of
firing $\pntransition$ on $\conf$.
%
We define $\movesto{}$ as
$\bigcup_{\pntransition\in\pntransitions}\movesto{\pntransition}$ and
use $\movesto{*}$ to denote the reflexive transitive closure of
$\movesto{}$.
%
For sets $\confs_1$ and $\confs_2$ of configurations, 
we use $\confs_1\movesto{}\confs_2$ to denote that
$\conf_1\movesto{}\conf_2$ for some $\conf_1\in\confs_1$ and $\conf_2\in\confs_2$.
%
% By  $\conf_1\movesto{}\confs_2$ we mean 
% $\set{\conf_1}\movesto{}\confs_2$.
% %
% We define $\conf_1\movesto{*}\confs_2$, $\confs_1\movesto{*}\confs_2$, etc in a similar manner to above.


%% ---------------------------------------
\SubSection{Modeling programs using Petri Nets}
%% ---------------------------------------

We model the flow of control in SML programs using Petri Nets.  A
concurrent program contains multiple separate threads of control and
each thread is assigned a particular task, so-called \emph{job type},
modeled by a petri net. SML programs have a finite number of job
types. Multiple instances of a job type will be modeled by multiple
tokens. Note that the structure of the petri net is then static.

Transitions in the petri net correspond to thread statements and
places are used to control the flow and the shared variable
dependencies.
%
This modeling formally captures the concurrency between threads using
the concurrency constructs of a petri net, captures synchronization
between threads (e.g. locks, access to shared variables, condition
variables, ...) using appropriate mechanisms in the net, and
formalizes the fact that data is abstracted in a sound manner.
%
In the following, a place will be represented by a
\tikz{\node[place,minimum size=1ex]{};} and by a
\tikz{\node[place,shared,minimum size=1.1ex]{};} when it is shared. A
transition will be represented as \tikz{\node[transition]{};}.

%% -------------------------------
\myparagraph{Reading and Writing a shared variable.}
%% -------------------------------
A shared variable $v$ is associated with two places,
$\LD{v}$ and $\ST{v}$. A thread places a token in $\LD{v}$
(resp. $\ST{v}$) if it is currently accessing the variable $v$ for
reading (resp. writing). We model read and write accesses to shared
variables with two transitions.

\noindent%
\begin{tikzpicture}[scale=0.9,show background rectangle,every place/.append style={scale=0.9}]
  \node[place,label=left:$in$] (in) at (0,0) {};
  \node[place](temp) at(0,-2){};
  \node[place,label=left:$out$](out) at(0,-4){};
  \node[place,shared,label=below:$\LD{x}$](Rx) at (2,-2){};
  \node[transition](t1) at (0,-1){} edge [pre] (in) edge [post] (temp);
  \node[transition](t2) at (0,-3){} edge [pre] (temp) edge [post] (out);
  \draw[post] (t1.-15) to[out=-90,in=165] (Rx);  
  \draw[pre] (t2.15) to[out=90,in=-165] (Rx);  

  \begin{pgfonlayer}{my background}
    \node[dottedrectangle,fit=(t1) (temp) (t2)] (r){};
  \end{pgfonlayer}

\end{tikzpicture}
\begin{tikzpicture}[scale=0.9,show background rectangle,every place/.append style={scale=0.9}]
  \node[place,label=right:$in$] (in) at (0,0) {};
  \node[place](temp) at(0,-2){};
  \node[place,label=right:$out$](out) at(0,-4){};
  \node[place,shared,label=above:$\ST{x}$](Rx) at (-2,-2){};
  \node[transition](t1) at (0,-1){} edge [pre] (in) edge [post] (temp);
  \node[transition](t2) at (0,-3){} edge [pre] (temp) edge [post] (out);
  \draw[post] (t1.-165) to[out=-90,in=15] (Rx);  
  \draw[pre] (t2.165) to[out=90,in=-15] (Rx);  

  \begin{pgfonlayer}{my background}
    \node[dottedrectangle,fit=(t1) (temp) (t2)] (r){};
  \end{pgfonlayer}

\end{tikzpicture}

%% -------------------------------
\myparagraph{Acquiring and releasing a lock.}
%% -------------------------------
There is a place $\pnplace[L]$ associated with each lock. Intuitively,
if $\pnplace[L]$ contains a token, the lock is free, otherwise it is
busy. This ensures that only one thread can hold the lock at a
time. Note that $\pnplace[L]$ is a global variable.

\noindent%
\begin{tikzpicture}[scale=0.9,show background rectangle,every place/.append style={scale=0.9}]
  \node[place,label=left:$in$] (in) at (0,0) {};
  \node[place,shared,label=right:$L$](L) at (2,0){};
  \node[place,label=left:$out$](out) at(0,-2){};

  \node[transition,label=right:$lock$](trans) at (0,-1){} edge [pre] (in) edge [post] (out);
  \draw[pre] (trans.15) to[out=90,in=-165] (L);  
\end{tikzpicture}
\begin{tikzpicture}[scale=0.9,show background rectangle,every place/.append style={scale=0.9}]
  \node[place,label=left:$in$] (in) at (0,0) {};
  \node[place,shared,label=right:$L$](L) at (2,-2){};
  \node[place,label=left:$out$](out) at(0,-2){};

  \node[transition,label=right:$unlock$](trans) at (0,-1){} edge [pre] (in) edge [post] (out);
  \draw[post] (trans.-15) to[out=-90,in=165] (L);  
\end{tikzpicture}

%% -------------------------------
\myparagraph{Waiting on a condition variable.}
%% -------------------------------
A condition variable is modeled with 3 places, namely $\pnplace[cond]$,
$\pnplace[signal]$ and $\pnplace[ready]$, and 3 transitions as follows.

\noindent%
\begin{tikzpicture}[scale=0.9,show background rectangle,every place/.append style={scale=0.9}]

  \node[place,label=left:$in$] (win)   at (1,3)  {};
  \node[place,shared,label=above:$L$]  (L)    at (-1,0)  {};
  \node[place,label=left:$out$](wout)  at (1,-2) {};
  \node[place]                 (temp) at (1,0) {};
  \node[place,shared](cond)  at (4,1.5) {};
  \node[place,shared](wakeUp)  at (5.2,1.5) {};
  \node[place,shared](ready)  at (4.6,-0.5) {};

  \node[above=3pt,inner sep=0pt](condLabel) at (cond.120) {$cond$};
  \node[above=1pt,inner sep=0pt](signalLabel) at (wakeUp.60) {$signal$};
  \node[right=1pt,inner sep=0pt](readyLabel) at (ready.east) {$ready$};

  \node[transition,label={[scale=0.8]left:$win$}](twin) at (1,2){} edge [pre] (win) edge [post] (temp);
  \node[transition,label={[scale=0.8]left:$wout$}](twout) at (1,-1){} edge [pre] (temp) edge [post] (wout);
  \node[transition,label={[scale=0.8]left:$tr$}](tready) at (4.6,0.5){} edge [post] (ready);

  \draw[post] (twin.-165) to[out=-90,in=15] (L);
  \draw[post] (twin.-15) to[out=-90,in=180] (cond);  
  \draw[pre] (twout.165) to[out=90,in=-15] (L);
  \draw[pre] (twout.15) to[out=90,in=-180] (ready);  

  \draw[pre] (tready.165) to[out=90,in=-90] (cond);
  \draw[pre] (tready.15) to[out=90,in=-90] (wakeUp);

  \node(cheating) at (6.5,0) {};
  \node(cheating2) at (-2,0) {};

  \begin{pgfonlayer}{my background}
    \node[dottedrectangle,fit=(cond) (wakeUp) (ready) (condLabel) (signalLabel) (readyLabel), inner sep=0pt] (r){};
    \node[below,scale=0.7] at (r.south) {Condition Variable};
    \node[dottedrectangle,fit=(temp) (twin) (twout)]{};
  \end{pgfonlayer}
\end{tikzpicture}

\noindent%
The transition $win$ releases the lock and places a token in the
$\pnplace[cond]$ place. The thread is then blocked since neither $wout$
nor $tr$ are enabled. If another thread places a token in
$\pnplace[signal]$, $tr$ is enabled. If $tr$ fires, the blocked thread
is ready to fire $wout$ to resume execution, with an eventual delay
for (re-)acquiring the lock.
% (Another thread might have indeed acquired the lock while this
% thread was waiting on the condition variable).

\pagebreak
%% -------------------------------
\myparagraph{Signaling a condition variable.}
%% -------------------------------
A thread can wake up another blocked thread by placing a token in the
$\pnplace[signal]$ place.

\noindent%
\begin{tikzpicture}[scale=0.9,show background rectangle,every place/.append style={scale=0.9}]

  \node[place,shared](cond)  at (3,1) {};
  \node[place,shared](wakeUp)  at (4,1) {};
  \node[place,shared](ready)  at (3.5,-1) {};

  \node[above=3pt,inner sep=0pt](condLabel) at (cond.120) {$cond$};
  \node[above=1pt,inner sep=0pt](signalLabel) at (wakeUp.60) {$signal$};
  \node[right=1pt,inner sep=0pt](readyLabel) at (ready.east) {$ready$};

  \node[transition,label={[scale=0.8]left:$tr$}](tready) at (3.5,0){} edge [post] (ready);
  \draw[pre] (tready.165) to[out=90,in=-90] (cond);
  \draw[pre] (tready.15) to[out=90,in=-90] (wakeUp);

  \node[place,label=right:$in$](sin)  at (8,2) {};
  \node[place,label=right:$out$](sout)  at (8,0) {};
  \node[transition,label={[scale=0.8]right:$ts$}](tsignal) at (8,1){} edge [pre] (sin) edge [post] (sout);
  \draw[post] (tsignal.-165) to[out=-90,in=0] (wakeUp);

  \node[transition,rotate=30,scale=0.7,label={[scale=0.8]right:$conflict$}](dead) at (5.7,-1){};
  \draw[pre] (dead) to[in=-30,out=120] (wakeUp.-30);

  \node(cheating) at (9.5,0) {};
  \node(cheating2) at (1,0) {};

  \begin{pgfonlayer}{my background}
    \node[dottedrectangle,fit=(cond) (wakeUp) (ready) (condLabel) (signalLabel) (readyLabel), inner sep=0pt] (r){};
    \node[below,scale=0.7] at (r.south) {Condition Variable};
  \end{pgfonlayer}

\end{tikzpicture}

\noindent%
Signaling a condition variable in the model while no thread is waiting
on it should have no effect. That is to say, the token in
$\pnplace[signal]$ should be consumed if the $\pnplace[cond]$ place is
empty. To alleviate the problem, we use a $\pntransition[conflict]$
transition. Recall that an enabled transition only \emph{may} fire. As
it is not possible to model the firing of a transition based on the
condition that a place is empty, we introduce an abstraction where
signals might be lost (i.e.\ signaling a condition variable might not
wake up other threads which are waiting on that condition variable).
%
Nevertheless, it is an over-approximation so we do not bias
correctness.
%\daz{rewrite that last sentence better}

%% -------------------------------
\myparagraph{Branching.}
%% -------------------------------
The \emph{if}, \emph{if-then-else} and \emph{while} statements are
easily implemented with \emph{branch} and \emph{goto}. We therefore
only model those latter.

\noindent%
\begin{tikzpicture}[scale=0.9,show background rectangle,every place/.append style={scale=0.9}]

  \node[place,label=left:$in$](in)  at (0,0) {};
  \node[place,label=left:$out_1$](out1)  at (-2,-2) {};
  \node[place,label=right:$out_2$](out2)  at (2,-2) {};

  \node(cheating) at (-4,0) {};
  \node(cheating2) at (4,0) {};

  \node[transition,label={[scale=0.8]left:$if\ true$}](btrue) at (-2,-1){} edge [pre] (in) edge [post] (out1);
  \node[transition,label={[scale=0.8]right:$if\ false$}](bfalse) at (2,-1){} edge [pre] (in) edge [post] (out2);

\end{tikzpicture}

\noindent%
Since we abstract away all data, we cannot determine the branching
based on a predicate evaluation. So we use an abstraction where the
program takes both branches and model it with two transitions. This
eventually introduces false behaviours, and increase the number of
false positives. Nevertheless, it is again an over-approximation and
does not bias the correctness argument. If, for example, the predicate
in the while loop evaluates while reading some shared variable
(e.g. $while(x<10)$), we would model it as a cascade of a read
instruction followed by a branch.
%
%\daz{write that better again.}


%% Cheating the formatting
\vspace{6cm}

%% -------------------------------
\myparagraph{Goto or jump.} 
%% -------------------------------
This is only introduced as a convenience to work in conjunction with
\emph{branch} and model the control flow.

\noindent%
\begin{tikzpicture}[scale=0.9,show background rectangle,every place/.append style={scale=0.9}]

  \node[place,label=left:$in$](in)  at (0,0) {};
  \node[place,label=right:$out$](out)  at (3,0) {};
  \node[transition,rotate=90](goto) at (1.5,0){} edge [pre] (in) edge [post] (out);

\end{tikzpicture}

%% -------------------------------
\myparagraph{Creating a new thread.}
%% -------------------------------

\noindent%
\begin{minipage}{0.59\linewidth}
  A new thread is associated with a job type and an initial
  place. Firing the $create$ transition produces a new token that we
  place in the initial place of the new job type. The calling thread
  will continue its execution. Note that $same$ and $new$ place can
  coincide. It would be easy to extend the transitions to multiple
  arcs.
\end{minipage}
\begin{minipage}{0.4\linewidth}
\hfill%
\begin{tikzpicture}[scale=0.9,show background rectangle,every place/.append style={scale=0.9}]

  \node[place,label=left:$in$](in)  at (0,0) {};
  \node[place,label=below:$same$](out1)  at (-1,-2) {};
  \node[place,label=below:$new$](out2)  at (1,-2) {};

  \node[transition,label={[scale=0.8]right:$create$}](t) at (0,-1){} edge [pre] (in);
  \draw[post] (t.-165) to[out=-90,in=15] (out1);
  \draw[post] (t.-15) to[out=-90,in=165] (out2);
\end{tikzpicture}
\end{minipage}%


%% -------------------------------
\SubSection{An example}
%% -------------------------------

As the petri net of a concurrent program written in the SML language
can quickly grow in size, we only show a short example presented in
Figure~\ref{fig:modelExampleCS}, which models the program from
Figure~\ref{fig:languageExampleCS}.

\begin{figure}[h]
  \begin{tikzpicture}[scale=0.9,show background rectangle,every place/.append style={scale=0.9}]

    \node(cheating) at(-2,0){};

    \node[place,label=right:$init$,tokens=2](in)  at (4,0) {};
    \node[place](lockOut) at (0,0) {};
    \node[place](temp1) at (0,-2) {};
    \node[place](readOut) at (0,-4) {};
    \node[place](temp2) at (0,-6) {};
    \node[place](writeOut) at (0,-8) {};
    \node[place,label=right:$end$](unlockOut) at (4,-8) {};

    \node[transition,rotate=90,label={[scale=0.9]above left:$lock$}](tlock) at (2,0){} edge [pre] (in) edge [post] (lockOut);
    \node[transition](treadIn) at (0,-1){} edge [pre] (lockOut) edge [post] (temp1);
    \node[transition](treadOut) at (0,-3){} edge [pre] (temp1) edge [post] (readOut);
    \node[transition](twriteIn) at (0,-5){} edge [pre] (readOut) edge [post] (temp2);
    \node[transition](twriteOut) at (0,-7){} edge [pre] (temp2) edge [post] (writeOut);
    \node[transition,rotate=90,label={[scale=0.9]below right:$unlock$}](tunlock) at (2,-8){} edge [pre] (writeOut) edge [post] (unlockOut);

    \node[place,shared,label=right:$L$,tokens=1] (L)  at (6,-4) {};
    \node[place,shared,label=right:$\LD{counter}$] (Rc)  at (2,-3.5) {};
    \node[place,shared,label=right:$\ST{counter}$] (Wc)  at (2,-4.5) {};

    \draw[pre] (tlock.-165) to[out=0,in=90] (L);
    \draw[post] (tunlock.-15) to[out=0,in=-90] (L);

    \draw[post] (treadIn.-15) to[out=-90,in=120] (Rc.120);
    \draw[pre] (treadOut.15) to[out=90,in=150] (Rc.150);

    \draw[post] (twriteIn.-15) to[out=-90,in=-150] (Wc.-150);
    \draw[pre] (twriteOut.15) to[out=90,in=-120] (Wc.-120);

  \begin{pgfonlayer}{my background}
    \node[dottedrectangle,fit=(Rc) (Wc)] (r){};
    \node[dottedrectangle,fit=(treadIn) (treadOut)] (read){};
    \node[dottedrectangle,fit=(twriteIn) (twriteOut)] (write){};
  \end{pgfonlayer}

  \end{tikzpicture}
  \caption{Modeling the critical section problem with petri nets.}
  \label{fig:modelExampleCS}
\end{figure}
