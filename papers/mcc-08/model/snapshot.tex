\section{Snapshot}

We describe in the section the information we require in order to
depict a snapshot of the system at a given time.\daz{Define system?
  Use \emph{model} instead?}

\smallskip\noindent%
\begin{minipage}{0.49\textwidth}
  \setlength{\parindent}{1em}%
  The system contains multiple separate threads of control and each
  thread is assigned a particular task, so-called \emph{job type},
  represented by a finite state automaton. We can assume, to start
  with, that the system has a finite number of job types.
\end{minipage}
\begin{minipage}{0.5\textwidth}
\begin{center}
\begin{tikzpicture}
    \node[inner xsep=0pt,shade, top color=green!50!blue,text width=2cm, text centered, above, draw](jt1){Job type 1};
    \node[inner xsep=0pt,text width=2cm, below=0mm of jt1, draw, text height=1.6cm](box1){};
    \node[mystate](init1) at ([shift={(7mm,4mm)}]box1.south west){};
    \node[mystate=red,above=5mm of init1](a1){};
    \node[mystate=blue,right=5mm of a1](b1){};
    \node[mystate=green,below=5mm of b1](c1){};
    \draw[transition,->] (init1.120) to [out=120,in=-120] (a1.-120);
    \draw[transition,->] (a1.60) to [out=60,in=120] (b1.120);
    \draw[transition,->] (b1.-30) to [out=-30,in=60] (c1.60);
    \draw[transition,->] (a1.west) .. controls +(left:5mm) and +(up:5mm) .. (a1.north);
    %
    \node(dots) at (15mm,0){...};
    %
    \node[inner xsep=0pt,shade, top color=green!50!blue,text width=2cm, text centered, above, draw](jt2) at (2.8cm,0){Job type 2};
    \node[inner xsep=0pt,text width=2cm, below=0mm of jt2, draw, text height=1.6cm](box2){};
    \node[mystate](init2) at ([shift={(7mm,4mm)}]box2.south west){};
    \node[mystate=green,above=5mm of init2](a2){};
    \node[mystate=yellow,right=5mm of a2](b2){};
    \node[mystate=orange,below=5mm of b2](c2){};
    \draw[transition,->] (init2.120) to [out=120,in=-120] (a2.-120);
    \draw[transition,->] (a2.60) to [out=60,in=120] (b2.120);
    \draw[transition,->] (b2.-30) to [out=-30,in=60] (c2.60);
    \draw[transition,->] (init2.60) to [out=60,in=-150] (b2.-150);
    % 
\end{tikzpicture}
\end{center}
\end{minipage}

\smallskip%
The system (ie the whole program) is associated with a set of shared
variables, so-called \emph{Globals}. The Globals might be read and
written by any threads.
%
A thread has a unique id and is associated with a set of local
variables, so-called \emph{Locals} (such as its stack).

\bigskip%
The communication between the threads is achieved by means of shared
memory. Synchronization (i.e. constraining all possible thread
inter-leavings to only the desirable ones) is achieved by means of
mutexes and condition variables. A mutex is a simple lock with 2
states: taken or free. Any thread requesting to take the lock when the
lock is taken will wait until the lock is freed. Condition variables
represent queues of all the sleeping threads on an associated
predicate.

\smallskip\noindent%
\begin{minipage}{0.7\textwidth}
  \setlength{\parindent}{1em}%
  Each shared variable is associated with a set of pairs, so-called
  \emph{access set}. A pair represents the thread id accessing that
  variable and a token corresponding to the type of access: either
  read (R) or write (W). If no thread is accessing the variable, the
  access set is empty.
\end{minipage}
%
% Right part
%
\hfill%
\begin{minipage}{0.29\textwidth}
\begin{center}
  \begin{tikzpicture}
    \node[fill=yellow,inner sep=1pt,circle](x){X};
    \node[ellipse,fill=yellow,minimum width=25mm, minimum height=10mm](set) at +(10mm,-15mm){};
    \filldraw[yellow] (x.-130) to[out=-50,in=20] (set.160) -- (set.90) to[out=180,in=-100] (x.-10) -- (x.-40) -- (x.-90);
    
    \node[pair,scale=0.7,left=3mm of set.center]{T$_i$\nodepart{lower}R};
    \node[pair,scale=0.7,right=-1mm of set.center]{T$_j$\nodepart{lower}W};
    \node[right=7mm of set.center]{...};
  \end{tikzpicture}
\end{center}
\end{minipage}

\smallskip%
A thread can voluntarily stop its activity and resume it only when
another thread terminates (either normally or abnormally).  An
\emph{on-termination} association is set up and the sleeping thread
gets notified and awaken when the awaited thread terminates.


\bigskip%
So, a snapshot of the system would be depicted as follows: the Globals
and a collection of thread with a given job type, a current state and
Locals, with eventual on-termination associations. The Globals
contains %the thread ids,
the mutexes, condition variables and all shared variables, with their
respective access set.


\bigskip%\noindent%
\begin{tikzpicture}[font=\scriptsize]
    \node[shade, top color=blue,text width=2cm, text centered, above, draw](jt1){Thread: Type 1};
    \node[text width=2cm, below=0mm of jt1, draw](loc1){Locals};
    \node[text width=2cm, below=0mm of loc1, draw, text height=2cm](box1){};
    \node(init1) at ([shift={(7mm,7mm)}]box1.south west){};
    \node[above=5mm of init1](a1){};
    \node[mystate=blue,right=5mm of a1](b1){};
    \node[below=5mm of b1](c1){};
    %
    %
    \node[shade, top color=blue,text width=2cm, text centered, above, draw](jt2) at (4cm,-2cm){Thread: Type 2};
    \node[text width=2cm, below=0mm of jt2, draw](loc2){Locals};
    \node[text width=2cm, below=0mm of loc2, draw, text height=2cm](box2){};
    \node(init2) at ([shift={(7mm,7mm)}]box2.south west){};
    \node[above=5mm of init2](a2){};
    \node[right=5mm of a2](b2){};
    \node[mystate=orange,below=5mm of b2](c2){};
    % 
    % 
    %\draw[dashed,->,shorten <=1mm,shorten >=1mm] (jt1.east) -- (jt2.west) node[midway,above,sloped]{child};
    \draw[dotted,<-,shorten <=1mm,shorten >=1mm] (loc1.east) -- (loc2.west) node[sloped,below,pos=0.6]{on-termination};
    %
    % 
    \node(dots) at (4cm,-1cm){...};

    \begin{scope}
      [xshift=7.2cm,
      every text node part/.style={font=\tiny},
      every second node part/.style={font=\tiny},
      every third node part/.style={font=\tiny},
      every fourth node part/.style={font=\tiny}
      ]
      \node[above,rounded corners, shade, top color=blue!50!red,inner sep=2mm]{Globals}; 
      \node[below,rounded corners, rectangle split, rectangle split parts=3, draw, text width=3.2cm] 
      {
        Shared variables: X, Y, Z, ... 
        \nodepart{second} 
        Mutexes: L$_1$, L$_2$, L$_3$, ...
        \nodepart{third}
        Condition Variables:\\
        Q$_1$, Q$_2$, Q$_3$, ... 
      }; 
    \end{scope}

    \begin{scope}
      [xshift=8.2cm,yshift=-0.2cm,
      ] 
      \node[fill=yellow,inner sep=1pt,circle](z){Z};
      \node[ellipse,fill=yellow,minimum width=25mm, minimum height=15mm](set) at +(0,-25mm){};
      \filldraw[yellow] (z.-110) to[out=-75,in=30] (set.160) -- (set.20) to[out=120,in=-90] (z.east) -- (z.south);

      \node[pair,left=3mm of set.center]{T$_i$\nodepart{lower}R};
      \node[pair,right=-1mm of set.center]{T$_j$\nodepart{lower}W};
      \node[right=7mm of set.center]{...};
    \end{scope}

\end{tikzpicture}

