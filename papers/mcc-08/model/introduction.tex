\section{Introduction}

A race condition is a situation in a shared-variable concurrent
program in which one process changes a variable that another process
has previously read and that other process doesn't get notified of the
change.\daz{or just doesn't see the change}

A process can, for example, write a variable that a second process
reads, but the first process continues execution -- namely races ahead
-- and changes the variable again before the second process sees the
result of the first change. Another example: when a process checks a
variable and takes action based on the content of the variable, it is
possible for another process to ``sneak in'' and change the variable
in between the check and the action in such a way that the action is
no longer appropriate.

Alternatively, one can define a race condition as the possibility of
incorrect results in the presence on unlucky timing in concurrent
programs, that is, getting the right answer relies on lucky timing.

\bigskip%
Race condition are of particular interest because they can lead to
some rather devious bugs. These bugs are extremely hard to track since
they are non-deterministic and difficulty reproducable. The kind of
errors caused by race condition are very subtle and often manifest
themselves in the form of corrupted or incorrect variable
data. Unfortunately, if often means that the error won't crash the
system immediately, but will wait until some other code is executed
and which relies on the data to be correct, making the process of
locating the original race condition even more difficult.

To avoid data corruption or incorrectness, the programmer uses
synchronization techniques to constraint all possible process
inter-leavings to only the desirable ones. Race condition usually
lead to an incorrectly synchronized program.

A precise way to detect race conditions would be to search the state
space (of a \emph{model} of the program) for a state where multiple
threads try to access and change the same variable.  The technique is
sound, but more importantly it seems to be much more precise than
other techniques (static analysis, dynamic detection, ...). It indeed
detects the race conditions themselves instead of detecting violations
of the locking discipline that can be used to prevent race conditions.
