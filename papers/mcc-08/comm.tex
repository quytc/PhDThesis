% put this in the beginning of a paragraph and Latex will
% automatically try to make it one line shorter, if possible

\newcommand{\X}{\looseness -1}

% - ignore command

\newcommand{\ignore}[1]{}

%% Having a note in the margin. Notes are numbered.
%% ---------------------------------------------
\newcommand{\NoteComment}[2]{%
  \stepcounter{NoteCounter#1}%
  {\scriptsize\bf$^{(\arabic{NoteCounter#1})}$}%
  \marginpar
  %[\fbox{
  %   \parbox{2cm}{\raggedleft
  %       \footnotesize$^{({\bf{\arabic{NoteCounter#1}{#1}}})}$%
  %       \footnotesize #2}}]%
  {\fbox{\parbox{25mm}{\raggedright 
      \footnotesize$^{({\bf{\arabic{NoteCounter#1}{#1}}})}$%
      \footnotesize #2}}}
}
\newcounter{NoteCounter}
\newcommand{\Note}[1]{\NoteComment{}{#1}}
% \renewcommand{\Note}[1]{} %% Uncomment to remove the comment
\newcommand{\parosh}[1]{\Note{Parosh: #1}}
\newcommand{\daz}[1]{\Note{Fred: #1}}

%% Local note as footnotes.
%% ---------------------------------------------
\newcommand{\localnote}[1]{\footnote{#1}}
\renewcommand{\localnote}[1]{}
%% Uncomment to remove the comment footnote


%% ---------------------------------------------

\renewcommand{\daz}[1]{\localnote{Fred: #1}}

\newcommand{\myparagraph}[1]{\noindent\textbf{#1}}
%\renewcommand{\myparagraph}[1]{\paragraph{#1}}


\newcommand{\intersect}{\ensuremath{\cap}}
\newcommand{\Intersect}{\ensuremath{\bigcap}}
\newcommand{\union}{\ensuremath{\cup}}
\newcommand{\Union}{\ensuremath{\bigcup}}

\newcommand{\tuple}[1]{\ensuremath{\left(#1\right)}}
\newcommand{\set}[1]{\ensuremath{\left\{#1\right\}}}
\newcommand{\setcomp}[2]{\ensuremath{\set{#1|\;#2}}}

\newcommand{\LD}[1]{\ensuremath{Read_{#1}}}
\newcommand{\ST}[1]{\ensuremath{Write_{#1}}}

%% ================================
%% Parosh
%% ================================

% multisets, words, etc
\newcommand{\msets}[1]{{#1}^{\circledast}}
\newcommand{\lst}[1]{\left[#1\right]}
\newcommand{\mset}{M}
\newcommand{\emptystring}{\varepsilon}
\newcommand{\emptyword}{\emptystring}
\newcommand{\word}{w}
\newcommand{\append}{\cdot}
\newcommand{\wordering}{\preceq}
\newcommand{\chordering}{\wordering}
\newcommand{\lastof}[1]{{\it last}\left(#1\right)}

% ordering
\newcommand{\uclosed}{U}
\newcommand{\uc}[1]{\widehat{#1}}
\newcommand{\genof}[1]{{\it gen}\left(#1\right)}
\newcommand{\subsumes}{\sqsubseteq}

% PN
\newcommand{\pn}{\ensuremath{{\mathcal N}}}
\newcommand{\pnplaces}{P}
\newcommand{\place}{p}
\newcommand{\places}{P}
\newcommand{\pnplace}[1][p]{#1}
\newcommand{\pntransition}[1][t]{#1}
\newcommand{\pntransitions}{T}
\newcommand{\flowrel}{F}
\newcommand{\inputof}[1]{I\left(#1\right)}
\newcommand{\outputof}[1]{O\left(#1\right)}
\newcommand{\pntuple}{\tuple{\pnplaces,\pntransitions,\flowrel}}

% problem definitions

\newcommand{\acovproblem}{{\sc APRX-RACE-COV}}
\newcommand{\covproblem}{{\sc RACE-COV}}

\newcounter{probboxcounter}
\setcounter{probboxcounter}{0}
\newcommand{\probbox}[3]{%
  \vspace{3mm}
  \noindent
  \framebox{
    \begin{minipage}{0.95\hsize}
      \noindent\underline{#1}

      \noindent{\bf Instance:}

      \begin{itemize}
        #2
      \end{itemize}

      \noindent{\bf Question:}
      #3
    \end{minipage}
  }
  \vspace{2mm}
}

% transition systems
\newcommand{\ts}{{\mathcal T}}
\newcommand{\tstuple}{\tuple{\confs,\movesto{},\ordering,\initconfs}}
\newcommand{\conf}{c}
\newcommand{\confs}{C}
\newcommand{\movesto}[1]{\stackrel{#1}{\longrightarrow}}
\newcommand{\bad}{{\it Bad}}
\newcommand{\finalconfs}{\confs_{\it fin}}
\newcommand{\initconf}{\conf_{\it init}}
\newcommand{\initconfs}{\confs_{\it init}}
\newcommand{\pre}{{\it Pre}}
\newcommand{\bmovesto}[1]{\stackrel{#1}{\leadsto}}
\newcommand{\ordering}{\preceq}
\newcommand{\tstriple}{\tuple{\confs,\movesto{},\ordering}}
\newcommand{\worklist}{{\tt WorkList}}
\newcommand{\explored}{{\tt Explored}}
\newcommand{\state}{s}
\newcommand{\initstate}{\state_{{\it init}}}
\newcommand{\states}{S}
\newcommand{\transition}{t}
\newcommand{\transitions}{T}
\newcommand{\rank}{{\it Rank}}

%% TikZ styles
\tikzset{background rectangle/.style={rounded corners,inner sep=1pt,bottom color=red!20,top color=white}}
% Place and transition for the petri nets
\tikzset{every place/.style={draw=blue,fill=blue!20,thick}}
\tikzset{shared/.style={draw=green,fill=green!30!gray,thick}}
\tikzset{every transition/.style={draw=red,fill=red!20,thick,minimum width=10mm,minimum height=1mm}}

\pgfdeclarelayer{my background} 
\pgfsetlayers{background,my background,main}

\tikzset{dottedrectangle/.style={rounded corners,draw,dotted,inner sep=0pt}}

%% --------------------------------------------------------------------------------------------------
\lstset{ %
  language=C,                     % choose the language of the code
  basicstyle=\scriptsize,         % the size of the fonts that are used for the code
  numbers=left,                   % where to put the line-numbers
  numberstyle=\tiny,              % the size of the fonts that are used for the line-numbers
  stepnumber=0,                   % the step between 5 line-numbers. If it's 1 each line will be numbered
%  numbersep=15pt,                 % how far the line-numbers are from the code
%  backgroundcolor=\color{white},  % choose the background color. You must add \usepackage{color}
  showspaces=false,               % show spaces adding particular underscores
  showstringspaces=false,         % underline spaces within strings
  showtabs=false,                 % show tabs within strings adding particular underscores
 % frame=single,                  % adds a frame around the code
%  tabsize=4,                      % sets default tabsize to 2 spaces
%  captionpos=b,                   % sets the caption-position to bottom
  breaklines=true,                % sets automatic line breaking
  breakatwhitespace=false,        % sets if automatic breaks should only happen at whitespace
  escapeinside={\%*}{*)}          % if you want to add a comment within your code
}
