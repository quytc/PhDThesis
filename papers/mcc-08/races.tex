%% ----------------------------------------------
\Section{Race-Freeness}
\label{section:races}

\SubSection{Bad states and ordering} 

\noindent%
\begin{minipage}{0.6\linewidth}
  A data race occurs when multiple threads access a shared variable
  and at least one has the intention of changing it, without any
  synchronization constraints (i.e. event ordering). Consequently,
  given a variable $v$, we notice that a bad state (i.e.\ a race on
  $v$) is one configuration that contains at least two tokens in
  either $\LD{v}$ or $\ST{v}$ with one of them in the $\ST{v}$ place.
\end{minipage}
\begin{minipage}{0.39\linewidth}
  \hfill%
  \begin{tikzpicture}
    [scale=0.9,show background rectangle,every place/.append style={scale=0.9}]

    \begin{scope}
      \node[place,shared,label=above:$\LD{v}$,tokens=1] (Rx) at (0,0) {};
      \node[place,shared,label=above:$\ST{v}$,tokens=1] (Wx) at (1.5,0) {};
      \begin{pgfonlayer}{my background}
        \node[dottedrectangle,fit=(Rx) (Wx)](r){};
      \end{pgfonlayer}
    \end{scope}

    \begin{scope}[shift={(0,-2)}]
      \node[place,shared,label=below:$\LD{v}$] (Rx2) at (0,0) {};
      \node[place,shared,label=below:$\ST{v}$,tokens=2] (Wx2) at (1.5,0) {};
      \begin{pgfonlayer}{my background}
        \node[dottedrectangle,fit=(Rx2) (Wx2)](r2){};
      \end{pgfonlayer}
    \end{scope}

     % the horizontal line
    \node (b) at (barycentric cs:Rx=1,Wx=1,Rx2=1,Wx2=1) {and};
    \draw (b) -- +(-1,0) (b) -- +(1,0);

  \end{tikzpicture}
\end{minipage}

We therefore introduce the following (partial) pre-order $\preceq$ on
configurations.  For two configurations $\conf$ and $\conf'$, we say
that $\conf\preceq\conf'$ if all the places in $\conf'$ contain more
tokens than the respective places in $\conf$. That is, we can obtain
$\conf$ by removing tokens from $\conf'$.

For a configuration $\conf$, we use $\uc{\conf}$ to denote the
\emph{upward closure} of $\conf$, i.e.\
$\uc{\conf}=\setcomp{\conf'}{\conf\preceq\conf'}$.
%
For a set $\confs$ of configurations, we define $\uc{\confs}$ as
$\bigcup_{\conf\in\confs}\uc{\conf}$.

Upward closed sets are attractive to use because they can be
characterized by their minimal elements, which often makes it possible
to have efficient symbolic representations of infinite sets of
configurations.


\SubSection{Safety property}

A set $\confs$ of configurations is said to be \emph{reachable} if
$\initconfs\movesto{*}\confs$.
%
Checking the safety property amounts to performing reachability
analysis: is it possible to reach a bad configuration from an initial
configuration through a sequence of actions performed by different
threads?
%
It can be shown using standard
techniques~\cite{VW:modelchecking,GoWo:safety}, that checking safety
properties can be translated into instances of the following coverability
problem.

\probbox{Coverability}
{%
  \item A set of initial configurations $\initconfs$.
  \item An upward closed set of bad configurations $\finalconfs$.
}{%
Is $\uc{\finalconfs}$ reachable? (i.e.\ $\initconfs\movesto{*}\uc{\finalconfs}$ ?)
}

The main idea is to perform symbolic backward reachability analysis,
using upward closed sets, to check for race-freeness for concurrent
programs written in the SML language. A negative answer guarantees the
absence of races in the program.
