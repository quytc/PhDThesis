%% ------------------------------------------------------------------------- 
\Section{Introduction}
%% ------------------------------------------------------------------------- 
\label{section:introduction}

The behaviours of concurrent (or multi-threaded) programs are highly
nontrivial and hard to predict. It is important to develop rigorous
methods to verify their correctness. It is now also widely accepted
that verification methods should be \emph{automatic}.
%
This would allow engineers to perform verification, without needing to
be familiar with the complex constructions and algorithms behind the
tools.
%

In this paper, we will concentrate on a particular
approach to  verification of concurrent programs, 
namely that of \emph{model checking}~\cite{CES:modelchecking,QuSi:cesar}.
%
The aim of model checking is to provide an algorithmic solution
to the verification problem.
%
Concurrent programs involve several complex features which often give
rise to infinite state spaces.
%
First, there is no bound \emph{a priori} on the
number of threads which may arise in a given run of the system.
%
In addition, each thread manipulates local variables which often
range over unbounded domains.
%
Furthermore, a system has a dynamic configuration in the sense that
threads can be created and terminated throughout execution of the
system.
%

%% -------------------------
We concentrate on checking a particular class of properties for
concurrent programs, namely \emph{safety properties}.
%
Intuitively, a safety property states that ``nothing bad will
ever occur during the execution of the system''.
%
In particular, we focus on \emph{race-freeness}, i.e. the absence of
race conditions (also called data races) in shared-variable concurrent
programs.
%

%% -------------------------
A race condition is a situation in which one process changes a
variable which another process has previously read and the other
process does not get notified of the change.
%
A process can, for example, write a variable that a second process
reads, but the first process continues execution -- namely races ahead
-- and changes the variable again before the second process sees the
result of the first change. Another example: when a process checks a
variable and takes action based on the content of the variable, it is
possible for another process to ``sneak in'' and change the variable
in between the check and the action in such a way that the action is
no longer appropriate.
%
Alternatively, one can define a race condition as the possibility of
incorrect results in the presence on unlucky timing in concurrent
programs, that is, getting the right answer relies on lucky timing.

%% -------------------------
%\bigskip%
Race conditions are of particular interest because they can lead to
rather devious bugs. These bugs are extremely hard to track since they
are non-deterministic and difficult to reproduce. The kind of errors
caused by race condition are very subtle and often manifest themselves
in the form of corrupted or incorrect variable data. Unfortunately, it
often means that the error will not harm the system immediately, but
it will manifest itself when some other code is executed, which relies
on the data to be correct. This makes the process of locating the
original race condition even more difficult.
%
To avoid data corruption or incorrectness, the programmer uses
synchronization techniques to constraint all possible process
interleavings to only the desirable ones. Race condition usually
unveil incorrectly synchronized program.

%% -------------------------
\paragraph{\bf Related Work.}
Existing race checkers fall into three main categories: on-the-fly,
ahead-of-time and post-mortem tools. They exhibit different strengths
and can perform race detection, while our method focuses at the moment
on race-freeness. The ahead-of-time approach encompasses static
analysis and compile-time heuristics, while on-the-fly approaches are
by nature dynamic. The post-mortem approach is a combination of static
and dynamic techniques.

Type-based solutions provide a strong assurance to the programmer, in
addition to the familiarity of a compiler-based
solution~\cite{FQ03}. If a program type-checks, it is guaranteed to be
race-free. It is however necessary to examine the source code as a
whole in order to draw conclusion about to find relations between the
locks and the shared data that they protect. This is often alleviated
by requiring annotations from the programmer, as to whether a function
has an effect clause or not, such as ``the caller must hold lock
L''. Moreover, this only forces a programming discipline and it can
also disallow some race-free programs.

Dynamic tools visit only feasible paths and have an accurate view of
the values of the shared data, while static tools must be
conservative. They are based on two main approaches: the lockset and
the happens-before approaches. The lockset algorithm works on the
assumption that shared variables should be protected by an appropriate
lock and any failure to comply to that discipline is reported. Lockset
tools tend to report many false positives. This technique has been
first implemented in Eraser~\cite{SBNSA07:Eraser}. The happens-before
technique~\cite{Lamport:ordering} watches for any accesses of shared
variables that do not have an implied ordering between them. This
technique is highly dependant on the actual thread execution ordering,
and instrumentation might bias the analysis. Moreover, it only reports
a subset of all race conditions (however real ones). These two main
approaches are often combined to get the best of both worlds, e.g.\
MultiRace and RaceTrack~\cite{YRC05:RaceTrack,PS03:MultiRace}.

Model checkers, such as Verisoft, Bandera and
KISS~\cite{HJM04:ContextInference, D03:Verisoft, QW04:Kiss,
  CDHR00:Bandera}, check concurrent programs with a finite and fixed
number of threads. They often require a user supplied abstraction.

A precise way to detect race conditions would be to search the state
space (of a \emph{model} of the program) for a state where multiple
threads try to access and change the same variable.  The technique is
sound (i.e. being able to prove the race-freeness of a concurrent
program), but more importantly it seems to be much more precise than
the other techniques. It indeed detects the race conditions themselves
instead of detecting violations of the locking discipline that can be
used to prevent race conditions.

%% -------------------------
\paragraph{\bf Verification Method.}
We first construct a model for concurrent programs, where each
configuration (snapshot) of the system is represented by a petri
net. The transitions in the petri net represent the instructions of
the program. The configuration of the system changes dynamically
during its execution, by firing transitions in the petri net. The
system starts running from an \emph{initial configuration} and we
characterize a set of \emph{bad configurations} which violate the
given safety property (i.e.\ configurations which should not occur
during the execution of the system).
%
%
Checking the safety property amounts then to performing
\emph{reachability analysis}: Is it possible to reach a bad
configuration from an initial configuration through a sequence of
actions performed by different threads?

%% -------------------------
We will follow a particular methodology which we have earlier developed
for model checking general classes of infinite-state systems
 \cite{Parosh:Bengt:Karlis:Tsay:general}.
%
The method is designed to be applied for verification of
safety properties for infinite-state systems 
which are \emph{monotonic} w.r.t. a \emph{well quasi-ordering} on the set
of configurations.
%
The main idea is to perform symbolic backward reachability analysis
to check safety properties for such systems.
%

\paragraph{\bf Outline.}
We present in Section~\ref{section:language} and
Section~\ref{section:model} the type of programs we consider and how
we extract a model from them. In Section~\ref{section:races}, we state
the class of safety properties we want to check, and in
Section~\ref{section:backwardreachability} a solution. We finish with
a small description of our experimental results in
Section~\ref{section:experiments} and a conclusion.
