
\chapter{Shape Analysis}
Pointers and heap-allocated storage are features of all modern imperative programming languages.
They are among the most
complicated features of imperative programming language:
updating pointer variables (or pointer-fields of records) may have large side effects.
For example, dereferencing a pointer that has been freed will lead to segmentation fault in a C or C++ program.
Such side effects also make program analysis harder, because they make it
difficult to compute aliasing relationships among different pointers in a program. Aliasing arises when the same memory location can be accessed using different names.
For instance, consider the instruction statement $\tt x.f := y$ in an imperative language, where $\tt x$ and $\tt y$ are pointer variables. Its effect is to assign the value of the pointer $\tt y$ to the cell pointed to by $\tt x.f$. In order to update aliasing information of $\tt y$. We have to require information about all the cell pointed by $\tt x.f$ which is not an easy task.
%\bjcom{The last two sentences do not make sense (e.g., x.f points to exactly one
%  cell, so what means ``all cells''?). Make a new try}

In verification and program analysis, it is a problem to deduce and describe  how the heap-allocated memory is organized. E.g., program invariants must often describe how the heap-allocated memory is structured in order to infer the effects of statements that dereference pointer fields. This is the topic of ``shape analysis''.


%\bjcom{First sentence must be rewritten}
Shape analysis is a generic term denoting static
program-analysis techniques that attempt to discover and verify properties of the heap contents in (usually imperative) programs. The shape analysis problem becomes more challenging in concurrent programs that manipulate pointers and dynamically allocated objects, which are usually complicated. 
In concurrent programs, different threads interact in complex ways, which are difficult to foresee in the analysis. Several approaches for representing the possible structures of the heap have been proposed.
\begin{figure}
\center 
\begin{tikzpicture}[]


\node[rounded corners,draw,name=cell1,minimum width=24pt, minimum height=20pt]{};
\node[minimum width=8pt, minimum height=10pt,anchor=north west,font=\tiny,inner sep=0pt,name=d1] at (cell1.north west){$\mkadata0$};
\node[minimum width=8pt, minimum height=10pt,anchor=south west,font=\tiny,inner sep=0pt,scale=0.8] at ($(cell1.south west)+(0.5pt,1pt)$){{--$\infty$}};
\node[draw,minimum width=8pt, minimum height=10pt,anchor=north,font=\tiny,inner sep=0pt]at (cell1.north){\cross};
\node[draw,minimum width=8pt, minimum height=10pt,anchor=south,font=\tiny,inner sep=0pt]at (cell1.south){\cross};
\node[name=succ1,circle,fill,minimum size=3pt,inner sep=0pt,outer sep=0pt] at ($(cell1.east)+(-4pt,0pt)$) {};
\node[anchor=north,font=\tiny,align=center] at ($(cell1.south)+(0pt,2pt)$) {{\tt head}};

\node[rounded corners,draw,name=cell2,minimum width=24pt, minimum height=20pt,anchor=west]
at ($(cell1.east)+(10pt,0pt)$){};
\node[minimum width=8pt, minimum height=10pt,anchor=north west,font=\tiny,inner sep=0pt,name=d2] at (cell2.north west){$\mkadata0$};
\node[minimum width=8pt, minimum height=10pt,anchor=south west,font=\tiny,inner sep=0pt] at (cell2.south west){$3$};
\node[draw,minimum width=8pt, minimum height=10pt,anchor=north,font=\tiny,inner sep=0pt]at (cell2.north){\cross};
\node[draw,minimum width=8pt, minimum height=10pt,anchor=south,font=\tiny,inner sep=0pt]at (cell2.south){\tick};
\node[name=succ2,circle,fill,minimum size=3pt,inner sep=0pt,outer sep=0pt] at ($(cell2.east)+(-4pt,0pt)$) {};
\draw[->] (succ1) -- (cell2);
\draw[->,dotted,line width=1pt,color=blue,out=90,in=120] ($(d1.north)+(0pt,0pt)$) to ($(d2.north west)+(1pt,-1pt)$);
\node[anchor=north,font=\tiny,align=center] at ($(cell2.south)+(0pt,2pt)$) {$\tuple{\thread_{\tt 1},{\tt p}}$\\$\tuple{\thread_{\tt 2},{\tt p}}$};



\node[rounded corners,draw,name=cell3,minimum width=24pt, minimum height=20pt,anchor=west]
at ($(cell2.east)+(10pt,20pt)$){};
\node[minimum width=8pt, minimum height=10pt,anchor=north west,font=\tiny,inner sep=0pt,name=d3] at (cell3.north west){$\mkadata1$};
\node[minimum width=8pt, minimum height=10pt,anchor=south west,font=\tiny,inner sep=0pt,name=d3a] at (cell3.south west){$4$};
\node[draw,minimum width=8pt, minimum height=10pt,anchor=north,font=\tiny,inner sep=0pt]at (cell3.north){\cross};
\node[draw,minimum width=8pt, minimum height=10pt,anchor=south,font=\tiny,inner sep=0pt]at (cell3.south){\cross};
\node[name=succ3,circle,fill,minimum size=3pt,inner sep=0pt,outer sep=0pt] at ($(cell3.east)+(-4pt,0pt)$) {};
\draw[->,out=0,in=180] (succ2) to (cell3.west);
\draw[->,dotted,line width=1pt,color=blue,out=60,in=180] ($(d2.north)+(3pt,0pt)$) to ($(d3.west)$);
\node[anchor=south,font=\tiny,align=center] at ($(cell3.north)+(0pt,-2pt)$) {$\tuple{\thread_{\tt 2},{\tt n}}$};



\node[rounded corners,draw,name=cell4,minimum width=24pt, minimum height=20pt,anchor=west]
at ($(cell3.east)+(10pt,-20pt)$){};
\node[minimum width=8pt, minimum height=10pt,anchor=north west,font=\tiny,inner sep=0pt,name=d4] at (cell4.north west){$\mkadata0$};
\node[minimum width=8pt, minimum height=10pt,anchor=south west,font=\tiny,inner sep=0pt,name=d4a] at (cell4.south west){$6$};
\node[draw,minimum width=8pt, minimum height=10pt,anchor=north,font=\tiny,inner sep=0pt]at (cell4.north){\cross};
\node[draw,minimum width=8pt, minimum height=10pt,anchor=south,font=\tiny,inner sep=0pt]at (cell4.south){\tick};
\node[name=succ4,circle,fill,minimum size=3pt,inner sep=0pt,outer sep=0pt] at ($(cell4.east)+(-4pt,0pt)$) {};
\draw[->,out=0,in=180] (succ3) to  ($(cell4.west)+(0pt,6pt)$);
\node[anchor=south,font=\tiny,align=center] at ($(cell4.north)+(0pt,-2pt)$) {$\tuple{\thread_{\tt 2},{\tt c}}$};


\node[rounded corners,draw,name=cell5,minimum width=24pt, minimum height=20pt,anchor=west]
at ($(cell4.east)+(10pt,0pt)$){};
\node[minimum width=8pt, minimum height=10pt,anchor=north west,font=\tiny,inner sep=0pt,name=d5] at (cell5.north west){$\mkadata0$};
\node[minimum width=8pt, minimum height=10pt,anchor=south west,font=\tiny,inner sep=0pt,name=d5a] at (cell5.south west){$9$};
\node[draw,minimum width=8pt, minimum height=10pt,anchor=north,font=\tiny,inner sep=0pt]at (cell5.north){\cross};
\node[draw,minimum width=8pt, minimum height=10pt,anchor=south,font=\tiny,inner sep=0pt]at (cell5.south){\cross};
\node[name=succ5,circle,fill,minimum size=3pt,inner sep=0pt,outer sep=0pt] at ($(cell5.east)+(-4pt,0pt)$) {};
\draw[->] (succ4) -- (cell5);
\draw[->,dotted,line width=1pt,color=blue,out=270,in=270] ($(d4a.south)+(0pt,0pt)$) to ($(d5a.south)+(-3pt,1pt)$);
\node[anchor=south,font=\tiny,align=center] at ($(cell5.north)+(0pt,-2pt)$) {$\tuple{\thread_{\tt 3},{\tt c}}$};


\node[rounded corners,draw,name=cell6,minimum width=24pt, minimum height=20pt,anchor=west,name=cell6]
at ($(cell5.east)+(10pt,0pt)$){};
\node[minimum width=8pt, minimum height=10pt,anchor=north west,font=\tiny,inner sep=0pt,name=d6] at (cell6.north west){$\mkadata0$};
\node[minimum width=8pt, minimum height=10pt,anchor=south west,font=\tiny,inner sep=0pt,scale=0.8,name=d6a] at ($(cell6.south west)+(1pt,1pt)$){$\infty$};
\node[draw,minimum width=8pt, minimum height=10pt,anchor=north,font=\tiny,inner sep=0pt]at (cell6.north){\cross};
\node[draw,minimum width=8pt, minimum height=10pt,anchor=south,font=\tiny,inner sep=0pt]at (cell6.south){\cross};
\draw ($(cell6.north east)+(-1pt,-2pt)$) -- ($(cell6.south east)+(-8pt,0pt)$);
\draw ($(cell6.north east)+(-8pt,0pt)$) -- ($(cell6.south east)+(-1pt,2pt)$);
\draw[->] (succ5) -- (cell6);
\draw[->,dotted,line width=1pt,color=blue,out=270,in=270] ($(d5a.south)+(3pt,0pt)$) to ($(d6a.south)+(0pt,-1pt)$);
\node[anchor=south,font=\tiny,align=center] at ($(cell6.north)+(0pt,-2pt)$) {{\tt tail}};

\node[rounded corners,draw,name=cell7,minimum width=24pt, minimum height=20pt,anchor=west]
at ($(cell2.east)+(7pt,-20pt)$){};
\node[minimum width=8pt, minimum height=10pt,anchor=north west,font=\tiny,inner sep=0pt,name=d7] at (cell7.north west){$\mkadata1$};
\node[minimum width=8pt, minimum height=10pt,anchor=south west,font=\tiny,inner sep=0pt] at (cell7.south west){$4$};
\node[draw,minimum width=8pt, minimum height=10pt,anchor=north,font=\tiny,inner sep=0pt]at (cell7.north){\tick};
\node[draw,minimum width=8pt, minimum height=10pt,anchor=south,font=\tiny,inner sep=0pt]at (cell7.south){\tick};
\node[name=succ7,circle,fill,minimum size=3pt,inner sep=0pt,outer sep=0pt] at ($(cell7.east)+(-4pt,0pt)$) {};
\draw[->,out=0,in=180] (succ7) to ($(cell4.west)+(0pt,-6pt)$);
\draw[<->,dotted,line width=1pt,color=blue,out=270,in=90] ($(d3a.south)$) to ($(d7.north)+(-2pt,0pt)$);
\draw[->,dotted,line width=1pt,color=blue,out=90,in=180] ($(d7.north)+(2pt,0pt)$) to ($(cell4.west)$);
\node[anchor=north,font=\tiny,align=center] at ($(cell7.south)+(0pt,2pt)$) {$\tuple{\thread_{\tt 1},{\tt c}}$};

\end{tikzpicture}
\caption{A heap configuration of the {\tt Lazy Set} Algorithm. The abstract data values, 
and the dotted arrows represent performing data abstraction.
There are three active threads $\thread_1$, $\thread_2$, and $\thread_3$,
running the methods ${\tt rmv}(4)$, ${\tt add}(4)$, and ${\tt ctn}(9)$ respectively.
%
The symbols \tick and \cross represent the Boolean values {\tt true} and {\tt false}.}
\label{lazy:list:heap:fig}
\end{figure}
%

 
TVLA (Three-Valued Logic Analysis) \cite{SagivRW02} is one of the first and one of the most popular shape analysis
methods. It is based on a three-valued first-order predicate logic with transitive closure. Intuitively, concrete heap structure is represented by a finite set of abstract summary cells, each of them representing a set of concrete cells. Summary cells are obtained by merging several heap cells that agree on the values of a chosen set of unary abstraction predicates.  %\bjcom{I think the explanation of summary cells should be improved to be understandable. You might consider putting a small illustration of what is a summary cell. }
A unique important aspect of TVLA is that it automatically generates the abstract transformers from the concrete semantics; these transformers are guaranteed to be sound, and precise enough to verify wide ranges of applications. However, it cannot fully automatically handle all programs, and that one may have to extend it by appropriate predicates etc. Its the synthesis of appropriate predicates that are able to express the invariants in the program. This problem is even more difficult with complicated heap structures such as skip-lists, trees, or arrays of lists.  

There are several other approaches
 based on the use of logics to present heap configurations. The logics can be separation logic \cite{John:SL, Stephen:SL,JoshCris:SL,Hongseok:SL,Kamil:SL,Chin:SL,Quang:SL, Ruzica:SL, Constrantin:SL}, monadic second-order
 % \bjcom{check how to separate or conjoin ``second'' ``order''}
 logic \cite{Ander:ML, Jakob:ML,Madhusudan:ML} and others \cite{Shmuel:Shape, Karen:Shape}. Among these works, the works based on separation logic are
 %such as \cite{JoshCris:SL,Hongseok:SL, Quang:SL} 
  more efficient than the others. 
  %\bjcom{The preceding sentence had no information. You can improve it} 
  The reason for that is that their approaches effectively decompose the heap into disjoint components and treat them independently. However, most of the techniques based on separation logic are either specialised for some particular data structure, or they need to be provided inductive definitions of the data structures. In addition, their entailment checking procedures are either for specific class of data structures or based on folding/unfolding inductive predicates in the formulae and trying to obtain a syntactic proof of the entailment. 

 
 This issue can be fixed addressed by automata-based techniques using the generality of the automata-based representation such as techniques using tree automata. Finite tree automata, for instance, have been shown to provide a good balance between efficiency and expressiveness. The work \cite{Ahmed:TreeAutomata} uses a finite tree automaton to describe a set of tree parts and represent non-tree edges of heaps by using regular “routing” expressions. Finite tree transducers
are used to compute set of reachable configurations, and symbolic configuration is abstracted
collapsing certain states of the automata. The refinement technique called counterexample-guided
abstraction refinement (CEGAR) technique is used during the run of the analysis. This technique
is fully automatically and can handle complex data structures such as binary trees with linked
leaves. However, it suffers from the inefficiency and it also can not handle concurrent programs.
%\bjcom{The preceding paragraph was interesting, but too hard to follow. Try to
%  explain in a more understandable way}

The problem of inefficiently of the previous technique can be solved by the approach based on forest automata \cite{foresterfull}. In such representation, a heap is split in a canonical way into several tree components whose roots are the so-called cut-points. Cut-points are cells which are pointed to by either program variables or having several incoming edges. The tree components can refer to the roots of each other, and hence they are “separated” much like heaps described by formulae joined by the separating conjunction in separation logic \cite{John:SL}. Using this decomposition, sets of heaps with a bounded number of cut-points can then be represented by so called forest automata (FA). Each of the tree
automata within the tuple accepts trees whose leaves may refer back to the
roots of any of these trees. A forest automaton then represents exactly the set
of heaps that may be obtained by taking a single tree from the language of each
of the component tree automata and by gluing the roots of the trees with the
leaves referring to them. %\bjcom{Previous sentences was a good explanation. Now
  %introduce the example, and discuss nested FAs only later}
 Moreover, they allow alphabets of FA to contain nested FA, leading to a hierarchical encoding of heaps, allowing us to represent even sets of heaps with an unbounded number of cut-points (e.g., sets of DLL, skiplist). \input FA
Let us take an example of how to split a heap into small tree components. Figure \ref{forestautomata} shows five tree components obtained by splitting the heap in figure \ref{lazy:list:heap:fig}. These components are named as 1,2,3,4 and 5 from left to right. Each root of a tree component is a cut-point in the heap in figure \ref{lazy:list:heap:fig}. These cut-points are cells pointed to by variables $\tt head$, $\tt tail$,  $\tt p$,  $\tt c$ and the cell which has two incoming pointers. In each tree component, the red node show which  tree component it refers. For instance, the tree component 1 refers to tree component 2, and both tree components 2 and 3 refer to tree component 4.
%\bjcom{The number of pictures is buggy in preceding paragraph.
%Check \url{http://www.terminally-incoherent.com/blog/2007/04/14/latex-fixing-wrong-figure-numbers/}}

This forest automata approach is fully automatic and able to verify various classes of data structures, including complicated structures such as trees and skip-lists with bounded number of levels. 
However, the approach can not verify properties related to data values of heap cells such as sortedness in the lazy set algorithm. Therefore, in this thesis, we extend their work to verify data properties. 
%bjcom{This paragraph can be extended so that it mentions the challenges that
%  you will address in papers I, II, and III: You can start from the FA approach
%  and mention what it cannot do. Thereafter, you will have sections for each
%  challenge in ``Our Approaches''. For instance, you can focus on
%data in paper I, concurrency in Paper II, and non-SLL structures in paper III}

The last approach that we will mention is based on graph grammars describing heap graphs \cite{Jonathan:Shape, Jonathan:Grammars}.  The approach is to model heap states hypergraphs, and represent both pointer operations and abstraction mappings by hypergraph transformations. The presented approaches differ in their degree of specialisation for a particular class of data structures, their efficiency, and their level of dependence on user assistance (such as definition of loop invariants or inductive predicates for the considered data structures).
  
\section*{Our Approaches}
%\bjcom{Use the present tense}
In this thesis, we propose three approaches for heap abstractions. In paper I, we propose a novel approach of extending the forest automata approach \cite{foresterfull} by expressing relationships between data elements associated with cells of the heap %\bjcom{Did you explain the term ``heap graph''?} 
by two classes of constraints.
%representing sets of heaps via tree automata (TA). In our representation, a heap is split in a canonical way into several tree components whose roots are the so-called cut-points. Cut-points are nodes pointed to by program variables or having several incoming edges. The tree components can refer to the roots of each other, and hence they are “separated” much like heaps described by formulae joined by the separating conjunction in separation logic [15]. Using this decomposition, sets of heaps with a bounded number of cut-points are then represented by the so called forest automata (FA) that are basically tuples of TA accepting tuples of trees whose leaves can refer back to the roots of the trees. Moreover, we allow alphabets of FA to contain nested FA, leading to a hierarchical encoding of heaps, allowing us to represent even sets of heaps with an unbounded number of cut-points (e.g., sets of DLL, skiplist).
%\bjcom{Separate the text so that previous FA work is in previous section (as
%  you have now). In this section, you describe how you extend the FA approach}
\input TA

 %\bjcom{You can be more clear by putting bullets here: e.g., one for local and one for global}
 \begin{itemize}
 	\item Local data constraints are associated with transitions of each TA and capture relationships between data of neighboring cells in a heap; they can be used, e.g., to represent ordering internal to data structures such as sorted linked lists and binary search trees.
 	\item Global data constraints are associated with states of TA and capture relationships between data in distant parts of the heap. Intuitively, a global data constraint between two TAs captures the data relationship between cells of two heaps accepted by these two TAs. 
 %\bjcom{At this point, the text needs an illustration. Only thereafter you explain what the approach can accomplish}
 \end{itemize}
  This approach was applied to verification of sequential heap manipulation programs. This approach is general and fully automatic, it can handle many types of sequential programs without any manual step. However, due to the complexity of tree automata operations, this approach is not suitable to handle concurrent programs where a large number of states and computation are needed. Figure \ref{figure:forest} shows an example of how to represent a heap in \ref{lazy:list:heap:fig} by a set of tree automata with added data constraints. In the figure, the local data constraints are located along the solid arrows between cells, whereas global constraints are located along the blue arrows. The global constraint $\tt \prec_{aa}$ means that data values of all cells in the left hand side are smaller than data values of all cells in the right hand side. The local constraint $\tt \prec_{rr}$ means that left hand side cell is smaller than data value of the cell in the right hand side.  We just show here small examples of data constraints, the detail about different types of constraints can be found in paper I.    

%\bjcom{Here, a new (sub)section is needed}
In order to verify concurrent data structures with unbounded number of threads, thread-modular is a promising approach for this challenge. Its high efficiency is achieved by
abstracting the interaction between threads. The approach verifies a concurrent data structure by generating an invariant that correlates the global state with the local state of an
arbitrary thread. In other words, it only keep track of the shape viewed by one thread, while abstracting away all the other threads. The thread-modular
approach includes a step where it takes the information about one thread and
intersect it with the information from another thread, in order to take into account the interference of all the other threads on the first thread. Forest automata approach is not suitable for this thread-modular. The reason is that computing intersection between FAs is not efficient in concurrent systems where the number of FAs is huge. 

Therefore, in paper II, we have to adapt FA to the new setting by providing a symbolic encoding of the heap structure, that is less precise than the forest automata approach in paper I. However it is precise enough to allow the verification of the concurrent data structures, and efficient enough to make the verification procedure feasible in practice.
% \bjcom{The text that introduce paper II does not give a good picture. First of all, you must introduce the thread-modular
%  approach, and illustrate it, i.e., one challenge is to cope with concurrency.
%Thereafter, you can say that you need to adapt FA to the new setting.
%You can say that your approach is now specialized for SLLs. You can introduce it by saying that TAs are replaced by ``heap segments''. (is it true that a heap
%segment can be characterized by a TA with data constraints?).
%Then you can explain whether cut points are now the same as for FAs or
%whether they are different.}
The main idea of the abstraction is to have a more precise description of the parts of the heap that are visible (reachable) from global variables, and to make a succinct representation of the parts that are local to the threads. Intuitively, a heap segment can be characterized by a TA with data constraints. More concretely, we will extract a set of heap segments between two cut-points which are same as cut-points in the forest automata approach. For each segment, we will store a summary of the content of the heap along the segment. This summary consists of three parts, each part contains different pieces of information, including 
\begin{itemize}
\item the values of data fields of cells along the segment which have finite values. Note that, these values do not include values of cut-points, and
\item the ordering among data values of fields of cells along the segment which have integer values. 
\item The sequence of observer registers that appear in the segment. 
\end{itemize}
\input SL	
%\bjcom{I could not understand the text in the preceding sentence} For each given program, the set of possible abstract shapes insight \bjcom{what do you mean here?} and hence the verification procedure is guaranteed to terminate.
%\bjcom{In all descriptions, separate the text into more paragraphs. You can never
%  have both explanation and assessment in the same. If you explain how the approach works, you can then illustrate. AFter that, you must make a new paragraph, where you can say what can be achieved with such an approach}

This approach is very efficient but it is not optimal for complicated concurrent data structures like trees, lists of lists or skiplists. The  approach is now specialized for SLLs. Figure \ref{heapsummary} gives our symbolic abstraction of the heap in figure \ref{lazy:list:heap:fig} where the observer register $\tt z$ is equal to 3. In this figure, there are four segments obtained from five cut-points. In each segment, the red box contains ordering information between data values of cells along the segment, the green box contains information about the values of fields {\tt mark}, and {\tt lock} of cells along the segment. Finally, the blue box contains the information about the sequence of observer registers. In this example, in the first segment between the two cells pointed to by $\tt head$ and $\tt p$. The sequence of observer registers is $\tt z^1$, it means that between two cut-points there is exactly one cell whose data value of $\tt val$ field is equal to the observer register $\tt z$.

In paper III, we present an approach which can handle concurrent programs implemented from simple to complex data structures. \bjcom{Be more specific than this} In our fragment abstraction,
\bjcom{introduce the term ``fragment abstraction'' better, before you explain how it works} we represent the part of the heap that is accessible to a thread by a set of fragments. A fragment represents a pair of heap cells (accessible to $\thread$)
that are connected by a pointer field, under the applied data abstraction. The fragment contains both
(i) {\em local} information about the cell's fields and variables that
  point to it, as well as
(ii) {\em global} information, representing how
  each cell in the pair can reach to and be reached from
  (by following a chain of pointers) a small set of globally significant
  heap cells.
 A set of fragments represents the set of heap
structures in which each pair of pointer-connected nodes is represented by some
fragment in the set.
Put differently, a set of fragments describes the set of heaps that can be formed by
``piecing together'' pairs of pointer-connected nodes that are represented
by some fragment in the set. This ``piecing together'' must
be both locally consistent (appending only fragments that agree on their
common node), and globally consistent (respecting the global reachability
information).
\input fragment

Let us illustrate how pairs of heap nodes can be represented by fragments. Figure \ref{fragment} shows the set of fragments abstracted from the heap in \ref{figure:forest}(a). In each fragment, the ordering between two keys of two nodes is shown as a label on the arrow between two tags. Above each tag is pointer variables. The first brown row under each tag is $\tt reachfrom$ information, whereas the second green row is $\tt reachto$ information.


\bjcom{After this, one would like to read about the point of fragment abstraction. Please show how it can be applied to skiplists}



