%% ---------------------------------------------
%% Math stuff
%% ---------------------------------------------

%\newtheorem{definition}{Definition}[chapter]

%\newcommand{\set}[1]{\left\{#1\right\}}
\newcommand{\set}[1]{\{#1\}}
%\newcommand{\setcomp}[2]{\{{#1}\mathrel{}\middle|\mathrel{}{#2}\}}
%\newcommand{\setcomp}[2]{\set{#1\mid\;#2}}
\newcommand{\setcomp}[2]{\{{#1}\mathrel{}\mid\mathrel{}{#2}\}}

\newcommand{\cross}{\textcolor{red}{\ding{55}}}
\newcommand{\tickk}{\textcolor{green}{\ding{52}}}

\newcommand{\nat}{\ensuremath{\mathbb N}}
\newcommand{\reals}{\ensuremath{\mathbb R}}
\newcommand{\sizeof}[1]{|#1|}
\newcommand{\union}{\cup}
%\newcommand{\minsetunion}{\sqcup}
\newcommand{\range}[2]{\llbracket{#1}{,}{#2}\rrbracket} %% {,} otherwise I get some spacing after the ','

% \newcommand{\updateby}[2]{\ensuremath{\left[#1\leftarrow#2\right]}}

%% ---------------------------------------------
%% Parameterized systems
%% ---------------------------------------------
\newcommand\tuple[1]{\left\langle#1\right\rangle}
\newcommand{\parsys}{\ensuremath{\mathcal P}}
\newcommand{\locs}{\ensuremath{Q}}
\newcommand{\rules}{\ensuremath{\Delta}}

\newcommand{\witnesses}{S}
\newcommand{\quantrule}[5]{ \mathbf{if}\ {#3}~j\,{#4}\,i:\,{#5}\ \mathbf{then}\ {#1}\trans{#2}}
\newcommand{\quantify}{\mathbb Q}

\newcommand{\confs}{\ensuremath{\mathcal C}}
\newcommand{\trans}{\ensuremath{\rightarrow}}
\newcommand{\transof}[1]{\stackrel{#1}{\trans}}
\newcommand{\transys}{\ensuremath{\mathcal T}}

\newcommand{\src}{\mathtt{src}}
\newcommand{\dst}{\mathtt{dst}}

%\newcommand{\bad}{\mathit{Bad}}
\newcommand{\Bad}{\ensuremath{\mathcal{B}}}
\newcommand{\minbad}{\ensuremath{\Bad_{min}}}
%\newcommand{\minbad}{B}
\newcommand{\Reach}{\ensuremath{\mathcal{R}}}
\newcommand{\Inits}{\ensuremath{\mathcal{I}}}

\newcommand{\domain}{\ensuremath{\mathcal{D}}}
\newcommand{\entails}{\preccurlyeq}
\newcommand{\preorder}{\leqslant}%\vartriangleleft

% \newcommand{\BinRel}{\mathcal B} % Binary relation
% \newcommand{\rel}{R}

\newcommand{\forrule}[5]{\ensuremath{\mathbf{if~foreach}\ j\mathrel{#1}i: {#2} \ \mathbf{then}\ {#3}\trans{#4}\ \mathbf{else}\ {#3}\trans{#5}}}

%% ---------------------------------------------
%% Monotonic abstraction
%% ---------------------------------------------
\newcommand{\thread}{{\tt th}}
\newcommand{\subword}{\sqsubseteq}
% \newcommand{\word}[3][0]{{#2}_{#1}  \ldots  {#2}_{#3}}
\newcommand{\ucl}[1]{\ensuremath{\lfloor{#1}\rfloor}} %% Upward-Closure
\newcommand{\gen}[1]{min(#1)}

\newcommand{\parabol}[3][]{\draw[black, very thin, fill=white,#1] (0,0) parabola[parabola height=#3, bend pos=0.5] ++(#2,0)}
\newcommand{\atrans}{\leadsto}
\newcommand{\atransof}[1]{\stackrel{#1}{\atrans}}

\newcommand{\abstrans}[1]{\hat{#1}}

\newcommand{\worklist}{\texttt{W}}
\newcommand{\visited}{\texttt{V}}
\newcommand{\inverse}[1]{\ensuremath{{#1}^{\textit{-}1}}} % Using hyphen and not the longer "minus".
% \newcommand{\pre}{\ensuremath{\inverse{\abstrans\rules}}}
% %\newcommand{\prestar}{\ensuremath{{\abstrans\rules}^{\textit{-}*}}}
% \newcommand{\prestar}{\ensuremath{(\pre)^{*}}}
\newcommand{\pre}{Pre}
\newcommand{\prestar}{\ensuremath{Pre^{*}}}


%% ---------------------------------------------
%% View abstraction
%% ---------------------------------------------
\newcommand{\dcl}[1]{\ensuremath{\lceil{#1}\rceil}} %% Downward-Closure

\newcommand{\Abs}{\alpha} 
\newcommand{\Conc}{\gamma}

\newcommand{\Absof}[1]{\Abs_{#1}}
\newcommand{\Concof}[1]{\Conc_{#1}}
%\newcommand{\Concoflim}[2]{\Conc_{#1}^{#2}}
\newcommand{\Concoflim}[2]{\oint_{#1}^{#2}}
%\newcommand{\minabstrof}[1]{\minsetof{\abstr_{#1}}}

\newcommand {\views}{\mathcal{V}}
\newcommand {\viewsof}[1]{\views_{#1}}
%\newcommand {\confsof}[1]{\confs_{#1}}

\newcommand {\badviewsof}[1]{\views_{#1}^{\mathit{bad}}}
\newcommand {\badviews}{\views^{\mathit{bad}}}

% \newcommand {\mk}[1]{{#1}^\bullet}
\newcommand {\proj}[2]{\Pi_{#1}(#2)}
% \newcommand {\tproj}[2]{\Pi_{#1}^\circ(#2)}
% \newcommand {\naproj}[2]{\Pi'_{#1}(#2)}

\newcommand {\post}{\mathit{post}}
\newcommand {\spost}{\mathit{spost}}
\newcommand {\apost}[1]{{\mathit{Apost}}_#1}
\newcommand {\sdelta}{\delta^\#}

\newcommand{\entailedby}{\succcurlyeq}
\newcommand{\minsetof}[1]{\lfloor #1 \rfloor}
\newcommand{\minunion}{\sqcup}

\newcommand{\base}{\mathtt{base}}
\newcommand{\ctx}{\mathtt{ctx}}

% \usepackage{wasysym}
% \newcommand{\aConcoflim}[2]{\ensuremath{\mathbin{\ooalign{\hspace{.2ex}\raisebox{.15ex}{\scalebox{.7}{\wasylozenge}}\cr$\int$\cr}}_{#1}^{#2}}}
%\newcommand{\aConcoflim}[2]{\ensuremath{\mathbin{\ooalign{\hspace{.2ex}\raisebox{.15ex}{\scalebox{.6}{$\square$}}\cr$\int_{#1}^{#2}$\cr}}}}
\newcommand{\aConcoflim}[2]{\sqint_{#1}^{#2}}
%\newcommand{\F}{\ensuremath{\mathtt{F}}}
\newcommand{\isbad}{\ensuremath{\mathtt{bad}}}

\newcommand{\tick}{\checkmark}
%\newcommand{\tickof}[2]{\checkmark({#1},{#2})}
\newcommand{\unticked}{\rho}%\xi \chi
\renewcommand{\next}{\mathit{next}}

\newcommand{\makehighgroup}[4][]{
  \draw[blue!70!red,#1] (#2.north west) ++(0,1mm) -- +(0,1mm) -- ([yshift=2mm]#3.north east) coordinate[midway](n#4) -- +(0,-1mm);
}
\newcommand{\makelowgroup}[4][]{
  \draw[#1] (#2.south west) ++(0,-1mm) -- +(0,-0.5mm) -- ([yshift=-1.5mm]#3.south east) coordinate[midway](p#4) -- +(0,0.5mm);
}
%% ---------------------------------------------
%% Shape Analysis
%% ---------------------------------------------
\newcommand{\frag}{\mathtt{v}}
\newcommand{\fragset}{V}
\newcommand\true{{\tt true}}
\newcommand\false{{\tt false}}
\newcommand*{\prgcode}[1]{\texttt{#1}}
\newcommand{\dset}{\mathbb{D}}
\newcommand\commitpoint[1]{\tikz{\node[commitpoint,#1]{};}}
\newcommand\stepa{\tikz{\node[draw, circle, fill = gray!20, draw = black, name = n1, minimum width=12pt, minimum height=12pt,anchor=south west,inner sep=0pt,scale=0.8]{\tt 1}}}
\newcommand\stepb{\tikz{\node[draw, circle, fill = gray!20, draw = black, name = n1, minimum width=12pt, minimum height=12pt,anchor=south west,inner sep=0pt,scale=0.8]{\tt 2}}}
\newcommand\stepc{\tikz{\node[draw, circle, fill = gray!20, draw = black, name = n1, minimum width=12pt, minimum height=12pt,anchor=south west,inner sep=0pt,scale=0.8]{\tt 3}}}
\newcommand\stepd{\tikz{\node[draw, circle, fill = gray!20, draw = black, name = n1, minimum width=12pt, minimum height=12pt,anchor=south west,inner sep=0pt,scale=0.8]{\tt 4}}}
\newcommand\stepe{\tikz{\node[draw, circle, fill = gray!20, draw = black, name = n1, minimum width=12pt, minimum height=12pt,anchor=south west,inner sep=0pt,scale=0.8]{\tt 5}}}
\newcommand\stepf{\tikz{\node[draw, circle, fill = gray!20, draw = black, name = n1, minimum width=12pt, minimum height=12pt,anchor=south west,inner sep=0pt,scale=0.8]{\tt 6}}}

\newcommand\threada{\tikz{
\node[draw, circle, fill = red!20, double = red, name = addT, minimum width=15pt, minimum height=15pt,anchor=south west,inner sep=0pt, scale=0.8]{{$\tt T_1$}};
}}

\newcommand\threadb{\tikz{
\node[draw, circle, fill = blue!20, double = blue, name = addT, minimum width=15pt, minimum height=15pt,anchor=south west,inner sep=0pt, scale=0.8]{{$\tt T_2$}};
}}

\newcommand\threadc{\tikz{
\node[draw, circle, fill = orange!20, double = orange, name = addT, minimum width=15pt, minimum height=15pt,anchor=south west,inner sep=0pt, scale=0.8]{{$\tt T_3$}};
}}


\newcommand\taa{\tikz{
\node[name = x, circle, color = red, draw = red, minimum width=12pt, minimum height=12pt,anchor=south west,inner sep=0pt,scale=0.8] at ($(cell1.south east)+(0pt,-9pt)$){{$\tt 1$}};
}}

\newcommand\tab{\tikz{
\node[name = x, circle, color = blue, draw = blue, minimum width=12pt, minimum height=12pt,anchor=south west,inner sep=0pt,scale=0.8] at ($(cell1.north east)+(204pt,63pt)$){{$\tt 2$}};
}}

\newcommand\tac{\tikz{
\node[name = x, circle, color = blue, draw = blue, minimum width=12pt, minimum height=12pt,anchor=south west,inner sep=0pt,scale=0.8] at ($(cell1.north east)+(244pt,63pt)$){{$\tt 3$}};
}}

\newcommand\tad{\tikz{
\node[name = x, circle, color = orange, draw = orange, minimum width=12pt, minimum height=12pt,anchor=south west,inner sep=0pt,scale=0.8] at ($(cell1.north east)+(284pt,63pt)$){{$\tt 4$}};
}}

\newcommand\tae{\tikz{
\node[name = x, circle, color = cyan, draw = cyan, minimum width=12pt, minimum height=12pt,anchor=south west,inner sep=0pt,scale=0.8] at ($(cell1.north east)+(324pt,63pt)$){{$\tt 5$}};
}}


\newcommand\nodea{\tikz{
\node[draw,rounded corners = 0.09cm, fill = red!30, draw = red, name = n1, minimum width=12pt, minimum height=12pt,anchor=south west,inner sep=0pt,scale=0.8] at ($(cell1.north east)+(0pt,0pt)$){{$\tt 1$}};
}}

\newcommand\nodeb{\tikz{
\node[draw, rounded corners = 0.09cm, fill = blue!30, draw = blue, name = n3, minimum width=12pt, minimum height=12pt,anchor=south west,inner sep=0pt,scale=0.8] at ($(cell1.north east)+(0pt,-25pt)$){{$\tt 4$}};
}}

\newcommand\nodec{\tikz{
\node[draw, rounded corners = 0.09cm, fill = orange!30, draw = orange, name = n6, minimum width=12pt, minimum height=12pt,anchor=south west,inner sep=0pt,scale=0.8] at ($(cell1.north east)+(0pt,-50pt)$){{$\tt 6$}};
}}

\newcommand\noded{\tikz{
\node[draw, rounded corners = 0.09cm, fill = blue!30, draw = blue, name = n4, minimum width=12pt, minimum height=12pt,anchor=south west,inner sep=0pt,scale=0.8] at ($(cell1.north east)+(40pt,-25pt)$){{$\tt 2$}};
}}

\newcommand\nodee{\tikz{
\node[draw, rounded corners = 0.09cm, fill = cyan!30, draw = cyan, name = n5, minimum width=12pt, minimum height=12pt,anchor=south west,inner sep=0pt,scale=0.8] at ($(cell1.north east)+(80pt,-25pt)$){{$\tt 8$}};
}}

\newcommand{\triggersym}{\blue{\bullet}}
\newcommand{\triggered}[1]{{#1}^{\triggersym}}

\newcommand{\pointsto}{\mapsto}
\newcommand{\reaches}{\dashrightarrow}
\newcommand{\pointedby}{\mapsfrom}
\newcommand{\reachedby}{\dashleftarrow}
\newcommand{\unrelated}{\Join}

\newcommand{\nullconst}{{\tt \#}}
\newcommand{\undefconst}{{\tt \bot}}
% \newcommand{\freeconst}{{\tt FREE}}
\newcommand{\Pred}{\mathit{Pred}}
%%%%%%  TiKZ styles    %%%%%

\tikzstyle{code-bg}=
[rounded corners,fill=cyan!2!white,draw=cyan!50!white,inner xsep=0pt]%

%\tikzstyle{inference-bg}=
%[rounded corners,fill=blue!5!white,draw=blue!50!white]%

\tikzstyle{inference-bg}=
[rounded corners,fill=gray!6!white,draw=gray!50!white,inner xsep=0pt]%

  
\tikzstyle{linenum}=[font={\footnotesize\tt},anchor=east]
\tikzstyle{linecode}=[font={\footnotesize\tt},anchor=west, scale = 0.9]
\tikzstyle{lpcode}=[anchor=west,text=blue]
\tikzstyle{ostate}=[fill=white,circle,minimum size=0pt,inner sep=0pt,text=black,,font=\tiny]
\tikzstyle{oedge}=[line width=1pt,->,>=stealth]
\tikzstyle{ctrlnode}=[font=\small,scale = 0.9]




%%%%%%  TiKZ layers    %%%%%
%\pgfdeclarelayer{background}
%\pgfdeclarelayer{bbackground}
%\pgfdeclarelayer{foreground}
%\pgfsetlayers{bbackground,background,main,foreground}
%% ---------------------------------------------
%% Experiments
%% ---------------------------------------------
\newcommand{\strue}{\ensuremath{\mathtt{true}}}
\newcommand{\sfalse}{\ensuremath{\mathtt{false}}}
\newcommand{\LD}[1]{\ensuremath{Read_{#1}}}
\newcommand{\ST}[1]{\ensuremath{Write_{#1}}}

