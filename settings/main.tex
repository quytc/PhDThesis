%\input{settings/builddir}
%% Check that we are not using XeTeX.
% Package to determine whether XeTeX is used
\usepackage{ifxetex}

%\RequireXeTeX
\ifxetex
   \begingroup
     \errorcontextlines=-1\relax
     \newlinechar=10\relax
     \errmessage{^^J^^J
     **********************************************^^J
     * This thesis should be compiled with pdflatex.^^J
     **********************************************^^J^^J}%
     \errmessage{^^J^^J
     **********************************************^^J
     * I'm telling you...^^J
     * It won't work with xelatex!^^J
     * Insisting? ^^J
     **********************************************^^J^^J}%
   \endgroup
\fi


% Plain LaTeX specific packages and settings
% Language, diacritics and hyphenation
\usepackage[swedish,french,english]{babel} 
\usepackage[dvipsnames]{xcolor}

%% -----------------------------------------------------------
%% Fonts settings
%% -----------------------------------------------------------
% %\defaultfontfeatures{Ligatures=TeX}
%% Main Font: Times New Roman (Regular, Italic, Bold, BoldItalic)
%% Roman font (regular font): Times New Roman
%% Sans Serif font: Arial
%% Mono font: Courier New
%% Brush: Papyrus
%% Chalk: Chalkduster
%% -----------------------------------------------------------
\usepackage[utf8]{inputenc}
%\usepackage[T1,OT1,LY1]{fontenc}
\usepackage[LY1,T1]{fontenc}
%\usepackage{type1cm}
%\usepackage{lmodern}% http://ctan.org/pkg/lm
\usepackage{times}
\usepackage{paralist}
%\usepackage{mathptmx} % I don't like this math font
% TeX ligature seem to be already enabled

\usepackage{texnansi}
%\input{settings/fonts/Chalkduster.fd}
%\input{settings/fonts/Papyrus.fd}
%\input{settings/fonts/Hyeenanhaukotus.fd}
% No font map, since I have run 'make fonts' before
% and my TEXMFVAR points to the right place

\newcommand{\Chalk}{\fontfamily{Chalkduster}\selectfont}
\newcommand{\Brush}{\fontfamily{Papyrus}\selectfont}
\newcommand{\TitleFont}{\fontfamily{Hyeenanhaukotus}\selectfont}

\newcommand{\Parosh}{\begingroup\Chalk Parosh\endgroup}
\newcommand{\Papa}{\begingroup\Brush Papa\endgroup}
\newcommand{\Maman}{\begingroup\Brush Maman\endgroup}
\newcommand{\Franck}{\begingroup\Brush Franck\endgroup}
\newcommand{\Alexandre}{\begingroup\Brush Alexandre\endgroup}

%% -----------------------------------------------------------
%% Packages
%% -----------------------------------------------------------

% Enable scaling of images on import
\usepackage[pdftex]{graphicx}
%\usepackage[pdftex]{graphics}

% Tables
\usepackage{booktabs}
\usepackage{tabularx}

\usepackage{hhline}
\usepackage{multirow}
\usepackage{multicol}

\usepackage{wrapfig}
\usepackage{enumitem}
%\usepackage{paralist}

\usepackage{nth} % For 1st, 2nd, 3rd 4th, ...

\usepackage{xcolor}
\usepackage{amsfonts,amscd,amssymb,amsmath,amsxtra,amsthm}
\usepackage{mathtools}
\usepackage{etoolbox}
%\usepackage{boxhandler}
\usepackage{ifthen}
\usepackage{stmaryrd}
\usepackage{url}
\usepackage{fancyvrb}

\usepackage{esint} % For the squared contour integral

\usepackage{pifont}% http://ctan.org/pkg/pifont

%\usepackage{fancyhdr,layout,appendix,subfigure}

%% For the acknowledgments
\usepackage{shapepar}


%% Index %%%%%%%%%%%%%%%%%%%%%%%%%%%%%%%
%% http://en.wikibooks.org/wiki/LaTeX/Indexing
\ifnoindex\else
\usepackage{makeidx}
\makeindex
\fi
%%%%%%%%%%%%%%%%%%%%%%%%%%%%%%%%%%%%%%%%

%% -----------------------------------------------------------
%% Citations - Doesn't work...
%% -----------------------------------------------------------
%\renewcommand*{\bibfont}{\footnotesize}
% \renewcommand*{\citesetup}{%
%   \footnotesize
%   \biburlsetup
%   \frenchspacing}

%% Making the citation smaller, and grey
\let\oldcite=\cite
\renewcommand{\cite}[1]{{\scriptsize\color{gray}\oldcite{#1}}}




%% -----------------------------------------------------------
%% For Tikz
%% -----------------------------------------------------------
\usepackage[version=latest]{pgf}

\usepackage{tikz}
\usetikzlibrary{%
  arrows,%
  arrows.meta,%
  calc,%
  %decorations,%
  decorations.pathmorphing,%
  %decorations.pathreplacing,%
  %calendar,%
  chains,%
  fit,%
  %shapes,%
  shapes.geometric,%
  shapes.misc,%
  %shapes.symbols,%
  %shapes.arrows,%
  %shapes.callouts,%
  shapes.multipart,%
  %backgrounds,%
  matrix,%
  %fadings,%
  %through,%
  positioning,%
  %scopes,%
  %decorations.shapes,%
  %decorations.pathmorphing,%
  %decorations.text,%
  shadows,%
  trees,%
  %snakes,% use decorations instead
  petri,%
  automata,%
  backgrounds,%
  patterns%
}
%% ================================
%% TikZ styles
%% ================================

\pgfdeclarelayer{my background} 
\pgfsetlayers{background,my background,main}

\tikzset{background rectangle/.style={rounded corners=1ex,draw=gray!5,thick,fill=gray!10,double}}

\tikzset{enumbullet/.style={draw,circle,double,inner sep=1pt}}
\tikzset{challenge/.style={draw,circle,double,scale=0.75,inner sep=1pt}}
\tikzset{subchallenge/.style={circle,fill=white,scale=0.6,inner sep=0.5pt,draw=black,very thin}}

\tikzset{myedge/.style={draw,shorten >=1pt,->,>=stealth',semithick}}
\tikzset{process/.style={circle,minimum width=2ex,inner sep=1pt,draw=blue!50,fill=blue!20,thick}}
\tikzset{mylabel/.style={inner xsep=2pt,draw=gray!10,fill=white,double,rounded corners=2pt,anchor=center}}
\tikzset{separation/.style={white,semithick}}

%------------------------------------------
%\tikzset{state/.style={circle,minimum size=3.2ex,inner sep=0pt,scale=0.75}}
\tikzset{state/.style={rectangle,rounded corners=0.5ex,minimum height=2ex,minimum size=3.2ex,inner sep=0pt,scale=0.75}}
\tikzset{word/.style={rectangle,rounded corners=0.5ex,thin,inner sep=0pt,draw=blue!50,fill=blue!50}}%inner xsep=1pt,inner ysep=1pt

% \tikzset{state-w/.style={top color=blue!30,bottom color=blue!30,middle color=blue!10}}
% \tikzset{word-w/.style={thin,fill=blue!30,draw=blue!10}}
\tikzset{state-w/.style={fill=none,draw=blue!10}}
\tikzset{word-w/.style={thin,fill=none,draw=blue!10}}

\tikzset{state-n/.style={top color=blue!30,bottom color=blue!30,middle color=blue!10}}
\tikzset{word-n/.style={thin,fill=blue!50,draw=blue!10}}

\tikzset{state-i/.style={top color=green!20,bottom color=green!20,middle color=green!10}}
\tikzset{word-i/.style={thin,fill=green!20,draw=green!30}}

\tikzset{state-b/.style={top color=red!50,bottom color=red!50,middle color=red!20}}
\tikzset{word-b/.style={thin,fill=red!50,draw=red!60}}

\tikzset{state-witness/.style={top color=green!20!orange,bottom color=green!20!orange,middle color=green!20!orange!10}}
\tikzset{word-witness/.style={thin,fill=green!20!orange,draw=green!20!orange!50}}

\tikzset{state--/.style={draw=none,fill=none}}
\tikzset{word--/.style={draw=none,fill=none}}

% \tikzset{word/.style={matrix of nodes,outer sep=0pt,inner ysep=1pt,inner xsep=1pt,rounded corners=2pt,
%                       thin,fill=blue!30,draw=blue!10,
%                       column sep=1pt,
%                       nodes={state-#1,rectangle,rounded corners=0pt,inner sep=1pt,outer sep=0pt,anchor=base,text height=1.5ex,%
%                              top color=blue!30,bottom color=blue!30,middle color=blue!10,
%                              %outer color=gray!80,inner color=gray!20,shading=radial,
%                              }%
%                       }}
% \tikzset{invisible/.style={rectangle,draw=none,outer sep=0pt,inner sep=0pt,fill=none,text width=0pt,text height=0pt,minimum height=0pt}}
%\tikzset{word/.style={matrix of nodes,outer sep=0pt,inner sep=0pt,column sep=-2pt,nodes={outer sep=0pt,anchor=base,state,state-#1}}}

\tikzset{trlabel/.style={scale=0.75}}

  
\tikzset{conf/.style={shading=axis,top color=gray!10,bottom color=gray!10,middle color=gray!50,shading angle=90,inner ysep=1pt}}
% \tikzset{view/.style={%draw,thin,rounded corners=2pt,inner xsep=2pt,
%                       shading=axis,top color=white,bottom color=white,middle color=gray!50,inner ysep=1pt}}%shading angle=90,
\tikzset{view/.style={word, draw=blue!50,double=gray!10,thin,inner xsep=0.4pt,%
                            shading=axis,top color=gray!10,bottom color=gray!10,middle color=gray!25,}}

\tikzfading[name=f, top color=transparent!100, bottom color=transparent!0, middle color=transparent!80]

%------------------------------------------
%% Petri Nets
\tikzset{every place/.style={circle,draw=blue,fill=blue!20,thick}}
\tikzset{shared/.style={draw=green,fill=green!30!gray,thick}}
\tikzset{every transition/.style={draw=red,fill=red!20,thick,minimum width=10mm,minimum height=1mm}}
\tikzset{dottedrectangle/.style={rounded corners,draw,dotted,inner sep=1ex}}

\tikzset{mysnake/.style={decorate,decoration={snake,amplitude=0.2mm,segment length=1mm,pre length=1mm,post length=1mm}}}


%------------------------------------------
%% Non atomic
\tikzset{looplabel/.style={scale=0.7,inner xsep=2pt,draw=gray!10,fill=white,rounded corners=2pt,anchor=center}}
 

%------------------------------------------
%% View Abstraction - Contexts

\tikzset{context/.style={rectangle,minimum width=3.2ex,minimum height=3.2ex,inner sep=0pt,fill=none,draw=black,thin,scale=0.6,anchor=center,fill=yellow!20!white}}
\tikzset{project/.style={draw,shorten >=1pt,>=stealth',very thin,blue!70!red}}
\tikzset{context-matrix/.style={matrix of nodes,inner xsep=0pt,column sep=1.5pt,column 1/.style={anchor=base}}}
\tikzset{na-looplabel/.style={inner sep=2pt,draw=gray,fill=white,rounded corners=2pt,anchor=base,yshift=-2pt}}

%------------------------------------------
%% Shape Analysis

\colorlet{gcolor}{green!30}
\colorlet{t1color}{yellow!50}
\colorlet{t2color}{pink!50}
\colorlet{t3color}{purple!50}

\tikzset{pointsto/.style={*-stealth,semithick}}%,decorate,decoration={snake,pre=lineto,pre length=1.5em,post=lineto,post length=1em,amplitude=2pt}},
\tikzset{varpointsto/.style={-stealth,thick,black!50!white,dotted}}
\tikzset{data/.style={circle,fill=#1}}
\tikzset{globals/.style={ellipse,fill=gcolor}}
\tikzset{thread1/.style={circle,fill=t1color}}
\tikzset{thread2/.style={circle,fill=t2color}}
\tikzset{thread3/.style={circle,fill=t3color}}
% >=stealth,
%\tikzset{age/.style={label={[color=white,fill=black,font=\scriptsize,circle,inner sep=0pt,scale=0.6,label distance=-4pt,on background layer]below right:${#1}$}}}
\tikzset{agebubble/.style={color=white,fill=black!30,font=\scriptsize,circle,inner sep=0pt,scale=0.6}}
\tikzset{age/.style={agebubble,anchor=north west,below right=-1pt of #1}}
\tikzset{var/.style={minimum size=1.8em,anchor=center}}

%% Observer states
\tikzset{obsstate/.style={circle,thick, minimum size=4ex, draw=#1!50,fill=#1!20}}
\tikzset{obsstate/.default=blue}
\tikzset{property/.style={rounded corners=1pt,inner xsep=1mm,draw=black!50,fill=white,double}}
\tikzset{initial/.style={initial by arrow,initial text=,initial where=left,fill=green!50,draw=green!60!black}}
\tikzset{accepting/.style={accepting by double,fill=red!80,draw=red}}

%% Commit points
\tikzset{commitpoint/.style={%
        shape=circle,
        inner sep=0pt,
        minimum size=2pt,
        draw=red,fill=red}}
\tikzset{infig/.style={scale=0.04}} % weird hack

\newlength{\blockwidth}
\setlength{\blockwidth}{0.5\linewidth}
\addtolength{\blockwidth}{-24pt}
\tikzset{codeblock/.style={text width=\blockwidth, inner xsep=12pt, inner ysep=4pt, %
                           rounded corners, draw = gray!20, below right,draw,fill=gray!1!white}}

%% ==============================
%% Experiments
%%      - MOSI message
\tikzset{message/.style={midway,draw,rounded corners,rectangle split, rectangle split parts=2, rectangle split part fill={yellow!10!white, pink!10!white}}}

%% ==============================
%% Conference name for the papers section
\tikzset{paper/.style={text=black,rectangle,rounded corners=0.5ex,thin,draw=black,minimum width=4ex,text centered,%
                       shading=axis,top color=gray!10,bottom color=gray!10,middle color=gray!30}}
\tikzset{conference/.style={scale=0.8,text=black,rectangle,rounded corners=0.5ex,thin,draw=blue!50,%
                            shading=axis,top color=blue!10,bottom color=blue!10,middle color=blue!20}}



%% -----------------------------------------------------------
%% Tikz input
%% -----------------------------------------------------------
\pgfrealjobname{thesis}

\newcommand{\tikzinput}[2][]{%
  \ifexternaltikz%
  \ifstrempty{#1}{%
    \beginpgfgraphicnamed{\builddir/#2}\input{#2}\endpgfgraphicnamed%
  }{\resizebox{#1}{!}{%
    \beginpgfgraphicnamed{\builddir/#2}\input{#2}\endpgfgraphicnamed%
  }}%
  \else%
    \ifstrempty{#1}{\input{#2}}{\resizebox{#1}{!}{\input{#2}}}%
  \fi%
}

\newcommand{\listinginput}[1]{\lstinputlisting[style=custom]{#1}}


%% -----------------------------------------------------------
%% Spaces around Floating elements (like figure)
%% See: http://tex.stackexchange.com/questions/60477/remove-space-after-figure-and-before-text
%% -----------------------------------------------------------
%%---- Spacing around the captions
\setlength{\abovecaptionskip}{0pt plus 2pt}
\setlength{\belowcaptionskip}{3pt plus 42pt}
%%---- length between two adjacent floats
\setlength\floatsep{1.25\baselineskip plus 3pt minus 2pt}
%%---- length between text and float placed at top or bottom.
\setlength\textfloatsep{1.25\baselineskip plus 3pt minus 2pt}
%%---- length between text and float placed in the middle (with 'here').
%\setlength\intextsep{1.25\baselineskip plus 3pt minus 2 pt}
\setlength{\intextsep}{0pt} %% Useful for the wrapfigure env

%\setlength{\columnsep}{1ex}

\newlength{\dazintextsep}
%\setlength\dazintextsep{1.25\baselineskip plus 3pt minus 2 pt}
\setlength\dazintextsep{\the\smallskipamount}


%% -----------------------------------------------------------
%% Algorithms
%% -----------------------------------------------------------
\usepackage[nofillcomment,noend,linesnumbered,noline,ruled]{algorithm2e}
%\usepackage[noend]{algorithmic}
\usepackage{listings}

\lstdefinestyle{custom}{
  showspaces=false,               % show spaces adding particular underscores
  showstringspaces=false,         % underline spaces within strings
  showtabs=false,                 % show tabs within strings adding particular underscores
  %belowcaptionskip=1\baselineskip,
  breaklines=true,
  frame=BT,                       % Lines above and below
  %frame=B,                        % Lines below only
  %xleftmargin=\parindent,
  language=C,
  basicstyle=\footnotesize\ttfamily,
  keywordstyle=\bfseries,%\color{green!40!black},
  commentstyle=\itshape\color{purple},
  %identifierstyle=\color{blue},
  %stringstyle=\color{orange},
  escapechar=@,
  %escapeinside={\%*}{*)}          % if you want to add a comment within your code 
  mathescape=true,
  captionpos=b,                   % sets the caption-position to bottom
  breaklines=true,                % sets automatic line breaking
  breakatwhitespace=false,        % sets if automatic breaks should only happen at whitespace
}

\fvset{fontfamily=helvetica,numbers=left,numbersep=5pt,stepnumber=1,firstnumber=0,numberblanklines=true,commandchars=\\\[\],codes={\catcode`$=3\catcode`^=7}}
%codes={\catcode`$=3}
\renewcommand{\theFancyVerbLine}{\tiny \arabic{FancyVerbLine}}

%% -----------------------------------------------------------
%% Comments
%% -----------------------------------------------------------
\ifcomments %% =====================================================

\setlength{\fboxsep}{2pt}

\newdimen\len
\len=\marginparwidth
\advance\len by -\marginparsep
\advance\len by -\marginparsep
%\advance\len by -\marginparsep

\newcounter{notec}
\newcommand{\note}[1]{%
  \stepcounter{notec}%
  {$^{\footnotesize\textcolor{red}\bf (\arabic{notec})}$}%
  \leavevmode%
  \marginpar[\fbox{\parbox{\len}{$^{\footnotesize\textcolor{red}\bf (\arabic{notec})}$ \footnotesize\raggedleft #1}}]%
  {\fbox{\parbox{\len}{$^{\footnotesize\textcolor{red}\bf (\arabic{notec})}$ \footnotesize\raggedright #1}}}%
}%

%% Local note, as footnote
\newcommand{\lnote}[1]{\footnote{#1}}

% For the large comments
\usepackage{framed}
\newenvironment{comment}{\begin{framed}}{\end{framed}}

%% Thesis Keywords (included in the index)
\usepackage{marginfix}
%\usepackage[noadjust]{marginnote}
\newcommand*{\KW}[1]{\leavevmode%
{%\mbox{}
\marginpar[\fbox{\parbox[b]{\len}{\raggedleft\footnotesize #1}}]{\fbox{\parbox[b]{\len}{\raggedright\footnotesize #1}}}%
%\marginnote[\fbox{\parbox[b]{\len}{\raggedleft\footnotesize #1}}]{\fbox{\parbox[b]{\len}{\raggedright\footnotesize #1}}}[0pt]%
%\index{THESIS KEYWORDS!#1}
}%
%\ignorespaces%
}
% \newcommand*{\KW}[1]{%
% {\mbox{}\marginpar{\tikz[baseline=(n.base)]\node[text width=\len,draw,fill=yellow!20,text centered,inner sep=3pt,rounded corners=4pt](n) at (0,0){\footnotesize #1};}%
% \index{THESIS KEYWORDS!#1}}%
% }


\else %% ==========================================================

\newcommand{\note}[1]{}
\newcommand{\lnote}[1]{}
\newcommand*{\KW}[1]{}

\newsavebox\commentb %% Saving and throwing away the comment env
\newenvironment{comment}{\setbox\commentb\hbox\bgroup}{\egroup}

\fi %% ============================================================


%% -----------------------------------------------------------
%% So far so good
%% -----------------------------------------------------------
\newcommand*{\sofarsogood}{\ifunderprogress\par\bigskip\noindent\hrulefill\begingroup\tiny\raisebox{-0.5ex}{\Chalk\ SO FAR SO GOOD\ }\endgroup\hrulefill\par\bigskip\fi}
\newcommand*{\sfsg}{\sofarsogood\endinput}
\newcommand*{\cutafter}{\endinput}

%% -----------------------------------------------------------
\ifunderprogress
\newenvironment{todo}{\par\bigskip\noindent\hrulefill\raisebox{-0.5ex}{\Chalk\ TODO\ }\hrulefill\par\vspace{1em}}{\par\vspace{1em}\noindent\hrulefill}
\else
\newsavebox\todob %% Saving and throwing away the todo env
\newenvironment{todo}{\setbox\todob\hbox\bgroup}{\egroup}
\fi


%% ///////////////////////////////////////////////////////////
%% -----------------------------------------------------------
%%                        Definitions                         
%% -----------------------------------------------------------
%% ///////////////////////////////////////////////////////////
%% Must be after styles
%% -----------------------------------------------------------

\newcommand{\lukasshort}{Luk\'a\v s}
\newcommand{\lukas}{Luk\'a\v s Hol\'ik}
\newcommand{\ahmed}{Ahmed Rezine}
\newcommand{\noomene}{Noomene Ben~Henda}
\newcommand{\jonathan}{Jonathan Cederberg}
\newcommand{\bengt}{Bengt Jonsson}
\newcommand{\parosh}{Parosh A. Abdulla}

%\newtheorem{lemma}{Lemma}%{\bfseries}{\itshape}

%% ---------------------------------------------
%% What We Learned in Chapter X
%% ---------------------------------------------
\newcommand{\whatwelearned}[1]{
  % \clearpage
  % \refstepcounter{section}\section*{\thesection\hspace{0,5em}What we learned in Chapter~\ref{chapter:parameterized:systems}}
  \chapter*{What we learned in Chapter~\thechapter}
  \KW{Summary Ch.~\thechapter}%
  \index{Chapter summaries}%
  \begin{description}
\item[A solution to the reachability problem] from
  Section~\ref{section:reachability:problem}.
\item[Abstraction and Concretization] functions define how to travel
  between sets of views and sets of configurations. An important
  notion to retain is that views work collectively to characterize
  configurations.
\item[Verfication Procedure] is composed of two nested loops, one of
  which is a~simple fixpoint. The other loop searches for a~cut-off
  point.
\item[Soundness.] The method computes an invariant that covers the
  reachable configurations of any size, using views of small sizes.
\item[Completeness.] The method is complete for WQO and for almost
  downward-closed invariants.
\item[Approximation] is introduced in order to leverage the
  entailement on views and makes it easier to compute.
\item[Acceleration] is achieved by seeding the fixpoint computation with more views.
\item[Requirements.] The procedure requires to be able to compute the
  initial views, test for the characterization of bad configurations
  (using the upward-closedness of the set of bad configurations nad
  checking for the presence of some ``bad'' views).
\item[Efficiency.] The method has proven to be very efficient as shown
  in the results (see Paper~\ref{paper:VMCAI13}
  and~\ref{paper:SAS14}). It exhibits the small model properties,
  i.e.\ most patterns occur in small instances.
\end{description}

}

%% ---------------------------------------------
\newenvironment{statement}{\begin{quote}\raggedleft}{\end{quote}}

%% Settings for enumerations and item lists
\setlist{noitemsep}
\setlist[enumerate,1]{itemsep=1ex}
%\setlist[enumerate]{align=right,labelindent=\parindent, leftmargin=*,widest*=4}
\setlist[enumerate]{align=right,leftmargin=*,widest*=4}
%\setlist[itemize]{labelindent=0pt,align=right,leftmargin=*}

%% Framing the lists
\usepackage[framemethod=TikZ]{mdframed}
\mdfdefinestyle{DazFrame}{%
    linecolor=gray!20!white,outerlinewidth=1pt,roundcorner=1ex,
    innertopmargin=1ex,innerrightmargin=2ex,innerbottommargin=1ex,innerleftmargin=1ex,
    backgroundcolor=white}

\newenvironment{strategy}{%
  \renewcommand{\labelenumi}{\protect\tikz[baseline=(n.base)]{\protect\node[enumbullet](n){\arabic{enumi}};}}
  % No need to redefine \theenumi since there is no cross-referencing
  \begin{mdframed}[style=DazFrame]%
  \begin{enumerate}}{\end{enumerate}\end{mdframed}}

\newenvironment{challenges}{%
  \renewcommand{\labelenumi}{\theenumi}%
  \renewcommand{\theenumi}{\protect\tikz[baseline=(n.base)]{\protect\path node[challenge](n){\Alph{enumi}};}}%
  % \smallskip\hrule\smallskip% 
  \begin{mdframed}[style=DazFrame]
    % \begin{enumerate}[leftmargin=0pt]
    \begin{enumerate}%
    }{\end{enumerate}%
    % \smallskip\hrule\smallskip
  \end{mdframed}%
}
\newenvironment{subchallenges}{%
  \renewcommand{\labelenumii}{\theenumii}%
  \renewcommand{\theenumi}{}%
  \renewcommand{\theenumii}{\protect\tikz[baseline=(n.base)]{\protect\path node[challenge](n){\Alph{enumi}} node[subchallenge,right=-1pt of n.south east]{\arabic{enumii}};}}%
  \begin{enumerate}}{\end{enumerate}}


%% ---------------------------------------------
%% Process state
%% ---------------------------------------------
\newcount\loopcounter
\newcommand{\w}[2][w]{%
  \loopcounter=-1%
\begin{tikzpicture}[baseline=(n0.base),start chain,node distance=0.4pt]%
  \foreach \c in {#2}{
    \global\advance\loopcounter by1
    \node[on chain,state,state-#1](n\the\loopcounter){\c};
  }
  \begin{pgfonlayer}{my background}
    \node[fit=(n0)(n\the\loopcounter),word,word-#1]{};
  \end{pgfonlayer}
\end{tikzpicture}%
}

% One state only
%\newcommand{\s}[2][w]{\ifnotikz\fbox{#1}\else\tikz[baseline=(n.base)]\node[state,state-#1](n){#2};\fi}
\newcommand{\s}[2][w]{\w[#1]{#2}}

%% Bad patterns: 2 states and some waves around
\newcommand{\badpattern}[2]{%
  \begin{tikzpicture}[baseline=(a.base),decoration={snake,segment length=0.8mm,amplitude=0.5pt}]%
    \node[state,state-b,outer sep=0pt](a){#1};
    \node[state,state-b,outer sep=0pt,right=3mm of a](b){#2};
    \coordinate[left=3mm of a](a');
    \coordinate[right=3mm of b](b');
    \draw[decorate] (a) -- (a') (a) -- (b) (b) -- (b');
    \begin{pgfonlayer}{my background}
      \node [fit=(a')(a)(b)(b'),rectangle,rounded corners=0.5ex,shading=axis,top color=white,bottom color=white,middle color=gray!30,inner ysep=1pt]{};%shading angle=90,
    \end{pgfonlayer}
  \end{tikzpicture}%
}


%% ---------------------------------------------
%% Switch example
%% ---------------------------------------------
\newcommand{\switch}[3][]{%
  \ifstrempty{#1}{%
    \mbox{\ensuremath{\llbracket\mathtt{#2}\!\mid\!\mathtt{#3}\rrbracket}}%
  }{%
    {\scriptsize\ensuremath{<\!\mathtt{#2},\mathtt{#3},\text{counter}=#1\!>}}%
  }%
}


%% -----------------------------------------------------------
%% Table of content
% -----------------------------------------------------------
% Numbering of headings down to the subsection level
%\numberingdepth{subsection} % from UUThesisTemplate.cls
\setcounter{secnumdepth}{2}
% Including headings down to the subsection level in contents
%\contentsdepth{section} % from UUThesisTemplate.cls
\setcounter{tocdepth}{1} %% Only Chapters and Sections

% \usepackage{minitoc}
% \newcommand{\initializepartialtoc}{\protect\dominitoc[n]}
% \newcommand{\adjusttocfornonnumberedchapters}{\mtcaddchapter\mtcaddchapter\mtcaddchapter} % Notifying Minitoc about the extra chapter* above
% \newcommand{\chaptertoc}{%
%   \adjustmtc%
%   \noindent\begin{tikzpicture}
%     \coordinate(c);%\node(c){In this chapter};
%     \node[anchor=north east,inner xsep=1ex, inner ysep=0pt](list) at (\linewidth,0){%inner sep=1ex,draw=gray!10!white,rounded corners=1ex, fill=white,
%       \parbox{0.85\linewidth}{\nomtcrule\minitoc[e]}%
%     };
%     \begin{pgfonlayer}{my background}
%       \path[rounded corners=1ex, shading=axis,top color=gray!10,bottom color=white]%,shading angle=90]
%       (c) -- (list.north east) -- (list.south east) -- (list.south west) to[out=90,in=0] ([yshift=-1em]c) -- cycle;
%       \path[shading=axis,top color=gray!20, bottom color=gray!10,shading angle=90] (c) ++(0.5em,-0.6em) rectangle ([shift={(-0.5em,-0.4em)}]list.north east);
%     \end{pgfonlayer}
%   \end{tikzpicture}\bigskip}
% %\mtcsettitle{minitoc}{In this chapter}
% \mtcsetdepth{minitoc}{1}
% \mtcsetoffset{minitoc}{0pt}
% \mtcsetfont{minitoc}{section}{\small\rmfamily\upshape}
% \renewcommand\mtcindent{0pt}
% \renewcommand\mtcskip{0pt}
% \renewcommand\kernafterminitoc{\kern0pt}
% \makeatletter
% \renewcommand{\mtc@strut}{}
% \renewcommand{\mtc@rule}{}
% \def\mtc@verse#1{\let\\=\@centercr
%  \list{}{%
%  %\itemsep=\z@
%    \itemindent=0pt \partopsep=0pt \listparindent=0pt \topsep=0pt \leftmargin=0pt \rightmargin=0pt \parsep=0pt
%  % \addtolength{\leftmargin}{+#1}%
%  % \addtolength{\rightmargin}{-#1}%
%  }%
%  \item[]}
% \makeatother

% \usepackage{titletoc}
% \newcommand{\initializepartialtoc}{}
% \newcommand{\adjusttocfornonnumberedchapters}{}
% \newcommand{\chaptertoc}{%
%   \addtocontents{toc}{\protect\setcounter{tocdepth}{2}}
%   \startcontents[chapters]
%   \noindent\begin{tikzpicture}
%     \coordinate(c);
%     \node[anchor=north east,inner ysep=1em,inner xsep=0pt](list) at (\linewidth,0){\parbox{0.85\linewidth}{\printcontents[chapters]{}{1}{}}};
%     \begin{pgfonlayer}{my background}
%       \path[rounded corners=1ex, shading=axis,top color=gray!10,bottom color=white]%,shading angle=90]
%       (c) -- (list.north east) -- (list.south east) -- (list.south west) to[out=90,in=0] ([yshift=-1em]c) -- cycle;
%       \path[shading=axis,top color=gray!20, bottom color=gray!10,shading angle=90] (c) ++(0.5em,-0.6em) rectangle +({\dimexpr\linewidth-1em},0.2em);
%     \end{pgfonlayer}
%   \end{tikzpicture}\bigskip%
% }
% \contentsmargin{0pt}
% % \titlecontents{section}
% %               [2.8em] % 2.3m + 0.5em
% %               {}
% %               {\contentslabel{2.3em}}
% %               {\hspace*{2.3em}}
% %               {\titlerule*[5pt]{.}\contentspage}

%  %% Redefining the chapter titles when they include a mini-ToC
% \makeatletter
% \newcommand\chapterwithtoc[1]{%
%   \if@openright\cleardoublepage\else\if@UU@chapterafterpart\cleardoublepage\else\clearpage\fi\fi
%   \@UU@chapterafterpartfalse
%   \thispagestyle{UU@chapter}
%   \suppressfloats[t]
%   \@startsection {chapter}{0}{\z@}{\z@}{1em plus 1em minus 1em}{%
%     \chapterfont%
%     \LARGE%
%     \UU@RaggedRight%
%     \hyphenpenalty=10000%
%   }{#1}%
%   \chaptertoc
% }
% \makeatother


%% If no mini-ToC
\newcommand{\initializepartialtoc}{}
\newcommand{\adjusttocfornonnumberedchapters}{}
\newcommand{\chaptertoc}{}
\let\chapterwithtoc\chapter


%% -----------------------------------------------------------
%% Hyperref
%% -----------------------------------------------------------
% Document links and bookmarks
%\def\texorpdfstring#1#2{#1}
\usepackage[pdftex,bookmarks]{hyperref}

\hypersetup{pdfauthor={Frédéric Haziza}}
\hypersetup{pdftitle={\@title}}
\hypersetup{pdfsubject={\@subtitle}}
\hypersetup{pdfkeywords={PhD Thesis, Parameterized Verification, Monotonic Abstraction, View Abstraction, Shape Analysis}}
\hypersetup{pdfcreator={pdflatex, bibtex, makeindex}}
%\hypersetup{pdfproducer={pdfLatex}}

%\usepackage[pdftex]{thumbpdf} % Create thumbnails

%% -----------------------------------------------------------
%% Other macros
%% -----------------------------------------------------------
%% ---------------------------------------------
%% Math stuff
%% ---------------------------------------------

%\newtheorem{definition}{Definition}[chapter]

%\newcommand{\set}[1]{\left\{#1\right\}}
\newcommand{\set}[1]{\{#1\}}
%\newcommand{\setcomp}[2]{\{{#1}\mathrel{}\middle|\mathrel{}{#2}\}}
%\newcommand{\setcomp}[2]{\set{#1\mid\;#2}}
\newcommand{\setcomp}[2]{\{{#1}\mathrel{}\mid\mathrel{}{#2}\}}

\newcommand{\cross}{\textcolor{red}{\ding{55}}}
\newcommand{\tickk}{\textcolor{green}{\ding{52}}}

\newcommand{\nat}{\ensuremath{\mathbb N}}
\newcommand{\reals}{\ensuremath{\mathbb R}}
\newcommand{\sizeof}[1]{|#1|}
\newcommand{\union}{\cup}
%\newcommand{\minsetunion}{\sqcup}
\newcommand{\range}[2]{\llbracket{#1}{,}{#2}\rrbracket} %% {,} otherwise I get some spacing after the ','

% \newcommand{\updateby}[2]{\ensuremath{\left[#1\leftarrow#2\right]}}

%% ---------------------------------------------
%% Parameterized systems
%% ---------------------------------------------
\newcommand\tuple[1]{\left\langle#1\right\rangle}
\newcommand{\parsys}{\ensuremath{\mathcal P}}
\newcommand{\locs}{\ensuremath{Q}}
\newcommand{\rules}{\ensuremath{\Delta}}

\newcommand{\witnesses}{S}
\newcommand{\quantrule}[5]{ \mathbf{if}\ {#3}~j\,{#4}\,i:\,{#5}\ \mathbf{then}\ {#1}\trans{#2}}
\newcommand{\quantify}{\mathbb Q}

\newcommand{\confs}{\ensuremath{\mathcal C}}
\newcommand{\trans}{\ensuremath{\rightarrow}}
\newcommand{\transof}[1]{\stackrel{#1}{\trans}}
\newcommand{\transys}{\ensuremath{\mathcal T}}

\newcommand{\src}{\mathtt{src}}
\newcommand{\dst}{\mathtt{dst}}


\newcommand{\borule}[4]{\mbox{\bf when }#1\mbox{ \bf provided }#2\mbox{ \bf broadcast }#3\mbox{ \bf emit }#4}



\newcommand{\rrule}[3]{\mbox{\bf when }#1\mbox{ \bf provided }#2\mbox{ \bf emit }#3}
\newcommand{\frule}[5]{\mbox{\bf from } #1 \mbox{ \bf when }#2\mbox{ \bf provided }#3\mbox{ \bf emit }#4\mbox{ \bf goto } #5}

\newcommand{\frulenostate}[3]{ \mbox{ \bf when }#1\mbox{ \bf provided }#2\mbox{ \bf emit }#3}

\newcommand{\eventseq}{w}
\newcommand{\veventseq}{v}


\newcommand{\rulename}{\rho}
\newcommand{\sndrule}[6]{\mbox{\bf from } #1 \mbox{ \bf when }#2\mbox{ \bf provided }#3\mbox{ \bf emit }#4\mbox{ \bf broadcast } #5 \mbox{ \bf goto } #6}
\newcommand{\sndrulenostate}[4]{\mbox{ \bf when }#1\mbox{ \bf provided }#2\mbox{ \bf emit }#3\mbox{ \bf broadcast } #4}


\newcommand{\rcvrule}[5]{\mbox{\bf from } #1  \mbox{ \bf when }#2\mbox{ \bf provided }#3\mbox{ \bf emit }#4 
\mbox{ \bf goto } #5}
\newcommand{\rcvrulenostate}[3]{\mbox{ \bf when }#1\mbox{ \bf provided }#2\mbox{ \bf emit }#3}

\newcommand{\rcvruletwostate}[6]{\mbox{\bf from } #1 {,} #2  \mbox{ \bf when }#3\mbox{ \bf provided }#4\mbox{ \bf emit }#5 \mbox{ \bf goto } #6}


\newcommand\snd[1]{{#1}!}
\newcommand\rcv[1]{{#1}?}
\newcommand{\ctrlof}[1]{{\tt cntrl}\left(#1\right)}
\newcommand{\augof}[1]{{\tt aug}\left(#1\right)}
\newcommand{\rvalues}{\theta}
\newcommand{\param}{p}
\newcommand{\ord}{{\tt ord}}
\newcommand\before{\tt before}
\newcommand\after{\tt after}

\newcommand{\subsumed}{\sqsubseteq}



%\newcommand{\bad}{\mathit{Bad}}
\newcommand{\Bad}{\ensuremath{\mathcal{B}}}
\newcommand{\minbad}{\ensuremath{\Bad_{min}}}
%\newcommand{\minbad}{B}
\newcommand{\Reach}{\ensuremath{\mathcal{R}}}
\newcommand{\Inits}{\ensuremath{\mathcal{I}}}

\newcommand{\domain}{\ensuremath{\mathcal{D}}}
\newcommand{\entails}{\preccurlyeq}
\newcommand{\preorder}{\leqslant}%\vartriangleleft

% \newcommand{\BinRel}{\mathcal B} % Binary relation
% \newcommand{\rel}{R}

\newcommand{\forrule}[5]{\ensuremath{\mathbf{if~foreach}\ j\mathrel{#1}i: {#2} \ \mathbf{then}\ {#3}\trans{#4}\ \mathbf{else}\ {#3}\trans{#5}}}

%% ---------------------------------------------
%% Monotonic abstraction
%% ---------------------------------------------
\newcommand{\thread}{{\tt th}}
\newcommand{\subword}{\sqsubseteq}
% \newcommand{\word}[3][0]{{#2}_{#1}  \ldots  {#2}_{#3}}
\newcommand{\ucl}[1]{\ensuremath{\lfloor{#1}\rfloor}} %% Upward-Closure
\newcommand{\gen}[1]{min(#1)}

\newcommand{\parabol}[3][]{\draw[black, very thin, fill=white,#1] (0,0) parabola[parabola height=#3, bend pos=0.5] ++(#2,0)}
\newcommand{\atrans}{\leadsto}
\newcommand{\atransof}[1]{\stackrel{#1}{\atrans}}

\newcommand{\abstrans}[1]{\hat{#1}}

\newcommand{\worklist}{\texttt{W}}
\newcommand{\visited}{\texttt{V}}
\newcommand{\inverse}[1]{\ensuremath{{#1}^{\textit{-}1}}} % Using hyphen and not the longer "minus".
% \newcommand{\pre}{\ensuremath{\inverse{\abstrans\rules}}}
% %\newcommand{\prestar}{\ensuremath{{\abstrans\rules}^{\textit{-}*}}}
% \newcommand{\prestar}{\ensuremath{(\pre)^{*}}}
\newcommand{\pre}{Pre}
\newcommand{\prestar}{\ensuremath{Pre^{*}}}


%% ---------------------------------------------
%% View abstraction
%% ---------------------------------------------
\newcommand{\dcl}[1]{\ensuremath{\lceil{#1}\rceil}} %% Downward-Closure

\newcommand{\Abs}{\alpha} 
\newcommand{\Conc}{\gamma}

\newcommand{\Absof}[1]{\Abs_{#1}}
\newcommand{\Concof}[1]{\Conc_{#1}}
%\newcommand{\Concoflim}[2]{\Conc_{#1}^{#2}}
\newcommand{\Concoflim}[2]{\oint_{#1}^{#2}}
%\newcommand{\minabstrof}[1]{\minsetof{\abstr_{#1}}}

\newcommand {\views}{\mathcal{V}}
\newcommand {\viewsof}[1]{\views_{#1}}
%\newcommand {\confsof}[1]{\confs_{#1}}

\newcommand {\badviewsof}[1]{\views_{#1}^{\mathit{bad}}}
\newcommand {\badviews}{\views^{\mathit{bad}}}

% \newcommand {\mk}[1]{{#1}^\bullet}
\newcommand {\proj}[2]{\Pi_{#1}(#2)}
% \newcommand {\tproj}[2]{\Pi_{#1}^\circ(#2)}
% \newcommand {\naproj}[2]{\Pi'_{#1}(#2)}

\newcommand {\post}{\mathit{post}}
\newcommand {\spost}{\mathit{spost}}
\newcommand {\apost}[1]{{\mathit{Apost}}_#1}
\newcommand {\sdelta}{\delta^\#}

\newcommand{\entailedby}{\succcurlyeq}
\newcommand{\minsetof}[1]{\lfloor #1 \rfloor}
\newcommand{\minunion}{\sqcup}

\newcommand{\base}{\mathtt{base}}
\newcommand{\ctx}{\mathtt{ctx}}

% \usepackage{wasysym}
% \newcommand{\aConcoflim}[2]{\ensuremath{\mathbin{\ooalign{\hspace{.2ex}\raisebox{.15ex}{\scalebox{.7}{\wasylozenge}}\cr$\int$\cr}}_{#1}^{#2}}}
%\newcommand{\aConcoflim}[2]{\ensuremath{\mathbin{\ooalign{\hspace{.2ex}\raisebox{.15ex}{\scalebox{.6}{$\square$}}\cr$\int_{#1}^{#2}$\cr}}}}
\newcommand{\aConcoflim}[2]{\sqint_{#1}^{#2}}
%\newcommand{\F}{\ensuremath{\mathtt{F}}}
\newcommand{\isbad}{\ensuremath{\mathtt{bad}}}

\newcommand{\tick}{\checkmark}
%\newcommand{\tickof}[2]{\checkmark({#1},{#2})}
\newcommand{\unticked}{\rho}%\xi \chi
\renewcommand{\next}{\mathit{next}}

\newcommand{\makehighgroup}[4][]{
  \draw[blue!70!red,#1] (#2.north west) ++(0,1mm) -- +(0,1mm) -- ([yshift=2mm]#3.north east) coordinate[midway](n#4) -- +(0,-1mm);
}
\newcommand{\makelowgroup}[4][]{
  \draw[#1] (#2.south west) ++(0,-1mm) -- +(0,-0.5mm) -- ([yshift=-1.5mm]#3.south east) coordinate[midway](p#4) -- +(0,0.5mm);
}
%% ---------------------------------------------
%% Shape Analysis
%% ---------------------------------------------
\newcommand{\frag}{\mathtt{v}}
\newcommand{\fragset}{V}
\newcommand\true{{\tt true}}
\newcommand\false{{\tt false}}
\newcommand*{\prgcode}[1]{\texttt{#1}}
\newcommand{\dset}{\mathbb{D}}
\newcommand\commitpoint[1]{\tikz{\node[commitpoint,#1]{};}}
\newcommand\stepa{\tikz{\node[draw, circle, fill = gray!20, draw = black, name = n1, minimum width=12pt, minimum height=12pt,anchor=south west,inner sep=0pt,scale=0.8]{\tt 1}}}
\newcommand\stepb{\tikz{\node[draw, circle, fill = gray!20, draw = black, name = n1, minimum width=12pt, minimum height=12pt,anchor=south west,inner sep=0pt,scale=0.8]{\tt 2}}}
\newcommand\stepc{\tikz{\node[draw, circle, fill = gray!20, draw = black, name = n1, minimum width=12pt, minimum height=12pt,anchor=south west,inner sep=0pt,scale=0.8]{\tt 3}}}
\newcommand\stepd{\tikz{\node[draw, circle, fill = gray!20, draw = black, name = n1, minimum width=12pt, minimum height=12pt,anchor=south west,inner sep=0pt,scale=0.8]{\tt 4}}}
\newcommand\stepe{\tikz{\node[draw, circle, fill = gray!20, draw = black, name = n1, minimum width=12pt, minimum height=12pt,anchor=south west,inner sep=0pt,scale=0.8]{\tt 5}}}
\newcommand\stepf{\tikz{\node[draw, circle, fill = gray!20, draw = black, name = n1, minimum width=12pt, minimum height=12pt,anchor=south west,inner sep=0pt,scale=0.8]{\tt 6}}}

\newcommand\threada{\tikz{
\node[draw, circle, fill = red!20, double = red, name = addT, minimum width=15pt, minimum height=15pt,anchor=south west,inner sep=0pt, scale=0.8]{{$\tt T_1$}};
}}

\newcommand\threadb{\tikz{
\node[draw, circle, fill = blue!20, double = blue, name = addT, minimum width=15pt, minimum height=15pt,anchor=south west,inner sep=0pt, scale=0.8]{{$\tt T_2$}};
}}

\newcommand\threadc{\tikz{
\node[draw, circle, fill = orange!20, double = orange, name = addT, minimum width=15pt, minimum height=15pt,anchor=south west,inner sep=0pt, scale=0.8]{{$\tt T_3$}};
}}


\newcommand\taa{\tikz{
\node[name = x, circle, color = red, draw = red, minimum width=12pt, minimum height=12pt,anchor=south west,inner sep=0pt,scale=0.8] at ($(cell1.south east)+(0pt,-9pt)$){{$\tt 1$}};
}}

\newcommand\tab{\tikz{
\node[name = x, circle, color = blue, draw = blue, minimum width=12pt, minimum height=12pt,anchor=south west,inner sep=0pt,scale=0.8] at ($(cell1.north east)+(204pt,63pt)$){{$\tt 2$}};
}}

\newcommand\tac{\tikz{
\node[name = x, circle, color = blue, draw = blue, minimum width=12pt, minimum height=12pt,anchor=south west,inner sep=0pt,scale=0.8] at ($(cell1.north east)+(244pt,63pt)$){{$\tt 3$}};
}}

\newcommand\tad{\tikz{
\node[name = x, circle, color = orange, draw = orange, minimum width=12pt, minimum height=12pt,anchor=south west,inner sep=0pt,scale=0.8] at ($(cell1.north east)+(284pt,63pt)$){{$\tt 4$}};
}}

\newcommand\tae{\tikz{
\node[name = x, circle, color = cyan, draw = cyan, minimum width=12pt, minimum height=12pt,anchor=south west,inner sep=0pt,scale=0.8] at ($(cell1.north east)+(324pt,63pt)$){{$\tt 5$}};
}}


\newcommand\nodea{\tikz{
\node[draw,rounded corners = 0.09cm, fill = red!30, draw = red, name = n1, minimum width=12pt, minimum height=12pt,anchor=south west,inner sep=0pt,scale=0.8] at ($(cell1.north east)+(0pt,0pt)$){{$\tt 1$}};
}}

\newcommand\nodeb{\tikz{
\node[draw, rounded corners = 0.09cm, fill = blue!30, draw = blue, name = n3, minimum width=12pt, minimum height=12pt,anchor=south west,inner sep=0pt,scale=0.8] at ($(cell1.north east)+(0pt,-25pt)$){{$\tt 4$}};
}}

\newcommand\nodec{\tikz{
\node[draw, rounded corners = 0.09cm, fill = orange!30, draw = orange, name = n6, minimum width=12pt, minimum height=12pt,anchor=south west,inner sep=0pt,scale=0.8] at ($(cell1.north east)+(0pt,-50pt)$){{$\tt 6$}};
}}

\newcommand\noded{\tikz{
\node[draw, rounded corners = 0.09cm, fill = blue!30, draw = blue, name = n4, minimum width=12pt, minimum height=12pt,anchor=south west,inner sep=0pt,scale=0.8] at ($(cell1.north east)+(40pt,-25pt)$){{$\tt 2$}};
}}

\newcommand\nodee{\tikz{
\node[draw, rounded corners = 0.09cm, fill = cyan!30, draw = cyan, name = n5, minimum width=12pt, minimum height=12pt,anchor=south west,inner sep=0pt,scale=0.8] at ($(cell1.north east)+(80pt,-25pt)$){{$\tt 8$}};
}}

\newcommand{\triggersym}{\blue{\bullet}}
\newcommand{\triggered}[1]{{#1}^{\triggersym}}

\newcommand{\pointsto}{\mapsto}
\newcommand{\reaches}{\dashrightarrow}
\newcommand{\pointedby}{\mapsfrom}
\newcommand{\reachedby}{\dashleftarrow}
\newcommand{\unrelated}{\Join}

\newcommand{\nullconst}{{\tt \#}}
\newcommand{\undefconst}{{\tt \bot}}
% \newcommand{\freeconst}{{\tt FREE}}
\newcommand{\Pred}{\mathit{Pred}}
%%%%%%  TiKZ styles    %%%%%

\tikzstyle{code-bg}=
[rounded corners,fill=cyan!2!white,draw=cyan!50!white,inner xsep=0pt]%

%\tikzstyle{inference-bg}=
%[rounded corners,fill=blue!5!white,draw=blue!50!white]%

\tikzstyle{inference-bg}=
[rounded corners,fill=gray!6!white,draw=gray!50!white,inner xsep=0pt]%

  
\tikzstyle{linenum}=[font={\footnotesize\tt},anchor=east]
\tikzstyle{linecode}=[font={\footnotesize\tt},anchor=west, scale = 0.9]
\tikzstyle{lpcode}=[anchor=west,text=blue]
\tikzstyle{ostate}=[fill=white,circle,minimum size=0pt,inner sep=0pt,text=black,,font=\tiny]
\tikzstyle{oedge}=[line width=1pt,->,>=stealth]
\tikzstyle{ctrlnode}=[font=\small,scale = 0.9]




%%%%%%  TiKZ layers    %%%%%
%\pgfdeclarelayer{background}
%\pgfdeclarelayer{bbackground}
%\pgfdeclarelayer{foreground}
%\pgfsetlayers{bbackground,background,main,foreground}
%% ---------------------------------------------
%% Experiments
%% ---------------------------------------------
\newcommand{\strue}{\ensuremath{\mathtt{true}}}
\newcommand{\sfalse}{\ensuremath{\mathtt{false}}}
\newcommand{\LD}[1]{\ensuremath{Read_{#1}}}
\newcommand{\ST}[1]{\ensuremath{Write_{#1}}}


